
Każdy plik DICOM posiada "Data Set", czyli zbiór informacji składający się z "Data Element".
"Data Element" jest strukturą przechowującą wszystkie dane dotyczące obrazu.

\paragraph{Budowa "Data Element"}

\begin{itemize}
    \item Data Element Tag - to unikalny identyfikator, złożony z dwóch liczb: grupy(uint16) i elementu(uint16) grupy.
    Obiekt używany do przechowywania taga to \gdcmclass{Tag}.
    Informuje o znaczeniu danych.
    Na przykład: jeżeli tag przyjmie wartość \dicomtag{PatientName}{0010}{0010}, oznacza to, że dane w "Data Element" zawierają nazwę pacjenta.
    \item Value Representation, w skrócie VR – typ danych umożliwiający poprawną interpretację danych.
    Informuje w jaki sposób dane są zapisane.
    Obiekt używany do przechowywania taga to \gdcmclass{VR}.
    Na przykład: Decimal String, w skrócie DS, oznacza liczbę zapisaną za pomocą teksu.
    Czasami to pole może być puste, wtedy należy się odnieść do VR przypisanego do taga, który określa standard.
    \item Value Length, w skrócie VL – rozmiar elementu
    \item Value Field (opcjonalne) – pole z właściwymi danymi
\end{itemize}

\paragraph{Konwerter}

Obiekt klasy \sokarclass{DataConverter} to obiekt zajmujący się konwersją danych z pliku DICOM na dane w formacie odpowiadającej przeglądarce.

Kilka rzeczy które się tyczą wszystkich danych i konwersji:
\begin{itemize}
    \item większość VR jest to zapisane jako tekst, kodowanie i dekodowanie tekstu jest zapewniane przez bibliotekę, a konkretniej przez klasę \gdcmclass{StringFilter}, dlatego nie przejmuje się takimi rzeczami jak zapisem LittleEndian i BigEndian.
    \item większość danych może mieć kilka wartości oddzielonych backslashem "\textbackslash", dlatego konwerter dla VR w, których standard przewiduje wiele wartości, zawsze zwraca \qtclass{QVector} z tymi wartościami
    \item wszystkie dane są zapisane parzystą ilością bajtów, w przypadku tekstu dodaje się znak spacji na końcu danych, taka spacja jest pomijana w analizie danych
\end{itemize}

Klasa obsługuje następujące VR:
\begin{itemize}
    \item AS - Age String - wiek lub długość życia

    Długość danych zawsze wynosi 4 bajty.
    Pierwsze trzy bajty to liczba całkowita zapisana za pomocą tekstu.
    Czwarty bajt to znaku określający jednostkę czasu.
    Standard definiuje cztery możliwe jednostki czasu: "D" jako dzień, "W" jako tydzień, "M" jako miesiąc, oraz "Y" jako jeden rok.
    Konwerter zmienia tą wartość na tekst w postaci czytelnej, np: "18 weeks" lub "3 years", oraz jest wrażliwy na obecny język aplikacji.
    
    Przykład: "018M" oznacza 18 miesięcy, "123D" oznacza 123 dni.

    \item AT - Attribute Tag - inny tag

    Długość danych to zawsze 32 bity, są to dwie 16 bitowe liczby.
    Odpowiednio grupa i element grupy.
    Ten VR jest używany kiedy wskazujemy na inny tag.
    Wartość nie jest nigdy pokazywana użytkownikowi, a jedynie używana w interpretacji przez algotyrm tworzenia filmu.
    
    Przykład: tag \dicomtag{FrameIncrementPointer}{0028}{0009} jest używany kiedy w pliku jest zapisana sekwencja kilku obrazów, wskazuje on na inny tag zawierający informacje w jaki sposób ta sekwencja ma być wyświetlona.
    
    \item DA - Date - data lub dzień

    Długość danych zawsze wynosi 8 bajtów.
    Data zapisana w formacie \texttt{YYYYMMDD}, gdzie: \texttt{YYYY} cztery cyfry roku, \texttt{MM} dwie cyfry miesiąca, \texttt{DD} dwie cyfry dnia w kalendarzu Gregoriańskim.
    Konwerter zamienia dane na obiekt klasy \qtclass{QDate}, który ma w sobie wbudowaną konwersję na tekst zależny od ustawień językowych aplikacji.
    
    Przykład: "\texttt{19800716}" oznacza 16 lipca 1980
    
    UWAGA: Standard "ACR-NEMA Standard 300", czyli poprzednik DICOM definiował date w sposób \texttt{"YYYY.MM.DD"}, według standardu DICOM, taki zapis jest nie poprawny, ale zdarzają się stare obrazy z takimi datami i \sokarclass{DataConverter} obsługuje taki format.

    \item DS - Decimal String - liczba zmienno przecinkowa lub ciąg kilku liczb zmienno przecinkowych zapisanych za pomocą tekstu w notacji wykładniczej

    Długość jednej liczby powinna maksymalne wynosić 16 bajtów.
    Dostępne znaki to "\texttt{0}"-"\texttt{9}", "\texttt{+}", "\texttt{-}", "\texttt{E}", "\texttt{e}", "\texttt{.}".
    Biblioteka QT posiada wbudowany konwerter liczb zapisanych w formacie wykładniczym, dlatego mój konwerter dzieli tekst i konwertuje za pomocą QT.
    
    Przykład: "\texttt{426\textbackslash468 }" oznacza dwie liczby 426 i 468. Proszę zwrócić uwagę na spacje na końcu.

    \item IS - Integer String - liczba całkowita 

    Długość jednej liczby powinna maksymalne wynosić 12 bajtów.
    Dostępne znaki to "\texttt{0}"-"\texttt{9}", "\texttt{+}", "\texttt{-}".
    Biblioteka QT posiada wbudowany konwerter liczb całkowitych, dlatego mój konwerter używa konwertera z QT.
    
    Przykład: "426 " oznacza liczbę 426.

    \item PN - Person Name - nazwa osoby

    Jako, że pacjenta, bądź obiekt badany można nazwać w sposób dowolny i odbiegający od polskiego standardu nazewnictwa, standard DICOM nie przewiduje rozdzielenie poszczególnych składowych nazwy na oznaczone fragmenty.
    "Person Name" dzieli nazwę na podane fragmenty, rozdzielony znakiem \texttt{\^{}} (94 znak kodu ASCII):
    \begin{itemize}
        \item family name complex - nazwisko, np. Smolik
        \item given name complex - imię, np. Adam
        \item middle name - środkowe imię, brak odpowiednika w polskim nazewnictwie
        \item name prefix - prefiks przed imieniem, np: mgr. inż.
        \item name suffix - sufiks po imieniu, brak odpowiednika
    \end{itemize}
    Długość jednego fragmenty powinna maksymalne wynosić 64 znaki.
    W przypadku mniejszej ilości segmentów, mamy założyć, że są puste.
    
    Przykład: "prof. dr. hab. inż. Waldemar Smolik pracownik ZEJIM" był by zapisany w sposób następujący: "\texttt{Smolik\^{}Waldemar\^{}\^{}prof. dr. hab. inż.\^{}pracownik ZEJIM}"

    \item SS - Signed Short - 16 bitowa liczba całkowita bez znaku
    
    Jako, że biblioteka GDCM zajmuje się konwersją Little i Big Endian, mogę dane zinterpretować jako typ \texttt{quint16}

    \item US - Unsigned Short - 16 bitowa liczba całkowita ze znakiem

    Jako, że biblioteka GDCM zajmuje się konwersją Little i Big Endian, mogę dane zinterpretować jako typ \texttt{qint16}
    
    \item UT - Unlimited Text - tekst o nieograniczonej długości.

    Zwykły tekst o długości maksymalnie 2\textsuperscript{32}-2 bajtów.
\end{itemize}