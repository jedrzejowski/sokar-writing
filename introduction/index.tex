
\section{Cel pracy}

Podstawowym celem jest zaliczenie studiów i osiągnięcie tytułu inżyniera.

Celem pracy jest zrobienie oprogramowania do przeglądania obrazów DICOM, która jest niezależna od systemu i środowiska w którym się znajduje

\section{Wprowadzenie}

\section{Obrazowe techniki diagnostyczne}

Diagnostyka obrazowa lub obrazowanie medyczne to dział diagnostyki medycznej zajmujący się tworzeniem i zbieraniem obrazów ludzkiego ciała za pomocą różnych rodzaju oddziaływań fizycznych.

\begin{itemize}
    \item Radiografia - RT
    \item Obrazowanie metodą rezonansu magnetycznego - MRI
    \item Medycyna nuklearna
    \item Ultrasonografia - USG

    \item Elastografia
    %http://www.endokrynologia.net/tarczyca/elastografia
    Metoda polegająca na pomiarze właściwości mechanicznych tkanek sprężystości tkanek

    \item Obrazowanie fotoakustyczne
    \item Tomografia
    \item Echokardiografia
    \item Funkcjonalna spektroskopia bliskiej podczerwieni
    \item Magnetyczne obrazowanie cząstek - MPI
\end{itemize}

\section{Obrazy diagnostyczne}

jakie cechy posinna spełniać przglądrka obrazów

\section{Wybór C++}

Jest wiele środowisk, które w łatwy, miły i przyjemny sposób pozwalają tworzyć oprogramowanie, które jest wstanie pracować na wielu platformach.
Do takich należą Java firmy Oracle, która po skompilowaniu, tworzy jednolity kod bajtowy, który może być uruchomiany na każdej platformie na której działa maszyna wirtualna Javy.
Jednakże takie rozwiązanie sprawia, że nie jesteśmy wstanie osiągnąć pełnego potencjału obliczeniowego maszyny przez pewien poziom wirtualizacji.
Rozwiązaniem jest więc C++, którego kod wynikowy, posiada wysoką wydajnością z bezpośrednim dostępem do zasobów sprzętowych i funkcji systemowych, łatwością dodawania innych bibliotek, napisanych w innych niż języku niż C++.
Dodatkowo jest niezależny od konkretnej platformy sprzętowej lub systemowej, co pozwala na przeniesienie kodu na inną platformę.

\section{Układ pracy}