
\section{Cel pracy}

Podstawowym celem jest zaliczenie studiów i osiągnięcie tytułu inżyniera.

Celem pracy jest zrobienie oprogramowania do przeglądania obrazów DICOM, która jest niezależna od systemu i środowiska w którym się znajduje

\section{Wprowadzenie}

\section{Obrazowe techniki diagnostyczne}

Diagnostyka obrazowa lub obrazowanie medyczne to dział diagnostyki medycznej zajmujący się tworzeniem i zbieraniem obrazów ludzkiego ciała za pomocą różnych rodzaju oddziaływań fizycznych.

Obrazowe techniki diagnostyczne to techniki i procesy tworzenia wizualnych reprezentacji wnętrza obiektu do analizy medycznej.
A także wizualne przedstawienie funkcjonowania narządów lub tkanek (fizjologia) w czasie, np. bicie serca.
Głównym celem obrazowanie medycznego jest ujawnienie wewnętrznych struktur ciała.
Obrazowanie medyczne pozwala również na agregacje danych o normalnej anatomii i fizjologii i ich zapisywania, w celu poźniejszej identyfikacji patologii poprzez porównanie jej ze zdrowymi narządami.

Istnieje wiele technik obrazowania wykorzystujące różne zjawiska fizyczne zachodzące w materii.

Kilka wybranych technik obrazowania medycznego:
\begin{itemize}
    \item Radiografia - RTG

    Najstarsza i najbardziej rozpoznawalna technika obrazowania.
    Pierwsze zdjęcie analogowe zostało wykonane przez Röntgena w 1896 roku.
    Polega na przepuszczeniu przez obiekt badany promieniowania, a następnie detekcji tego promieniowania za obiektem badanym.
    W praktyce rejestrujemy współczynnik osłabienia promieniowania rentgenowskiego przez badany obiekt.
    Wyróżniamy dwa typu radiografii: analogowy i cyfrowy.
    Radiografia analogowa odchodzi powoli w zapomnienie.
    W radiografii cyfrowej obrazowana jest ilość promieniowania X przenikające przez badany obiekt.
    Kontrast zależy od położenia obiektu między źródłem a detektorem (położenie optymalne), napięcie anodowe, filtracja, grubość okładek wzmacniających.
    Rozdzielczość zależy od rozdzielczości detektora i rozmiaru ogniska lampy.

    W standardzie DICOM radiografia cyfrowa jest oznaczana jako \quotett{RT}.

    \item Obrazowanie metodą rezonansu magnetycznego - MRI

    Sposób tworzenie obrazu MRI jest wysoce skomplikowanym procesem i ciężko opisać go w kilku zdaniach.
    Obrazowana jest sumaryczna gęstość atomów wodoru (protonów) w badanym obiekcie.
    Kontrast zależy od gęstości protonów, czasu relaksacji podłużnej i poprzecznej, prędkości przepływu płynu.
    Rozdzielczość zależy od parametrów skanera (rozmiar woksela).
    
    W standardzie DICOM obraz rezonansu magnetycznego jest oznaczana jako \quotett{MR}.
    
    \item Ultrasonografia
    
    Jest to badanie, które wszyscy kojarzą z badaniem płodu podczas ciąży z obrazem w kształcie łuku na, którym nic nie widać.
    Badanie ultrasonograficzne polega na wygenerowaniu fali akustycznej o wysokich częstotliwości, a następnie wprowadzeniu jej do ciała pacjenta.
    Następnie nasłuchuje się echa po tej fali.
    Obrazowana jest odbita fala ultradźwiękowa, osłabienia po odbiciach, zmienna częstotliwość i opóźnienie w czasie.
    Kontrast zależy od częstotliwości fali, głębokości badanego obiektu, ilości piezoelektryków w głowicy, obrazowanej struktury.
    Rozdzielczość zależy od czasu trwania impulsu zaburzenia oraz od szerokości wiązki ultradźwiękowej (powierzchnia czynna przetworników).

    W standardzie DICOM obraz ultrasonograficzny jest oznaczana jako \quotett{US}.

    \item Tomografia komputerowa - CT
    
    Agregacja w tomografii komputerowej jest podobna do badania RTG, ale w CT wykonujemy wiele pomiarów w różnych pozycjach względem obiektu badanego i pod różnym kontem.
    Następnie z tych pomiarów tworzymy obraz przez zastosowanie odpowiednich algorytmów tworzących obraz.
    Rejestrujemy współczynnik osłabienia promieniowania rentgenowskiego przez badany obiekt.
    Kontrast zależy od rozmiarów szczegółów badanego obiektu, napięcie anodowe, przyłożone masy (prąd katodowy i czas akwizycji).
    Rozdzielczość zależy od geometrii pomiaru, rozmiaru ogniska lampy rentgenowskiej, przestrzenna rozdzielczość matrycy detektora, liczby detektorów, dyskretyzację i filtru rekonstrukcyjnego.

    W standardzie DICOM obraz ultrasonograficzny jest oznaczana jako \quotett{CT}.

    \item Tomografii PET
    Technika obrazowania w której rejestruje się promieniowanie powstające podczas anihilacji pozytonów (antyelektronów).
    Źródłem promieniowania(pozytonów) jest podana pacjentowi substancja promieniotwórcza, ulegająca rozpadowi beta plus
    Rejestrujemy fotony powstające podczas anihilacji pozytonów.
    Kontrast zależy od wydajności detektorów, odległość detektora od obiektu oraz położenie obiektu.
    Na rozdzielczość ma wpływ przestrzenna rozdzielczość matrycy detektora, liczby detektorów, dyskretyzację.

    W standardzie DICOM obraz ultrasonograficzny jest oznaczana jako \quotett{PT}.
\end{itemize}

Istnieją też techniki, które są połączeniem kilku innych technik.
Takie jak:
\begin{itemize}
    \item PET-CT, PET/CT - połączenie PET z wielorzędowym tomografem komputerowym
    \item PET-MRI, PET/MRI - połączenie PET z rezonansem magnetycznym
\end{itemize}

\section{Obrazy diagnostyczne}

Wszystkie techniki mają na wspólny koniec.
Należy zdobyte dane wyświetlić w sposób zrozumiały człowiekowi, tak aby ten był wstanie je zinterpretować.
Kolokwialnie, rzecz biorąc trzeba wyświetlić obrazek.
Obraz taki powinien być odpowiednio opisany i czytelny.

\subsection{Podstawowe parametry obraz}

\begin{itemize}
    \item Użyteczność
    
    Podstawowym subiektywnym parametrem obrazu jest jego użyteczność.
    Z obrazu trzeba coś wywnioskować, aby na podstawie tych wniosków podjąć odpowiednie czynności, np: zaplanować operacje.

    \item Rozdzielczość przestrzenna.
    \item Kontrast

    Kontrast oznacza różnicę.
    W przypadku obrazu miarą kontrastu jest różnica pomiędzy poziomem szarości lub kolorem różnych jego fragmentów.

    \item Stosunek sygnału użytecznego do szumu
\end{itemize}

\subsection{Zapisywanie obrazu}

\subsection{Wyświetlanie obrazu}

Posiadanie wielu obrazów wiąże się z potrzebą ich przeglądania i porównywania.
Należy mieć, więc jakieś narzędzie do wyświetlenia w sposób poprawny, najlepiej jednym i tym samym programem.

jakie cechy posinna spełniać przglądrka obrazów

\section{Wybór C++}

Jest wiele środowisk, które w łatwy, miły i przyjemny sposób pozwalają tworzyć oprogramowanie, które jest wstanie pracować na wielu platformach.
Do takich należą Java firmy Oracle, która po skompilowaniu, tworzy jednolity kod bajtowy, który może być uruchomiany na każdej platformie na której działa maszyna wirtualna Javy.
Jednakże takie rozwiązanie sprawia, że nie jesteśmy wstanie osiągnąć pełnego potencjału obliczeniowego maszyny przez pewien poziom wirtualizacji.
Rozwiązaniem jest więc C++, którego kod wynikowy, posiada wysoką wydajnością z bezpośrednim dostępem do zasobów sprzętowych i funkcji systemowych, łatwością dodawania innych bibliotek, napisanych w innych niż języku niż C++.
Dodatkowo jest niezależny od konkretnej platformy sprzętowej lub systemowej, co pozwala na przeniesienie kodu na inną platformę.

\section{Układ pracy}