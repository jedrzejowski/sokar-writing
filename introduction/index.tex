
\section{Cel pracy}

Podstawowym celem jest zaliczenie studiów i osiągnięcie tytułu inżyniera.

Celem pracy jest zrobienie oprogramowania do przeglądania obrazów DICOM, która jest niezależna od systemu i środowiska w którym się znajduje

\section{Wprowadzenie}

\section{Obrazowe techniki diagnostyczne}

Diagnostyka obrazowa lub obrazowanie medyczne to dział diagnostyki medycznej zajmujący się tworzeniem i zbieraniem obrazów ludzkiego ciała za pomocą różnych rodzaju oddziaływań fizycznych.

Obrazowe techniki diagnostyczne to techniki i procesy tworzenia wizualnych reprezentacji wnętrza obiektu do analizy medycznej.
A także wizualne przedstawienie funkcjonowania narządów lub tkanek (fizjologia) w czasie, np. bicie serca.
Głównym celem obrazowanie medycznego jest ujawnienie wewnętrznych struktur ciała.
Obrazowanie medyczne pozwala również na agregacje danych o normalnej anatomii i fizjologii i ich zapisywania, w celu poźniejszej identyfikacji patologii poprzez porównanie jej ze zdrowymi narządami.

Istnieje wiele technik obrazowania wykorzystujące różne zjawiska fizyczne zachodzące w materii.

Kilka wybranych technik obrazowania medycznego:
\begin{itemize}
    \item Radiografia - RTG

    Najstarsza i najbardziej rozpoznawalna technika obrazowania.
    Pierwsze zdjęcie analogowe zostało wykonane przez Röntgena w 1896 roku.
    Polega na przepuszczeniu przez obiekt badany promieniowania, a następnie detekcji tego promieniowania za obiektem badanym.
    W praktyce rejestrujemy współczynnik osłabienia promieniowania rentgenowskiego przez badany obiekt.
    Wyróżniamy dwa typu radiografii: analogowy i cyfrowy.
    Radiografia analogowa odchodzi powoli w zapomnienie.
    W radiografii cyfrowej obrazowana jest ilość promieniowania X przenikające przez badany obiekt.
    Kontrast zależy od położenia obiektu między źródłem a detektorem (położenie optymalne), napięcie anodowe, filtracja, grubość okładek wzmacniających.
    Rozdzielczość zależy od rozdzielczości detektora i rozmiaru ogniska lampy.

    W standardzie DICOM radiografia cyfrowa jest oznaczana jako \quotett{RT}.

    \item Obrazowanie metodą rezonansu magnetycznego - MRI

    Sposób tworzenie obrazu MRI jest wysoce skomplikowanym procesem i ciężko opisać go w kilku zdaniach.
    Obrazowana jest sumaryczna gęstość atomów wodoru (protonów) w badanym obiekcie.
    Kontrast zależy od gęstości protonów, czasu relaksacji podłużnej i poprzecznej, prędkości przepływu płynu.
    Rozdzielczość zależy od parametrów skanera (rozmiar woksela).
    
    W standardzie DICOM obraz rezonansu magnetycznego jest oznaczana jako \quotett{MR}.
    
    \item Ultrasonografia
    
    Jest to badanie, które wszyscy kojarzą z badaniem płodu podczas ciąży z obrazem w kształcie łuku na, którym nic nie widać.
    Badanie ultrasonograficzne polega na wygenerowaniu fali akustycznej o wysokich częstotliwości, a następnie wprowadzeniu jej do ciała pacjenta.
    Następnie nasłuchuje się echa po tej fali.
    Obrazowana jest odbita fala ultradźwiękowa, osłabienia po odbiciach, zmienna częstotliwość i opóźnienie w czasie.
    Kontrast zależy od częstotliwości fali, głębokości badanego obiektu, ilości piezoelektryków w głowicy, obrazowanej struktury.
    Rozdzielczość zależy od czasu trwania impulsu zaburzenia oraz od szerokości wiązki ultradźwiękowej (powierzchnia czynna przetworników).

    W standardzie DICOM obraz ultrasonograficzny jest oznaczana jako \quotett{US}.

    \item Tomografia komputerowa - CT
    
    Agregacja w tomografii komputerowej jest podobna do badania RTG, ale w CT wykonujemy wiele pomiarów w różnych pozycjach względem obiektu badanego i pod różnym kontem.
    Następnie z tych pomiarów tworzymy obraz przez zastosowanie odpowiednich algorytmów tworzących obraz.
    Rejestrujemy współczynnik osłabienia promieniowania rentgenowskiego przez badany obiekt.
    Kontrast zależy od rozmiarów szczegółów badanego obiektu, napięcie anodowe, przyłożone masy (prąd katodowy i czas akwizycji).
    Rozdzielczość zależy od geometrii pomiaru, rozmiaru ogniska lampy rentgenowskiej, przestrzenna rozdzielczość matrycy detektora, liczby detektorów, dyskretyzację i filtru rekonstrukcyjnego.

    W standardzie DICOM obraz ultrasonograficzny jest oznaczana jako \quotett{CT}.

    \item Tomografii PET
    Technika obrazowania w której rejestruje się promieniowanie powstające podczas anihilacji pozytonów (antyelektronów).
    Źródłem promieniowania(pozytonów) jest podana pacjentowi substancja promieniotwórcza, ulegająca rozpadowi beta plus
    Rejestrujemy fotony powstające podczas anihilacji pozytonów.
    Kontrast zależy od wydajności detektorów, odległość detektora od obiektu oraz położenie obiektu.
    Na rozdzielczość ma wpływ przestrzenna rozdzielczość matrycy detektora, liczby detektorów, dyskretyzację.

    W standardzie DICOM obraz ultrasonograficzny jest oznaczana jako \quotett{PT}.
\end{itemize}

Istnieją też techniki, które są połączeniem kilku innych technik.
Takie jak:
\begin{itemize}
    \item PET-CT, PET/CT - połączenie PET z wielorzędowym tomografem komputerowym
    \item PET-MRI, PET/MRI - połączenie PET z rezonansem magnetycznym
\end{itemize}

\section{Obrazy diagnostyczne}

\subsection{Parametry obrazów}

\subsubsection{Wartość diagnostyczna obrazu}

W obrazowaniu medycznym chodzi o wyciągnięcie wniosków z obrazów i postawienie diagnozy.
Jest to kluczowy element obrazowania.
Brak możliwości stwierdzenia co na obrazie się znajduje, stawia sens takiego obrazowania pod znakiem zapytania.
Poco nam obraz w 4K na, którym można zobaczyć wyraźne plamy niczego.

Warość diagnostyczną można określić na podstawie następujących parametrów
\begin{itemize}
    \item Jakości obrazu
    
    Parametry jakościowe obrazów są szczegółowo opisane w sekcji \ref{sec:image-quality}

    \item Warunków obserwacji obrazu

    W brew pozorom warunki obserwacji mają kluczowe znaczenie dla wartości diagnostycznej.
    Jeżeli będziemy mieli dobry obraz, który wyświetlimy na budżetowym monitorze RGB, który w rzeczywistości posiada 6-bite kanały RGB i tworzy odcienie za pomocą techniki dithering'u, to niewiele zobaczymy.

    \item wiarygodności diagnostycznej obrazów

    \item charakterystyki pracy lekarza-specjalisty

\end{itemize}

\paragraph{Krzywa ROC}

\subsubsection{Jakość obrazów}
\label{sec:image-quality}

\begin{itemize}
    \item kontrast

    \item Rozdzielczość przestrzenna

    Rozdzielczość przestrzenna obrazu to najmniejsza odległość między dwoma punktami obrazu, które można rozróżnić.
    W radiografii rozdzielczość określa się zazwyczaj jako liczbę równoległych linii, czarnych i białych, które można rozróżnić ma 1 milimetrze obrazu(paralinie na milimetr).

    Porównanie zdolności rozdzielczych różnych technik obrazowania:
    \begin{itemize}
        \item scyntygrafia - 
        \item USG - 
        \item MRI -
        \item CT -
        \item radiografia -
    \end{itemize}
    TUTAJ COŚ WPISAĆ

    \item Stosunek sygnału użytecznego do szumu (SNR)

    W obrazach zawsze występuje szum, widoczny w różnych postaciach, na przykład w postaci cyfrowego ziarna.
    Rodzaj i poziom szumu zależy od techniki obrazowania.
    Stosunek sygnału użytecznego ma decydująy wpływ na widoczności obiektów, kontrast oraz percepcję szczegółów w obrazie.

    \item Poziom artefaktów
    
    Artefakty to zjawiska fałszujące obraz poprzez tworzeni nie istniejących struktur w obrazie.
    Problemem występującym w różnych technikach obrazowania.
    Najbardziej widocznymi artefaktami są warkocz komety i odbicie zwierciadlane w obrazach USG.

    \item Poziom zniekształceń przestrzennych
    
    Zniekształcenia przestrzenne powstają w wyniku geometrycznego ułożenia i kształtu obiektu badanego i aparat pomiarowego.
    Przykładem takiego zniekształcenia mogą być różne powiększenia obiektów zależne od głębokości ich ułożenia w USG, zmiana pozycji pacjenta(przez ruchy klatki piersiowej w czasie badani), czy deformacja obrazu spowodowana zmianami rozkładu pola magnetycznego przez metalowe obiekty w znaldujące się w tym samym pomieszczeniu, co MRI.

\end{itemize}

\subsection{Zapisywanie obrazóœ}

Jak już mamy obraz, to pojawia się problem zapisania tego obrazu w takiej formie aby nie było, żadnej straty informacji.

W obecnej chwili standard DICOM definiuje 81 różnych typów badań, w tym

\subsection{Wyświetlanie obrazów}




Posiadanie wielu obrazów wiąże się z potrzebą ich przeglądania i porównywania.
Należy mieć, więc jakieś narzędzie do wyświetlenia w sposób poprawny, najlepiej jednym i tym samym programem.

jakie cechy posinna spełniać przglądrka obrazów


\section{Wybór C++}

Jest wiele środowisk, które w łatwy, miły i przyjemny sposób pozwalają tworzyć oprogramowanie, które jest wstanie pracować na wielu platformach.
Do takich należą Java firmy Oracle, która po skompilowaniu, tworzy jednolity kod bajtowy, który może być uruchomiany na każdej platformie na której działa maszyna wirtualna Javy.
Jednakże takie rozwiązanie sprawia, że nie jesteśmy wstanie osiągnąć pełnego potencjału obliczeniowego maszyny przez pewien poziom wirtualizacji.
Rozwiązaniem jest więc C++, którego kod wynikowy, posiada wysoką wydajnością z bezpośrednim dostępem do zasobów sprzętowych i funkcji systemowych, łatwością dodawania innych bibliotek, napisanych w innych niż języku niż C++.
Dodatkowo jest niezależny od konkretnej platformy sprzętowej lub systemowej, co pozwala na przeniesienie kodu na inną platformę.

\section{Układ pracy}