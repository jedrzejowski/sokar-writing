
\section{Cel pracy}

Podstawowym celem jest zaliczenie studiów i osiągnięcie tytułu inżyniera.

Celem pracy jest zrobienie oprogramowania do przeglądania obrazów DICOM, która jest niezależna od systemu i środowiska w którym się znajduje

\section{Wprowadzenie}

\section{Obrazowe techniki diagnostyczne}

Diagnostyka obrazowa lub obrazowanie medyczne to dział diagnostyki medycznej zajmujący się tworzeniem i zbieraniem obrazów ludzkiego ciała za pomocą różnych rodzaju oddziaływań fizycznych.

\begin{itemize}
    \item Radiografia - RT
    \item Obrazowanie metodą rezonansu magnetycznego - MRI
    \item Medycyna nuklearna
    \item Ultrasonografia - USG

    \item Elastografia
    %http://www.endokrynologia.net/tarczyca/elastografia
    Metoda polegająca na pomiarze właściwości mechanicznych tkanek sprężystości tkanek

    \item Obrazowanie fotoakustyczne
    \item Tomografia
    \item Echokardiografia
    \item Funkcjonalna spektroskopia bliskiej podczerwieni
    \item Magnetyczne obrazowanie cząstek - MPI
\end{itemize}

\section{Obrazy diagnostyczne}

\subsection{Parametry obrazów}

\subsubsection{Wartość diagnostyczna obrazu}

W obrazowaniu medycznym chodzi o wyciągnięcie wniosków z obrazów i postawienie diagnozy.
Jest to kluczowy element obrazowania.
Brak możliwości stwierdzenia co na obrazie się znajduje, stawia sens takiego obrazowania pod znakiem zapytania.
Poco nam obraz w 4K na, którym można zobaczyć wyraźne plamy niczego.

Warość diagnostyczną można określić na podstawie następujących parametrów
\begin{itemize}
    \item Jakości obrazu
    
    Parametry jakościowe obrazów są szczegółowo opisane w sekcji \ref{sec:image-quality}

    \item Warunków obserwacji obrazu

    W brew pozorom warunki obserwacji mają kluczowe znaczenie dla wartości diagnostycznej.
    Jeżeli będziemy mieli dobry obraz, który wyświetlimy na budżetowym monitorze RGB, który w rzeczywistości posiada 6-bite kanały RGB i tworzy odcienie za pomocą techniki dithering'u, to niewiele zobaczymy.

    \item wiarygodności diagnostycznej obrazów

    \item charakterystyki pracy lekarza-specjalisty

\end{itemize}

\paragraph{Krzywa ROC}

\subsubsection{Jakość obrazów}
\label{sec:image-quality}

\begin{itemize}
    \item kontrast

    \item Rozdzielczość przestrzenna

    Rozdzielczość przestrzenna obrazu to najmniejsza odległość między dwoma punktami obrazu, które można rozróżnić.
    W radiografii rozdzielczość określa się zazwyczaj jako liczbę równoległych linii, czarnych i białych, które można rozróżnić ma 1 milimetrze obrazu(paralinie na milimetr).

    Porównanie zdolności rozdzielczych różnych technik obrazowania:
    \begin{itemize}
        \item scyntygrafia - 
        \item USG - 
        \item MRI -
        \item CT -
        \item radiografia -
    \end{itemize}
    TUTAJ COŚ WPISAĆ

    \item Stosunek sygnału użytecznego do szumu (SNR)

    W obrazach zawsze występuje szum, widoczny w różnych postaciach, na przykład w postaci cyfrowego ziarna.
    Rodzaj i poziom szumu zależy od techniki obrazowania.
    Stosunek sygnału użytecznego ma decydująy wpływ na widoczności obiektów, kontrast oraz percepcję szczegółów w obrazie.

    \item Poziom artefaktów
    
    Artefakty to zjawiska fałszujące obraz poprzez tworzeni nie istniejących struktur w obrazie.
    Problemem występującym w różnych technikach obrazowania.
    Najbardziej widocznymi artefaktami są warkocz komety i odbicie zwierciadlane w obrazach USG.

    \item Poziom zniekształceń przestrzennych
    
    Zniekształcenia przestrzenne powstają w wyniku geometrycznego ułożenia i kształtu obiektu badanego i aparat pomiarowego.
    Przykładem takiego zniekształcenia mogą być różne powiększenia obiektów zależne od głębokości ich ułożenia w USG, zmiana pozycji pacjenta(przez ruchy klatki piersiowej w czasie badani), czy deformacja obrazu spowodowana zmianami rozkładu pola magnetycznego przez metalowe obiekty w znaldujące się w tym samym pomieszczeniu, co MRI.

\end{itemize}

\subsection{Zapisywanie obrazóœ}

Jak już mamy obraz, to pojawia się problem zapisania tego obrazu w takiej formie aby nie było, żadnej straty informacji.

W obecnej chwili standard DICOM definiuje 81 różnych typów badań, w tym

\subsection{Wyświetlanie obrazów}



jakie cechy posinna spełniać przglądrka obrazów


\section{Wybór C++}

Jest wiele środowisk, które w łatwy, miły i przyjemny sposób pozwalają tworzyć oprogramowanie, które jest wstanie pracować na wielu platformach.
Do takich należą Java firmy Oracle, która po skompilowaniu, tworzy jednolity kod bajtowy, który może być uruchomiany na każdej platformie na której działa maszyna wirtualna Javy.
Jednakże takie rozwiązanie sprawia, że nie jesteśmy wstanie osiągnąć pełnego potencjału obliczeniowego maszyny przez pewien poziom wirtualizacji.
Rozwiązaniem jest więc C++, którego kod wynikowy, posiada wysoką wydajnością z bezpośrednim dostępem do zasobów sprzętowych i funkcji systemowych, łatwością dodawania innych bibliotek, napisanych w innych niż języku niż C++.
Dodatkowo jest niezależny od konkretnej platformy sprzętowej lub systemowej, co pozwala na przeniesienie kodu na inną platformę.

\section{Układ pracy}