\section{Wiadomości ogólne o przeglądarkach obrazów DICOM}

\subsection{Standard DICOM}

Transfert Syntax?
Conformance Statment?

\subsection{Zakres możliwość przeglądarek}

Przeglądarki to jest jeden z rodzajów oprogramowania używanego w obrazowaniu medycznym

Przetwarzanie
Analiza
Mierzenia

Czym rózni się od wybou medycznego

Każda dostępna przeglądarka na rynku, zarówno OpenSourcowym jak i komercyjnym jest inna niepowtarzalna.
Ale można wypunktować pewne cechy wspólne ich Możliwości:

\begin{itemize}
    \item Wyświetlanie obrazu
    \begin{itemize}
        \item powiększenie
        \item obrót
        \item przesuniecie
        \item skalowanie
    \end{itemize}
    \item Analiza obrazu
    \begin{itemize}
        \item Rekonstrukcja 3D z sekwencji obrazów
    \end{itemize}
\end{itemize}


\subsection{Określenia możliwość mojej przeglądarki}

Obsługa DIOCM

Modalności

Typy obrazu


\subsubsection{Wspierane typu obrazów}

W bazie danych tag \dicomtag{Photometric Interpretation}{0028}{0004} określa w jaki sposób należy interpretować wartość piksela w obrazie.
W standardzie zdefiniowano 13 różnych przestrzeni barw \dicomtag{Photometric Interpretation}{0028}{0004} z czego używanych jest tylko 5.

\begin{itemize}
    \item Monochrome1

    Skala szarości, w której 0 oznacza biel a wartość maksymalna(definiowana przez inny parametr) czerń.

    \item Monochrome2

    Skala szarości, w której 0 oznacza czerń a wartość maksymalna(definiowana przez inny parametr) biel.

    \item RGB

    Najbardziej rozpoznawana przestrzeń barw, składająca się z kanałów czerwonego, zielonego i niebieskiego

    \item Palette Color

    Tego nie wspieram na razie, bo nie ograniam jak to działa

    \item YBR Full


    \item Niewspierane
    \begin{itemize}
        \item HSV - \textit{retired}, dosłownie z angielskiego, emerytowany, modalność,
        \item ARGB - \textit{retired}
        \item CMYK - \textit{retired}
        \item Pochodne YBR
        \begin{itemize}
            \item YBR Full 422
            \item YBR Partial 422
            \item YBR Partial 420
            \item YBR ICT
            \item YBR RCT 
        \end{itemize}
    \end{itemize}

\end{itemize}

\subsubsection{Jak będzie pokazywany obraz}

\subsubsection{Sposób wyświetlania}

Kino

\subsubsection{Obsługa błędów}

Standard DICOM był zaprojektowany w ten sposób, aby mógł być poszerzany o nowe parametry w przyszłości, oraz tak aby ten nowe parametry nie ingerowały w strukturę już istniejących
