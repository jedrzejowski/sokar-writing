\section{Implementacja}

\subsection{Wieloplatformowość}

Przeglądarka jest napisana w taki sposób, że jej implementacja nie uwzględnia systemu operacyjnego na którym pracuje

Zróżnych perspektyw

\subsection{Język programowania}

Przeglądarka została napisana w C++ w standardzie z 2017 roku w skrócie C++17

\subsection{Środowisko programistyczne}

Do programowania głównie używałem CLion, IDE stworzonego przez firmę JetBrians.
Zdecydowana większość czasu przeglądarka była testowana i debugowana na aktualizowanym systemie ArchLinux.

\subsection{Obiektowy model w oprogramowaniu}

Cały projekt jest zaprojektowany w sposób obiektowy, za wyjątkiem kilku pomniejszych elementów.


\subsubsection{Konwertowanie i analiza danych w tagach}
\input{VR-explanation}

\subsubsection{DicomScene}

Klasa dziedzicząca pośrednio po \qtclass{QGraphicsScene} przez \sokarclass{Scene}.
Jest to obiektem jednej ramki obrazu i jest odpowiedzialna za pośrednie wygenerowanie obrazu oraz jego wyświetlenie na ekranie.

\paragraph{Informacje wyświetlane na scenie}

Informacje na scenie są wyświetlane za pomocą obiektów, które dziedziczą po klasie \sokarclass{SceneIndicator}.
Obiekty te mają dostęp do obiektu konwertera.
Obiekty dziedziczące po \sokarclass{SceneIndicator} implementują róznież swoją pozycje na scenie i są wstanie ją zmieniać w raz ze zmianą wielkości sceny.


Domyślnie obiekty wyświetlające informacje (tytuły punktów to nazwy klas):
\subsubsection{\sokarclass{PatientDataIndicator}}

Obiekt wyświetlający dane pacjenta, pojawia się zawsze na scenie w lewym górnym rogu i zawiera następujące linie:
\begin{itemize}
    \item Nazwa pacjenta oraz płeć

          Nazwa pacjenta znajduje się w \dicomtag{PatientName}{0010}{0010} o \dicomvr{PN}.

          Płeć, zapisana jest w \dicomtag{PatientSex}{0010}{0040} i może mieć następujące wartości:
          \begin{itemize}
              \item \dataword{M } - oznacza mężczyznę, wyświetlana jako O
              \item \dataword{F } - oznacza kobietę, wyświetlana jako O
              \item \dataword{O } - oznacza inną płeć i nie jest wyświetlana
          \end{itemize}

          W przypadku określenia inne płci niż jest w standardzie bądź braku tagu płeć nie będzie widoczna.

          Przykład: \dataword{Adam Jędrzejowski O}.

    \item Identyfikator pacjenta

          Unikalny identyfikator pacjenta z tagu \dicomtag{PatientID}{0010}{0020} wyświetlane w takiej formie jakiej jest zapisane, bez żadnej obróbki.
          W praktyce najczęściej jest to numer z systemu używanego w danym szpitalu, rzadziej numer PESEL.

          Przykład: \dataword{HIS/000000}.

    \item Data urodzenia oraz wiek pacjenta w trakcie badania

          Data urodzenia znajdująca się w \dicomtag{PatientBirthDate}{0010}{0030} i jest zamieniana na format \dataword{YYYY-MM-DD}.
          Dodatkowo, jeżeli tag \dicomtag{PatientAge}{0010}{1010} jest obecny, wyświetlany jest także wiek pacjenta w czasie badania.

          Przykład: \dataword{born 1982-08-09, 28 years}.

    \item Opis wykonany przez instytucję opis lub klasyfikację badania (komponentu)

          Tekst brany z \dicomtag{StudyDescription}{0008}{1030} i wyświetlany bez żadnej obróbki.

          UWAGA: Ta wartość jest wpisywana przez technika, operatora lub lekarza wykonującego badanie, więc wartość ta może być nie przewidywalna.

    \item Opis serii

          Tekst brany z \dicomtag{SeriesDescription}{0008}{103E} i wyświetlany bez żadnej obróbki.

          UWAGA: Ta wartość jest wpisywana przez technika, operatora lub lekarza wykonującego badanie, więc wartość ta może być nie przewidywalna.
\end{itemize}

Przykład pełnego teksu:

\texttt{\\
    \textbf{Adam Jędrzejowski} O\\
    HIS/123456\\
    born 1996-07-16, 19 years\\
    Kregoslup ledzwiowy a-p + boczne\\
    AP
}

\subsubsection{\sokarclass{HospitalDataIndicator}}

Obiekt wyświetlający dane szpitala/instytucji, pojawia się zawsze na scenie w prawym górnym rogu i zawiera następujące linie:
\begin{itemize}
    \item Nazwa instytucji

          Tekst brany z \dicomtag{InstitutionalDepartmentName}{0008}{1040} i wyświetlany bez żadnej obróbki.

\end{itemize}

\subsubsection{\sokarclass{ImageOrientationIndicator}}

Obiekt wyświetlający cztery litery oznaczające orientacje obrazu w stosunku do pacjenta.
Obiekt posiada cztery pola: lewe, górne, prawe i dolne.

Każda z sześciu możliwych liter oznacza kierunek oraz zwrot w jakim jest ułożony pacjent:
\begin{itemize}
    \item \dataword{R} --- right --- część prawa pacjenta
    \item \dataword{L} --- left --- część
    \item \dataword{A} --- anterior --- przód pacjenta
    \item \dataword{P} --- posterior --- tył pacjenta
    \item \dataword{F} --- feet --- część dolna
    \item \dataword{H} --- head --- część górna
\end{itemize}

Pary poszczególnych liter tworzą osie:
\begin{itemize}
    \item \quotett{x} --- oś przechodząca od prawej do lewej strony pacjenta, \dataword{L} oznacza zwrot zgodny z osią, a \dataword{R} oznacza zwrot przeciwny

    \item \quotett{y} --- oś przechodząca od przodu do tyłu pacjenta, \dataword{P} oznacza zwrot zgodny z osią, a \dataword{A} oznacza zwrot przeciwny

    \item \quotett{z} --- oś przechodząca od dołu do góry pacjenta, \dataword{H} oznacza zwrot zgodny z osią, a \dataword{F} oznacza zwrot przeciwny

\end{itemize}

\begin{figure}[!htbp]
    \centering
    \includegraphics[width=0.7\textwidth]{img/imageorientationindicator-003.pdf}
    \caption{Wizualizacja układu osi współrzędnych pacjenta. Zdjęcie własne.}
    \label{fig:imageorientationindicator2}
\end{figure}

Informacje o orientacji oraz pozycji względem pacjenta znajdują się w odpowiednio w tagach \dicomtag{ImageOrientation}{0020}{0037} i \dicomtag{ImagePosition}{0020}{0032}.
Wartość \dicomtag{ImageOrientation}{0020}{0037} składa się z sześciu liczb, opowiednio oznaczanych dalej X\textsubscript{x}, X\textsubscript{y}, X\textsubscript{z}, Y\textsubscript{x}, Y\textsubscript{y}, Y\textsubscript{z}.

Standard DICOM definiuje, że te dane mają być z interpretowane w następujący sposób:
\[
    \begin{bmatrix}
        P_x \\ P_y \\ P_z \\ 1
    \end{bmatrix}
    =
    \begin{bmatrix}
        X_x\Delta_i & Y_x\Delta_j & 0 & S_x \\
        X_y\Delta_i & Y_y\Delta_j & 0 & S_y \\
        X_z\Delta_i & Y_z\Delta_j & 0 & S_z \\
        0           & 0           & 0 & 1
    \end{bmatrix}
    \begin{bmatrix}
        i \\ j \\ 0 \\ 1
    \end{bmatrix}
    =
    M
    \begin{bmatrix}
        i \\ j \\ 0 \\ 1
    \end{bmatrix}
\]
gdzie:
\begin{itemize}
    \item $P_{xyz}$ --- koordynaty woksela (i,j) w macierzy obrazu wyrażone w milimetrach
    \item $S_{xyz}$ --- trzy wartości z elementu ze znacznikiem \dicomtag{ImagePosition}{0020}{0032}. Oznacza punkt pozycji pacjenta wyrażony w milimetrach w stosunku do urządzenia wykonującego pomiar.
    \item $X_{xyz}$ --- trzy pierwsze wartości z \dicomtag{ImageOrientation}{0020}{0037}
    \item $Y_{xyz}$ --- trzy ostatnie wartości z \dicomtag{ImageOrientation}{0020}{0037}
    \item $i$ i $j$ --- oznaczają współrzędne na macierzy obrazu, odpowiednio kolumnę i wiersz. Zero oznacza początek.
    \item $\Delta_i$ i $\Delta_j$ --- rzeczywista wielkość piksela obrazu wyrażoną w milimetrach, w algorytmie wyznaczania strony pacjenta ta wartość, może wynosić 1, ponieważ odpowiada za skale
\end{itemize}

Praktycznie rzecz biorąc, pierwsza macierz to wektor reprezentujący pozycję pacjenta.
Druga jest to transformata.
Trzecia to pozycja na obrazie.

Interesuje nas wyznaczenie pozycji sześciu (punktów) na płaszczyźnie obrazu, o następujących współrzędnych, dalej używanych pod nazwą $PatientPosition$:
\begin{itemize}
    \item \quotett{R} - $[-1, 0, 0, 1]$
    \item \quotett{L} - $[+1, 0, 0, 1]$
    \item \quotett{A} - $[0, -1, 0, 1]$
    \item \quotett{P} - $[0, +1, 0, 1]$
    \item \quotett{F} - $[0, 0, -1, 1]$
    \item \quotett{H} - $[0, 0, +1, 1]$
\end{itemize}

UWAGA: Wszystkie obliczenia odbywają się w współrzędnych jednorodnych.

Wykonuje takie przekształcenie:
\[PatientPosition = imgMatrix * ScenePosition\]
\[imgMatrix^{-1} * PatientPosition = imgMatrix^{-1} * imgMatrix * ScenePosition\]
\[imgMatrix^{-1} * PatientPosition = ScenePosition\]
\[ScenePosition = imgMatrix^{-1} * PatientPosition\]
gdzie:
\begin{itemize}
    \item $imgMatrix$ --- macierz przekształcenia obrazu, o której będzie dalej
    \item $ScenePosition$ --- pozycja na obrazie, która naz interesuje
    \item $PatientPosition$ --- któryś z punktów względem pacjenta.
\end{itemize}

Wygląd macierzy $imgMatrix$:
\[
    \begin{bmatrix}
        X_x & Y_x & 0 & 0 \\
        X_y & Y_y & 0 & 0 \\
        X_z & Y_z & 0 & 0 \\
        0   & 0   & 0 & 1
    \end{bmatrix}
\]
Powyższa macierz różni się od macierzy definiowanej w standardzie.
Po pierwsze PikselSpacing został pominięty, a konkretniej nadałem mu wartość 1.
Po drugie pozycja z \dicomtag{ImagePosition}{0020}{0032} została zrównana do punktu zerowego, dzięki temu, wynik też będzie względem punktu zero.
Wyznaczenie macierzy $imgMatrix$ jest jednorazowe.

Po wyznaczeniu sześciu punktów $ScenePosition$, po jednej dla każdego punktu względem pacjenta są zapisywane. $ScenePosition$ odpowiada pozycji punktów na obrazie w pozycji startowej.

Na scenie, której jest wyświetlany obraz, użytkownik, może obracać obraz o dowolny kąt, według własnego uznania.
Te przekształcenia, są realizowane za pomocą macierzy rotacji, dalej znana jako $rotateTransform$.
Macierz $rotateTransform$ jest przesyłana do naszego obiektu \sokarclass{ImageOrientationIndicator} za każdym razem kiedy zostanie zmieniona.

Ostateczne wyznaczenie pozycji punktów pacjent na obrazie odbywa sie przez przemnożenie lewostronne $rotateTransform$ i $ScenePosition$.
\[rotateTransform * ScenePosition\]
Wyznaczane jest w ten sposób pozycja sześciu punktów pacjenta na płaszczyźnie sceny wyświetlanej.
Następnie określane jest na, której z ośmiu części płaszczyzny jest umieszczony dany punkt, podział płaszczyzny jest widoczny na rysunku \ref{fig:imageorientationindicator4}.
Tej płaszczyźnie nadawany jest tytuł w postaci litery, która oznacza stronę pacjenta.
Jeżeli punkt znajduje się w centrum, na przecięciu osi, to oznacza, że punkt znajduje się za lub przed ekranem, więc jest pomijany.
Następnie do czterech pól wyświetlających zostają wstawione następujące teksty:
\begin{itemize}
    \item lewe pole: tytuł części 7, tytuł części 0 i tytuł części 1
    \item górne pole: tytuł części 1, tytuł części 2 i tytuł części 3
    \item prawe pole: tytuł części 3, tytuł części 4 i tytuł części 5
    \item dolne pole: tytuł części 7, tytuł części 6 i tytuł części 5
\end{itemize}

Przykład:\\
Punkt \quotett{H}, czyli punkt reprezentujący kierunek głowy, został przypisany do części 1 i odpowiednio \quotett{L} do części 7, \quotett{R} do części 3 i \quotett{F} do części 5.
Punkty \quotett{A} i \quotett{P} zostały pominięte ponieważ znalazły się na środku.
Do lewego pola wstawiany jest tekst \quotett{HL}, do górnego \quotett{HR}, do prawego \quotett{RF} i do dolnego \quotett{LF}.

\begin{figure}[!htbp]
    \centering
    \includegraphics[width=\textwidth]{img/imageorientationindicator-004.png}
    \caption{Podział płaszczyzny sceny. Wyróżniono osiem części. Zdjęcie własne.} 
    \label{fig:imageorientationindicator4}
\end{figure}

Przykład można zobaczyć na rysunku \ref{fig:imageorientationindicator1}.
Na obrazie widzimy, że lewa strona pacjenta znajduje się po prawej stronie obrazu, prawa strona pacjenta po lewej, góra pacjenta na górnej części obrazu.
Wynika z tego, że obraz przedstawia pacjenta skierowanego twarzą do nas.

\begin{figure}[!htbp]
    \centering
    \includegraphics[width=100mm]{img/imageorientationindicator-002.png}
    \caption{Przykładowy obraz medyczny (przekrój głowy MR) z oznaczeniem orientacji obrazu z apomocą liter A, P, R, L, F, H. Zdjęcie własne.}
    \label{fig:imageorientationindicator1}
\end{figure}

\subsubsection{\sokarclass{PixelSpacingIndicator}}

Obiekt wyświetlający podziałkę informującą jakich rozmiarów jest obiekt na obrazie w rzeczywistości, pojawia się na dole i po prawie stronie sceny, gdy tag \dicomtag{PixelSpacing}{0028}{0030} jest obecny.
Wygląd podziałki można zaobserwować na rysunku \ref{fig:imageorientationindicator1}.

Podziałka dostosowuje swoją wielkość do obecnej sceny, jak i do innych elementów na scenie.
Wartości wyświetlane biorą pod uwagę transformatę skali i rotacji obrazu.

\subsubsection{\sokarclass{ModalityIndicator}}

Obiekt wyświetla informacje o akwizycji obrazu.
Dane różnią się w zależności od modalności obrazu.
Domyślnie zawierają następujące linie:
\begin{itemize}
    \item bla bla bla
    \item bla bla bla
    \item bla bla bla
    \item bla bla bla
\end{itemize}

W przypadku następujących modalności zawierają również następujące informacje:
\begin{itemize}
    \item bla bla bla
    \item bla bla bla
    \item bla bla bla
    \item bla bla bla
\end{itemize}

\paragraph{Podział obowiązków w klasie}
\sokarclass{DicomScene} dzieli obowiązki wyświetlania poszczególnych rzeczy innym obiektom, które dziedziczą po \sokarclass

\paragraph{Interfejs generowanie obrazu}
\sokarclass{DicomScene} to klasa abstrakcyjna i nie implementuje sposoby generowania obrazu, ale implementuje interfejs do generowania obrazu, który wykorzystują klasy dziedziczące po \sokarclass{DicomScene}.
Sposoby generowania obrazu są opisane w sekcji ... ;

\paragraph{Informacje o obrazie}
\sokarclass{DicomScene} jest odpowiedzialny za wyświetlenie informacji o obrazie, które nie są zależne od interpretacji woksela obrazu.
Czyli takie jak wielkość obrazka w skali rzeczywistej zapisana w \dicomtag{PixelSpacing}{0028}{0030} określający wielkość woksela w milimetrach.
Szczegóły wyświetlania informacji są zawarte w sekcji ... ;

Funkcjonalność(zajżenć do UML)

Współbierzność

Struktura danych

Dlaczego wektor, a nie drzewo

\subsection{Generowanie obrazów z danych}

Wszystkie obrazy pod koniec nalezy wyświetlić na ekranie. 
Biblioteka QT wymaga aby obrazy do wyświetlania podawać jej za pomocą obiektu klasy \qtclass{QPixmap}, którego można utworzyć z \qtclass{QImage}, który z koleji można wytworzyć z tabilicy wypełnionej pikselami.
Tabilca ta musi mieć strukture RGB888, czyli trzy ośmio bitowe inty, odpowiednio: czerwony, zielony i niebieski.
Tablica ta dalej będzie zawana jako tablicą obrazu(tablica obrazu).

\subsubsection{Monochorme}

Obraz monochromatyczny to obraz w odcieniach szarości, od białego do czarnego lub od czarnego do białego. Dane są zapisane w sposób ciągły wartość po wartości.


Algorytm składa się z dwóch głównych części.
Pierwsza część to wygenerowanie tablicy konwersji, dana na kolor.
Druga to iteracja po wszystkich danych i zamiany ich na kolory za pomocą tablicy wygenerowanej na w kroku pierwszym.


\paragraph{Palety} 
Klasa \sokarclass{Palette} reprezentuje palety kolorów używanych do kolorowania obrazu monochromatycznego.
Mianowicie po paleta przerabia liczbe zmiennoprzecinokowa od zera do jedynki na jakiś kolor, zwracając \sokarclass{Pixel}, który z koleji jest wkładany do tablicy okienka.

\paragraph{Tablica okienka, wartość na kolor}

Tablica konwersji to tablica, która jest tak na prawde tablicą LUT(Look Up Table), tylko, że w naszym przypadku zmienia ona wartość obrazu na piksel.
Alokowana jest tablica struktur \sokarclass{Pixel} o długości 2 do potęgi N, gdzie N jest ilością zaalokowanych bitów, która jest pobrana z \dicomtag{BitsAllocated}{0028}{0100}.
Czyli w przypadku 16 bitowego int'a, nie zależnie od posiadania znaku czy nie, zostanie zaalokowanych 65536 struktur \sokarclass{Pixel}, czyli 196608 bajtów.
Ta operacja jest jedno razowa.

\paragraph{Wyznaczanie okienka}
Najpierw wyznaczam okienko, które zmienia wartości obrazu na skale od zera do jeden:
\[x_0 = center - width / 2\]
\[x_1 = center + width / 2\]
\[y_1 = 0.0\]
\[y_0 = 1.0\]
gdzie:
\begin{conditions}
center  &   środek okienka \\
width   &   szerokość okienka \\
x0, y0  &   współżędne pierwszego punktu \\
x1, y1  &   współżędne drugego punktu
\end{conditions}
Przeglądarka pozwala na inwersje okienka.
Dlatego kiedy użytkownik zażyczy sobie inwersji, zmienne y\textsubscript{0} i y\textsubscript{1} zamienią się wartoścami.

Standart DICOM przewiduje, że wszystkie dane powinny być wyskalowane, za pomocą wzoru.
\[OutputUnits = m*SV + b\]
gdzie:
\begin{conditions}
m           &    warość z \dicomtag{RescaleSlope}{0028}{1053} \\
b           &    warość z \dicomtag{RescaleIntercept}{0028}{1052} \\
SV          &    stored values - warość pixela z pliku  \\
OutputUnits &    wartość wynikowa
\end{conditions}

Wartości okienka odnoszą się do wartości już wyskalowanej, a ponieważ skalowanie całego obrazu jest czasochłonne, przeskalowaie okienka da taki sam efekt:
\[(OutputUnits - b ) / m = SV \]
więc:
\[x_0 -= rescaleIntercept\]
\[x_1 -= rescaleIntercept\]
\[x_0 /= rescaleSlope\]
\[x_1 /= rescaleSlope\]

Posiadamy, teraz dwa punkty okienka odnoszące się do wartośći obrazu.
Wyznaczam parametry prostej przechodzącej przez dwa punkty:
\[a = (y_1 - y_0) / (x_1 - x_0)\]
\[b = y_1 - a * x_1\]

Teraz iterujemy po wszystkich możliwych wartościach wartośćiach obrazu i wykonujemy takie operacje.
\begin{itemize}
    \item wyznaczenie wartości okienka.
    \[y = a * x + b\]
    \item y zostaje obcięcie do 1.0 lub 0.0 jeżeli wyjdzie poza zakres od 1.0 do 0.0
    \item pobranie z palety piksel odpowiadający wartości
    \item wsadzenie piksela do tablicy, tak aby najmniejsza wartości obrazu miała indeks 0 a największy ostani
\end{itemize}

\subsubsection{RGB}

Obrazów zapisanych w RGB nie trzeba w żaden sposób obrabiać, dane już są prawie gotowe do wyświetlenia, należy je tylko odpowiednio posortować, tak jak wymaga biblioteka QT.
Sposób posortowania wartości w pilku określa \dicomtag{PlanarConfiguration}{0x0028}{0006}. Może o przyjąć dwie następujące wartośći:

\begin{itemize}
    \item 0 - oznacza to, że wartości pikseli są ułożone w taki sposób
        \[R1, G1, B1, R2, G2, B2, R3, G3, B3, R4, G4, B4,  ...\]
    \item 1 - oznacza to, że wartości pikseli są ułożone w taki sposób
        \[R1, R2, R3, R4, ... , G1, G2, G3, G4, ..., B1, B2, B3, B4, ...\]
\end{itemize}
gdzie:
\begin{conditions}
Rn  &   wartość czerwonego kanału \\
Gn  &   wartość zielonego kanału \\
Bn  &   wartość niebieskiego kanału
\end{conditions}

Wartości obrazu są przepisywane do buffora dla biblioteki QT.

\subsubsection{YBR}

Skórt YBR odpowiada skrótowi YCbCr.
Wartości są ułożone w taki sposób.
\[Y1, B1, R1, Y2, B2, R2, Y3, B3, R3, Y4, B4, R4,  ...\]

Ponieważ wartości te reprezentują kolory, są już w pewnym sensie są obrazem, ale nie można go wyświetlić, ponieważ komputery bazują na kolorach RGB.
Dlatego odpowieni algorytm konwertuje kolor YBR na kolor RGB, iterując po wszystkich wartościach obrazu.

\paragraph{Konwersja koloru YBR na kolor RGB}

YBR albo YCbCr to model przestrzeni kolorów do przechowywania obrazów i wideo.
Wykorzystuje do tego trzy typy danych: Y – składową luminancji, B lub Cb – składową różnicową chrominancji Y-B, stanowiącą różnicę między luminancją a niebieskim, oraz R lub Cr – składową chrominancji Y-R, stanowiącą różnicę między luminancją a czerwonym.
Kolor zielony jest uzyskiwany na podstawie tych trzech wartości.
YBR nie pokrywa w całości RGB, tak jak RGB nie pokrywa YBR.
Posiadają one część wspólną, co uniemożliwia wyświetlenie obrazu w stu procentach bez zniekształceń.
