
\def\qtclassExplanations{
    \begin{infobox}
        \begin{zeroindent}
            W dokumencie są wielokrotnie zawarte odniesienia do klas z biblioteki Qt.
            Dlatego aby zwiększyć czytelność pracy, została zastosowana konwencja poprzedzania klas z biblioteki Qt przedrostkiem \textit{\qtprefix}.
            Przykład poniżej:

            \begin{center}
                \qtclass{QObject}
            \end{center}

            Wszystkie funkcje wewnątrz klas są oznaczone następująco:

            \begin{center}
                \qtfunction{QObject}{connect}
            \end{center}

            Dodatkowo w dokumencie PDF można kliknąć na nazwę klasy i użytkownik zostanie przekierowany do oficjalnej dokumentacji Qt znajdującej się pod adresem \url{\qtdocurl}.
        \end{zeroindent}
    \end{infobox}
}

\def\gdcmclassExplanations{
    \begin{infobox}
        \begin{zeroindent}
            W dokumencie są wielokrotnie zawarte odniesienia do klas z biblioteki GDCM.
            Dlatego aby zwiększyć czytelność pracy, została zastosowana konwencja poprzedzania klas z biblioteki Qt przedrostkiem \textit{\gdcmprefix}, który za razem jest przestrzenią nazw biblioteki.
            Przykład poniżej:

            \begin{center}
                \gdcmclass{ImageReader}
            \end{center}

            Wszystkie funkcje wewnątrz klas są oznaczone następująco:

            \begin{center}
                \gdcmfunction{ImageReader}{GetImage}
            \end{center}

            Dodatkowo w dokumencie PDF można kliknąć na nazwę klasy i użytkownik zostanie przekierowany do oficjalnej dokumentacji GDCM znajdującej się pod adresem \url{\gdcmdocurl}.
        \end{zeroindent}
    \end{infobox}
}

\def\sokarclassExplanations{
    \begin{infobox}
        \begin{zeroindent}
            W dokumencie są wielokrotnie zawarte odniesienia do klas z przeglądarki obrazów.
            Dlatego aby zwiększyć czytelność pracy, została zastosowana konwencja poprzedzania klas z aplikacji przedrostkiem \textit{\sokarprefix}, który za razem jest przestrzenią nazw programu.
            Przykład poniżej:

            \begin{center}
                \sokarclass{DataConverter}
            \end{center}

            Wszystkie funkcje wewnątrz klas są oznaczone następująco:

            \begin{center}
                \sokarfunction{DataConverter}{toString}
            \end{center}

            Dodatkowo w dokumencie PDF można kliknąć na nazwę klasy i użytkownik zostanie przekierowany do TU WYMYŚLIĆ DO CZEGO
        \end{zeroindent}
    \end{infobox}
}

\def\dicomtagExplanations{
    \begin{infobox}
        \begin{zeroindent}
            W dokumencie są wielokrotnie zawarte odniesienia do znaczników DICOM.
            Dlatego aby zwiększyć czytelność pracy, została zastosowana konwencja poprzedzania znaczników przedrostkiem \dicomtagprefix i sufiksem składającym się z numeru grupy i elementu grupy zapisanych heksadecymalnie.
            Przykład poniżej:

            \begin{center}
                \dicomtag{PatientName}{0010}{0010}
            \end{center}

            Dodatkowo w dokumencie PDF można kliknąć na nazwę klasy i użytkownik zostanie przekierowany do TU WYMYŚLIĆ DO CZEGO
        \end{zeroindent}
    \end{infobox}
}