\documentclass[aspectratio=169]{beamer}
\usepackage{polski}
\usepackage[utf8]{inputenc} % utf8
\usepackage{xspace}
\newcommand{\dicomtag}[3] {#1(#2, #3)}

\newcommand{\wiki}[1]{https://pl.wikipedia.org/wiki/Teksel}

\newcommand{\gdcmclass}[1]{gdcm::#1}

\newcommand{\sokarclass}[1]{Sokar::#1}

\newcommand{\qtclass}[1]{Qt::#1}

\newcommand{\dicomvr}[1]{VR #1}

\newenvironment{conditions}
  {\par\vspace{\abovedisplayskip}\noindent\begin{tabular}{>{$}l<{$} @{${}={}$} l}}
  {\end{tabular}\par\vspace{\belowdisplayskip}}

\newcommand{\quotett}[1]{\enquote{\texttt{#1}}}

\newcommand{\cppcode}[1]{{\color{blue}\texttt{#1}}}

\newcommand*{\thead}[1]{\multicolumn{1}{|c|}{\bfseries #1}}
\def\dicomtagprefix{}

\usetheme{Boadilla}
\usecolortheme{crane}

\title{Wieloplatformowa przeglądarka obrazów DICOM w C++}
\author{Adam Jędrzejowski}
\date{\today}

\begin{document}

\begin{frame}
    \titlepage
\end{frame}

\begin{frame}
    \frametitle{Obrazowe techniki medyczne}

    \begin{itemize}
        \item Radiografia --- RTG
        \item Tomografia komputerowa --- CT
        \item Obrazowanie metodą rezonansu magnetycznego --- MRI
        \item Ultrasonografia
        \item Scyntygrafia
        \item Tomografia SPECT
        \item Tomografia PET
    \end{itemize}
\end{frame}

\begin{frame}
    \frametitle{Standard DICOM}
    \tiny

    \begin{columns}
        \column{0.4\textwidth}

        \column{0.6\textwidth}
        \begin{tabular}{|l|c|c|l|}
            \hline
            \thead{Nazwa}             & \thead{Identyfikator} & \thead{Typ danych} & \thead{Opis}                \\\hline
            SpecificCharacterSet      & (0008,0005)           & CS                 & Używana specyfikacja        \\\hline
            InstitutionName           & (0008,0080)           & LO                 & Miejsce wykonywania badania \\\hline
            Manufacturer              & (0008,0070)           & LO                 & Producent aplikacji         \\\hline
            StationName               & (0008,1010)           & SH                 & Nazwa urządzenia            \\\hline
            PatientID                 & (0010,0020)           & LO                 & Identyfikator pacjenta      \\\hline
            PatientsName              & (0010,0010)           & PN                 & Nazwisko pacjenta           \\\hline
            PatientsBirthDate         & (0010,0030)           & DA                 & Data urodzin pacjenta       \\\hline
            PatientsSex               & (0010,0040)           & CS                 & Płeć pacjenta               \\\hline
            PatientsAge               & (0010,1010)           & AS                 & Wiek pacjenta               \\\hline
            BodyPartExamined          & (0018,0015)           & CS                 & Badana część ciała          \\\hline
            StudyDate                 & (0008,0020)           & DA                 & Data badania                \\\hline
            PhotometricInterpretation & (0028,0004)           & CS                 & Format zapisu obrazu        \\\hline
            Rows                      & (0028,0010)           & US                 & Wysokość zdjęcia            \\\hline
            Columns                   & (0028,0011)           & CS                 & Szerokość zdjęcia           \\\hline
        \end{tabular}
        \vspace{-1em}
        \includegraphics[trim={1.2cm 0.4cm 1.2cm 0.2cm},clip,width=0.7\textwidth]{img/dicom-dataelement002.pdf}
    \end{columns}
\end{frame}

\begin{frame}
    \frametitle{Cel pracy i założenia}
\end{frame}

\begin{frame}[t]
    \frametitle{Narzędzia i biblioteki}

    % Tekst
    \begin{columns}[t]
        \column{0.33\textwidth}
        \textbf{\LARGE CMake}
        \scriptsize
        to wieloplatformowe narzędzie do automatycznego zarządzania procesem kompilacji programu.
        Jest to niezależne od kompilatora narzędzie pozwalające napisać jeden plik, z którego można wygenerować odpowiednie pliki budowania dla dowolnej platformy.

        \column{0.34\textwidth}
        \textbf{\LARGE Qt}
        \scriptsize
        jest zbiorem bibliotek i narzędzi programistycznych dedykowanych dla języków C++, QML i Java.
        Qt posiada bibliotekę do tworzenia interfejsu graficznego, oraz wiele innych rozwiązań ułatwiających programowanie obiektowe i zdarzeniowe.

        Posiadanych normy: IEC 62304:2015, IEC 61508:2010-3 7.4.4, ISO 9001:2015.

        Posiada systemy rodzicielstwa i sygnałów.

        \column{0.33\textwidth}
        \textbf{\LARGE GDCM}
        \scriptsize
        to biblioteka do obsługi standardu DICOM.
        Posiada możliwość wczytywania plików z dysku jak i z lokalizacji sieciowych oraz wczytywania plików DICOMDIR.
        Ma wbudowaną dekompresje obrazów i obsługi różnych kodowań tekstu.

    \end{columns}

    % Obrazki

    % \begin{columns}[c]
    %     \column{0.33\textwidth}

    %     \begin{figure}
    %         \includegraphics[width=0.5\textwidth]{img/logo-cmake.png}
    %     \end{figure}

    %     \column{0.34\textwidth}
    %     \begin{figure}
    %         \includegraphics[width=0.5\textwidth]{img/logo-qt.png}
    %     \end{figure}

    %     \column{0.33\textwidth}
    %     \begin{figure}
    %         \includegraphics[width=1\textwidth]{img/logo-gdcm.png}
    %     \end{figure}

    % \end{columns}
\end{frame}

\begin{frame}
    \frametitle{Projekt interfejsu graficznego}
    \begin{figure}
        \includegraphics[height=0.8\textheight]{img/sokar-gui-002.png}
    \end{figure}
\end{frame}

\begin{frame}
    \frametitle{Projekt interfejsu graficznego - scena}
    \begin{figure}
        \includegraphics[height=0.8\textheight]{img/sokar-gui-003.png}
    \end{figure}
\end{frame}

\begin{frame}
    \frametitle{Budowa obiektowa programu}

    \begin{columns}[c]
        \column{0.5\textwidth}

        \begin{figure}
            \includegraphics[width=0.95\textwidth]{img/uml/global-sturcture.png}
        \end{figure}
        \kern-2em
        \begin{figure}
            \includegraphics[width=0.95\textwidth]{img/uml/sokar-scene-sets.png}
        \end{figure}

        \column{0.5\textwidth}
        \begin{figure}
            \includegraphics[width=0.95\textwidth]{img/uml/dicom-view.png}
        \end{figure}

    \end{columns}

\end{frame}

\begin{frame}[t]
    \frametitle{Dekoder DICOM}
    \begin{center}
        \footnotesize
        \def\arraystretch{1.5}
        \setlength{\tabcolsep}{5pt}
        \begin{tabular}{|l|c|c|c|c|c|c|c|c|c|c|c|c|c|c|c|c|c|c|c|c|}
            \hline
            char*  & $8b_1$                        & $8b_2$                        & $8b_3$                        & $8b_4$                        & $8b_5$                        & $8b_6$                        & $8b_7$                        & $8b_8$                        & $8b_9$ & $8b_10$ & $8b_{11}$ & $8b_{12}$ & $8b_{13}$ & $8b_{14}$ & $8b_{15}$ & $8b_{16}$ \\
            \hline
            int8*  & $8b_1$                        & $8b_2$                        & $8b_3$                        & $8b_4$                        & $8b_5$                        & $8b_6$                        & $8b_7$                        & $8b_8$                        & $8b_9$ & $8b_10$ & $8b_{11}$ & $8b_{12}$ & $8b_{13}$ & $8b_{14}$ & $8b_{15}$ & $8b_{16}$ \\
            \hline
            int16* & \multicolumn{2}{|c|}{$16b_1$} & \multicolumn{2}{|c|}{$16b_2$} & \multicolumn{2}{|c|}{$16b_3$} & \multicolumn{2}{|c|}{$16b_4$} & \multicolumn{2}{|c|}{$16b_5$} & \multicolumn{2}{|c|}{$16b_6$} & \multicolumn{2}{|c|}{$16b_7$} & \multicolumn{2}{|c|}{$16b_8$}                                                                                            \\
            \hline
            int32* & \multicolumn{4}{|c|}{$32b_1$} & \multicolumn{4}{|c|}{$32b_2$} & \multicolumn{4}{|c|}{$32b_3$} & \multicolumn{4}{|c|}{$32b_4$}                                                                                                                                                                                                                            \\
            \hline
            int64* & \multicolumn{8}{|c|}{$64b_1$} & \multicolumn{8}{|c|}{$64b_2$}                                                                                                                                                                                                                                                                                            \\
            \hline
        \end{tabular}
    \end{center}

    auto origin = (quint16 *) \&originBuffer[0];
\end{frame}

\begin{frame}[t]

    \frametitle{Okienkowanie}

    \vspace{-1em}

    \begin{columns}[t]
        \column{0.5\textwidth}
        \vspace{-1.0em}
        \begin{figure}
            \includegraphics[width=1\textwidth]{img/windowing-chart.png}
        \end{figure}
        \vspace{-1.0em}
        \begin{figure}
            \includegraphics[trim={0 1cm 0 2cm},clip,width=1\textwidth]{img/monochrome-002.png}
        \end{figure}
        \vspace{-2.0em}
        \begin{figure}
            \includegraphics[trim={0 1cm 0 2cm},clip,width=1\textwidth]{img/monochrome-003.png}
        \end{figure}

        \column{0.5\textwidth}
        \tiny

        Standard \DICOM przewiduje, że wszystkie dane powinny być wyskalowane za pomocą wzoru:
        \[OutputUnits = m*SV + b\]
        \vspace{-2em}
        \begin{itemize}
            \item $m$ --- wartość z \dicomtag{RescaleSlope}{0028}{1053},
            \item $b$ --- wartość z \dicomtag{RescaleIntercept}{0028}{1052},
            \item $SV$ --- stored values --- wartość woksela z pliku,
            \item $OutputUnits$ --- wartość wynikowa.
        \end{itemize}

        \vspace{1em}
        {\normalsize Implementacja}

        \[x_0 = center - width / 2 \qquad y_0 = 1.0\]
        \vspace{-2em}
        \[x_1 = center + width / 2 \qquad y_1 = 0.0\]
        \[(OutputUnits - b ) / m = SV \]
        \[x_0 -= rescaleIntercept \qquad x_0 /= rescaleSlope\]
        \vspace{-2em}
        \[x_1 -= rescaleIntercept \qquad x_1 /= rescaleSlope\]
        \[a = (y_1 - y_0) / (x_1 - x_0) \qquad b = y_1 - a * x_1\]

        {\normalsize Tablica LUT}
        \begin{center}
            \begin{tabular}{ |c|c|c|c|c| }
                \hline
                $8 b$   & $12 b$   & $16 b$   & $32 b$    & $64 b$        \\
                \hline
                $768 B$ & $196 kB$ & $196 kB$ & $12,5 GB$ & $55*10^{6}TB$ \\
                \hline
            \end{tabular}
        \end{center}
    \end{columns}
\end{frame}

\begin{frame}[t]
    \frametitle{Orientacja pacjenta}
    \begin{columns}[t]

        \column{0.5\textwidth}
        Definicja standardu DICOM
        \tiny
        \[
            \begin{bmatrix}
                P_x \\ P_y \\ P_z \\ 1
            \end{bmatrix}
            =
            \begin{bmatrix}
                X_x\Delta_i & Y_x\Delta_j & 0 & S_x \\
                X_y\Delta_i & Y_y\Delta_j & 0 & S_y \\
                X_z\Delta_i & Y_z\Delta_j & 0 & S_z \\
                0           & 0           & 0 & 1
            \end{bmatrix}
            \begin{bmatrix}
                i \\ j \\ 0 \\ 1
            \end{bmatrix}
            =
            M
            \begin{bmatrix}
                i \\ j \\ 0 \\ 1
            \end{bmatrix}
        \]
        gdzie:
        \begin{itemize}
            \item $P_{xyz}$ --- koordynaty woksela (i,j) we współrzędnych obrazu wyrażone w milimetrach,
            \item $S_{xyz}$ --- trzy wartości z elementu ze znacznikiem \dicomtag{Image Position}{0020}{0032}. Oznacza on punkt pozycji pacjenta wyrażony w milimetrach w stosunku do urządzenia wykonującego pomiar.
            \item $X_{xyz}$ --- trzy pierwsze wartości ze znacznika \dicomtag{Image Orientation}{0020}{0037},
            \item $Y_{xyz}$ --- trzy ostatnie wartości ze znacznika \dicomtag{Image Orientation}{0020}{0037},
            \item $i$ i $j$ --- oznaczają współrzędne na macierzy obrazu, odpowiednio kolumnę i wiersz. Zero oznacza początek,
            \item $\Delta_i$ i $\Delta_j$ --- rzeczywista wielkość piksela obrazu wyrażona w milimetrach.
                  W algorytmie wyznaczania strony pacjenta ta wartość może wynosić 1, ponieważ odpowiada za skale.
        \end{itemize}

        \column{0.5\textwidth}
        Implementacja
        \tiny
        \[PatientPosition = imgMatrix * ScenePosition\]
        \[imgMatrix^{-1} * PatientPosition = imgMatrix^{-1} * imgMatrix * ScenePosition\]
        \[imgMatrix^{-1} * PatientPosition = ScenePosition\]
        \[ScenePosition = imgMatrix^{-1} * PatientPosition\]
        gdzie:
        \begin{itemize}
            \item $imgMatrix$ --- macierz przekształcenia obrazu, o której będzie później opisana,
            \item $ScenePosition$ --- pozycja na obrazie, która nas interesuje,
            \item $PatientPosition$ --- jeden z punktów względem pacjenta.
        \end{itemize}
        \par
        \[
            rotateTransform*
            (
            \begin{bmatrix}
                X_x & Y_x & 0 & 0 \\
                X_y & Y_y & 0 & 0 \\
                X_z & Y_z & 0 & 0 \\
                0   & 0   & 0 & 1
            \end{bmatrix}
            * PatientPosition)
        \]
    \end{columns}


\end{frame}

\begin{frame}
    \frametitle{Funckje przeglądarki}
\end{frame}

\begin{frame}
    \frametitle{Wnioski}
\end{frame}

\end{document}
