\documentclass[aspectratio=169]{beamer}
\usepackage{polski}
\usepackage[utf8]{inputenc} % utf8
\usepackage{xspace}
\usepackage[autostyle]{csquotes} 
\usepackage{textpos}
\usepackage{animate}
\newcommand{\dicomtag}[3] {#1(#2, #3)}

\newcommand{\wiki}[1]{https://pl.wikipedia.org/wiki/Teksel}

\newcommand{\gdcmclass}[1]{gdcm::#1}

\newcommand{\sokarclass}[1]{Sokar::#1}

\newcommand{\qtclass}[1]{Qt::#1}

\newcommand{\dicomvr}[1]{VR #1}

\newenvironment{conditions}
  {\par\vspace{\abovedisplayskip}\noindent\begin{tabular}{>{$}l<{$} @{${}={}$} l}}
  {\end{tabular}\par\vspace{\belowdisplayskip}}

\newcommand{\quotett}[1]{\enquote{\texttt{#1}}}

\newcommand{\cppcode}[1]{{\color{blue}\texttt{#1}}}


%%%%%%%%%%%%%%%%%%%%%%%%%%%%%%%%%%%%%%%%%%%%%%%%%%%%%%%%%%%%%%%%%%%%%%%%%%%%%%%%
% Pakiet do kolorowania składni

\usepackage{listings}

%https://www.overleaf.com/learn/latex/Code_listing
\lstdefinestyle{cppstyle}{
    language=C++,
    backgroundcolor=\color{white},   
    commentstyle=\color{teal},
    keywordstyle=\color{blue},
    numberstyle=\tiny\color{black},
    stringstyle=\color{red},
    basicstyle=\ttfamily\scriptsize,
    breakatwhitespace=false,
    breaklines=true,
    captionpos=b,
    keepspaces=true,
    numbers=left,
    numbersep=5pt,
    showspaces=false,
    showstringspaces=false,
    showtabs=false,
    tabsize=4,
    inputencoding=utf8,
    extendedchars=true,
    morekeywords={
        override,
        qreal, emit, slots, signals,
        quint, qint,
        qint8, qint16, qint32, qint64,
        quint8, quint16, quint32, quint64,
    },
    literate={ą}{{\k{a}}}1 {ć}{{\'{c}}}1 {ę}{{\k{e}}}1 {Ł}{{\L{}}}1 {ł}{{\l{}}}1 {ń}{{\'{n}}}1 {ó}{{\'{o}}}1 {ś}{{\'{s}}}1 {ż}{{\.{z}}}1 {ź}{{\'{z}}}1 ,
}

\lstset{style=cppstyle}


\newcommand*{\thead}[1]{\multicolumn{1}{|c|}{\bfseries #1}}
\def\dicomtagprefix{}

\usetheme{Boadilla}
\definecolor{WEITI_grafitowy}{rgb}{0.235294118, 0.235294118, 0.298039216}
\usecolortheme[named=WEITI_grafitowy]{structure}
\beamertemplatenavigationsymbolsempty

\title{Wieloplatformowa przeglądarka obrazów DICOM w C++}
\author{Adam Jędrzejowski}
\date{\today}
\institute[]{
    Wydział Elektroniki i Technik Informacyjnych Politechniki Warszawskiej\\
    Instytut Radioelektroniki i Technik Multimedialnych\\ 
    Zakład Elektroniki Jądrowej i Medycznej}

\begin{document}

\begin{frame}

    \begin{textblock*}{0mm}(0.5cm,0cm)
        \includegraphics[height=0.8cm]{img/logo-weiti.png}
    \end{textblock*}
    \vspace{1cm}
    \titlepage
\end{frame}

% logo wydziału
\addtobeamertemplate{frametitle}{}{%
    \begin{textblock*}{0mm}(11.2cm,-0.5cm)
        \includegraphics[height=0.8cm]{img/logo-weiti.png}
    \end{textblock*}}

\addtobeamertemplate{frametitle}{\vspace*{0cm}}{\vspace*{0.3cm}}


\begin{frame}[t]
    \frametitle{DICOM}
    \normalsize
    \textbf{D}igital \textbf{I}maging and \textbf{Co}mmunications in \textbf{M}edicine

    \vspace{1em}
    \scriptsize
    Standard DICOM jest odpowiedzią społeczności radiologów, radiofarmaceutów, fizyków medycznych na potrzebę wymiany danych pomiędzy różnymi systemami komputerowymi, przeglądarek obrazów, stacji do przetwarzania i analizowania obrazów medycznych.

    \vspace{1em}
    Standard \DICOM v3 definiuje ujednolicony sposób zapisu i przekazywania danych medycznych reprezentujących lub związanych z obrazami diagnostycznymi w medycynie.


    \begin{columns}[T]
        \column{0.6\textwidth}
        \vspace{1em}
        \normalsize
        \hspace*{0.3em}Obrazowe techniki medyczne:
        \scriptsize
        \begin{itemize}
            \item Radiografia, Tomografia komputerowa -- współczynnik natężenia promieniowania X przez obiekt
            \item Obrazowanie metodą rezonansu magnetycznego -- sumaryczna gęstość atomów wodoru
            \item Ultrasonografia -- różnica gęstości poszczególnych warstw znajdujących się w obiekcie
            \item Scyntygrafia, Tomografia SPECT, Tomografia PET -- rozkład radiofarmaceutyka w obiekcie
        \end{itemize}

        \column{0.4\textwidth}
        \includegraphics[height=0.4\textheight]{img/usg.png}


    \end{columns}
\end{frame}

\begin{frame}[t]
    \frametitle{Standard DICOM}
    \begin{columns}[T]
        \column{0.4\textwidth}
        \scriptsize
        Plik w formacie \DICOM można traktować jako zbiór elementów danych.
        Zbiór nazywa się \enquote{Data Set} i składa się z rekordów, które nazywają się \enquote{Data Element}.
        Elementy danych są ułożone w postaci listy.
        Element danych może zawierać w sobie listę elementów danych.

        \vspace{1em}
        Element danych składa się ze znacznika, interpretacji danych, długości danych i danych.

        \vspace{1em}
        Znacznik składa się z dwóch liczb (grupy i elementu grupy) i informuje co dane zapisują.
        \DICOM definiuje dwa typy znaczników: publiczne i prywatne.

        \vspace{1em}
        W obecnej chwili standard DICOM definiuje 81 różnych typów badań i ponad 4000 publicznych znaczników.

        \column{0.6\textwidth}
        \centering
        \tiny
        \vspace{-0.1cm}
        \includegraphics[trim={1.2cm 1cm 1.2cm 0.2cm},clip,width=0.7\textwidth]{img/dicom-dataelement002.pdf}
        \begin{tabular}{|l|c|c|l|}
            \hline
            \thead{Nazwa}             & \thead{Identyfikator} & \thead{Typ danych} & \thead{Opis}                \\\hline
            SpecificCharacterSet      & (0008,0005)           & CS                 & Używana specyfikacja        \\\hline
            InstitutionName           & (0008,0080)           & LO                 & Miejsce wykonywania badania \\\hline
            Manufacturer              & (0008,0070)           & LO                 & Producent aplikacji         \\\hline
            StationName               & (0008,1010)           & SH                 & Nazwa urządzenia            \\\hline
            PatientID                 & (0010,0020)           & LO                 & Identyfikator pacjenta      \\\hline
            PatientsName              & (0010,0010)           & PN                 & Nazwisko pacjenta           \\\hline
            PatientsBirthDate         & (0010,0030)           & DA                 & Data urodzin pacjenta       \\\hline
            PatientsSex               & (0010,0040)           & CS                 & Płeć pacjenta               \\\hline
            PatientsAge               & (0010,1010)           & AS                 & Wiek pacjenta               \\\hline
            BodyPartExamined          & (0018,0015)           & CS                 & Badana część ciała          \\\hline
            StudyDate                 & (0008,0020)           & DA                 & Data badania                \\\hline
            PhotometricInterpretation & (0028,0004)           & CS                 & Format zapisu obrazu        \\\hline
            Rows                      & (0028,0010)           & US                 & Wysokość zdjęcia            \\\hline
            Columns                   & (0028,0011)           & US                 & Szerokość zdjęcia           \\\hline
        \end{tabular}

    \end{columns}
\end{frame}

\begin{frame} % jako 3 slajd
    \frametitle{Cel pracy i założenia}

    \begin{alertblock}{Cel pracy}
        Opracowanie przeglądarki obrazów DICOM, działającej w wielu systemach operacyjnych
    \end{alertblock}

    Założenia:
    \begin{itemize}
        \item Wieloplatformowość:
              \begin{itemize}
                  \item eliminacja start wydajności spowodowanych wykonaniem kodu wirtualnego (wirtualizacja kodu binarnego z pomocą maszyny wirtualnej JVM; języki skryptowe, których interpretacja kodu jest równoległa z wykonywaniem)
                  \item kod źródłowy z możliwością kompilacji na wskazane platformy: Linux, MacOS, Windows.
              \end{itemize}
        \item Obsługa obrazów statycznych i dynamicznych, serii przekrojów tomograficznych
        \item Realizacja podstawowych funkcji typowej przeglądarki obrazów medycznych: zmiana kontrastu z pomocą \enquote{okienkowanie}, skalowanie, rotacja.
    \end{itemize}

\end{frame}

\begin{frame}[t]
    \frametitle{Biblioteki i narzędzia}

    % Tekst
    \begin{columns}[t]
        \column{0.34\textwidth}
        \textbf{\LARGE Qt}
        \scriptsize
        jest zbiorem bibliotek i narzędzi programistycznych dedykowanych dla języków C++, QML i Java.
        Qt posiada bibliotekę do tworzenia interfejsu graficznego, oraz wiele innych rozwiązań ułatwiających programowanie obiektowe i zdarzeniowe.

        Normy: IEC 62304:2015, IEC 61508:2010-3 7.4.4, ISO 9001:2015.

        Posiada systemy rodzicielstwa i sygnałów.

        \column{0.33\textwidth}
        \textbf{\LARGE GDCM}
        \scriptsize
        to biblioteka do obsługi standardu DICOM.
        Posiada możliwość wczytywania plików z dysku jak i z lokalizacji sieciowych oraz wczytywania plików DICOMDIR.
        Ma wbudowaną dekompresję obrazów i obsługi różnych kodowań tekstu.

        \column{0.33\textwidth}
        \textbf{\LARGE CMake}
        \scriptsize
        to wieloplatformowe narzędzie do automatycznego zarządzania procesem kompilacji programu.
        Jest to niezależne od kompilatora narzędzie pozwalające napisać jeden plik, z którego można wygenerować odpowiednie pliki budowania dla dowolnej platformy.

    \end{columns}

    % Obrazki

    \begin{columns}[c]

        \column{0.33\textwidth}
        \begin{figure}
            \includegraphics[width=0.5\textwidth]{img/logo-qt.png}
        \end{figure}

        \column{0.34\textwidth}
        \begin{figure}
            \includegraphics[width=1\textwidth]{img/logo-gdcm.png}
        \end{figure}

        \column{0.33\textwidth}
        \begin{figure}
            \includegraphics[width=0.5\textwidth]{img/logo-cmake.png}
        \end{figure}

    \end{columns}
\end{frame}

\begin{frame}
    \frametitle{Projekt interfejsu graficznego}
    \centering
    przeglądarka struktury katalogów i plików
    \includegraphics[height=0.8\textheight]{img/sokar-gui-004.png}
\end{frame}

\begin{frame}
    \frametitle{Projekt interfejsu graficznego -- scena}
    \tiny
    \begin{columns}[t]

        \column{0.15\textwidth}

        Dane pacjenta:\\
        - imię i nazwisko\\
        - identyfikator\\
        - data urodzenia i wiek\\
        - opis badania\\
        - opis serii\\

        \vspace{8em}
        Litera orientacji

        \vspace{10em}
        Dane akwizycji badania\\
        różnią się w zależności od modalności


        \column{0.7\textwidth}
        \centering
        Litera orientacji
        \includegraphics[width=\textwidth]{img/sokar-gui-003.png}
        Podziałka i litera orientacji

        \column{0.15\textwidth}

        Dane szpitala:\\
        - nazwa instytucji\\
        - producent i model urządzenia\\
        - lekarz\\
        - operator

        \vspace{7em}
        Litera orientacji i podziałka

        \vspace{11em}
        Parametry z jakimi jest wyświetlany obraz

    \end{columns}
\end{frame}

\begin{frame}
    \frametitle{Budowa obiektowa programu -- perspektywy}

    \begin{columns}[c]
        \def\umlsize{0.9\textwidth}
        \column{0.5\textwidth}
        \centering
        Układ globalnej struktury programu
        \includegraphics[width=\umlsize]{img/uml/global-sturcture.png}

        \vspace{-0.3cm}
        Struktura przechowywania ramek obrazowych
        \includegraphics[width=\umlsize]{img/uml/sokar-scene-sets.png}

        \column{0.5\textwidth}
        \centering
        Obiekt zakładki wyświetlający sekwencje ramek
        \includegraphics[width=\umlsize]{img/uml/dicom-view.png}

    \end{columns}

\end{frame}

\begin{frame}[t,fragile]
    \frametitle{Dekoder DICOM}
    \vspace{-1em}
    \begin{columns}[t]
        \column{0.5\textwidth}
        Odczyt danych nieobrazowych
        \begin{lstlisting}[basicstyle=\ttfamily\tiny,numbers=none]
gdcm::Reader reader;
reader.SetFileName("/path/to/file");
if (!reader.Read()) return 1;

/* Pobranie obiektu pliku */
const gdcm::File &file = reader.GetFile();

/* Pobranie obiektu zbioru danych */
const gdcm::DataSet &dataset = file.GetDataSet();

/* Inicjalizacja klasy konwersji danych na string */
gdcm::StringFilter stringFilter;
stringFilter.SetFile(file);

/* Definicja znaczników wybranych elementów */
const static gdcm::Tag
    TagPatientName(0x0010, 0x0010),
    TagWindowCenter(0x0028, 0x1050);

/* Odczyt danej tekstowej */
if (dataset->FindDataElement(TagPatientName))
  string name = stringFilter.GetString(TagPatientName);

/* Odczyt danej binarnej */
if (dataset->FindDataElement(TagWindowCenter)) {
  auto& ele = dataset->GetDataElement(TagWindowCenter);
  quint16 center = ele.GetByteValue()->GetPointer(); }
\end{lstlisting}

        \column{0.5\textwidth}
        Odczyt obrazów
        \begin{lstlisting}[basicstyle=\ttfamily\tiny,numbers=none]
gdcm::ImageReader ir;
ir.SetFileName("/path/to/file");
if (!ir.Read()) return 1;

/* Pobranie obiektu obrazu */
const gdcm::Image &gimage = ir.GetImage();

/* Utworzenie buffora i jego wypełnienie */
std::vector<char> imgbuffer;
imgbuffer.resize(gimage.GetBufferLength());
gimage.GetBuffer(&imgbuffer[0]);

/* Odczyt parametrów obrazu */
const quint* dimension = gimage.GetDimensions();
quint dimX = dimension[0];
quint dimY = dimension[1];

if (gimage.GetPhotometricInterpretation() == 
    gdcm::PhotometricInterpretation::RGB)
    cout << "Jest to obraz RGB";

if (gimage.GetPhotometricInterpretation() == 
    gdcm::PhotometricInterpretation::MONOCHROME2)  
    cout << "Jest to obraz monochromatyczny 2";

if (gimage.GetPixelFormat() == gdcm::PixelFormat::UINT8)
    cout << "Pixel jest 8-bitową liczbą całkowitą";
\end{lstlisting}
    \end{columns}
\end{frame}

\begin{frame}[t]

    \frametitle{Poprawa kontrastu za pomocą \enquote{okienkowania}}

    \vspace{-1em}

    \begin{columns}[t]
        \column{0.5\textwidth}
        \vspace{-2.0em}
        \begin{figure}
            \includegraphics[width=1\textwidth]{img/windowing-chart2.png}
        \end{figure}
        \vspace{-1.0em}
        \begin{figure}
            \includegraphics[trim={0 1cm 0 2cm},clip,width=1\textwidth]{img/monochrome-002.png}
        \end{figure}
        \vspace{-2.0em}
        \begin{figure}
            \includegraphics[trim={0 1cm 0 2cm},clip,width=1\textwidth]{img/monochrome-003.png}
        \end{figure}

        \column{0.5\textwidth}
        \tiny

        Standard \DICOM przewiduje, że wszystkie dane powinny być wyskalowane za pomocą wzoru:
        \[OutputUnits = m*SV + b\]
        \vspace{-2em}
        \begin{itemize}
            \item $m$ --- wartość z \dicomtag{RescaleSlope}{0028}{1053},
            \item $b$ --- wartość z \dicomtag{RescaleIntercept}{0028}{1052},
            \item $SV$ --- stored values --- wartość woksela z pliku,
            \item $OutputUnits$ --- wartość wynikowa.
        \end{itemize}

        \vspace{1em}
        {\normalsize Implementacja}

        \[P_0: \qquad x_0 = center - width / 2 \qquad y_0 = 1.0\]
        \vspace{-2em}
        \[P_1: \qquad x_1 = center + width / 2 \qquad y_1 = 0.0\]
        \[(OutputUnits - b ) / m = SV \]
        \[x_0 -= rescaleIntercept \qquad x_0 /= rescaleSlope\]
        \vspace{-2em}
        \[x_1 -= rescaleIntercept \qquad x_1 /= rescaleSlope\]
        \[a = (y_1 - y_0) / (x_1 - x_0) \qquad b = y_1 - a * x_1\]

        {\normalsize Tablica LUT}
        \begin{center}
            \begin{tabular}{ |c|c|c|c|c| }
                \hline
                $8 b$   & $12 b$   & $16 b$   & $32 b$    & $64 b$        \\
                \hline
                $768 B$ & $196 kB$ & $196 kB$ & $12,5 GB$ & $55*10^{6}TB$ \\
                \hline
            \end{tabular}
        \end{center}
    \end{columns}
\end{frame}

\begin{frame}[t]
    \frametitle{Orientacja pacjenta}
    \begin{columns}[t]

        \column{0.5\textwidth}
        Zapis informacji o orientacji w DICOM
        \tiny
        \vspace{1em}
        \begin{columns}
            \column{0.6\textwidth}
            \vspace{-2em}
            \[
                \begin{bmatrix}
                    P_x \\ P_y \\ P_z \\ 1
                \end{bmatrix}
                =
                \begin{bmatrix}
                    X_x\Delta_i & Y_x\Delta_j & 0 & S_x \\
                    X_y\Delta_i & Y_y\Delta_j & 0 & S_y \\
                    X_z\Delta_i & Y_z\Delta_j & 0 & S_z \\
                    0           & 0           & 0 & 1
                \end{bmatrix}
                \begin{bmatrix}
                    i \\ j \\ 0 \\ 1
                \end{bmatrix}
            \]
            \column{0.4\textwidth}
            \includegraphics[width=\textwidth]{img/imageorientationindicator-003.pdf}
        \end{columns}

        \begin{itemize}
            \item $P$ -- koordynaty woksela we współrzędnych obrazu,
            \item $S$ -- trzy wartości z elementu ze znacznikiem Image Position,
            \item $X$, $Y$ -- położenie pierwszego wiersza i kolumny względem układu współrzędnych związanych z pacjentem
            \item $i$ i $j$ --- oznaczają współrzędne na macierzy obrazu,
            \item $\Delta_i$ i $\Delta_j$ --- rzeczywista wielkość piksela obrazu w $mm$.
        \end{itemize}
        \vspace{2em}
        {\normalsize Implementacja}
        \tiny
        \[PatientPosition = imgMatrix * ScenePosition\]
        \vspace{-2em}
        \[imgMatrix^{-1} * PatientPosition = imgMatrix^{-1} * imgMatrix * ScenePosition\]
        \vspace{-1.5em}
        \[imgMatrix^{-1} * PatientPosition = ScenePosition\]
        \vspace{-1.5em}
        \[ScenePosition = imgMatrix^{-1} * PatientPosition\]

        \column{0.5\textwidth}
        \tiny
        \[
            rotateTransform*
            (
            \begin{bmatrix}
                X_x & Y_x & 0 & 0 \\
                X_y & Y_y & 0 & 0 \\
                X_z & Y_z & 0 & 0 \\
                0   & 0   & 0 & 1
            \end{bmatrix}^{-1}
            * PatientPosition)
        \]

        \includegraphics[width=0.49\textwidth]{img/imageorientationindicator-004.pdf}
        \includegraphics[width=0.49\textwidth]{img/imageorientationindicator-005.png}
    \end{columns}
\end{frame}

\begin{frame}[t]
    \frametitle{Funkcje przeglądarki}
    \begin{columns}[T]
        \column{0.5\textwidth}
        \footnotesize
        \begin{itemize}
            \item Wprowadzono obsługę serii obrazów jako całości, włączając w to przegląd obrazów w serii, animacje, wspólne okna w skali barwnej oraz wspólne przekształcenia macierzowe.
            \item Obsługa podstawowych operacji na obrazie ułatwiających jego oglądanie i ocenienie, takich jak: przenoszenie, skalowanie, obrót.
            \item Dynamiczna zmiana parametrów wyświetlania obrazu, polepszanie kontrstu za pomocą \enquote{okienkowania} z możliwością zastosowania wielokolorowych palet.
            \item Wyliczanie orientacji pacjenta w stosunku do obrazu.
            \item Przeglądanie obrazów w systemie.
        \end{itemize}

        \column{0.50\textwidth}
        \includegraphics[width=\textwidth]{img/test004.png}
    \end{columns}
\end{frame}

\begin{frame}[t]
    \frametitle{Testy}

    \vspace{-3mm}
    \begin{columns}[T]
        \column{0.64\textwidth}

        \begin{itemize}
            \item Zapewniono jednolity sposób kompilacji na platformach przy użyciu narzędzia CMake.
            \item Napotkano problem z biblioteką GDCM -- biblioteka wymaga kompilacji przez użytkownika.
            \item Skompilowano i przetestowano na platformach: Linux, MacOS i Windows.
            \item Wykonano testy na obrazach różnych modalności:
                  \begin{itemize}
                      \item CT z zakładowego tomografu
                      \item MR z zakładowego tomografu
                      \item USG zapisanego w formacie RGB
                      \item Medycyna nuklearna -- obsługa obrazów multiframe
                  \end{itemize}
            \item Wykonano testy algorytmu wyznaczania orientacji pacjenta na obrazie -- porównanie z innymi przeglądarkami
        \end{itemize}
        \column{0.36\textwidth}
        \includegraphics[width=\textwidth]{img/test001.png}
        \includegraphics[width=\textwidth]{img/test002.png}
    \end{columns}

\end{frame}

\begin{frame}
    \frametitle{Wnioski}
    \begin{itemize}
        \item Opracowano aplikację do obsługi obrazów \DICOM w C++ z możliwością kompilacji na wiele platform.
        \item Zastosowano biblioteki dostępne na różnych platformach: Qt i GDCM, które również zostały napisane w C++, dzięki czemu uzyskano jednolity program napisany w jednym języku.
        \item Jednolity wygląd aplikacji zapewniła biblioteka Qt, co sprawia, że interfejs aplikacji jest prawie taki sam na każdym systemie.
        \item Przeprowadzone testy pokazały poprawność implementacji algorytmów.
    \end{itemize}

\end{frame}

\end{document}
