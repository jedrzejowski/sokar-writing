\usepackage{polski}

\usepackage{titlesec}
\usepackage[T1]{fontenc}
\usepackage[utf8]{inputenc} % utf8
\usepackage{helvet}
\usepackage[autostyle]{csquotes} % cudzysłowy
\usepackage{xcolor} % kolorowanie tekstu
\usepackage[pdftex]{hyperref}
\usepackage{listings} % kolorowanie składni
\usepackage{array}
\usepackage[polish]{babel}
\usepackage{graphicx}
\usepackage{amsmath}
\usepackage{indentfirst}% wcięcia przed akapitami


% Dobry tutorial do ramek https://shearnrylan.wordpress.com/2015/01/02/latex-information-boxes/
\usepackage[framemethod=TikZ]{mdframed}% ramki

\usepackage{prmag2017}

\usepackage{todonotes}

% Ustawienie głębokości numerowania
\setcounter{secnumdepth}{5}
\setcounter{tocdepth}{5}

%https://www.overleaf.com/learn/latex/Code_listing
\lstdefinestyle{cppstyle}{
    backgroundcolor=\color{white},   
    commentstyle=\color{teal},
    keywordstyle=\color{blue},
    numberstyle=\tiny\color{black},
    stringstyle=\color{red},
    basicstyle=\ttfamily\scriptsize,
    breakatwhitespace=false,
    breaklines=true,
    captionpos=b,
    keepspaces=true,
    numbers=left,
    numbersep=5pt,
    showspaces=false,
    showstringspaces=false,
    showtabs=false,
    tabsize=4,
    inputencoding=utf8,
    extendedchars=true,
    literate={ą}{{\k{a}}}1 {ć}{{\'{c}}}1 {ę}{{\k{e}}}1 {Ł}{{\L{}}}1 {ł}{{\l{}}}1 {ń}{{\'{n}}}1 {ó}{{\'{o}}}1 {ś}{{\'{s}}}1 {ż}{{\.{z}}}1 {ź}{{\'{z}}}1 ,
}

\lstset{style=cppstyle}

% Enter po tytule patagrafu
% https://latex.org/forum/viewtopic.php?t=1383
\makeatletter
\renewcommand\paragraph{%
    \@startsection{paragraph}{4}{0mm}%
       {-\baselineskip}%
       {.5\baselineskip}%
       {\normalfont\normalsize\bfseries}}
\makeatother
 
