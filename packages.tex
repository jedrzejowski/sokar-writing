\usepackage{polski}

\usepackage{titlesec}
\usepackage[T1]{fontenc}
\usepackage[utf8]{inputenc} % utf8
\usepackage{helvet}
\usepackage[autostyle]{csquotes} % cudzysłowy
\usepackage{xcolor} % kolorowanie tekstu
\usepackage[pdftex]{hyperref}
\usepackage{array}
\usepackage[polish]{babel}
\usepackage{graphicx}
\usepackage{amsmath}
\usepackage{indentfirst} % wcięcia przed akapitami
\usepackage{xspace} % spacja po komendzie
\usepackage{url}
\usepackage{titling} % dla \thetitle


% Marginesy
\usepackage[margin=25mm]{geometry}
\addtolength{\oddsidemargin}{5mm}
\addtolength{\evensidemargin}{-5mm}


% Naprzemienne
\usepackage{titleps}
\renewpagestyle{plain}{%
\sethead{}{}{}
\setfoot[\thepage][][]{}{}{\thepage}
}%
\pagestyle{plain}


% Dobry tutorial do ramek https://shearnrylan.wordpress.com/2015/01/02/latex-information-boxes/
\usepackage[framemethod=TikZ]{mdframed}% ramki

\usepackage{prmag2017}

% Ustawienie głębokości numerowania
% \setcounter{secnumdepth}{3}
% \setcounter{tocdepth}{3}

% Ustawienie mnijeszego tekstu pod rysunkami
\usepackage{caption}
\captionsetup[figure]{
    font=small,
    labelfont=bf
}


%%%%%%%%%%%%%%%%%%%%%%%%%%%%%%%%%%%%%%%%%%%%%%%%%%%%%%%%%%%%%%%%%%%%%%%%%%%%%%%%
% Pakiet do kolorowania składni

\usepackage{listings}

%https://www.overleaf.com/learn/latex/Code_listing
\lstdefinestyle{cppstyle}{
    language=C++,
    backgroundcolor=\color{white},   
    commentstyle=\color{teal},
    keywordstyle=\color{blue},
    numberstyle=\tiny\color{black},
    stringstyle=\color{red},
    basicstyle=\ttfamily\scriptsize,
    breakatwhitespace=false,
    breaklines=true,
    captionpos=b,
    keepspaces=true,
    numbers=left,
    numbersep=5pt,
    showspaces=false,
    showstringspaces=false,
    showtabs=false,
    tabsize=4,
    inputencoding=utf8,
    extendedchars=true,
    morekeywords={
        override,
        qreal, emit, slots, signals,
        qint8, qint16, qint32, qint64,
        quint8, quint16, quint32, quint64,
    },
    literate={ą}{{\k{a}}}1 {ć}{{\'{c}}}1 {ę}{{\k{e}}}1 {Ł}{{\L{}}}1 {ł}{{\l{}}}1 {ń}{{\'{n}}}1 {ó}{{\'{o}}}1 {ś}{{\'{s}}}1 {ż}{{\.{z}}}1 {ź}{{\'{z}}}1 ,
}

\lstset{style=cppstyle}

%%%%%%%%%%%%%%%%%%%%%%%%%%%%%%%%%%%%%%%%%%%%%%%%%%%%%%%%%%%%%%%%%%%%%%%%%%%%%%%%
% Entery po tytule paragrafie

% https://latex.org/forum/viewtopic.php?t=1383
\makeatletter
\renewcommand\paragraph{%
    \@startsection{paragraph}{4}{0mm}%
       {-\baselineskip}%
       {.5\baselineskip}%
       {\normalfont\normalsize\bfseries}}
\makeatother
 
\makeatletter
\renewcommand\subparagraph{%
    \@startsection{subparagraph}{4}{0mm}%
       {-\baselineskip}%
       {.5\baselineskip}%
       {\normalfont\normalsize\bfseries}}
\makeatother
 

\DeclareUnicodeCharacter{2640}{\utffemale}