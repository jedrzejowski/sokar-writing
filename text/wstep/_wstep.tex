W dzisiejszych czasach obrazowanie diagnostyczne odgrywa kluczową rolę w medycynie.
Obrazy medyczne pomagają lekarzom w podejmowaniu decyzji w sprawie dalszych losów pacjenta.
Nie można pozwolić aby błąd techniczny wpłynął negatywnie na zdrowie lub życie pacjenta.
Dlatego tak bardzo ważnym aspektem obrazowania jest jego dokładność, rzetelność, niezawodność oraz powtarzalność.

Diagnostyka obrazowa lub obrazowanie medyczne to dział diagnostyki medycznej zajmujący się tworzeniem i zbieraniem obrazów ludzkiego ciała za pomocą różnych rodzaju oddziaływań fizycznych.
Obrazowe techniki diagnostyczne to techniki i procesy tworzenia wizualnych reprezentacji wnętrza obiektu do analizy medycznej.
A także wizualne przedstawienie funkcjonowania narządów lub tkanek (fizjologia) w czasie, np. bicie serca.
Głównym celem obrazowanie medycznego jest ujawnienie wewnętrznych struktur ciała.
Obrazowanie medyczne pozwala również na agregacje danych o normalnej anatomii i fizjologii i ich zapisywania, w celu poźniejszej identyfikacji patologii poprzez porównanie jej ze zdrowymi narządami.

Proces obrazowania diagnostycznego można podzielić na dwa główne etapy, pierwszy to wykonanie pomiarów na pacjencie, a drugi to analiza badania przez personel medyczny i decyzja o podjęciu działań.

Wygląd pierwszego etapu, który jest wykonywany najczęściej przez technika, jest zależny od techniki badania.
Technika badania to sposób na wejście w oddziaływanie fizyczne z ciałem pacjenta a następnie, a następnie na agregacji, czyli zbierania, danych pomiarowych.
Przykładami najbardziej popularnych badań są: radiografia, tomografia rentgenowska, obrazowanie metodą rezonansu magnetycznego, ultrasonografia, scyntygrafia, tomografia SPECT oraz tomografii PET.
Każda z wymienionych technik jest szerzej opisana w sekcji \ref{sec:basic-imaging-technics}.
Podczas badania zapisywanie są też wszystkie parametry dotyczące badania oraz tego w jakich warunkach zostało przeprowadzone.
Parametrami badania, są na przykład dane pozwalające zidentyfikować pacjenta w sposób jednoznaczny oraz jego płeć, date urodzenia, wiek w trakcie badania, sposób ułożenia ciała w urządzeniu pomiarowym.
Parametrem badania jest też model urządzenia, unikalny identyfikator urządzenia, nazwę producenta, date badania, date ostatniego przeglądu urządzenia oraz dane personalne osoby wykonującej badanie.

Na końcu tego procesu, wszystkie dane oraz parametry są zapisywane do pliku o formacie zgodnym ze standardem DICOM.
DICOM definiuje między innymi format zapisu danych i parametrów obrazowania w pliku w postaci cyfrowej.
Standard DICOM został opracowany przez dwie organizacje American College of Radiology (w skrócie ACR) i National Electrical Manufacturers Association (w skrócie NEMA) i opublikowany w swojej ostatecznej wersji w 1993.
W obecnym czasie jest to jedyny wiodący standard zapisu w obrazowaniu medycznym.

Drugi etap, czyli analiza danych i parametrów badania przez personel medyczny, sprowadza się do wyświetlenia danych i parametrów badania w taki sposób aby były zrozumiane i miały wartość diagnostyczną.
Nie jest to trywialny proces i posiada wiele aspektów.
Standard DICOM przewiduje praktyczną możliwość zapisania danych i parametrów w jakiejkolwiek formie z jakiejkolwiek techniki obrazowania.
Sprawia to, że po wczytaniu parametrów badania i ich analizie należy podjąć decyzje w jaki sposób mamy je wyświetlić użytkownikowi.
Wszystkie aspekty analizy dokonywanej przez program będą wyjaśnione w dalszych częściach dokumentu.

%rozpendówka o przetwarzaniu

Dlatego celem tej pracy inżynierskiej jest zrobienie niezawodnej przeglądarki działającej niezależnie od platformy i mogącej obsłużyć wiele typów obrazów medycznych.
Implementacje takiego rozwiązania, można zrealizować na wiele sposobów.

Możliwość uruchomienia na wielu platformach można uzyskać w wiele różnych sposobów.
Pierwszym rozwiązaniem z jest napisanie oprogramowania w środowisku, który pozwala na uruchomienie na wielu platformach.
Do takich należą Java firmy Oracle, która po skompilowaniu, tworzy jednolity kod bajtowy, który może być uruchomiany na każdej platformie na której działa maszyna wirtualna Javy.
Jednakże takie rozwiązanie sprawia, że nie jesteśmy wstanie osiągnąć pełnego potencjału obliczeniowego maszyny przez pewien poziom wirtualizacji.

Drugim rozwiązaniem jest napisanie jednolitego kodu, który można skompilować do kodu natywnego na każdą z docelowych platform.
Taką możliwość daje C++, którego kod wynikowy, posiada wysoką wydajnością z bezpośrednim dostępem do zasobów sprzętowych i funkcji systemowych, łatwością dodawania innych bibliotek, napisanych w innych niż języku niż C++.

Wybrany został drugi sposób ponieważ został w pewnym sensie narzucony przez tytuł pracy ale daje też większą możliwość wykorzystania możliwości obliczeniowych dzisiejszych komputerów.

Do pisania programu zostały wykorzystane dwie biblioteki. Qt do tworzenia interfejsu graficznego w stylu obecnej platformy. Oraz Grassroots DICOM library(w skrócie GDCM) do obsługi plików DICOM.

