
\section{Narzędzia potrzebne do kompilacji}

\par
Do kompilacji wystarczą podstawowe narzędzia budowania dostosowane do systemu operacyjnego.

\begin{itemize}
    \item Windows --- Visual Studio w wersji 2017 lub nowszej
    \item Linux --- pakiety zawierające następujące komendy: make; cmake (w wersji 3.10 lub nowszej), g++ (w wersji 8 lub nowszej)
    sudo apt-get install build-essential libgl1-mesa-dev
sudo pacman -Syu qt5-base
    \item MacOS --- to samo co w Linuxie, program Xcode instaluje wszystkie potrzebne komendy
\end{itemize}

Kod źródłowy został skompilowany za pomocą powyższych narządzi.
Ale w kodzie programu nie występują, żadne elementy odbiegające od standardu C++17, więc nie powinno być problemów z użyciem niższych wersji wspierających standard C++17

\section{Biblioteki potrzebne do kompilacji}

Do kompilacji są potrzebne biblioteki Qt i GDCM.


\subsection{Instalacja Qt}

Program był testowany na wersji 5.12.

\subsubsection*{Linux}

Nie istnieje dystrybucja Linuxa, w której repozytoriach nie było by biblioteki Qt.
Dlatego instalacja Qt sprowadza się pobrania jej z repozytoriów za pomocą menadżera pakietów.
\par
Komendy pozwalające zainstalować bibliotekę Qt na wybranych dystrybucjach:\\
Ubuntu: \texttt{sudo apt-get install qt5-default}\\
ArchLinux: \texttt{sudo pacman -Syu qt5-base}

\subsubsection*{Windows i MacOS}

\par
W celu instalacji Qt należy udać się na oficjalną stronę biblioteki Qt.
W prawym gónym rogu kliknąć zielony przycisk \enquote{Download. Try. Buy.}.
Na dole kolumny zatytułowanej \enquote{Open Source} kliknąć zielony przycisk \enquote{Go open source}.
Zostanie automatycznie pobrany plik instalacyjny.
Po pobraniu należy go otworzyć i postępować zgodnie z instalacją.
\par
W pewnym momencie użytkownik może zostać poproszony o dane kontaktowe.
Nie jest to wymagane i można kliknąć przycisk \enquote{Skip}.
\par
Następnie należy wybrać komponenty do zainstalowania.
W przypadku Windowsa należy zainstalować wersje \enquote{Qt 5.12.X MSVC 2017 64 bit}.
Z kolei na MacOS należy zainstalować \enquote{Qt 5.12.X clang\_x64}.
Należy odhaczyć wszystkie inne opcje, nie są one wymagane do kompilacji programu.

\subsection{Pobranie kodu źródłowego GDCM}

W przypadku biblioteki GDCM, należy udać się na stronę \url{https://github.com/malaterre/GDCM/releases/tag/v2.8.9}.
UWAGA: program był testowany na wersji 2.8.9, wersja 3.0.0 wyszła po skończeniu pisania kodu i nie była testowana.
Na stronie należy pobrać plik \enquote{Source code (zip)}, a następnie go rozpakować.

\subsection{Pobranie kodu źródłowego Sokar}

Kod źródłowy aplikacji można pobrać repozytorium git znajdującego się pod adresem \url{https://gl.ire.pw.edu.pl/ajedrzejowski/sokar-app} lub z nośnika danych dołączonego do pracy.

\section{Przygotowanie katalogów}

Należy utworzyć folder w którym będą znajdowały się wszystkie foldery z plikami, dalej ten folder będzie nazywany \enquote{/path/}.
Kod źródłowy GDCM umieść w katalogu \enquote{/path/gdcm/}.
Kod źródłowy Sokar umieść w katalogu \enquote{/path/sokar-app/}.
Utwórz również foldery \enquote{/path/gdcm-bin/} i \enquote{/path/sokar-app-bin/}.

\section{Kompilacja GDCM}

Uruchom CMake z menu programów lub za pomocą \enquote{cmake-gui}.
W polu \enquote{Where is the source code:} wpisz \enquote{/path/gdcm/}.
W polu \enquote{Where to build the binnaries:} wpisz \enquote{/path/gdcm-bin/}.
Kliknij \enquote{Configure}.
\begin{itemize}
    \item Windows
    \item Linux i MacOS
\end{itemize}
Odznacz wszystkie wartość za wyjątkiem \enquote{GDCM\_BUILD\_SHARED\_LIBS}.

\section{Kompilacja Sokar}

\section{Przeniesienie plików do jednego folderu}