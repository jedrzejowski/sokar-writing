
\par
Po analizie możliwości przeglądarek plików DICOM dostępnych na rynku postanowiłem zaimplementować następujące komponenty w mojej przeglądarce:
\begin{itemize}
      \item Przesuwanie \fromEng{pan}

      \item Skalowaniu lub powiększenie

      \item Rotacja i odbicia lustrzane

      \item Okienkowanie i pseudokolorowanie

      \item Obsługa następujących obrazów o interpretacji fotometrycznej:

            \begin{itemize}
                  \item \dataword{MONOCHROME1}
                  \item \dataword{MONOCHROME2}
                  \item \dataword{RGB}
                  \item \dataword{YBR}
            \end{itemize}

      \item Wczytanie wielu plików i ich połączenie w formie filmu
\end{itemize}

\par
Najbardziej rozpoznawalne dwie przeglądarki Osirix i Horus, posiadają swoje nazwy od bogów egipskich.
Odpowiednio od Ozyrysa, boga śmierci i Horusa, boga nieba.
Dlatego postanowiłem nazwać swoja przeglądarkę w podobny sposób: Sokar.
\par
Sokar w mitologii egipskiej to bóstwo dokonujące przyjęcia i oczyszczenia zmarłego władcy oraz przenoszący go na swej barce do niebios, patron metalurgów, rzemieślników i tragarzy (nosicieli lektyk) oraz wszelkich przewoźników.
