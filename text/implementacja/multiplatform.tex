Przeglądarka jest napisana w taki sposób, że jej implementacja nie ogranicza możliwości kompilacji na konkretny systemy operacyjnego.


\subsection{Język programowania}

Przeglądarka została napisana w C++ w standardzie ISO/IEC 14882 z 2018, w skrócie C++17

\subsection{Środowisko programistyczne}

Do pisania kodu oraz debugowania używałem głównie CLion, IDE stworzonego przez firmę JetBrians.
Zdecydowaną większość czasu przeglądarka była testowana i debugowana na aktualizowanym systemie ArchLinux.

\subsection{Obiektowy model w oprogramowaniu}

Praca jest zaprojektowany w sposób obiektowy, w taki sposób aby była możliwość jej rozbudowy i dodawania nowych funkcji.