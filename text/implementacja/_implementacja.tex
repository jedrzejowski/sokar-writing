
\par
Najbardziej rozpoznawalne dwie przeglądarki to Osirix i Horus.
Ich nazwy zaczerpnięto od nazw egipskich bogów: odpowiednio od Ozyrysa, boga śmierci i Horusa, boga nieba.
Nazwa przeglądarki omawianej w pracy będzie miała nazwę: Sokar.
\par
Sokar w mitologii egipskiej to bóstwo dokonujące przyjęcia i oczyszczenia zmarłego władcy oraz przenoszący go na swej barce do niebios, patron metalurgów, rzemieślników i tragarzy (nosicieli lektyk) oraz wszelkich przewoźników.

\section{Zakres implementacji}

\par
Po analizie możliwości przeglądarek plików \DICOM dostępnych na rynku postanowiono zaimplementować następujące komponenty w opracowywanej przeglądarce:
\begin{itemize}
    \item Obsługa obrazów bez względu na ich modalność, ale z ograniczeniem do następujących interpretacji fotometrycznej:

          \begin{itemize}
              \item \dataword{MONOCHROME1}
              \item \dataword{MONOCHROME2}
              \item \dataword{RGB}
              \item \dataword{YBR}
          \end{itemize}

    \item Przesuwanie \fromEng{pan}.

    \item Skalowanie lub powiększenie poprzez decymacje i interpolacje liniowe.

    \item Rotacja i odbicia lustrzane.

    \item Okienkowanie i pseudokolorowanie, zarówno w skali szarości jak i z użyciem wielokolorowych palet.

    \item Obsługa serii obrazów jako całości
          \begin{itemize}
              \item przegląd obrazów w serii
              \item animacje
              \item wspólne okna w skali barwnej
              \item wspólne przekształcenia macierzowe
          \end{itemize}
\end{itemize}


\section{Wieloplatformowość}
Przeglądarka jest napisana w taki sposób, że jej implementacja nie ogranicza możliwości kompilacji na konkretny systemy operacyjnego.


\subsection{Język programowania}

Przeglądarka została napisana w C++ w standardzie ISO/IEC 14882 z 2018, w skrócie C++17

\subsection{Środowisko programistyczne}

Do pisania kodu oraz debugowania używałem głównie CLion, IDE stworzonego przez firmę JetBrians.
Zdecydowaną większość czasu przeglądarka była testowana i debugowana na aktualizowanym systemie ArchLinux.

\subsection{Obiektowy model w oprogramowaniu}

Praca jest zaprojektowany w sposób obiektowy, w taki sposób aby była możliwość jej rozbudowy i dodawania nowych funkcji.

\section{Graficzny interfejs użytkownika}
\sokarclassExplanations

\par
Po uruchomieniu programu użytkownikowi ukazuje się jedno okno (rysunek \ref{sec:gui-window}), które zawiera 3 elementy: menu (obiekt klasy \qtclass{QMenuBar}), drzewa plików (obiekt klasy \sokarclass{FileTree}), obiekt zakładek z obrazami (obiekt klasy \sokarclass{DicomTabs}).
\par
Użytkownik może otworzyć plik DICOM na trzy sposoby: z menu na górze, z drzewem ze strukturą pików i poprzez przeciągnięcie.

\section{Projekt struktury obiektowej programu}

\par
W tej sekcji wyjaśniona jest ogólna struktura programu, z pominięciem dokładnych opisów poszczególnych elementów.
Ich szczegółowy opis znajduje się w następnych sekcjach.
\par
Obiekt okna, klasy \sokarclass{MainWindow} posiada 3 elementy: menu (klasy \qtclass{QMenuBar}), drzewa plików (klasy \sokarclass{FileTree}), obiekt zakładek (klasy \sokarclass{DicomTabs}).
Zakładki obiektu zakładek są implementowane prze klasę \sokarclass{DicomView}.
Obiekt zakładki posiada abstrakcyjną kolekcję scen, implementowaną przez \sokarclass{DicomSceneSet}.
Kolekcja scen odpowiada za przechowywanie obrazów i scen, obiektów klasy \sokarclass{DicomScene}.
Sceny nie posiadają bezpośredniego dostępu do pliku, a jednie wskaźniki do odpowiednich miejsc w pamięci, gdzie obrazy są przechowywane.
Ogólny diagram klas znajduje się na rysunku \ref{fig:uml-global-sturcture}.

\begin{figure}[!htbp]
    \centering
    \includegraphics[width=\textwidth]{img/uml/global-sturcture.png}
    \caption{Diagram klas UML globalnej struktury programu.}
    \label{fig:uml-global-sturcture}
\end{figure}


\section{Struktury danych}

\subsection{Konwertowanie danych ze znaczników}

\par
Każdy plik \DICOM posiada zbiór elementów danych.
Zapisane elementy danych należy przekonwertować na obiekty danych odpowiadające potrzebom programu.
Dlatego został zaimplementowany obiekt klasy \sokarclass{DataConverter} zajmujący się konwersją danych z pliku \DICOM na dane w formacie odpowiadającym programowi.

\par
Obiekt konwertera jest tworzony na podstawie pliku \DICOM i przy wywoływaniu konwersji należy podać tylko znacznik, który nas interesuje.
Takie rozwiązanie pozwala na przesłanie do wszystkich obiektów jednego względnie małego obiektu konwertera, co ułatwia zarządzanie dostępem do pliku \DICOM.

\par
Klasa \sokarclass{DataConverter} posiada następujące funkcje, pozwalające na konwertowanie danych:
\begin{itemize}
    \item \sokarfunction{DataConverter}{toString}

          Funkcja konwertuje element na obiekt tekstu \qtclass{QString}.

    \item \sokarfunction{DataConverter}{toAttributeTag}

          Funkcja konwertuje element o znaczniku typu \dicomvr{AT} na obiekt znacznika \gdcmclass{Tag}.

    \item \sokarfunction{DataConverter}{toAgeString}

          Funkcja konwertuje element o znaczniku typu \dicomvr{AS} na tekst w postaci czytelnej, np: „18 weeks” lub „3 years”.

    \item \sokarfunction{DataConverter}{toDate}

          Funkcja konwertuje element o znacznik typu \dicomvr{DA} na obiekt klasy \qtclass{QDate}, który ma w sobie wbudowaną konwersję na tekst zależny od ustawień językowych aplikacji.

    \item \sokarfunction{DataConverter}{toDecimalString}

          Funkcja konwertuje element o znacznik typu \dicomvr{DS} na obiekt wektora posiadającego liczby rzeczywiste.
          \cppcode{qreal} jest aliasem do typu zmiennoprzecinkowego, na systemach 64-bitowy jest to \cppcode{double}.

    \item \sokarfunction{DataConverter}{toIntegerString}

          Funkcja konwertuje element o znacznik typu \dicomvr{IS} na 32-bitową liczbę całkowitą (\cppcode{qint32}).

    \item \sokarfunction{DataConverter}{toPersonName}

          Funkcja konwertuje element o znacznik typu \dicomvr{PN} na obiekt tekst zawierający imię w formie pisanej.

    \item \sokarfunction{DataConverter}{toShort}

          Funkcja konwertuje element o znacznik typu \dicomvr{SS} na 16-bitowa liczbę całkowitą ze znakiem (\cppcode{qint16}).

    \item \sokarfunction{DataConverter}{toUShort}

          Funkcja konwertuje element o znacznik typu \dicomvr{US} na 16-bitowa liczbę całkowitą bez znaku (\cppcode{quint16}).

\end{itemize}
Oprócz powyższych funkcji jest jeszcze kilka innych funkcji pobocznych oraz kilka aliasów.

\par
Ogólne zasady konwersji, które się tyczą wszystkich danych:
\begin{itemize}
    \item Większość VR jest to zapisanych jako tekst, kodowanie i dekodowanie tekstu jest zapewniane przez bibliotekę.
    \item Większość danych może mieć kilka wartości oddzielonych backslashem \quotett{\textbackslash}, dlatego konwerter dla VR, w których standard przewiduje wiele wartości, zawsze zwraca wektor z tymi wartościami.
    \item Wszystkie dane są zapisane parzystą ilością bajtów, w przypadku tekstu dodaje się znak spacji na końcu danych.
          Taka spacja jest pomijana w analizie danych.
\end{itemize}



\subsection{Scena}
\label{sec:sokar-dicomscene}

Jest to obiekt jednej ramki obrazu i jest odpowiedzialna za pośrednie wygenerowanie obrazu oraz jego wyświetlenie na ekranie.
Implementowany jest przez klasę \sokarclass{DicomScene}, dziedzicząca po \sokarclass{Scene}, natomiast \sokarclass{Scene} dziedziczy po \qtclass{QGraphicsScene}.
Diagram klas UML znajduje się na rysunku \ref{uml:sokar-dicomscene}

\begin{figure}[!htbp]
    \centering
    \includegraphics[width=\textwidth]{img/uml/dicom-scene.png}
    \caption{Diagram klas UML dziedziczenia klasy \sokarclass{DicomScene}.}
    \label{uml:sokar-dicomscene}
\end{figure}

\subsubsection{Wyświetlanie sceny}
\par
Qt zapewnia własny silnik graficzny, który pozwala na łatwą wizualizację przedmiotów, z obsługą obrotu i powiększania.
Silnik ten jest implementowany w postaci scen za pomocą \qtclass{QGraphicsScene}.
Natomiast klasa \qtclass{QGraphicsView} dostarcza element interfejsu graficznego, który jest miejscem do wyświetlania scen.
\par
Na scenie mogą być wyświetlana obiekty dziedziczące po \qtclass{QGraphicsItem}.
Obiekty te mogą być dodawane, usuwane i przesuwane ze sceny w czasie rzeczywistym.
Dodatkowo można  na tym obiektach używać transformat we współrzędnych jednorodnych, szerzej opisanych w sekcji \ref{sec:sokar-dicomscene-tranformat}.
Przykłady obiektów używanych w scenie \sokarclass{DicomScene}:
\begin{itemize}
    \item \qtclass{QGraphicsTextItem} --- element wyświetlający tekst, obsługuje on możliwość wyświetlania podstawowych znaczników HTML
    \item \qtclass{QGraphicsLineItem} --- element wyświetlający prostą linie z punktu $A$ do $B$
    \item \qtclass{QGraphicsPixmapItem} --- element wyświetlający obrazy graficzne, obiety klasy \qtclass{QPixmap}
    \item \qtclass{QGraphicsItemGroup} --- element grupujący wiele elementów, pozwala na łatwą implementacje bardziej złożonych struktur
\end{itemize}

\subsubsection{Informacje wyświetlane na scenie}

\par
Wszystkie elementy wyświetlające dane z pliku \DICOM dziedziczą po klasie \sokarclass{SceneIndicator}.
Diagram klas UML znajduje się na rysunku \ref{uml:sokar-scene-indicators}.

\begin{figure}[!htbp]
    \centering
    \includegraphics[width=\textwidth]{img/uml/dicom-scene-indicators.png}
    \caption{Diagram klas UML dziedziczenia klasy \sokarclass{SceneIndicator}.}
    \label{uml:sokar-scene-indicators}
\end{figure}

\par
Domyślnie obiekty wyświetlające informacje (tytuły punktów to nazwy klas):

\paragraph{Dane pacjenta}

Dane pacjenta są implementowane przez \sokarclass{PatientDataIndicator} i pojawiają się zawsze na scenie w lewym górnym rogu.
Zawierają następujące linie:
\begin{itemize}
    \item Nazwa pacjenta oraz płeć

          Nazwa pacjenta znajduje się w \dicomtag{Patient Name}{0010}{0010} o \dicomvr{PN}.

          Płeć, zapisana jest w \dicomtag{Patient Sex}{0010}{0040} i może mieć następujące wartości:
          \begin{itemize}
              \item \dataword{M } --- oznacza mężczyznę, wyświetlana jako znak \utfMaleSign
              \item \dataword{F } --- oznacza kobietę, wyświetlana jako znak \utfFemaleSign
              \item \dataword{O } --- oznacza inną płeć i nie jest wyświetlana
          \end{itemize}

          Przykład: \enquote{Adam Jędrzejowski \utfMaleSign}.

    \item Identyfikator pacjenta

          Unikalny identyfikator pacjenta ze znacznika \dicomtag{Patient ID}{0010}{0020} wyświetlany jest w takiej formie, w jakiej jest zapisany.
          W praktyce najczęściej jest to numer z systemu używanego w danym szpitalu, rzadziej numer PESEL.

          Przykład: \enquote{HIS/000000}.

    \item Data urodzenia oraz wiek pacjenta w trakcie badania

          Data urodzenia znajdująca się w \dicomtag{Patient Birth Date}{0010}{0030} i jest zamieniana na format \dataword{YYYY-MM-DD}.
          Dodatkowo, jeżeli tag \dicomtag{PatientAge}{0010}{1010} jest obecny, wyświetlany jest także wiek pacjenta w czasie badania.

          Przykład: \enquote{born 1982-08-09, 28 years}.

    \item Opis wykonany przez instytucję lub klasyfikację badania (komponentu)

          Tekst brany z \dicomtag{Study Description}{0008}{1030} i wyświetlany bez ingerencji.

          UWAGA: Ta wartość jest wpisywana przez technika, operatora lub lekarza wykonującego badanie, więc wartość ta może być nie przewidywalna.

    \item Opis serii

          Tekst brany z \dicomtag{Series Description}{0008}{103E} i wyświetlany bez ingerencji.

          UWAGA: Ta wartość jest wpisywana przez technika, operatora lub lekarza wykonującego badanie, więc wartość ta może być nie przewidywalna.
\end{itemize}

Przykład pełnego teksu:

\begin{center}
    \begin{tabular}{l}
        \textbf{Adam Jędrzejowski} \utfMaleSign \\
        HIS/123456                              \\
        born 1996-07-16, 19 years               \\
        Kregoslup ledzwiowy a-p + boczne        \\
        AP
    \end{tabular}
\end{center}

\paragraph{Dane jednostki organizacyjnej}

Są implementowane przez \sokarclass{HospitalDataIndicator}.
Pojawia się zawsze na scenie w prawym górnym rogu i zawiera następujące linie:
\begin{itemize}
    \item Nazwa instytucji

          Tekst jest obierany z \dicomtag{Institutional Department Name}{0008}{1040} i wyświetlany bez ingerencji.

    \item Producent wyposażenia wraz z modelem urządzenia

          Tekst jest obierany z \dicomtag{Manufacturer}{0008}{0070} i \dicomtag{Manufacturer Model Name}{0008}{1070}, oddzielony spacją i wyświetlany bez ingerencji.

    \item Nazwisko lekarza wykonującego badanie

          Tekst jest obierany z \dicomtag{Referring Physician Name}{0008}{0090} i wyświetlany bez ingerencji.

    \item Nazwisko operatora wspierającego badanie

          Tekst jest obierany z \dicomtag{Operators Name}{0008}{1070} i wyświetlany bez ingerencji.
\end{itemize}

\paragraph{Orientacja obrazu}

\par
Jest implementowana przez \sokarclass{ImageOrientationIndicator}.
Obiekt wyświetla cztery litery oznaczające orientację obrazu w stosunku do pacjenta.
Obiekt posiada cztery pola: lewe, górne, prawe i dolne.

\par
Każda z sześciu możliwych liter oznacza kierunek oraz zwrot w jakim jest ułożony pacjent:
\begin{itemize}
    \item \enquote{R} --- right --- część prawa pacjenta
    \item \enquote{L} --- left --- część
    \item \enquote{A} --- anterior --- przód pacjenta
    \item \enquote{P} --- posterior --- tył pacjenta
    \item \enquote{F} --- feet --- część dolna
    \item \enquote{H} --- head --- część górna
\end{itemize}

\par
Pełny opis implementacji algorytmu wyznaczania stron znajduje się w sekcji \ref{sec:algorithm-imageorientationindicator}.

\paragraph{Podziałka}

Jest implementowana przez \sokarclass{PixelSpacingIndicator}.
Obiekt wyświetla podziałkę informującą o rzeczywistych rozmiarach obiektu na obrazie.
Pojawia się na dole i po prawej stronie sceny, gdy znacznik \dicomtag{PixelSpacing}{0028}{0030} jest obecny.
Wygląd podziałki można zaobserwować na rysunku \ref{fig:imageorientationindicator1}.

Podziałka dostosowuje swoją wielkość do obecnej sceny, jak i do innych elementów na scenie.
Wartości wyświetlane biorą pod uwagę transformatę skali i rotacji obrazu.

\paragraph{Dodatkowe informacje o modalności}

Są implementowane przez \sokarclass{ModalityIndicator}.
Obiekt wyświetla informacje o akwizycji obrazu.
Dane różnią się w zależności od modalności obrazu.
Domyślnie zawierają następujące linie:
\begin{itemize}
    \item \enquote{Modality} --- Modalność --- pobierana ze znacznika \dicomtag{Modality}{0008}{0060}.
    \item \enquote{Series} --- Numer serii --- pobierany ze znacznika \dicomtag{Series Number}{0020}{0011}.
    \item \enquote{Instance number} --- Numer instancji w serii --- pobierany ze znacznika \dicomtag{Instance Number}{0020}{0013}.
    \item Wartości odnoszące się do właściwości plastra obrazu.
          \enquote{Slice thickness} --- Grubość plastra --- pobierana ze znacznika \dicomtag{Slice Thickness}{0018}{0050}.
          \enquote{Slice location} --- Pozycja plastra --- pobierana ze znacznika \dicomtag{Slice Location}{0020}{1041}.
\end{itemize}

W przypadku następujących modalności zawierają również następujące informacje:
\begin{itemize}
    \item CT --- tomografia komputerowa
          \begin{itemize}
              \item \enquote{KVP} --- Szczytowe napięcie wyjściowe generatora promieniowania rentgenowskiego --- wyrażone w kilo voltach, pobierane z \dicomtag{KVP}{0018}{0060}
              \item \enquote{Exposure time} --- Czas ekspozycji --- pobierany ze znacznika \dicomtag{Exposure Time}{0018}{1150}.
              \item \enquote{Exposure} --- Ekspozycja --- wyrażona w mAs, pobierana ze znacznika \dicomtag{Exposure}{0018}{1152}.
          \end{itemize}

    \item RT/CR --- radiologia analogowa i cyfrowa
          \begin{itemize}
              \item \enquote{Exposure time} --- Czas ekspozycji --- pobierany ze znacznika \dicomtag{Exposure Time}{0018}{1150}.
              \item \enquote{KVP} --- Szczytowe napięcie wyjściowe generatora promieniowania rentgenowskiego --- wyrażone w kilo voltach, pobierane z \dicomtag{KVP}{0018}{0060}
          \end{itemize}

    \item MR --- rezonans magnetyczny
          \begin{itemize}
              \item \enquote{Repetition time} --- Czas repetycji --- pobierany ze znacznika \dicomtag{Repetition Time}{0018}{0080}.
              \item \enquote{Echo time} --- Czas echa --- pobierany ze znacznika \dicomtag{Echo Time}{0018}{0081}.
              \item \enquote{Magnetic field} --- Pole magnetyczne --- nominalna wartość pola magnetycznego wyrażona w teslach pobierana ze znacznika \dicomtag{Magnetic Field Strength}{0018}{0087}.
              \item \enquote{SAR} --- Swoiste tempo pochłaniania energii --- pobierane ze znacznika \dicomtag{SAR}{0018}{1316}.
          \end{itemize}
\end{itemize}

\subsubsection{Generowanie obrazów z danych}

Klasa \sokarclass{DicomScene} jest klasą abstrakcyjną i nie generuje obrazu, pozostawia to klasą dziedziczących po niej.
Dokładna analiza cyklu generowania obrazów jest opisana w sekcji \ref{sec:algorithm-pixmap-generate}.

\subsubsection{Transformaty obrazu}

\par
Wygenerowany obraz można wyświetlić na scenie bez większego problemu.
Wyświetlanie \cppcode{pixmap}, obiektu klasy \qtclass{QPixmap}, odbywa się za pomocą obiektu \cppcode{pixmapItem}, obiektu klasy \qtclass{QGraphicsPixmapItem}, który dziedziczy po \qtclass{QGraphicsItem}.
Ta ostatnia klasa ma w sobie zaimplementowaną funkcję pozwalającą na nałożenie przekształcenia na obraz.
Transformata to obiekt klasy \qtclass{QTransform}, który reprezentuje transformatę dwu wymiarowa na obiekt, praktycznie jest to macierz 3 na 3 reprezentująca przekształcenie we współrzędnych jednorodnych.

Zostało zdefiniowanych 4 transformat, które działają na obiekt obrazu wyświetlany na scenie:
\begin{itemize}
    \item \cppcode{centerTransform} --- transformata wyśrodkowująca, zadanie tego przekształcenia jest przeniesienie obrazu na środek sceny
    \item \cppcode{panTransform} --- transformata przesunięcia
    \item \cppcode{scaleTransform} --- transformata skali
    \item \cppcode{rotateTransform} --- transformata rotacji
\end{itemize}

\par
Transformaty, podczas interakcji z użytkownikiem, mogą ulegać zmianom na dwa sposoby.
Pierwszym sposobem jest odebranie sygnału od przycisków z paska zadań, szerzej opisanego w sekcji \ref{sec:sokar-dicomtoolbar}, znajdującego się nad sceną.
Drugi sposób to przechwycenie ruchów myszki gdy wciśnięty jest lewy przycisk myszy.
\par
Pełny algorytm tworzenia transformat i ich zmian poprzez interakcje z użytkownikiem, znajduje się w sekcji \ref{sec:algorithm-pixmap-transformat}.

\subsection{Kolekcje scen}
\input{text/implementacja/strukturydanych/dicom-scene-sets.tex}

\subsection{Zakładka}
\label{sec:sokar-dicomview}
\par
Każda zakładka z obrazem lub obrazami jest implementowana przez klasę \sokarclass{DicomView}.

\par
Interfejs graficzny \sokarclass{DicomView} wyświetla następujące elementy:
\begin{itemize}
    \item pasek narzędzi znajdujący się na górze --- implementowany za pomocą klasy \sokarclass{DicomToolBar}, opisany w sekcji \ref{sec:sokar-dicomtoolbar}
    \item miejsce na scene z obrazem DICOM na środku --- implementowany za pomocą klasy \sokarclass{DicomGraphics}, opisany w sekcji \ref{sec:sokar-dicomgraphics}
    \item suwak filmu w dolnej części --- implementowany za pomocą klasy \sokarclass{MovieBar}, opisany w sekcji \ref{sec:sokar-moviebar}
    \item podgląd miniaturek obrazów w prawej części --- implementowany za pomocą klasy \sokarclass{FrameChooser}, opisany w sekcji \ref{sec:sokar-framechooser}
\end{itemize}

\par
Dodatkowo posiada obiekt kolekcji scen, który jest zbiorem obrazów opisany w sekcji \ref{sec:sokar-scenesets}.

\begin{figure}[!htbp]
    \centering
    \includegraphics[width=\textwidth]{img/sokar-dicomview-001.png}
    \caption{Wygląd DicomView wraz z numeracją elementów interfejsu. Zdjęcie własne.}
    \label{fig:sokar-dicomview001}
\end{figure}

\subsubsection{Pasek narzędzi}
\label{sec:sokar-dicomtoolbar}
\par
Pasek narzędzi znajdujący się na górze, implementowany przez klasę \sokarclass{DicomToolBar}, dziedziczącą po klasy \qtclass{QToolBar}.
Posiada on zespół ikonek z rozwijalnymi menu kontekstowymi.

\par
Kliknięcie odpowiedniej ikony spowoduje wysłanie sygnału do obecnie wyświetlanej sceny.
Są dwa sygnały możliwe do wysłania \sokarfunction{DicomToolBar}{stateToggleSignal} lub \sokarfunction{DicomToolBar}{actionTriggerSignal}.
Pierwszy sygnał oznacza zmianę stanu paska, czyli sposób obsługi myszki i zawiera jeden argument: stan (typu \cppcode{enum}).
Sygnał ten okazał się bezużyteczny i nie jest obecnie wykorzystywany przez scene.
Drugi oznacza akcję, która powinna być wykonana na przez scenę.
Zawiera dwa argumenty: typ akcji (typu \cppcode{enum}) i stan akcji (typu \cppcode{bool} z domyślną wartością \cppcode{false}).

Ikony na pasku:
\begin{itemize}
    \item Okienkowanie (1)

          Stan: \cppcode{Windowing}.
          Oznacza, że horyzontalny ruch myszki powinien zmieniać szerokość okna, a wertykalny środek okna.
          Przycisk jest aktywny tylko wtedy, gdy obecna scena posiada obraz monochromatyczny.

    \item Przesuwanie (2)

          Stan: \cppcode{Pan}.
          Oznacza, że ruch myszki powinien przesuwać obraz na scenie w prawo, lewo, góra, dół, kiedy jest wciśnięty klawisz myszy.

          Rozwijalne menu zawiera tylko jedne element \enquote{Move To Center} wysyłający sygnał akcji z argumentem \cppcode{ClearPan}.

    \item Skalowanie (3)

          Stan: \cppcode{Zoom}.
          Oznacza, że ruch myszki powinien skalować obraz kiedy jest wciśnięty klawisz myszy.

          Menu rozwijalne:
          \begin{itemize}
              \item Fit To Screen --- Dopasuj do ekranu

                    Akcja: \cppcode{Fit2Screen}.

                    Po otrzymaniu sygnału obraz na scenie powinien dopasować swoją wielkość do wielkości sceny

              \item Original Resolution --- Skala jeden do jednego

                    Akcja: \cppcode{OriginalResolution}.

                    Po otrzymaniu sygnału obraz na scenie powinien dopasować swoją wielkość jeden do jednego w stosunku do piksela na ekranie.

          \end{itemize}

    \item Rotacja (4)

          Stan: \cppcode{Rotate}.
          Oznacza, że ruch myszki powinien obracać obrazem znajdującym się na scenie.

          Menu rozwijalne:
          \begin{itemize}
              \item Rotate Right --- Obróć w prawo

                    Akcja: \cppcode{RotateRight90}.

                    Po otrzymaniu sygnału obraz na scenie powinien obróć się o 90 stopni w prawo.

              \item Rotate Left --- Obróć w lewo

                    Akcja: \cppcode{RotateLeft90}.

                    Po otrzymaniu sygnału obraz na scenie powinien obróć się o 90 stopni w lewo.

              \item Flip Horizontal --- Odbij lustrzanie poziomo

                    Akcja: \cppcode{FlipHorizontal}.

                    Po otrzymaniu sygnału obraz na scenie powinien odbić się lustrzanie poziomo.

              \item Flip Vertical --- Odbij lustrzanie pionowo

                    Akcja: \cppcode{FlipVertical}.

                    Po otrzymaniu sygnału obraz na scenie powinien odbić się lustrzanie pionowo.

              \item Clear Transformation --- Wyczyść przekształcenia obrotu

                    Akcja: \cppcode{ClearRotate}.

                    Po otrzymaniu sygnału obraz na scenie powinien wyczyścić transformatę obrotu.

          \end{itemize}
    \item Informacje na obrazie (5)

          Ten element potrafi wyłączyć wyświetlanie niektórych elementów na scenie.
          Kliknięcie go odznacza lub zaznacza wszystkie pozycje w menu kontekstowym.
          Wszystkie pozycje są pozycjami odznaczanymi.

          Menu rozwijalne:
          \begin{itemize}
              \item Patient Data --- Dane pacjenta

                    Akcja: \cppcode{PatientData}.

                    Po otrzymaniu sygnału obiekt klasy \sokarclass{PatientDataIndicator} znajdujący się na scenie powinien pokazać lub ukryć się w zależności od stanu pozycji.

              \item Hospital Data --- Dane szpitala

                    Akcja: \cppcode{HospitalData}.

                    Po otrzymaniu sygnału obiekt klasy \sokarclass{HospitalDataIndicator} znajdujący się na scenie powinien pokazać lub ukryć się w zależności od stanu pozycji.
              \item Image Acquisition --- Dane akwizycji

                    Akcja: \cppcode{ModalityData}.

                    Po otrzymaniu sygnału obiekt klasy \sokarclass{ModalityIndicator} znajdujący się na scenie powinien pokazać lub ukryć się w zależności od stanu pozycji.

          \end{itemize}

    \item Tagi (5)

          Akcja: \cppcode{OpenDataSet}.

          Kliknięcie tego przycisku wyśle prośbę o otworzenie okna ze zbiorem elementów danych pliku obrazu, który jest obecnie wyświetlany na scenie.

\end{itemize}

\subsubsection{\sokarclass{DicomGraphics}}
\label{sec:sokar-dicomgraphics}

\subsubsection{Pasek filmu}
\label{sec:sokar-moviebar}

\par
Pasek filmu znajduje się w dolnej części zakładki i jest implementowany prze klasę \sokarclass{MovieBar},.
Ma dostęp do sekwencji scen i ukrywa swoją obecność przed użytkownikiem, kiedy w sekwencji jest tylko jedna scena.

\par
Pasek jest podzielony na trzy części: trzy przyciski znajdujące się po lewej, pasek pokazujący postępu sekwencji na środku i prządka z trzema przyciskami po prawej.

\par
Trzy lewe przyciski odpowiadają za poruszanie się po sekwencji.
Wciśniecie pierwszego przycisku (z indeksem 8 na rysunku \ref{fig:sokar-dicomview001}) powoduje zatrzymanie upływu sekwencji i wysłanie sygnału \sokarfunction{SceneSequence}{stepBackward} do sekwencji.
Wciśniecie drugiego przycisku (9) powoduje włączenie lub wyłączenie upływu sekwencji.
Wciśniecie trzeciego przycisku (10) powoduje zatrzymanie upływu sekwencji i wysłanie sygnału \sokarfunction{SceneSequence}{stepForward} do sekwencji.
\par
Pasek (11) pokazujący postępu sekwencji jest obiektem klasy \qtclass{QSlider}.
Odświeżanie paska jest wrażliwe na sygnał \sokarfunction{SceneSequence}{steped} of sekwencji.
\par
Elementy po prawej stronie definiuje parametry trybu filmowego.
Prządka (12), element do wprowadzania liczby zmiennoprzecinkowej klasy \qtclass{QDoubleSpinBox}.
Im większa wartość liczby tym klatki filmu są dłużej wyświetlane.
Drugi (13) przycisk pozwala zmienić sposób przemiatania.
Trzeci (14) przycisk wymusza tryb jednego okna dla wszystkich klatek filmu.
Jeżeli mamy załadowanych wile obrazów tego samego badania, to nie koniecznie muszą mieć to samo okno.
Dodatkowo ten tryb pozwala wprowadzić jednolite okienko dla wszystkich klatek po zmianie parametrów tego okienka na jednej klatce.
Czwarty (15) i ostatni przycisk służy do użycia jednej macierzy transformaty na wszystkich klatkach.

\paragraph*{Tryb filmowy}

\par
Tryb filmowy można aktywować jedynie wtedy gdy w sekwencji scen jest więcej niż jedna scena.
Włączenie trybu filmowego polega na stworzeniu obiektu klasy \sokarclass{MovieMode}.
Obiekt ten zapisuje wskaźnik go obecnie wyświetlanej sceny, to czy powinno być użyte to samo okno, oraz to czy powinna być używana ta sama transformata.
Następnie obiekt ten jest wysyłany do wszystkich scen w sekwencji.
Uruchamiany jest timer, obiekt klasy \qtclass{QTimer}, na czas równy czasu trwania sceny zapisanego w kroku przemnożonego przez liczbę z prządki.
Po upływie timera, wstawiana jest nowa scena za pomocą sygnały \sokarfunction{MovieBar}{setStep}, a timer jest ustawiany nan nowo.

\subsubsection{\sokarclass{FrameChooser}}
\label{sec:sokar-framechooser}

Ten element to wybór scen za pomocą ikon, implementowany przez klasę \sokarclass{DicomView}.
Element, podobnie jak pasek filmu ma dostęp do sekwencji scen i ukrywa swoją obecność przed użytkownikiem, kiedy w sekwencji jest tylko jedna scena.
Po wciśnięciu ikony jest zmieniana scena.

\subsection{Obiekt zakładek}
\label{sec:sokar-dicomtabs}

\par
Obiekt zakładek, implementowany za pomocą klasy \sokarclass{DicomTabs}, odpowiada za wyświetlanie wielu obiektów zakładek w jednym obiekcie interfejsu.
Obsługuje również wczytanie nowych plików.

\subsubsection{Sposoby uzyskania nowych plików}

\par
Otworzenie nowego pliku może odbyć się z następujących źródeł: obiektu drzewa ze strukturą plików w systemie (opisanego w \ref{sec:sokar-framechooser}), menu programu (opisanego w \ref{sec:sokar-window-menu}), lub poprzez przeciągnięcie i upuszczenie pliku.
Z dwóch pierwszych można wczytać tylko po jednym pliku, natomiast trzecim sposobem można wczytać zarówno jeden jak i wiele plików.
Wysyłanie prośby odbywa się za pomocą dwóch funkcji: \sokarfunction{DicomTabs}{addDicomFile} i \sokarfunction{DicomTabs}{addDicomFiles}.
Każda z tych funkcji ma dwa przeciążenia, jedno z parametrem ścieżki a drugie z wczytanym plikiem.

\subsubsection{Wczytywanie plików}

\par
Po dostarczeniu ścieżek do obiektu, pliki zostają wczytane za pomocą funkcji \gdcmclass{ImageReader}.
W przypadku błędu proces wczytywania się kończy.
Po wczytaniu wszystkich plików zostaje utworzony obiekt kolekcji ramek obrazu lub kolekcji plików \DICOM za pomocą funkcji \sokarfunction{DicomFileSet}{create}, opisanej w sekcji \ref{sec:sokar-dicomfileset-create}.


\subsection{Okno główne programu}
\label{sec:sokar-window}

\par
Główne okno programu jest implementowane przez \sokarclass{MainWindow}.
Jest wywoływane od razu po uruchomieniu programu.
\par
Zawiera w sobie 4 elementy: menu, drzewo ze strukturą plików, obiekt z zakładkami oraz w dolnej części okna sugestie, aby nie używać programu w celach medycznych.

\subsubsection{Drzewo katalogów i zakładki}

\par
W lewej części okna znajduje się element listy, implementowany przez \sokarclass{FileTree}, zawiera on w sobie model drzewa plików systemu, który z kolei jest implementowany przez klasę \qtclass{QFileSystemModel}.
Po wybraniu pliku ścieżka jest przesyłana do obiektu z zakładkami.
\par
W środkowej części programu znajduje się obiekt z zakładkami, szczegółowo opisany w sekcji \ref{sec:sokar-dicomtabs}.

\subsubsection{Menu programu}
\label{sec:sokar-window-menu}

\par
W górnej części okna programu znajduje się menu, obiekt klasy \sokarclass{QMenuBar}.
Struktura Menu programu:
\begin{itemize}
    \item File
          \begin{itemize}
              \item Open --- otwiera okienko wyboru plików, implementowane przez \qtfunction{QFileDialog}{getOpenFileName}, następnie wczytuje plik
              \item Open Recent --- program zapisuje ostatnio wczytane pliki i pozwala ja ich ponowne wczytanie z tego menu
              \item Export as --- zapisanie obrazu w formacie JPEG, BMP, GIF lub PNG.
                    Zapisywanie jest zaimplementowane przez funkcje \qtfunction{QImage}{save}, która umożliwia zapisanie obrazu do pliku.
              \item Exit --- wyjście z aplikacji
          \end{itemize}
    \item Help
          \begin{itemize}
              \item About Qt --- otwiera okno informacji o bibliotece Qt.
                    Biblioteka Qt ma wbudowane takie okno w postaci \qtfunction{QMessageBox}{aboutQt}
              \item About GDCM --- otwiera okno z informacjami o bibliotece GDCM, implementowane przez funkcje \sokarfunction{About}{GDCM}
              \item About Sokar --- otwiera okno z informacjami o aplikacji, implementowane przez funkcje \sokarfunction{About}{Sokar}
          \end{itemize}
\end{itemize}


\section{Algorytmy}


\subsection{Cykl generowania obrazów}
\label{sec:algorithm-pixmap-generate}

Klasa \sokarclass{DicomScene} dostarcza następujące obiekty do generowania obrazu:
\begin{itemize}
    \item \cppcode{processing}, obiekt klasy \qtclass{QMutex}, muteks do zablokowania podczas generowania obrazu, aby parametry obrazu nie mogły być zmienianie podczas jego generowania.

    \item \cppcode{imgDimX} zmienna typu \cppcode{uint}, oznacza szerokość obrazu w pikselach.

    \item \cppcode{imgDimY} zmienna typu \cppcode{uint}, oznacza wysokość obrazu w pikselach.

    \item \cppcode{targetBuffer} wektor docelowego obrazu RGB o długości $imgDimX*imgDimY$, typu \cppcode{std::vector<Pixel>}.

          \sokarclass{Pixel} to struktura reprezentujące piksel.
          Nie jest to w żadnym wypadku obiekt, a jedynie twór ułatwiający zarządzanie kodem.

          \begin{lstlisting}
struct Pixel {
    quint8 red = 0;   
    quint8 green = 0;    
    quint8 blue = 0;   
}\end{lstlisting}

        \par
        C++ od standardu C++03 przewiduje, że elementy znajdujące się w \stdclass{vector}{container/vector} są ułożone ciągiem, jeden za drugim.
        Dlatego odwołując się do wskaźnika pierwszego elementu w ten sposób \cppcode{\&targetBuffer[0]}, mogę potraktować to jako tablicę.

    \item \cppcode{originBuffer} wektor danych wypełniona danymi z jednej ramki o długośći iloczynu $imgDimX*imgDimY$ i ilości bajtów jednego piksela obrazu.

    \item \cppcode{qImage} obiekt obrazu klasy \qtclass{QImage}.

          \qtclass{QImage} można zrobić z istniejącego bufora, w tym przypadku jest to \cppcode{targetBuffer}.
          Format obrazu to \qtclass{QImage::Format\_RGB888}, czyli trzy bajty, każdy na jeden kanał.
          Proszę zwrócić uwagę, że struktura \sokarclass{Pixel} odpowiada temu formatowi.
          Według dokumentacji Qt obiekt ten po utworzeniu z istniejącego bufora powinien z niego dalej korzystać, dlatego zmiany \cppcode{targetBuffer} nie wymagają odświeżania \cppcode{qImage}.

    \item \cppcode{pixmap} obiekt obrazu do wyświetlania, klasy \qtclass{QPixmap}.

          Obiektów klasy \qtclass{QImage} nie da się wyświetlić, nie jest on przystosowany do wyświetlania.
          Natomiast klasa \qtclass{QPixmap} to reprezentacja obrazu dostosowana do wyświetlania ekranie, która może być używana jako urządzenie do malowania w bibliotece Qt.

    \item \cppcode{iconPixmap} obiekt obrazu ikonu, klasy \qtclass{QPixmap}, docelowo powinien mieć 128 pikseli na 128 pikseli.

\end{itemize}

Generowanie obrazu jest robione przez czysto wirtualną funkcje \sokarfunction{DicomScene}{generatePixmap}.
Po wywołaniu funkcji obiekt \cppcode{targetBuffer} powinien zawierać obraz wygenerowany z obecnymi parametrami.
Funkcja zwraca również wartość logiczną, który informuje nas czy \cppcode{targetBuffer} rzeczywiście został zmieniony.
Następnie obiekt \cppcode{pixmap} jest na nowo generowany na bazie \cppcode{qImage}.

Całe odświeżanie obrazu jest implementowane w funkcji \sokarfunction{DicomScene}{reloadPixmap}.
Funkcja wywołuje \sokarfunction{DicomScene}{generatePixmap} i odświeża \cppcode{pixmapItem} kiedy zajdzie taka potrzeba

Generowanie poszczególnych typów obrazów jest wyjaśnione poniżej.


\subsubsection{Obraz monochromatyczny}
\par
Obraz monochromatyczny to obraz w odcieniach szarości, od białego do czarnego lub od czarnego do białego.
Generowanie takiego obrazu odbyta się poprzez pseudokolorowanie.
Cały proces jest wyjaśniony w sekcji \ref{sec:algorithm-pixmap-monochrome}.

\subsubsection{RGB}
Obrazów zapisanych w RGB nie trzeba w żaden sposób obrabiać, dane już są prawie gotowe do wyświetlenia, należy je tylko odpowiednio posortować, jeżeli zachodzi taka potrzeba.
Sposób posortowania wartości w pliku określa \dicomtag{PlanarConfiguration}{0x0028}{0006}.
Może o przyjąć dwie następujące wartości:

\begin{itemize}
    \item 0 --- oznacza to, że wartości pikseli są ułożone w taki sposób
        \[R_1, G_1, B_1, R_2, G_2, B_2, R_3, G_3, B_3, R_4, G_4, B_4,  ...\]
    \item 1 --- oznacza to, że wartości pikseli są ułożone w taki sposób
        \[R_1, R_2, R_3, R_4, ... , G_1, G_2, G_3, G_4, ..., B_1, B_2, B_3, B_4, ...\]
\end{itemize}
gdzie:
\begin{itemize}
    \item $R_n$ --- wartość czerwonego kanału
    \item $G_n$ --- wartość zielonego kanału
    \item $B_n$ --- wartość niebieskiego kanału
\end{itemize}

Wartości obrazu są przepisywane do \cppcode{targetBuffer} dla biblioteki QT.

\subsubsection{YBR}
\label{sec:algorithm-pixmap-ybr}

\par
YBR albo YC\textsubscript{b}C\textsubscript{r} to model przestrzeni kolorów do przechowywania obrazów i wideo.
Wykorzystuje do tego trzy typy danych: Y – składową luminancji, B lub Cb – składową różnicową chrominancji Y-B, stanowiącą różnicę między luminancją a niebieskim, oraz R lub Cr – składową chrominancji Y-R, stanowiącą różnicę między luminancją a czerwonym.
Kolor zielony jest uzyskiwany na podstawie tych trzech wartości.
YBR nie pokrywa w całości RGB, tak jak RGB nie pokrywa YBR.
Posiadają one część wspólną, a część która nie jest wspólna ulega zniekształceniu.

\par
Wartości w pliku DICOM są ułożone w taki sposób.
\[Y_1, B_1, R_1, Y_2, B_2, R_2, Y_3, B_3, R_3, Y_4, B_4, R_4,  ...\]

\par
Ponieważ wartości te reprezentują kolory, są już formą obrazu, ale nie można jeszcze wyświetlić na monitorze RGB.
Dlatego należy przekonwertować kolor YBR na kolor RGB, iterując po wszystkich wartościach obrazu.

\par
Poniżej przedstawiono kod źródłowy funkcji zamiany kolory YBR na RGB.

\begin{lstlisting}
Sokar::Pixel ybr2Pixel(quint8 y, quint8 b, quint8 r) {
    qreal red, green, blue;

    red = green = blue = (255.0 / 219.0) * (y - 16.0);

    red += 255.0 / 224 * 1.402 * (r - 128);
    green -= 255.0 / 224 * 1.772 * (b - 128) * (0.114 / 0.587);
    green -= 255.0 / 224 * 1.402 * (r - 128) * (0.299 / 0.587);
    blue += 255.0 / 224 * 1.772 * (b - 128);

    /* W tym miejscu jest dokonywana utrata danych */
    red = qBound(0.0, red, 255.0);
    green = qBound(0.0, green, 255.0);
    blue = qBound(0.0, blue, 255.0);

    return Sokar::Pixel(quint8(red), quint8(green), quint8(blue));
}
\end{lstlisting}


\subsection{Generowania obraz monochromatycznego}
\label{sec:algorithm-pixmap-monochrome}

Obraz monochromatyczny to obraz w odcieniach szarości, od białego do czarnego lub od czarnego do białego. Dane są zapisane w sposób ciągły wartość po wartości.

\subsubsection{Pseudokolorowanie obrazu}

Mamy obraz, którego piksele to n-bitowe liczby, na przykład 16 bitowa liczba całkowita.
W takiej postaci wyświetlemoe obrazu na monitorze RGB lub nawet na profesjonalnym 10-bitowym jest niemożliwe.
Należy taką liczbę przerobić na trzy liczby, reprezentujące 3 kanały RGB, czerwony, zielony i niebieski.
Dlatego do wyświetlania obrazów monochromatycznych o dużym kontraście stosuję się twór zwany okienkiem.
Jest to funkcja, która mapuje n-bitwy obraz na 8-bitowy obraz w skali szarości.
8-bitów, ponieważ monitor RGB jest wstanie wyświetlić 256 odcieni szarości.

\paragraph*{Zwiększanie kontrastu za pomocą \enquote{funckji okna}}
Przyjeło się, że \enquote{okno} definiuje się dwoma liczbami: środkiem, oznaczanym jako $center$ i długością, oznaczaną jako $width$.
Wyznaczamy zakres okienka $x_0$ i $x_1$ ze środka okienka $center$ i długości $width$.
\[x_0 = center - width / 2\]
\[x_1 = center + width / 2\]
Wyznaczamy parametry $a$ i $b$, prostej przechodzącej przez dwa punkty $(x_0, y_0)$ i $(x_1, y,_1)$.
Gdzie $y_0$ jest równe 0, a $y_1$ jest równe 255.
Funkcja \enquote{okna} wygląda następująco:
\[
    f(v)=
    \begin{cases}
        0     & \text{gdy $0 \le v \wedge v \le x_0$ } \\
        a*x+b & \text{gdy $x_0 < v \wedge v < x_1$}    \\
        255   & \text{gdy $x_1 \le v \wedge v \le 1$ }
    \end{cases}
\]

gdzie $v$ to wartość piksela danych obrazu.

Następnie iterujemy przez wszystkie woksele obrazu i używamy na nich funkcji \enquote{okna} i otrzymujemy obraz w skali od $0$ do $255$.
Taki obraz w skali można już wyświetlić.
Natomiast standard DICOM przewiduje, że obraz można jeszcze wyświetlić w wielokolorowej palecie barw.
Przykład takiej palety HotIron w porównaniu do skali szarości można zobaczyć na rysunku .
Taka paleta barw nie koniecznie musi mieć 256 odcieni, dlatego lepiej jest zrobić aby okienko, mapowało na liczbę od 0 do 1, a później paleta mapowała na kolor RGB.

\begin{figure}[!htbp]
    \centering
    \includegraphics[width=0.7\textwidth]{img/monochrome-001.png}
    \caption{Paleta HotIron w porównaniu do palety w skali szarości. Zdjęcie ze standardu DICOM dostępne pod adresem \url{http://dicom.nema.org/medical/dicom/2019a/output/chtml/part06/chapter_B.html}.}
    \label{fig:monochrome1}
\end{figure}

Teraz iterujemy po wszystkich możliwych wartościach wartośćiach obrazu i wykonujemy takie operacje.
\begin{itemize}
    \item wyznaczenie wartości okienka.
          \[y = a * x + b\]
    \item y zostaje obcięcie do 1.0 lub 0.0 jeżeli wyjdzie poza zakres od 1.0 do 0.0
    \item pobranie z palety piksel odpowiadający wartości
    \item wsadzenie piksela do tablicy, tak aby najmniejsza wartości obrazu miała indeks 0 a największy ostani
\end{itemize}


\subsubsection{Implementacja algorytmu}

\paragraph{Opis}
\par
Implementacja powyżej przedstawionego algorytmu w sposób dosłowny byłaby mało optymalna dla maszyny i wymagała by wielu pobocznych tablic oraz względnie dużej ilości mnożenia.
Trzeba też zauważyć, że do wyliczenie jakiegoś piksela nie potrzeba liczyć, żadnego innego piksela, co skutkuje, że każdy piksel można wyliczyć oddzielnie.
Dlatego najlepiej było by współbieżnie przelecieć po całym obrazie i zamienić dane na piksele.
Ale do zamiany dane na piksel, musimy mnożyć i dzielić liczby zmiennoprzecinkowe, a to do najszybszych nie należy.
Dlatego dobrym pomysłem jest zrobienie mniejszej tablicy typu LookUpTable, wypełnienie jej wszystkimi możliwymi wartościami obrazu, a następnie przerobić obraz z tablicą LUT.
Ale ponieważ tablica LUT posiada wszystkie możliwe kombinacje wartości, jej rozmiar można wyznaczyć wzorem: $2^N*3$, gdzie N to liczba bitów liczby.
Standard DICOM definiuje, że liczby mogą mieć $8$, $12$, $16$, $32$ i $64$ bity, jednakże, $12$ bitowe i tak się zapisuje w postaci 16-bitowych w pamięci RAM.
Dlatego możliwe wartości wielkości tablicy LUT to w przybliżeniu: $768$ bajtów, $196$ kilobajtów, $12,5$ gigabajtów i $56$ eksabajta($55*10^{6}$ terabajtów).
Alokowanie dwóch największych wartości może być lekko problematyczne, dlatego zrobiłem dwie implementacje algorytmu: z tablicą LUT(dla 8 i 16 bitowych obrazów i bez tablicy LUT(dla 32 i 64 bitowych obrazów).
Algorytm składa się z 3 części: wyznaczenie parametrów okna, przygotowanie okna (tylko gdy jest tablica LUT), wielowątkowa iteracja po obrazie.
\par
Okno z LUT jest implementowane przez \sokarclass{Monochrome}{WindowIntDynamic}.
Okno bez LUT jest implementowane przez \sokarclass{Monochrome}{WindowIntDynamic}.
Obie klasy dziedziczą po abstrakcyjnej klasie \sokarclass{Monochrome}{Window}, która z kolei dziedziczy po \sokarclass{SceneIndicator}, dlatego od razu może wyświetlać obecne wartości okna.
Decyzja o używanym oknie jest podejmowana podczas wczytywania obrazu przez klasę \sokarclass{Monochrome{\scopedots}Scene}
\par
UWAGA: Standard DICOM zakłada, że danymi mogą być liczby całkowite(\cppcode{int}) oraz zmiennoprzecinkowe(\cppcode{float} lub \cppcode{double}), ale praktycznie, nie ma takich aparatów medycznych, które zapisywały by takie obrazy, gdzie dane to liczby zmiennoprzecinkowe. Dlatego założyłem, że takie obrazy nie istnieją.

\paragraph{Wyznaczenie parametrów okna}
\par
Najpierw wyznaczam okienko, które zmienia wartości obrazu na skale od zera do jeden:
\[x_0 = center - width / 2\]
\[x_1 = center + width / 2\]
\[y_1 = 0.0\]
\[y_0 = 1.0\]
gdzie:
\begin{itemize}
    \item $center$ --- środek okienka
    \item $width$ --- szerokość okienka
    \item $x0$ i $y0$ --- współrzędne pierwszego punktu
    \item $x1$ i $y1$ --- współrzędne drugego punktu
\end{itemize}
Przeglądarka pozwala na inwersje okienka.
Dlatego kiedy użytkownik zażyczy sobie inwersji, zmienne $y0$ i $y1$ zamienią się wartoścami.

Standard DICOM przewiduje, że wszystkie dane powinny być wyskalowane, za pomocą wzoru.
\[OutputUnits = m*SV + b\]
gdzie:
\begin{itemize}
    \item $m$ --- wartość z \dicomtag{RescaleSlope}{0028}{1053}
    \item $b$ --- wartość z \dicomtag{RescaleIntercept}{0028}{1052}
    \item $SV$ --- stored values - warość pixela z pliku
    \item $OutputUnits$ --- wartość wynikowa
\end{itemize}

Wartości okienka odnoszą się do wartości już wyskalowanej, a ponieważ skalowanie całego obrazu jest czasochłonne, przeskalowaie okienka da taki sam efekt:
\[(OutputUnits - b ) / m = SV \]
więc:
\[x_0 -= rescaleIntercept\]
\[x_1 -= rescaleIntercept\]
\[x_0 /= rescaleSlope\]
\[x_1 /= rescaleSlope\]

Posiadamy, teraz dwa punkty okienka odnoszące się do wartośći obrazu.
Wyznaczam parametry prostej przechodzącej przez dwa punkty:
\[a = (y_1 - y_0) / (x_1 - x_0)\]
\[b = y_1 - a * x_1\]

\par
Teraz algorytm się rozdwaja.
Pobieranie wartości z okienka odbywa się za pomocą funkcji \sokarclass{Monochrome}{Window\zerospace::{\zerospace}getPixel()}.

\paragraph{Implementacja dynamiczna bez tablicy LUT}

\par
W tej wersji funkcja \sokarfunction{Monochrome{\scopedots}Window}{getPixel} wygląda następująco:
\par
\begin{lstlisting}
inline const Pixel &getPixel(quint64 value) override {
    if (value < x0) {
        return background;
    } else if (value > x1) {
        return foreground;
    } else {
        return palette->getPixel(a * value + b);
    }
}
\end{lstlisting}
\par
Widzimy tutaj, że funkcja najpierw sprawdza czy zakres okienka został przekroczony, następnie wylicza wartość obrazu i pobiera kolor z palety.
\par
UWAGA: ponieważ nie dysponuje rzeczywistym obrazem o pikselu danych 32-bitowym lub 64-bitowych, implementacja dynamiczna nie była testowana w warunkach rzeczywistych.

\paragraph{Implementacja statyczna z tablicą LUT}
\par
W wersji z LUT, podczas tworzenia okienka jest alokowany wektor obiektów \sokarclass{Pixel} klasy \stdclass{vector}{container/vector}.
Standard DICOM przewiduje, że woksele mogą mieć wartości ujemne, więc tablica powinna mieć możliwość posiadania takich wartości indeksów, ale C++ nie przewiduje takiej możliwości.
Dlatego wprowadzono dwie zmienne pomocnicze \cppcode{maxValue} i \cppcode{signedMove}.
\cppcode{maxValue} jest to maksymalna wartość jaką dane mogą przyjąć, jest ona równa $2^N$, gdzie $N$ to liczba bitów brana z \dicomtag{BitsStored}{0028}{0101}.
A \cppcode{signedMove} to liczba przesunięcia liczb, przyjmuje wartość zero gdy dane wokseli są całkowite nieujemne lub wartość przeciwną do \cppcode{maxValue} gdy woksele mogą być ujemne.
Długość wektora pikseli jest sumą \cppcode{maxValue} i \cppcode{signedMove}.
A indeks woksela w wektorze ma wartość tego woksela zwiększoną o \cppcode{signedMove}.
\par
Wypełnienie wektora wartościami odbywa się poprzez iteracje po wszystkich możliwych wartościach, przeliczenie ich przez funkcje okna, a następnie wstawienie ich do wektora.
W celu poprawy szybkości, zastosowano sprawdzanie czy wartości są w zakresie okna.
Poniżej kod funkcji:
\begin{lstlisting}
bool genLUT() override {

    if (WindowInt::genLUT()) {

        /* Przeskalowanie wektora, gdy jest to wymagane */
        if (arraySize != signedMove + maxValue) {
            arraySize = signedMove + maxValue;
            arrayVector.resize(arraySize);
        }

        /* Wyliczenie najmniejszej wartości */
        qreal x = qreal(signedMove) * -1;

        auto &background = isInversed() ? palette->getForeground() : palette->getBackground();
        auto &foreground = isInversed() ? palette->getBackground() : palette->getForeground();

        /* Iteracja */
        pixelArray = &arrayVector[0];
        for (int i = 0; i <= arraySize; i++) {

            if (x < x0) {
                *pixelArray = background;
            } else if (x > x1) {
                *pixelArray = foreground;
            } else {
                *pixelArray = palette->getPixel(a * x + b);
            }

            x++;
            pixelArray++;
        }

        pixelArray = &arrayVector[0];

        updateLastChange();

        return true;
    }
    return false;
}
\end{lstlisting}

\par
Funkcja pobierania piksela z \enquote{okna} prezentuje się następująco:
\begin{lstlisting}
inline const Pixel &getPixel(quint64 value) override {
    return *(pixelArray + signedMove + value);
}
\end{lstlisting}

\paragraph{Iterowanie po obrazie}
\par
Po przygotowaniu okienka, należy przeiterować obraz przez funkcje \enquote{okna}.
Do zokienkowania jednego piksela nie potrzeba innego piksela dlatego w celu przyspieszenia procesu okienkowania, iteracja po obrazie odbywa się w wielu wątkach.
\par
W C++ typy zmiennych muszą być zdefiniowane przed kompilacją, co jest pewnym problemem.
Mając dwa typy okienek, każde odsługujące 4 typy liczb całkowitych, musiało by zostać zaimplementowanych 8 prawie identycznych funkcji.
Dlatego podział ten został zaimplementowany za pomocą 4 funkcji z szablonami:
\begin{itemize}
    \item \sokarfunction{Monochrome{\scopedots}Scene}{genQPixmapOfTypeWidthWindowThread}

          Jest funkcją jednego wątku, który iteruje po obrazie.
          Jego parametrami są zakresy podane w indeksach wokseli po któych ma iterować.
          \cppcode{IntType} to jest typ zmiennej woksela obrazu.
          \cppcode{WinClass} klasa okienka.
          Nazewnictwo będzie kontynuowane w następnych punktach.
          Kod funkcji:
          \begin{lstlisting}
template<typename IntType, typename WinClass>
void Monochrome::Scene::genQPixmapOfTypeWidthWindowThread(quint64 from, quint64 to) {

	auto buffer = &targetBuffer[from];
	auto origin = (IntType *) &originBuffer[0];
	auto windowPtr = (WinClass *) getCurrentWindow();

	origin += from;

	for (quint64 i = from; i < to; i++, origin++) {
		*buffer++ = windowPtr->getPixel(*origin);
	}
}
\end{lstlisting}

    \item \sokarfunction{Monochrome{\scopedots}Scene}{genQPixmapOfTypeWidthWindow}

          Jest funkcją, która dzieli obraz na wątki, tworzy je i uruchamia.
          Ilość wątków jest ustalana za pomocą funkcji \qtfunction{QThread}{idealThreadCount}.
          Wątki działają na zakresach o długości ilości wokseli podzielonej prze ilość wątków.
          Kod funkcji:
\begin{lstlisting}
template<typename IntType, typename WinClass>
void Monochrome::Scene::genQPixmapOfTypeWidthWindow() {

    /* Tworzenie wektora wątków */
    std::vector<std::thread> threads;

    quint64 max = imgDimX * imgDimY;
    quint64 step = max / QThread::idealThreadCount();

    for (int i = 1; i < QThread::idealThreadCount(); i++) {
        std::thread t(&Scene::genQPixmapOfTypeWidthWindowThread<IntType, WinClass>,
                      this,
                      i * step,
                      std::min((i + 1) * step, max));

        threads.push_back(std::move(t));
    }

    /* W celu zmniejszenia ilości wątków wątke obecny też zostanie wykorzystany */
    genQPixmapOfTypeWidthWindowThread<IntType, WinClass>(0, step);

    /* Czekanie na wszystkie wątki */
    for (auto &t: threads) t.join();
}
\end{lstlisting}

    \item \sokarfunction{Monochrome{\scopedots}Scene}{genQPixmapOfType}
 
    Jest ot funkcja pomocnicza ustalająca obecne klasę obecnego okna aby móc wykonać funkcje \sokarfunction{Monochrome{\scopedots}Scene}{genQPixmapOfTypeWidthWindow}.
    Kod funkcji:
\begin{lstlisting}
template<typename IntType>
void Monochrome::Scene::genQPixmapOfType() {

    switch (getCurrentWindow()->type()) {
        case Window::IntDynamic:
            genQPixmapOfTypeWidthWindow<IntType, WindowIntDynamic>();
            break;

        case Window::IntStatic:
            genQPixmapOfTypeWidthWindow<IntType, WindowIntStatic>();
            break;

        default:
            throw WrongScopeException(__FILE__, __LINE__);
    }
}
\end{lstlisting}

    \item \sokarfunction{Monochrome{\scopedots}Scene}{generatePixmap}
    
    Funkcja odświeża okienko i sprawdza czy odświeżenie obrazu jest konieczne, następnie sprawdza typ liczby woksela i uruchamia \sokarfunction{Monochrome{\scopedots}Scene}{genQPixmapOfType}.
    Kod funkcji:
\begin{lstlisting}
bool Monochrome::Scene::generatePixmap() {

    /* Odświeżamy okno i sprawdzamy czy odświeżenie obrazu jest konieczne */
    getCurrentWindow()->genLUT();
    if (lastPixmapChange >= getCurrentWindow()->getLastChange()) return false;

    /* Sprawdzamy typ liczby woksela obraau */
    switch (gdcmImage.GetPixelFormat()) {
        case gdcm::PixelFormat::INT8:
            genQPixmapOfType<qint8>();
            break;
        case gdcm::PixelFormat::UINT8:
            genQPixmapOfType<quint8>();
            break;
        case gdcm::PixelFormat::INT16:
            genQPixmapOfType<qint16>();
            break;
        case gdcm::PixelFormat::UINT16:
            genQPixmapOfType<quint16>();
            break;
        case gdcm::PixelFormat::INT32:
            genQPixmapOfType<qint32>();
            break;
        case gdcm::PixelFormat::UINT32:
            genQPixmapOfType<quint32>();
            break;
        case gdcm::PixelFormat::INT64:
            genQPixmapOfType<qint64>();
            break;
        case gdcm::PixelFormat::UINT64:
            genQPixmapOfType<quint64>();
            break;

        default: /* W przypadku innych jest zwracany wyjątek */
            throw Sokar::ImageTypeNotSupportedException();
    }

    pixmap.convertFromImage(qImage);
    return true;
}
\end{lstlisting}
\end{itemize}

\subsubsection{Palety}
Klasa \sokarclass{Palette} reprezentuje palety kolorów używanych do pseudokolorwania brazu monochromatycznego.
Paleta przerabia liczbę zmiennoprzecinkową od zera do jedynki na kolor RGB, zwracając \sokarclass{Pixel}.
Paleta nie ma zdefiniowanej długości minimalne i maksymalnej.

\par
Palety są wczytywane z plików XML w czasie uruchamiania programu i można definiować własne palety nie będące częścią standardem.
Przykładowy wygląd pliku palety HotIron:
\begin{lstlisting}[language=XML]
<palette display="Hot Iron" name="HOT_IRON">

    <entry red="0" green="0" blue="0"/>
    <entry red="2" green="0" blue="0"/>
    <entry red="4" green="0" blue="0"/>

    ...

    <entry red="254" green="0" blue="0"/>
    <entry red="255" green="0" blue="0"/>
    <entry red="255" green="2" blue="0"/>

    ...

    <entry red="255" green="250" blue="248"/>
    <entry red="255" green="252" blue="252"/>
    <entry red="255" green="255" blue="255"/>
</palette>
\end{lstlisting}

\subsection{Generowania obraz YBR}
\label{sec:algorithm-pixmap-ybr}

\par
YBR albo YC\textsubscript{b}C\textsubscript{r} to model przestrzeni kolorów do przechowywania obrazów i wideo.
Wykorzystuje do tego trzy typy danych: Y – składową luminancji, B lub Cb – składową różnicową chrominancji Y-B, stanowiącą różnicę między luminancją a niebieskim, oraz R lub Cr – składową chrominancji Y-R, stanowiącą różnicę między luminancją a czerwonym.
Kolor zielony jest uzyskiwany na podstawie tych trzech wartości.
YBR nie pokrywa w całości RGB, tak jak RGB nie pokrywa YBR.
Posiadają one część wspólną, a część która nie jest wspólna ulega zniekształceniu.

\par
Wartości w pliku DICOM są ułożone w taki sposób.
\[Y_1, B_1, R_1, Y_2, B_2, R_2, Y_3, B_3, R_3, Y_4, B_4, R_4,  ...\]

\par
Ponieważ wartości te reprezentują kolory, są już formą obrazu, ale nie można jeszcze wyświetlić na monitorze RGB.
Dlatego należy przekonwertować kolor YBR na kolor RGB, iterując po wszystkich wartościach obrazu.

\par
Poniżej przedstawiono kod źródłowy funkcji zamiany kolory YBR na RGB.

\begin{lstlisting}
Sokar::Pixel ybr2Pixel(quint8 y, quint8 b, quint8 r) {
    qreal red, green, blue;

    red = green = blue = (255.0 / 219.0) * (y - 16.0);

    red += 255.0 / 224 * 1.402 * (r - 128);
    green -= 255.0 / 224 * 1.772 * (b - 128) * (0.114 / 0.587);
    green -= 255.0 / 224 * 1.402 * (r - 128) * (0.299 / 0.587);
    blue += 255.0 / 224 * 1.772 * (b - 128);

    /* W tym miejscu jest dokonywana utrata danych */
    red = qBound(0.0, red, 255.0);
    green = qBound(0.0, green, 255.0);
    blue = qBound(0.0, blue, 255.0);

    return Sokar::Pixel(quint8(red), quint8(green), quint8(blue));
}
\end{lstlisting}

\subsection{Tworzenie transformat i ich użycie na obrazie}
\label{sec:algorithm-pixmap-transformat}

\subsubsection{Współrzędne jednorodne}

Współrzędne jednorodne definiuje się jako sposób reprezentacji punktów n-wymiarowej przestrzeni rzutowej za pomocą układu $n+1$ współrzędnych.
W bibliotece Qt jedną z implementacji współrzędnych jednorodnych jest klasa \qtclass{QTransform}.
Implementuje ona podstawowe zachowania macierzy 3 na 3, jak również wbudowane operacje takie jak: przesuwanie implementowane prze \qtfunction{QTransform}{translate}, obrót implementowany przez funkcję \qtfunction{QTransform}{rotate} i skalowanie implementowane przez \qtfunction{QTransform}{scale}.

Przykład działania:
\begin{lstlisting}
QTransform transform;
transform.translate(50, 50);
transform.rotate(45);
transform.scale(0.5, 1.0);
\end{lstlisting}
Powyższe przekształcenie macierzowe skaluje obiekt na 50\% szerokości, obraca o 45 stopni, przesuwa o 50 punktów na osi $x$ i $y$.

\par
Taką macierz można nałożyć na obiekty klasy \qtclass{QGraphicsPixmapItem}.

\subsubsection{Interakcja z użytkownikiem}

Trzy macierze (bez wyśrodkowującej) są zmieniane w trakcie interakcji z użytkownikiem.
Są zmieniane w dwóch przypadkach: po odebraniu sygnału od paska zadań, obiektu klasy \sokarclass{DicomToolbar} lub podczas ruchu myszki, gdy wciśnięty jest prawy przycisk myszy.

\paragraph{Zmiany poprzez oderanie sygnału}

\par
Na pasku zadań, nad sceną znajduje się szereg przycisków, które po wciśnięciu wysyłają sygnał do obecnej sceny poprzez obiekt klasy \sokarclass{DicomView}.
Sposób wysyłania sygnałów jest szerzej opisany w sekcji \ref{sec:sokar-dicomtoolbar}.

\par
Po otrzymaniu odpowiedniego sygnału jest wykonywana operacja na macierzy.
Odbiór wszystkich sygnałów jest implementowany przez wirtualną funkcję \sokarfunction{DicomScene}{toolBarActionSlot}, która jest slotem.

\par
Lista opisów reakcji na sygnały (stan zerowy macierzy, to stan w którym macierz nie wykonuje żadnych operacji):
\begin{itemize}

    \item \cppcode{ClearPan} --- przywraca macierz przesunięcia do stanu zerowego

    \item \cppcode{Fit2Screen} --- przywraca macierz skali do stanu zerowego, następnie wylicza nową skalę w zależności od wymiarów obrazu i sceny

    \item \cppcode{OriginalResolution} --- przywraca macierz skali do stanu zerowego

    \item \cppcode{RotateRight90} --- na macierzy rotacji zostaje użyta funkcja \qtfunction{QTransform}{rotate} z parametrem $90$.

    \item \cppcode{RotateLeft90} --- na macierzy rotacji zostaje użyta funkcja \qtfunction{QTransform}{rotate} z parametrem $-90$.

    \item \cppcode{FlipHorizontal} --- na macierzy rotacji zostaje użyta funkcja \qtfunction{QTransform}{scale} z parametrami $1$ i $-1$.

    \item \cppcode{FlipVertical} --- na macierzy rotacji zostaje użyta funkcja \qtfunction{QTransform}{scale} z parametrami $-1$ i $1$.

    \item \cppcode{ClearRotate} --- przywraca macierz rotacji do stanu zerowego

\end{itemize}
Po jakiejkolwiek zmianie macierzy jest wywoływana funkcja \sokarfunction{DicomScene}{updatePixmapTransformation}, która odświeża macierz przekształcenia na obiekcie \cppcode{pixmapItem}.

\paragraph{Zmiany poprzez obsługę myszki}

\par
\qtclass{QGraphicsScene} dostarcza możliwość obsługi myszki poprzez wirtualną funkcję \qtfunction{QGraphicsScene}{mouseMoveEvent}.
Dzięki temu obsługa myszki może być rozszerzana przez wszystkie klasy dziedziczące po tej klasie.
Dodatkowo funkcja ta dostarcza obiekt klasy \qtclass{QGraphicsSceneMouseEvent}, w którym znajdują się informacje zarówno o obecnej jak i poprzedniej pozycji myszki.

\par
Jeżeli jest wykryty ruch myszki z wciśniętym lewym przyciskiem myszy, to w zależności od stanu paska narzędzi, wywoływana jest odpowiednia akcja.
Akcje są obsługiwane przez klasy \sokarclass{DicomScene} i \sokarclass{Monochrome{\scopedots}Scene}.
Każda z nich obsługuje pewną pulę stanów.
Lista obsługiwanych stanów paska narzędzi:
\begin{itemize}
    \item \cppcode{Pan} --- stan przesuwania, obsługiwany przez \sokarclass{DicomScene}

          Na macierzy przesuwania wywoływana jest funkcja przesunięcia \qtfunction{QTransform}{translate} z parametrami odpowiadającymi przesunięciu myszki.

    \item \cppcode{Zoom} --- stan skalowania, obsługiwany przez \sokarclass{DicomScene}

          Na macierzy skalowania wywoływana jest funkcja skalowania \qtfunction{QTransform}{scale} z parametrem \cppcode{scale} wyliczanym podanym wzorem:

          \[scale=1\]
          \[scale=scale-{\Delta}y*0.01\]
          \[scale=scale-{\Delta}x*0.001\]

          Sprawia to, że ruch pionowy jest bardziej czuły na zmianę niż ruch poziomy.
          Teoretycznie jest możliwość implementacji odrębnego skalowania w dwóch osiach, jednakże jest to nieintuicyjne.

    \item \cppcode{Rotate} --- stan rotacji, obsługiwany przez \sokarclass{DicomScene}

          Na macierzy rotacji wywoływana jest funkcja rotacji \qtfunction{QTransform}{rotate} z parametrem \cppcode{rotate} wyliczanym podanym wzorem:

          \[rotate = 0\]
          \[rotate = rotate + {\Delta}y * 0.5;\]
          \[rotate = rotate + {\Delta}x * 0.1;\]

          Sprawia to, że ruch pionowy jest bardziej czuły na zmianę niż ruch poziomy.

    \item \cppcode{Windowing} --- stan okienkowania, obsługiwany przez \sokarclass{Monochrome{\scopedots}Scene}

          Do obiektu okienka są wysyłane zmiany poprzez funkcje: \sokarfunction{Window}{mvVertical} z parametrem ${\Delta}y$ i \sokarfunction{Window}{mvHorizontal} z parametrem ${\Delta}x$.
          Następnie ponownie jest generowany obraz z uwzględnieniem zmiany okienka.

\end{itemize}

\paragraph{Połączenie macierzy}
\par
Ostatnim krokiem jest połączenie macierzy w jedną.
Dlatego cztery macierze są mnożone za pomocą wirtualnej funkcji \sokarfunction{DicomScene}{getPixmapTransformation}.
Kod funkcji:
\begin{lstlisting}
QTransform DicomScene::getPixmapTransformation() {
	QTransform transform;
	transform *= centerTransform;
	transform *= scaleTransform;
	transform *= rotateTransform;
	transform *= panTransform;
	return transform;
}
\end{lstlisting}
\qtclass{QTransform} posiada operator mnożenia, dlatego można mnożyć obiekty tej klasy jak liczby.
Realizuje to następujące równanie:
\[panTransform*rotateTransform*scaleTransform*centerTransform\]


\subsection{Ustalanie pozycji pacjenta względem sceny}
\label{sec:algorithm-imageorientationindicator}

Pary poszczególnych liter tworzą osie:
\begin{itemize}
    \item \quotett{x} --- oś przechodząca od prawej do lewej strony pacjenta, \dataword{L} oznacza zwrot zgodny z osią, a \dataword{R} oznacza zwrot przeciwny

    \item \quotett{y} --- oś przechodząca od przodu do tyłu pacjenta, \dataword{P} oznacza zwrot zgodny z osią, a \dataword{A} oznacza zwrot przeciwny

    \item \quotett{z} --- oś przechodząca od dołu do góry pacjenta, \dataword{H} oznacza zwrot zgodny z osią, a \dataword{F} oznacza zwrot przeciwny

\end{itemize}

\begin{figure}[!htbp]
    \centering
    \includegraphics[width=0.7\textwidth]{img/imageorientationindicator-003.pdf}
    \caption{Wizualizacja układu osi współrzędnych pacjenta. Zdjęcie własne.}
    \label{fig:imageorientationindicator2}
\end{figure}

Informacje o orientacji oraz pozycji względem pacjenta znajdują się w odpowiednio w tagach \dicomtag{ImageOrientation}{0020}{0037} i \dicomtag{ImagePosition}{0020}{0032}.
Wartość \dicomtag{ImageOrientation}{0020}{0037} składa się z sześciu liczb, opowiednio oznaczanych dalej X\textsubscript{x}, X\textsubscript{y}, X\textsubscript{z}, Y\textsubscript{x}, Y\textsubscript{y}, Y\textsubscript{z}.

Standard DICOM definiuje, że te dane mają być z interpretowane w następujący sposób:
\[
    \begin{bmatrix}
        P_x \\ P_y \\ P_z \\ 1
    \end{bmatrix}
    =
    \begin{bmatrix}
        X_x\Delta_i & Y_x\Delta_j & 0 & S_x \\
        X_y\Delta_i & Y_y\Delta_j & 0 & S_y \\
        X_z\Delta_i & Y_z\Delta_j & 0 & S_z \\
        0           & 0           & 0 & 1
    \end{bmatrix}
    \begin{bmatrix}
        i \\ j \\ 0 \\ 1
    \end{bmatrix}
    =
    M
    \begin{bmatrix}
        i \\ j \\ 0 \\ 1
    \end{bmatrix}
\]
gdzie:
\begin{itemize}
    \item $P_{xyz}$ --- koordynaty woksela (i,j) w macierzy obrazu wyrażone w milimetrach
    \item $S_{xyz}$ --- trzy wartości z elementu ze znacznikiem \dicomtag{ImagePosition}{0020}{0032}. Oznacza punkt pozycji pacjenta wyrażony w milimetrach w stosunku do urządzenia wykonującego pomiar.
    \item $X_{xyz}$ --- trzy pierwsze wartości z \dicomtag{ImageOrientation}{0020}{0037}
    \item $Y_{xyz}$ --- trzy ostatnie wartości z \dicomtag{ImageOrientation}{0020}{0037}
    \item $i$ i $j$ --- oznaczają współrzędne na macierzy obrazu, odpowiednio kolumnę i wiersz. Zero oznacza początek.
    \item $\Delta_i$ i $\Delta_j$ --- rzeczywista wielkość piksela obrazu wyrażoną w milimetrach, w algorytmie wyznaczania strony pacjenta ta wartość, może wynosić 1, ponieważ odpowiada za skale
\end{itemize}

Praktycznie rzecz biorąc, pierwsza macierz to wektor reprezentujący pozycję pacjenta.
Druga jest to transformata.
Trzecia to pozycja na obrazie.

Interesuje nas wyznaczenie pozycji sześciu (punktów) na płaszczyźnie obrazu, o następujących współrzędnych, dalej używanych pod nazwą $PatientPosition$:
\begin{itemize}
    \item \quotett{R} - $[-1, 0, 0, 1]$
    \item \quotett{L} - $[+1, 0, 0, 1]$
    \item \quotett{A} - $[0, -1, 0, 1]$
    \item \quotett{P} - $[0, +1, 0, 1]$
    \item \quotett{F} - $[0, 0, -1, 1]$
    \item \quotett{H} - $[0, 0, +1, 1]$
\end{itemize}

UWAGA: Wszystkie obliczenia odbywają się w współrzędnych jednorodnych.

Wykonuje takie przekształcenie:
\[PatientPosition = imgMatrix * ScenePosition\]
\[imgMatrix^{-1} * PatientPosition = imgMatrix^{-1} * imgMatrix * ScenePosition\]
\[imgMatrix^{-1} * PatientPosition = ScenePosition\]
\[ScenePosition = imgMatrix^{-1} * PatientPosition\]
gdzie:
\begin{itemize}
    \item $imgMatrix$ --- macierz przekształcenia obrazu, o której będzie dalej
    \item $ScenePosition$ --- pozycja na obrazie, która naz interesuje
    \item $PatientPosition$ --- któryś z punktów względem pacjenta.
\end{itemize}

Wygląd macierzy $imgMatrix$:
\[
    \begin{bmatrix}
        X_x & Y_x & 0 & 0 \\
        X_y & Y_y & 0 & 0 \\
        X_z & Y_z & 0 & 0 \\
        0   & 0   & 0 & 1
    \end{bmatrix}
\]
Powyższa macierz różni się od macierzy definiowanej w standardzie.
Po pierwsze PikselSpacing został pominięty, a konkretniej nadałem mu wartość 1.
Po drugie pozycja z \dicomtag{ImagePosition}{0020}{0032} została zrównana do punktu zerowego, dzięki temu, wynik też będzie względem punktu zero.
Wyznaczenie macierzy $imgMatrix$ jest jednorazowe.

Po wyznaczeniu sześciu punktów $ScenePosition$, po jednej dla każdego punktu względem pacjenta są zapisywane. $ScenePosition$ odpowiada pozycji punktów na obrazie w pozycji startowej.

Na scenie, której jest wyświetlany obraz, użytkownik, może obracać obraz o dowolny kąt, według własnego uznania.
Te przekształcenia, są realizowane za pomocą macierzy rotacji, dalej znana jako $rotateTransform$.
Macierz $rotateTransform$ jest przesyłana do naszego obiektu \sokarclass{ImageOrientationIndicator} za każdym razem kiedy zostanie zmieniona.

Ostateczne wyznaczenie pozycji punktów pacjent na obrazie odbywa sie przez przemnożenie lewostronne $rotateTransform$ i $ScenePosition$.
\[rotateTransform * ScenePosition\]
Wyznaczane jest w ten sposób pozycja sześciu punktów pacjenta na płaszczyźnie sceny wyświetlanej.
Następnie określane jest na, której z ośmiu części płaszczyzny jest umieszczony dany punkt, podział płaszczyzny jest widoczny na rysunku \ref{fig:imageorientationindicator4}.
Tej płaszczyźnie nadawany jest tytuł w postaci litery, która oznacza stronę pacjenta.
Jeżeli punkt znajduje się w centrum, na przecięciu osi, to oznacza, że punkt znajduje się za lub przed ekranem, więc jest pomijany.
Następnie do czterech pól wyświetlających zostają wstawione następujące teksty:
\begin{itemize}
    \item lewe pole: tytuł części 7, tytuł części 0 i tytuł części 1
    \item górne pole: tytuł części 1, tytuł części 2 i tytuł części 3
    \item prawe pole: tytuł części 3, tytuł części 4 i tytuł części 5
    \item dolne pole: tytuł części 7, tytuł części 6 i tytuł części 5
\end{itemize}

Przykład:\\
Punkt \quotett{H}, czyli punkt reprezentujący kierunek głowy, został przypisany do części 1 i odpowiednio \quotett{L} do części 7, \quotett{R} do części 3 i \quotett{F} do części 5.
Punkty \quotett{A} i \quotett{P} zostały pominięte ponieważ znalazły się na środku.
Do lewego pola wstawiany jest tekst \quotett{HL}, do górnego \quotett{HR}, do prawego \quotett{RF} i do dolnego \quotett{LF}.

\begin{figure}[!htbp]
    \centering
    \includegraphics[width=\textwidth]{img/imageorientationindicator-004.png}
    \caption{Podział płaszczyzny sceny. Wyróżniono osiem części. Zdjęcie własne.} 
    \label{fig:imageorientationindicator4}
\end{figure}

Przykład można zobaczyć na rysunku \ref{fig:imageorientationindicator1}.
Na obrazie widzimy, że lewa strona pacjenta znajduje się po prawej stronie obrazu, prawa strona pacjenta po lewej, góra pacjenta na górnej części obrazu.
Wynika z tego, że obraz przedstawia pacjenta skierowanego twarzą do nas.

\begin{figure}[!htbp]
    \centering
    \includegraphics[width=100mm]{img/imageorientationindicator-002.png}
    \caption{Przykładowy obraz medyczny (przekrój głowy MR) z oznaczeniem orientacji obrazu z apomocą liter A, P, R, L, F, H. Zdjęcie własne.}
    \label{fig:imageorientationindicator1}
\end{figure}

\subsubsection{\sokarclass{PixelSpacingIndicator}}

Obiekt wyświetlający podziałkę informującą jakich rozmiarów jest obiekt na obrazie w rzeczywistości, pojawia się na dole i po prawie stronie sceny, gdy tag \dicomtag{PixelSpacing}{0028}{0030} jest obecny.
Wygląd podziałki można zaobserwować na rysunku \ref{fig:imageorientationindicator1}.

Podziałka dostosowuje swoją wielkość do obecnej sceny, jak i do innych elementów na scenie.
Wartości wyświetlane biorą pod uwagę transformatę skali i rotacji obrazu.