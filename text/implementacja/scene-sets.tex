\label{sec:sokar-scenesets}

Abstrakcyjna klasa \sokarclass{DicomSceneSet} implementuje kolekcje scen za pomocą wektora \qtclass{QVector}.
Jest to obiekt, który grupuje w jakiś sposób sceny a następnie tworzy obiekt \sokarclass{SceneSequence}, który jest rzeczywistą sekwencją scen, ułożoną w taki sposób, jaki obrazy powinny być wyświetlane.
Są dwie implementacje zbioru scen: zbiór plików i zbiór ramek z jednego pliku

\subsection{Sekwencja scen}
\label{sec:sokar-scenesequence}

\par
Sekwencja scen implementuje strukturę danych informującą o przejściach pomiędzy scenami poprzez klasę \sokarclass{SceneSequence}.
Sekwencja to wektor zawierającą kroki z dodatkowymi informacjami o stanie sekwencji.
Indeksem w którym obecnie znajduje się sekwencja.
Kierunkiem sekwencji, sekwencja może iść w stronę początku lub kocań.
Rodzajem przemiatania, jest to wartość logiczna informująca w jaki sposób ma zachować się gdy sekwencja dojdzie do końca, lub początku.
Po dojściu do końca sekwencja skoczy do pierwszego elementu lub może zmienić kierunek i zacząć iść do tyłu.

\par
Kroki, implementowane, przez klasą \sokarclass{Step}, zawierają następujące informacje: wskaźnik do sceny oraz czas trwania sceny.

\par
Sekwencja ma wbudowane funkcje zapewniające przesuwanie się po indeksie na wektorze:
\begin{itemize}
    \item \sokarfunction{SceneSequence}{stepForward} --- krok do przodu, zwiększa indeks tym samym wykonuje krok w stronę końca sekwencji
    \item \sokarfunction{SceneSequence}{stepBackward} --- krok do tyłu, zmniejsza indeks tym samym wykonując krok w stronę początku sekwencji
    \item \sokarfunction{SceneSequence}{step} --- wykonuje krok w tył lub przód w zależności od kierunku sekwencji
\end{itemize}
Wszystkie powyższe funkcje są zarazem slotami dla sygnałów oraz emitują sygnał \sokarfunction{SceneSequence}{steped}.

\subsection{Zbiór ramek DICOM}
\label{sec:sokar-dicomframeset}

\par
Zbiory ramek są implementowane przez \sokarclass{DicomFrameSet} i są tworzone z jednego wczytanego pliku DICOM.
Klasa tworzy obiekt konwertera i pobiera liczbę ramek w obrazie.
Tworzy jeden buffor na wszystkie ramki obrazów, a następnie dzieli go na ilość ramek.
Biblioteka GDCM nie daje dostępu do oryginalnego bufora, dlatego wymagany jest bufor pośredni.
Następnie jest tworzonych tyle obiektów scen ile jest ramek.
\par
Kolejność sekwencji scen jest taka sama jak kolejność ramek.
Natomiast czas wyświetlania ramki może być zapisany w różnych znacznikach.
To w którym tagu został zapisany informuje element o znaczniku \dicomtag{FrameIncrementPointer}{0028}{0009}, zawiera on wskaźnik do elementu o zadanym znaczniku i w zależności od znacznika.
Została zaimplementowana obsługa poniższy znaczników:
\begin{itemize}
    \item \dicomtag{FrameTime}{0018}{1063} --- element z tym znacznikiem zawiera czas trwania jednej ramki w milisekundach, każdemu krokowi jest przypisywana ta wartość trwania

    \item \dicomtag{FrameTimeVector}{0018}{1065} --- zawiera tablice z przyrostami czasu w milisekundach między n-tą ramką a poprzednią klatką. Pierwsza ramka ma zawsze przyrost czasu równy 0.
    
    \item \dicomtag{CineRate}{0018}{0040} --- zawiera ilość klatek wyświetlanych na sekunda, każdemu krokowi jest przypisywana wartość odwrotna do tej
\end{itemize}
W przypadku braku znacznika lub gdy zostaje wskazany znacznik nieznany, czas trwania ramki wynosi $83.3$ milisekundy, co odpowiada 12 klatkom na sekundę.


\subsection{Zbiór plików DICOM}
\label{sec:sokar-dicomfileset}
\par
Zbiory plików są implementowane prze \sokarclass{DicomFileSet} i służą do przechowywania wielu wczytanych plików DICOM.
Na początku pliki są sortowane na podstawie liczby zawartej w elemencie o znaczniku \dicomtag{TagInstanceNumber}{0020}{0x0013}.
Dla każdego pliku jest tworzony obiekt \sokarclass{DicomFrameSet}.
\par
Sekwencja jest tworzona na połączenie sekwencji poszczególnych obrazów.

\subsubsection{Segregowanie obrazów}
\label{sec:sokar-dicomfileset-create}
\par
W przypadku kiedy mamy do czynienia z wieloma plikami, należy jest rozdzielić na serie i uporządkować w odpowiedniej kolejności.
Unikalny identyfikator serii jest zawarty w elemencie danych znaczniku \dicomtag{TagSeriesInstanceUID}{0020}{000E}.
Kolejności obrazów w serii to liczba zawarta w elemencie danych o znaczniku \dicomtag{TagInstanceNumber}{0020}{0x0013}.
\par
Segregacja odbywa się za pomocą funkcji \sokarfunction{DicomFileSet}{create}.
Do funkcji jest przesyłany wektor z wczytanymi plikami DICOM, następnie dzieli ona pliki na zbiory zawierające zdjęcia tej samej serii, tworzy obiekty zbiorów plików DICOM, ostatecznie zwraca ona wektor z gotowymi obiektami zbiorów plików DICOM.
Sortowanie plików DICOM według ich kolejności odbywa się za pomocą funkcji \stdclass{sort} wewnątrz konstruktora klasy \sokarclass{DicomFileSet}, który nie jest publiczny.