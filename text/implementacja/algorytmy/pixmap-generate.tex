\label{sec:algorithm-pixmap-generate}

Klasa \sokarclass{DicomScene} dostarcza następujące obiekty do generowania obrazu:
\begin{itemize}
    \item \cppcode{processing}, obiekt klasy \qtclass{QMutex}, muteks do zablokowania podczas generowania obrazu, aby parametry obrazu nie mogły być zmienianie podczas jego generowania.

    \item \cppcode{imgDimX} zmienna typu \cppcode{uint}, oznacza szerokość obrazu w pikselach.

    \item \cppcode{imgDimY} zmienna typu \cppcode{uint}, oznacza wysokość obrazu w pikselach.

    \item \cppcode{targetBuffer} wektor docelowego obrazu RGB o długości $imgDimX*imgDimY$, typu \cppcode{std::vector<Pixel>}.

          \sokarclass{Pixel} to struktura reprezentujące piksel.
          Nie jest to w żadnym wypadku obiekt, a jedynie twór ułatwiający zarządzanie kodem.

          \begin{lstlisting}
struct Pixel {
    quint8 red = 0;   
    quint8 green = 0;    
    quint8 blue = 0;   
}\end{lstlisting}

    \item \cppcode{originBuffer} wektor danych wypełniona danymi z jednej ramki o długośći iloczynu $imgDimX*imgDimY$ i ilości bajtów jednego piksela obrazu.

    \item \cppcode{qImage} obiekt obrazu klasy \qtclass{QImage}.

          \qtclass{QImage} można zrobić z istniejącego bufora, w tym przypadku jest to \cppcode{targetBuffer}.
          Format obrazu to \qtclass{QImage::Format\_RGB888}, czyli trzy bajty, każdy na jeden kanał.
          Proszę zwrócić uwagę, że struktura \sokarclass{Pixel} odpowiada temu formatowi.
          Według dokumentacji Qt obiekt ten po utworzeniu z istniejącego bufora powinien z niego dalej korzystać, dlatego zmiany \cppcode{targetBuffer} nie wymagają odświeżania \cppcode{qImage}.

    \item \cppcode{pixmap} obiekt obrazu do wyświetlania, klasy \qtclass{QPixmap}.

          Obiektów klasy \qtclass{QImage} nie da się wyświetlić, nie jest on przystosowany do wyświetlania.
          Natomiast klasa \qtclass{QPixmap} to reprezentacja obrazu dostosowana do wyświetlania ekranie, która może być używana jako urządzenie do malowania w bibliotece Qt.

    \item \cppcode{iconPixmap} obiekt obrazu ikonu, klasy \qtclass{QPixmap}, docelowo powinien mieć 128 pikseli na 128 pikseli.

\end{itemize}

Generowanie obrazu jest robione przez czysto wirtualną funkcje \sokarfunction{DicomScene}{generatePixmap}.
Po wywołaniu funkcji obiekt \cppcode{targetBuffer} powinien zawierać obraz wygenerowany z obecnymi parametrami.
Funkcja zwraca również wartość logiczną, który informuje nas czy \cppcode{targetBuffer} rzeczywiście został zmieniony.
Następnie obiekt \cppcode{pixmap} jest na nowo generowany na bazie \cppcode{qImage}.

Całe odświeżanie obrazu jest implementowane w funkcji \sokarfunction{DicomScene}{reloadPixmap}.
Funkcja wywołuje \sokarfunction{DicomScene}{generatePixmap} i odświeża \cppcode{pixmapItem} kiedy zajdzie taka potrzeba

Generowanie poszczególnych typów obrazów jest wyjaśnione poniżej.


\subsubsection{Obraz monochromatyczny}
\par
Obraz monochromatyczny to obraz w odcieniach szarości, od białego do czarnego lub od czarnego do białego.
Generowanie takiego obrazu odbyta się poprzez pseudokolorowanie.
Cały proces jest wyjaśniony w sekcji \ref{sec:algorithm-pixmap-monochrome}.

\subsubsection{RGB}
Obrazów zapisanych w RGB nie trzeba w żaden sposób obrabiać, dane już są prawie gotowe do wyświetlenia, należy je tylko odpowiednio posortować, tak jak wymaga biblioteka QT.
Sposób posortowania wartości w pilku określa \dicomtag{PlanarConfiguration}{0x0028}{0006}. Może o przyjąć dwie następujące wartośći:

\begin{itemize}
    \item 0 --- oznacza to, że wartości pikseli są ułożone w taki sposób
        \[R1, G1, B1, R2, G2, B2, R3, G3, B3, R4, G4, B4,  ...\]
    \item 1 --- oznacza to, że wartości pikseli są ułożone w taki sposób
        \[R1, R2, R3, R4, ... , G1, G2, G3, G4, ..., B1, B2, B3, B4, ...\]
\end{itemize}
gdzie:
\begin{itemize}
    \item Rn --- wartość czerwonego kanału
    \item Gn --- wartość zielonego kanału
    \item Bn --- wartość niebieskiego kanału
\end{itemize}

Wartości obrazu są przepisywane do bufora dla biblioteki QT.


\subsubsection{YBR}
\label{sec:algorithm-pixmap-ybr}

\par
YBR albo YCbCr to model przestrzeni kolorów do przechowywania obrazów i wideo.
Wykorzystuje do tego trzy typy danych: Y – składową luminancji, B lub Cb – składową różnicową chrominancji Y-B, stanowiącą różnicę między luminancją a niebieskim, oraz R lub Cr – składową chrominancji Y-R, stanowiącą różnicę między luminancją a czerwonym.
Kolor zielony jest uzyskiwany na podstawie tych trzech wartości.
YBR nie pokrywa w całości RGB, tak jak RGB nie pokrywa YBR.
Posiadają one część wspólną, co uniemożliwia wyświetlenie obrazu w stu procentach bez zniekształceń.

\par
Wartości w pliku DICOM są ułożone w taki sposób.
\[Y1, B1, R1, Y2, B2, R2, Y3, B3, R3, Y4, B4, R4,  ...\]

\par
Ponieważ wartości te reprezentują kolory, są już w pewnym sensie są obrazem, ale nie można go wyświetlić na monitorze RGB.
Dlatego należy przekonwertować kolor YBR na kolor RGB, iterując po wszystkich wartościach obrazu.

\par
Poniżej przedstawiono kod źródłowy funkcji zamiany kolory YBR na RGB.

\begin{lstlisting}
Sokar::Pixel ybr2Pixel(quint8 y, quint8 b, quint8 r) {
    qreal red, green, blue;

    red = green = blue = (255.0 / 219.0) * (y - 16.0);

    red += 255.0 / 224 * 1.402 * (r - 128);
    green -= 255.0 / 224 * 1.772 * (b - 128) * (0.114 / 0.587);
    green -= 255.0 / 224 * 1.402 * (r - 128) * (0.299 / 0.587);
    blue += 255.0 / 224 * 1.772 * (b - 128);

    /* W tym miejscu jest dokonywana utrata danych */
    red = qBound(0.0, red, 255.0);
    green = qBound(0.0, green, 255.0);
    blue = qBound(0.0, blue, 255.0);

    return Sokar::Pixel(quint8(red), quint8(green), quint8(blue));
}
\end{lstlisting}
