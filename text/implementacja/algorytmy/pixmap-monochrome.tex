\label{sec:algorithm-pixmap-monochrome}

Obraz monochromatyczny to obraz w odcieniach szarości, od białego do czarnego lub od czarnego do białego. Dane są zapisane w sposób ciągły wartość po wartości.

\subsubsection{Pseudokolorowanie obrazu}

Mamy obraz, którego piksele to n-bitowe liczby, na przykład 16 bitowa liczba całkowita.
W takiej postaci wyświetlemoe obrazu na monitorze RGB lub nawet na profesjonalnym 10-bitowym jest niemożliwe.
Należy taką liczbę przerobić na trzy liczby, reprezentujące 3 kanały RGB, czerwony, zielony i niebieski.
Dlatego do wyświetlania obrazów monochromatycznych o dużym kontraście stosuję się twór zwany okienkiem.
Jest to funkcja, która mapuje n-bitwy obraz na 8-bitowy obraz w skali szarości.
8-bitów, ponieważ monitor RGB jest wstanie wyświetlić 256 odcieni szarości.

\paragraph*{Zwiększanie kontrastu za pomocą \enquote{funckji okna}}
Przyjeło się, że \enquote{okno} definiuje się dwoma liczbami: środkiem, oznaczanym jako $center$ i długością, oznaczaną jako $width$.
Wyznaczamy zakres okienka $x_0$ i $x_1$ ze środka okienka $center$ i długości $width$.
\[x_0 = center - width / 2\]
\[x_1 = center + width / 2\]
Wyznaczamy parametry $a$ i $b$, prostej przechodzącej przez dwa punkty $(x_0, y_0)$ i $(x_1, y,_1)$.
Gdzie $y_0$ jest równe 0, a $y_1$ jest równe 255.
Funkcja \enquote{okna} wygląda następująco:
\[
    f(v)=
    \begin{cases}
        0     & \text{gdy $0 \le v \wedge v \le x_0$ } \\
        a*x+b & \text{gdy $x_0 < v \wedge v < x_1$}    \\
        255   & \text{gdy $x_1 \le v \wedge v \le 1$ }
    \end{cases}
\]

gdzie $v$ to wartość piksela danych obrazu.

Następnie iterujemy przez wszystkie woksele obrazu i używamy na nich funkcji \enquote{okna} i otrzymujemy obraz w skali od $0$ do $255$.
Taki obraz w skali można już wyświetlić.
Natomiast standard DICOM przewiduje, że obraz można jeszcze wyświetlić w wielokolorowej palecie barw.
Przykład takiej palety HotIron w porównaniu do skali szarości można zobaczyć na rysunku .
Taka paleta barw nie koniecznie musi mieć 256 odcieni, dlatego lepiej jest zrobić aby okienko, mapowało na liczbę od 0 do 1, a później paleta mapowała na kolor RGB.

\begin{figure}[!htbp]
    \centering
    \includegraphics[width=0.7\textwidth]{img/monochrome-001.png}
    \caption{Paleta HotIron w porównaniu do palety w skali szarości. Zdjęcie ze standardu DICOM dostępne pod adresem \url{http://dicom.nema.org/medical/dicom/2019a/output/chtml/part06/chapter_B.html}.}
    \label{fig:monochrome1}
\end{figure}

Teraz iterujemy po wszystkich możliwych wartościach wartośćiach obrazu i wykonujemy takie operacje.
\begin{itemize}
    \item wyznaczenie wartości okienka.
          \[y = a * x + b\]
    \item y zostaje obcięcie do 1.0 lub 0.0 jeżeli wyjdzie poza zakres od 1.0 do 0.0
    \item pobranie z palety piksel odpowiadający wartości
    \item wsadzenie piksela do tablicy, tak aby najmniejsza wartości obrazu miała indeks 0 a największy ostani
\end{itemize}


\subsubsection{Implementacja algorytmu}

\paragraph{Opis}
\par
Implementacja powyżej przedstawionego algorytmu w sposób dosłowny byłaby mało optymalna dla maszyny i wymagała by wielu pobocznych tablic oraz względnie dużej ilości mnożenia.
Trzeba też zauważyć, że do wyliczenie jakiegoś piksela nie potrzeba liczyć, żadnego innego piksela, co skutkuje, że każdy piksel można wyliczyć oddzielnie.
Dlatego najlepiej było by współbieżnie przelecieć po całym obrazie i zamienić dane na piksele.
Ale do zamiany dane na piksel, musimy mnożyć i dzielić liczby zmiennoprzecinkowe, a to do najszybszych nie należy.
Dlatego dobrym pomysłem jest zrobienie mniejszej tablicy typu LookUpTable, wypełnienie jej wszystkimi możliwymi wartościami obrazu, a następnie przerobić obraz z tablicą LUT.
Ale ponieważ tablica LUT posiada wszystkie możliwe kombinacje wartości, jej rozmiar można wyznaczyć wzorem: $2^N*3$, gdzie N to liczba bitów liczby.
Standard DICOM definiuje, że liczby mogą mieć $8$, $12$, $16$, $32$ i $64$ bity, jednakże, $12$ bitowe i tak się zapisuje w postaci 16-bitowych w pamięci RAM.
Dlatego możliwe wartości wielkości tablicy LUT to w przybliżeniu: $768$ bajtów, $196$ kilobajtów, $12,5$ gigabajtów i $56$ eksabajta($55*10^{6}$ terabajtów).
Alokowanie dwóch największych wartości może być lekko problematyczne, dlatego zrobiłem dwie implementacje algorytmu: z tablicą LUT(dla 8 i 16 bitowych obrazów i bez tablicy LUT(dla 32 i 64 bitowych obrazów).
Algorytm składa się z 3 części: wyznaczenie parametrów okna, przygotowanie okna (tylko gdy jest tablica LUT), wielowątkowa iteracja po obrazie.
\par
Okno z LUT jest implementowane przez \sokarclass{Monochrome}{WindowIntDynamic}.
Okno bez LUT jest implementowane przez \sokarclass{Monochrome}{WindowIntDynamic}.
Obie klasy dziedziczą po abstrakcyjnej klasie \sokarclass{Monochrome}{Window}, która z kolei dziedziczy po \sokarclass{SceneIndicator}, dlatego od razu może wyświetlać obecne wartości okna.
Decyzja o używanym oknie jest podejmowana podczas wczytywania obrazu przez klasę \sokarclass{Monochrome{\scopedots}Scene}
\par
UWAGA: Standard DICOM zakłada, że danymi mogą być liczby całkowite(\cppcode{int}) oraz zmiennoprzecinkowe(\cppcode{float} lub \cppcode{double}), ale praktycznie, nie ma takich aparatów medycznych, które zapisywały by takie obrazy, gdzie dane to liczby zmiennoprzecinkowe. Dlatego założyłem, że takie obrazy nie istnieją.

\paragraph{Wyznaczenie parametrów okna}
\par
Najpierw wyznaczam okienko, które zmienia wartości obrazu na skale od zera do jeden:
\[x_0 = center - width / 2\]
\[x_1 = center + width / 2\]
\[y_1 = 0.0\]
\[y_0 = 1.0\]
gdzie:
\begin{itemize}
    \item $center$ --- środek okienka
    \item $width$ --- szerokość okienka
    \item $x0$ i $y0$ --- współrzędne pierwszego punktu
    \item $x1$ i $y1$ --- współrzędne drugego punktu
\end{itemize}
Przeglądarka pozwala na inwersje okienka.
Dlatego kiedy użytkownik zażyczy sobie inwersji, zmienne $y0$ i $y1$ zamienią się wartoścami.

Standard DICOM przewiduje, że wszystkie dane powinny być wyskalowane, za pomocą wzoru.
\[OutputUnits = m*SV + b\]
gdzie:
\begin{itemize}
    \item $m$ --- wartość z \dicomtag{RescaleSlope}{0028}{1053}
    \item $b$ --- wartość z \dicomtag{RescaleIntercept}{0028}{1052}
    \item $SV$ --- stored values - warość pixela z pliku
    \item $OutputUnits$ --- wartość wynikowa
\end{itemize}

Wartości okienka odnoszą się do wartości już wyskalowanej, a ponieważ skalowanie całego obrazu jest czasochłonne, przeskalowaie okienka da taki sam efekt:
\[(OutputUnits - b ) / m = SV \]
więc:
\[x_0 -= rescaleIntercept\]
\[x_1 -= rescaleIntercept\]
\[x_0 /= rescaleSlope\]
\[x_1 /= rescaleSlope\]

Posiadamy, teraz dwa punkty okienka odnoszące się do wartośći obrazu.
Wyznaczam parametry prostej przechodzącej przez dwa punkty:
\[a = (y_1 - y_0) / (x_1 - x_0)\]
\[b = y_1 - a * x_1\]

\par
Teraz algorytm się rozdwaja.
Pobieranie wartości z okienka odbywa się za pomocą funkcji \sokarclass{Monochrome}{Window\zerospace::{\zerospace}getPixel()}.

\paragraph{Implementacja dynamiczna bez tablicy LUT}

\par
W tej wersji funkcja \sokarfunction{Monochrome{\scopedots}Window}{getPixel} wygląda następująco:
\par
\begin{lstlisting}
inline const Pixel &getPixel(quint64 value) override {
    if (value < x0) {
        return background;
    } else if (value > x1) {
        return foreground;
    } else {
        return palette->getPixel(a * value + b);
    }
}
\end{lstlisting}
\par
Widzimy tutaj, że funkcja najpierw sprawdza czy zakres okienka został przekroczony, następnie wylicza wartość obrazu i pobiera kolor z palety.
\par
UWAGA: ponieważ nie dysponuje rzeczywistym obrazem o pikselu danych 32-bitowym lub 64-bitowych, implementacja dynamiczna nie była testowana w warunkach rzeczywistych.

\paragraph{Implementacja statyczna z tablicą LUT}
\par
W wersji z LUT, podczas tworzenia okienka jest alokowany wektor obiektów \sokarclass{Pixel} klasy \stdclass{vector}{container/vector}.
Standard DICOM przewiduje, że woksele mogą mieć wartości ujemne, więc tablica powinna mieć możliwość posiadania takich wartości indeksów, ale C++ nie przewiduje takiej możliwości.
Dlatego wprowadzono dwie zmienne pomocnicze \cppcode{maxValue} i \cppcode{signedMove}.
\cppcode{maxValue} jest to maksymalna wartość jaką dane mogą przyjąć, jest ona równa $2^N$, gdzie $N$ to liczba bitów brana z \dicomtag{BitsStored}{0028}{0101}.
A \cppcode{signedMove} to liczba przesunięcia liczb, przyjmuje wartość zero gdy dane wokseli są całkowite nieujemne lub wartość przeciwną do \cppcode{maxValue} gdy woksele mogą być ujemne.
Długość wektora pikseli jest sumą \cppcode{maxValue} i \cppcode{signedMove}.
A indeks woksela w wektorze ma wartość tego woksela zwiększoną o \cppcode{signedMove}.
\par
Wypełnienie wektora wartościami odbywa się poprzez iteracje po wszystkich możliwych wartościach, przeliczenie ich przez funkcje okna, a następnie wstawienie ich do wektora.
W celu poprawy szybkości, zastosowano sprawdzanie czy wartości są w zakresie okna.
Poniżej kod funkcji:
\begin{lstlisting}
bool genLUT() override {

    if (WindowInt::genLUT()) {

        /* Przeskalowanie wektora, gdy jest to wymagane */
        if (arraySize != signedMove + maxValue) {
            arraySize = signedMove + maxValue;
            arrayVector.resize(arraySize);
        }

        /* Wyliczenie najmniejszej wartości */
        qreal x = qreal(signedMove) * -1;

        auto &background = isInversed() ? palette->getForeground() : palette->getBackground();
        auto &foreground = isInversed() ? palette->getBackground() : palette->getForeground();

        /* Iteracja */
        pixelArray = &arrayVector[0];
        for (int i = 0; i <= arraySize; i++) {

            if (x < x0) {
                *pixelArray = background;
            } else if (x > x1) {
                *pixelArray = foreground;
            } else {
                *pixelArray = palette->getPixel(a * x + b);
            }

            x++;
            pixelArray++;
        }

        pixelArray = &arrayVector[0];

        updateLastChange();

        return true;
    }
    return false;
}
\end{lstlisting}

\par
Funkcja pobierania piksela z \enquote{okna} prezentuje się następująco:
\begin{lstlisting}
inline const Pixel &getPixel(quint64 value) override {
    return *(pixelArray + signedMove + value);
}
\end{lstlisting}

\paragraph{Iterowanie po obrazie}
\par
Po przygotowaniu okienka, należy przeiterować obraz przez funkcje \enquote{okna}.
Do zokienkowania jednego piksela nie potrzeba innego piksela dlatego w celu przyspieszenia procesu okienkowania, iteracja po obrazie odbywa się w wielu wątkach.
\par
W C++ typy zmiennych muszą być zdefiniowane przed kompilacją, co jest pewnym problemem.
Mając dwa typy okienek, każde odsługujące 4 typy liczb całkowitych, musiało by zostać zaimplementowanych 8 prawie identycznych funkcji.
Dlatego podział ten został zaimplementowany za pomocą 4 funkcji z szablonami:
\begin{itemize}
    \item \sokarfunction{Monochrome{\scopedots}Scene}{genQPixmapOfTypeWidthWindowThread}

          Jest funkcją jednego wątku, który iteruje po obrazie.
          Jego parametrami są zakresy podane w indeksach wokseli po któych ma iterować.
          \cppcode{IntType} to jest typ zmiennej woksela obrazu.
          \cppcode{WinClass} klasa okienka.
          Nazewnictwo będzie kontynuowane w następnych punktach.
          Kod funkcji:
          \begin{lstlisting}
template<typename IntType, typename WinClass>
void Monochrome::Scene::genQPixmapOfTypeWidthWindowThread(quint64 from, quint64 to) {

	auto buffer = &targetBuffer[from];
	auto origin = (IntType *) &originBuffer[0];
	auto windowPtr = (WinClass *) getCurrentWindow();

	origin += from;

	for (quint64 i = from; i < to; i++, origin++) {
		*buffer++ = windowPtr->getPixel(*origin);
	}
}
\end{lstlisting}

    \item \sokarfunction{Monochrome{\scopedots}Scene}{genQPixmapOfTypeWidthWindow}

          Jest funkcją, która dzieli obraz na wątki, tworzy je i uruchamia.
          Ilość wątków jest ustalana za pomocą funkcji \qtfunction{QThread}{idealThreadCount}.
          Wątki działają na zakresach o długości ilości wokseli podzielonej prze ilość wątków.
          Kod funkcji:
\begin{lstlisting}
template<typename IntType, typename WinClass>
void Monochrome::Scene::genQPixmapOfTypeWidthWindow() {

    /* Tworzenie wektora wątków */
    std::vector<std::thread> threads;

    quint64 max = imgDimX * imgDimY;
    quint64 step = max / QThread::idealThreadCount();

    for (int i = 1; i < QThread::idealThreadCount(); i++) {
        std::thread t(&Scene::genQPixmapOfTypeWidthWindowThread<IntType, WinClass>,
                      this,
                      i * step,
                      std::min((i + 1) * step, max));

        threads.push_back(std::move(t));
    }

    /* W celu zmniejszenia ilości wątków wątke obecny też zostanie wykorzystany */
    genQPixmapOfTypeWidthWindowThread<IntType, WinClass>(0, step);

    /* Czekanie na wszystkie wątki */
    for (auto &t: threads) t.join();
}
\end{lstlisting}

    \item \sokarfunction{Monochrome{\scopedots}Scene}{genQPixmapOfType}
 
    Jest ot funkcja pomocnicza ustalająca obecne klasę obecnego okna aby móc wykonać funkcje \sokarfunction{Monochrome{\scopedots}Scene}{genQPixmapOfTypeWidthWindow}.
    Kod funkcji:
\begin{lstlisting}
template<typename IntType>
void Monochrome::Scene::genQPixmapOfType() {

    switch (getCurrentWindow()->type()) {
        case Window::IntDynamic:
            genQPixmapOfTypeWidthWindow<IntType, WindowIntDynamic>();
            break;

        case Window::IntStatic:
            genQPixmapOfTypeWidthWindow<IntType, WindowIntStatic>();
            break;

        default:
            throw WrongScopeException(__FILE__, __LINE__);
    }
}
\end{lstlisting}

    \item \sokarfunction{Monochrome{\scopedots}Scene}{generatePixmap}
    
    Funkcja odświeża okienko i sprawdza czy odświeżenie obrazu jest konieczne, następnie sprawdza typ liczby woksela i uruchamia \sokarfunction{Monochrome{\scopedots}Scene}{genQPixmapOfType}.
    Kod funkcji:
\begin{lstlisting}
bool Monochrome::Scene::generatePixmap() {

    /* Odświeżamy okno i sprawdzamy czy odświeżenie obrazu jest konieczne */
    getCurrentWindow()->genLUT();
    if (lastPixmapChange >= getCurrentWindow()->getLastChange()) return false;

    /* Sprawdzamy typ liczby woksela obraau */
    switch (gdcmImage.GetPixelFormat()) {
        case gdcm::PixelFormat::INT8:
            genQPixmapOfType<qint8>();
            break;
        case gdcm::PixelFormat::UINT8:
            genQPixmapOfType<quint8>();
            break;
        case gdcm::PixelFormat::INT16:
            genQPixmapOfType<qint16>();
            break;
        case gdcm::PixelFormat::UINT16:
            genQPixmapOfType<quint16>();
            break;
        case gdcm::PixelFormat::INT32:
            genQPixmapOfType<qint32>();
            break;
        case gdcm::PixelFormat::UINT32:
            genQPixmapOfType<quint32>();
            break;
        case gdcm::PixelFormat::INT64:
            genQPixmapOfType<qint64>();
            break;
        case gdcm::PixelFormat::UINT64:
            genQPixmapOfType<quint64>();
            break;

        default: /* W przypadku innych jest zwracany wyjątek */
            throw Sokar::ImageTypeNotSupportedException();
    }

    pixmap.convertFromImage(qImage);
    return true;
}
\end{lstlisting}
\end{itemize}

\subsubsection{Palety}
Klasa \sokarclass{Palette} reprezentuje palety kolorów używanych do pseudokolorwania brazu monochromatycznego.
Paleta przerabia liczbę zmiennoprzecinkową od zera do jedynki na kolor RGB, zwracając \sokarclass{Pixel}.
Paleta nie ma zdefiniowanej długości minimalne i maksymalnej.

\par
Palety są wczytywane z plików XML w czasie uruchamiania programu i można definiować własne palety nie będące częścią standardem.
Przykładowy wygląd pliku palety HotIron:
\begin{lstlisting}[language=XML]
<palette display="Hot Iron" name="HOT_IRON">

    <entry red="0" green="0" blue="0"/>
    <entry red="2" green="0" blue="0"/>
    <entry red="4" green="0" blue="0"/>

    ...

    <entry red="254" green="0" blue="0"/>
    <entry red="255" green="0" blue="0"/>
    <entry red="255" green="2" blue="0"/>

    ...

    <entry red="255" green="250" blue="248"/>
    <entry red="255" green="252" blue="252"/>
    <entry red="255" green="255" blue="255"/>
</palette>
\end{lstlisting}