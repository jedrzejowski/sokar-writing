\label{sec:algorithm-pixmap-ybr}

\par
YBR albo YCbCr to model przestrzeni kolorów do przechowywania obrazów i wideo.
Wykorzystuje do tego trzy typy danych: Y – składową luminancji, B lub Cb – składową różnicową chrominancji Y-B, stanowiącą różnicę między luminancją a niebieskim, oraz R lub Cr – składową chrominancji Y-R, stanowiącą różnicę między luminancją a czerwonym.
Kolor zielony jest uzyskiwany na podstawie tych trzech wartości.
YBR nie pokrywa w całości RGB, tak jak RGB nie pokrywa YBR.
Posiadają one część wspólną, co uniemożliwia wyświetlenie obrazu w stu procentach bez zniekształceń.

\par
Wartości w pliku DICOM są ułożone w taki sposób.
\[Y1, B1, R1, Y2, B2, R2, Y3, B3, R3, Y4, B4, R4,  ...\]

\par
Ponieważ wartości te reprezentują kolory, są już w pewnym sensie są obrazem, ale nie można go wyświetlić na monitorze RGB.
Dlatego należy przekonwertować kolor YBR na kolor RGB, iterując po wszystkich wartościach obrazu.

\par
Poniżej przedstawiono kod źródłowy funkcji zamiany kolory YBR na RGB.

\begin{lstlisting}
Sokar::Pixel ybr2Pixel(quint8 y, quint8 b, quint8 r) {
    qreal red, green, blue;

    red = green = blue = (255.0 / 219.0) * (y - 16.0);

    red += 255.0 / 224 * 1.402 * (r - 128);
    green -= 255.0 / 224 * 1.772 * (b - 128) * (0.114 / 0.587);
    green -= 255.0 / 224 * 1.402 * (r - 128) * (0.299 / 0.587);
    blue += 255.0 / 224 * 1.772 * (b - 128);

    /* W tym miejscu jest dokonywana utrata danych */
    red = qBound(0.0, red, 255.0);
    green = qBound(0.0, green, 255.0);
    blue = qBound(0.0, blue, 255.0);

    return Sokar::Pixel(quint8(red), quint8(green), quint8(blue));
}
\end{lstlisting}