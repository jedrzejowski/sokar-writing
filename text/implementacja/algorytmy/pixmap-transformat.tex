\label{sec:algorithm-pixmap-transformat}

\subsubsection{Współrzędne jednorodne}

Współrzędne jednorodne definiuje się jako sposób reprezentacji punktów n-wymiarowej przestrzeni rzutowej za pomocą układu $n+1$ współrzędnych.
W bibliotece Qt jedną z implementacji współrzędnych jednorodnych jest klasa \qtclass{QTransform}.
Implementuje ona podstawowe zachowania macierzy 3 na 3, jak również wbudowane operacje takie jak: przesuwanie implementowane prze \qtfunction{QTransform}{translate}, obrót implementowany przez funkcję \qtfunction{QTransform}{rotate} i skalowanie implementowane przez \qtfunction{QTransform}{scale}.

Przykład działania:
\begin{lstlisting}
QTransform transform;
transform.translate(50, 50);
transform.rotate(45);
transform.scale(0.5, 1.0);
\end{lstlisting}
Powyższe przekształcenie macierzowe skaluje obiekt na 50\% szerokości, obraca o 45 stopni, przesuwa o 50 punktów na osi $x$ i $y$.

\par
Taką macierz można nałożyć na obiekty klasy \qtclass{QGraphicsPixmapItem}.

\subsubsection{Interakcja z użytkownikiem}

Trzy macierze (bez wyśrodkowującej) są zmieniane w trakcie interakcji z użytkownikiem.
Są zmieniane w dwóch przypadkach: po odebraniu sygnału od paska zadań, obiektu klasy \sokarclass{DicomToolbar} lub podczas ruchu myszki, gdy wciśnięty jest prawy przycisk myszy.

\paragraph{Zmiany poprzez oderanie sygnału}

\par
Na pasku zadań, nad sceną znajduje się szereg przycisków, które po wciśnięciu wysyłają sygnał do obecnej sceny poprzez obiekt klasy \sokarclass{DicomView}.
Sposób wysyłania sygnałów jest szerzej opisany w sekcji \ref{sec:sokar-dicomtoolbar}.

\par
Po otrzymaniu odpowiedniego sygnału jest wykonywana operacja na macierzy.
Odbiór wszystkich sygnałów jest implementowany przez wirtualną funkcję \sokarfunction{DicomScene}{toolBarActionSlot}, która jest slotem.

\par
Lista opisów reakcji na sygnały (stan zerowy macierzy, to stan w którym macierz nie wykonuje żadnych operacji):
\begin{itemize}

    \item \cppcode{ClearPan} --- przywraca macierz przesunięcia do stanu zerowego,

    \item \cppcode{Fit2Screen} --- przywraca macierz skali do stanu zerowego, następnie wylicza nową skalę w zależności od wymiarów obrazu i sceny,

    \item \cppcode{OriginalResolution} --- przywraca macierz skali do stanu zerowego,

    \item \cppcode{RotateRight90} --- na macierzy rotacji zostaje użyta funkcja \qtfunction{QTransform}{rotate} z parametrem $90$,

    \item \cppcode{RotateLeft90} --- na macierzy rotacji zostaje użyta funkcja \qtfunction{QTransform}{rotate} z parametrem $-90$,

    \item \cppcode{FlipHorizontal} --- na macierzy rotacji zostaje użyta funkcja \qtfunction{QTransform}{scale} z parametrami $1$ i $-1$,

    \item \cppcode{FlipVertical} --- na macierzy rotacji zostaje użyta funkcja \qtfunction{QTransform}{scale} z parametrami $-1$ i $1$,

    \item \cppcode{ClearRotate} --- przywraca macierz rotacji do stanu zerowego.

\end{itemize}
Po jakiejkolwiek zmianie macierzy jest wywoływana funkcja \sokarfunction{DicomScene}{updatePixmapTransformation}, która odświeża macierz przekształcenia na obiekcie \cppcode{pixmapItem}.

\paragraph{Zmiany poprzez obsługę myszki}

\par
\qtclass{QGraphicsScene} dostarcza możliwość obsługi myszki poprzez wirtualną funkcję \qtfunction{QGraphicsScene}{mouseMoveEvent}.
Dzięki temu obsługa myszki może być rozszerzana przez wszystkie klasy dziedziczące po tej klasie.
Dodatkowo funkcja ta dostarcza obiekt klasy \qtclass{QGraphicsSceneMouseEvent}, w którym znajdują się informacje zarówno o obecnej jak i poprzedniej pozycji myszki.

\par
Jeżeli jest wykryty ruch myszki z wciśniętym lewym przyciskiem myszy, to w zależności od stanu paska narzędzi, wywoływana jest odpowiednia akcja.
Akcje są obsługiwane przez klasy \sokarclass{DicomScene} i \sokarclass{Monochrome{\scopedots}Scene}.
Każda z nich obsługuje pewną pulę stanów.
Lista obsługiwanych stanów paska narzędzi:
\begin{itemize}
    \item \cppcode{Pan} --- stan przesuwania, obsługiwany przez \sokarclass{DicomScene}

          Na macierzy przesuwania wywoływana jest funkcja przesunięcia \qtfunction{QTransform}{translate} z parametrami odpowiadającymi przesunięciu myszki.

    \item \cppcode{Zoom} --- stan skalowania, obsługiwany przez \sokarclass{DicomScene}

          Na macierzy skalowania wywoływana jest funkcja skalowania \qtfunction{QTransform}{scale} z parametrem \cppcode{scale} wyliczanym podanym wzorem:

          \[scale=1\]
          \[scale=scale-{\Delta}y*0.01\]
          \[scale=scale-{\Delta}x*0.001\]

          Sprawia to, że ruch pionowy jest bardziej czuły na zmianę niż ruch poziomy.
          Teoretycznie jest możliwość implementacji odrębnego skalowania w dwóch osiach, jednakże jest to nieintuicyjne.

    \item \cppcode{Rotate} --- stan rotacji, obsługiwany przez \sokarclass{DicomScene}

          Na macierzy rotacji wywoływana jest funkcja rotacji \qtfunction{QTransform}{rotate} z parametrem \cppcode{rotate} wyliczanym podanym wzorem:

          \[rotate = 0\]
          \[rotate = rotate + {\Delta}y * 0.5;\]
          \[rotate = rotate + {\Delta}x * 0.1;\]

          Sprawia to, że ruch pionowy jest bardziej czuły na zmianę niż ruch poziomy.

    \item \cppcode{Windowing} --- stan okienkowania, obsługiwany przez \sokarclass{Monochrome{\scopedots}Scene}

          Do obiektu okienka są wysyłane zmiany poprzez funkcje: \sokarfunction{Window}{mvVertical} z parametrem ${\Delta}y$ i \sokarfunction{Window}{mvHorizontal} z parametrem ${\Delta}x$.
          Następnie ponownie jest generowany obraz z uwzględnieniem zmiany okienka.

\end{itemize}

\paragraph{Połączenie macierzy}
\par
Ostatnim krokiem jest połączenie macierzy w jedną.
Dlatego cztery macierze są mnożone za pomocą wirtualnej funkcji \sokarfunction{DicomScene}{getPixmapTransformation}.
Kod funkcji:
\begin{lstlisting}
QTransform DicomScene::getPixmapTransformation() {
	QTransform transform;
	transform *= centerTransform;
	transform *= scaleTransform;
	transform *= rotateTransform;
	transform *= panTransform;
	return transform;
}
\end{lstlisting}
\qtclass{QTransform} posiada operator mnożenia, dlatego można mnożyć obiekty tej klasy jak liczby.
Realizuje to następujące równanie:
\[panTransform*rotateTransform*scaleTransform*centerTransform\]
