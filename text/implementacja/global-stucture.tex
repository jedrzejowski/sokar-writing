
W tej sekcji jest wyjaśniona ogólna struktura programu, z pominięciem dokładnych opisów poszczególnych elementów, co znajduje się w następnych sekcjach.

JA TO JESZCZE ROZWINE
\par
Cały program zawiera się w jednym oknie, które zawiera 3 elementy: menu (obiekt klasy \qtclass{QMenu}), drzewa plików (obiekt klasy \sokarclass{FileTree}), obiekt zakładek z obrazami (obiekt klasy \sokarclass{DicomTabs}).
Menu oraz drzewo plików wysyła sygnały do obiektu zakładek, dodatkowo obiekt ten może wysłać sygnał sam do siebie jeżeli dostanie nowe pliki opuszczone przez użytkownika.
\par
\sokarclass{DicomTabs} jest to klasa wyświetlająca obrazu w zakładkach za pośrednictwem \sokarclass{DicomView}.
Obiekt zakładek, gdy dostanie sygnał, aby wczytać nowe pliki to wczytuje te pliki, sprawdza ich poprawność, grupuje je odpowiednio, tworząc odpowiednie obiekty klasy dziedziczącej po \sokarclass{DicomSceneSet}, a następnie tworzy jeden lub wiele obiektów klasy \sokarclass{DicomView}.
\par
\sokarclass{DicomSceneSet} jest abstrakcyjną klasą, która przechowuje obraz lub obrazy w zbiorze odpowiadającej ich realnemu położeniu.
\par
\sokarclass{DicomView} jest obiektem wyświetlającym obraz lub obrazy za pośrednictwem klas dziedziczących po \sokarclass{DicomScene}.
Nie implementuje on ułożenia obrazów względem siebie a jedynie tworzy warunki do wyświetlenia sekwencji.
\par
\sokarclass{DicomScene} jest to klasa generują i wyświetlająca jedną ramkę obrazu.

