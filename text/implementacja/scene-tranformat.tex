\par
Już wygenerowany obraz można wyświetlić bez większego problemu.
Wyświetlanie odbywa \cppcode{pixmap}, obiektu klasy \qtclass{QPixmap}, odbywa się za pomocą obiektu \cppcode{pixmapItem}, obiektu klasy \qtclass{QGraphicsPixmapItem}, który dziedziczy po \qtclass{QGraphicsItem}.
Ta ostatnia klasa ma w sobie zaimplementowaną funkcjonalność pozwalającą na nałożenie transformaty na nią.
Transformata to obiekt klasy \qtclass{QTransform}, który reprezentuje transformatę dwu wymiarowa na obiekt, praktycznie jest to macierz 3 na 3 reprezentująca przekształcenie w współrzędnych jednorodnych.
\par

Zostało zdefiniowanych 4 transformat:
\begin{itemize}
    \item \cppcode{centerTransform} --- transformata wyśrodkowująca, zadanie tego przekształcenia jest przeniesienie obrazu na środek sceny
    \item \cppcode{panTransform} --- transformata przesunięcia
    \item \cppcode{scaleTransform} --- transformata skali
    \item \cppcode{rotateTransform} --- transformata rotacji
\end{itemize}

Te cztery transformaty są łączone za pomocą wirtualnej funkcji \sokarfunction{DicomScene}{getPixmapTransformation}.
Kod funkcji:
\begin{lstlisting}
QTransform DicomScene::getPixmapTransformation() {
	QTransform transform;
	transform *= centerTransform;
	transform *= scaleTransform;
	transform *= rotateTransform;
	transform *= panTransform;
	return transform;
}
\end{lstlisting}
Realizuje on takie równanie:
\[panTransform*rotateTransform*scaleTransform*centerTransform\]


\subsubsection{Współżędne jednorodne}

\subsubsection{Interakcja z użytkownikiem}

Trzy transformaty (bez wyśrodkowującej) są zmieniane w trakcie interakcji z użytkownikiem.
Są zmieniane w dwóch przypadkach: po odebraniu sygnału od paska zadań, obiektu klasy \sokarclass{DicomToolbar} lub podczas ruchu myszki, gdy wciśnięty jest prawy przycisk.