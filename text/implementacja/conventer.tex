
Każdy plik DICOM posiada \quotett{Data Set}, czyli zbiór informacji składający się z \quotett{Data Element}.
\quotett{Data Element} jest strukturą przechowującą wszystkie dane dotyczące obrazu.

\paragraph{Konwerter}

Obiekt klasy \sokarclass{DataConverter} to obiekt zajmujący się konwersją danych z pliku DICOM na dane w formacie odpowiadającej przeglądarce.

Kilka rzeczy które się tyczą wszystkich danych i konwersji:
\begin{itemize}
    \item większość VR jest to zapisane jako tekst, kodowanie i dekodowanie tekstu jest zapewniane przez bibliotekę, a konkretniej przez klasę \gdcmclass{StringFilter}, dlatego nie przejmuje się takimi rzeczami jak zapisem LittleEndian i BigEndian.
    \item większość danych może mieć kilka wartości oddzielonych backslashem \quotett{\textbackslash}, dlatego konwerter dla VR w, których standard przewiduje wiele wartości, zawsze zwraca \qtclass{QVector} z tymi wartościami
    \item wszystkie dane są zapisane parzystą ilością bajtów, w przypadku tekstu dodaje się znak spacji na końcu danych, taka spacja jest pomijana w analizie danych
\end{itemize}
