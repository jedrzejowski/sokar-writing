
\par
Każdy plik \DICOM posiada zbiór elementów danych.
Zapisane elementy danych należy przekonwertować na obiekty danych odpowiedające potrzebą programu.
Dlatego został zaimplementowany obiekt klasy \sokarclass{DataConverter} zajmujący się konwersją danych z pliku \DICOM na dane w formacie odpowiadającym programowi.

\par
Obiekt konwertera jest tworzony na podstawie pliku \DICOM i przy wywoływaniu konwersji należy podać tylko znacznik, który nas interesuje.
Taki rozwiązanie pozwala na przesłanie do wszystkich obiektów, jednego względnie małego obiektu konwertera co ułatwia zarządzanie plikiem \DICOM.

\par
Klasa \sokarclass{DataConverter} posiada następujące funkcje:
\begin{itemize}
    \item \cppcode{QString toString(const gdcm::Tag \&tag);}

          Funkcja konwertuje element na obiekt tekstu.

    \item \cppcode{gdcm::Tag toAttributeTag(const gdcm::Tag \&tag);}

          Funkcja konwertuje element o znaczniku typu \dicomvr{AT} na obiekt znacznika.

    \item \cppcode{QString toAgeString(const gdcm::Tag \&tag);}

          Funkcja konwertuje element o znaczniku typu \dicomvr{AS} na tekst w postaci czytelnej, np: „18 weeks” lub „3 years”, oraz jest wrażliwy na obecny język aplikacji.

    \item \cppcode{QDate toDate(const gdcm::Tag \&tag);}

          Funkcja konwertuje element o znacznik typu \dicomvr{DA} na obiekt klasy \qtclass{QDate}, który ma w sobie wbudowaną konwersję na tekst zależny od ustawień językowych aplikacji.

    \item \cppcode{QVector<qreal> toDecimalString(const gdcm::Tag \&tag);}

          Funkcja konwertuje element o znacznik typu \dicomvr{DS} na obiekt wektora posiadającego liczby rzeczywiste.
          \cppcode{qreal} jest aliasem do typu zmiennoprzecinkowego, na systemach 64-bitowy jest to \cppcode{double}.

    \item \cppcode{qint32 toIntegerString(const gdcm::Tag \&tag);}

          Funkcja konwertuje element o znacznik typu \dicomvr{IS} na obiekt 32-bitowa liczbę całkowitą.

    \item \cppcode{QString toPersonName(const gdcm::Tag \&tag);}

          Funkcja konwertuje element o znacznik typu \dicomvr{PN} na obiekt tekst zawierający imię w formie pisanej.

    \item \cppcode{qint16 toShort(const gdcm::Tag \&tag);}

          Funkcja konwertuje element o znacznik typu \dicomvr{SS} na obiekt 16-bitowa liczbę całkowitą ze znakiem.

    \item \cppcode{quint16 toUShort(const gdcm::Tag \&tag);}

          Funkcja konwertuje element o znacznik typu \dicomvr{US} na obiekt 16-bitowa liczbę całkowitą bez znaku.

\end{itemize}
Oprócz powyższych funkcji jest jeszcze kilka innych funkcji pobocznych oraz kilka aliasów.

\par
Kilka rzeczy które się tyczą wszystkich danych i konwersji:
\begin{itemize}
    \item Większość VR jest to zapisane jako tekst, kodowanie i dekodowanie tekstu jest zapewniane przez bibliotekę.
    \item Większość danych może mieć kilka wartości oddzielonych backslashem \quotett{\textbackslash}, dlatego konwerter dla VR w, których standard przewiduje wiele wartości, zawsze zwraca wektor z tymi wartościami.
    \item Wszystkie dane są zapisane parzystą ilością bajtów, w przypadku tekstu dodaje się znak spacji na końcu danych, taka spacja jest pomijana w analizie danych.
\end{itemize}

