
Obrazów zapisanych w RGB nie trzeba w żaden sposób obrabiać, dane już są prawie gotowe do wyświetlenia, należy je tylko odpowiednio posortować, tak jak wymaga biblioteka QT.
Sposób posortowania wartości w pilku określa \dicomtag{PlanarConfiguration}{0x0028}{0006}. Może o przyjąć dwie następujące wartośći:

\begin{itemize}
    \item 0 - oznacza to, że wartości pikseli są ułożone w taki sposób
        \[R1, G1, B1, R2, G2, B2, R3, G3, B3, R4, G4, B4,  ...\]
    \item 1 - oznacza to, że wartości pikseli są ułożone w taki sposób
        \[R1, R2, R3, R4, ... , G1, G2, G3, G4, ..., B1, B2, B3, B4, ...\]
\end{itemize}
gdzie:
\begin{conditions}
Rn  &   wartość czerwonego kanału \\
Gn  &   wartość zielonego kanału \\
Bn  &   wartość niebieskiego kanału
\end{conditions}

Wartości obrazu są przepisywane do buffora dla biblioteki QT.
