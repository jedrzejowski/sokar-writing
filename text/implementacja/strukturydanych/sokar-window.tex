\label{sec:sokar-window}

\par
Główne okno programu jest implementowane przez \sokarclass{MainWindow}.
Jest wywoływane od razu po uruchomieniu programu.
\par
Zawiera w sobie 4 elementy: menu, drzewo ze strukturą plików, obiekt z zakładkami oraz w dolnej części okna sugestie, aby nie używać programu w celach medycznych.

\subsubsection{Drzewo katalogów i zakładki}

\par
W lewej części okna znajduje się element listy, implementowany przez \sokarclass{FileTree}, zawiera on w sobie model drzewa plików systemu, który z kolei jest implementowany przez klasę \qtclass{QFileSystemModel}.
Po wybraniu pliku ścieżka jest przesyłana do obiektu z zakładkami.
\par
W środkowej części programu znajduje się obiekt z zakładkami, szczegółowo opisany w sekcji \ref{sec:sokar-dicomtabs}.

\subsubsection{Menu programu}
\label{sec:sokar-window-menu}

\par
W górnej części okna programu znajduje się menu, obiekt klasy \sokarclass{QMenuBar}.
Struktura Menu programu:
\begin{itemize}
    \item File
          \begin{itemize}
              \item Open --- otwiera okienko wyboru plików, implementowane przez \qtfunction{QFileDialog}{getOpenFileName}, następnie wczytuje plik
              \item Open Recent --- program zapisuje ostatnio wczytane pliki i pozwala ja ich ponowne wczytanie z tego menu
              \item Export as --- zapisanie obrazu w formacie JPEG, BMP, GIF lub PNG.
                    Zapisywanie jest zaimplementowane przez funkcje \qtfunction{QImage}{save}, która umożliwia zapisanie obrazu do pliku.
              \item Exit --- wyjście z aplikacji
          \end{itemize}
    \item Help
          \begin{itemize}
              \item About Qt --- otwiera okno informacji o bibliotece Qt.
                    Biblioteka Qt ma wbudowane takie okno w postaci \qtfunction{QMessageBox}{aboutQt}
              \item About GDCM --- otwiera okno z informacjami o bibliotece GDCM, implementowane przez funkcje \sokarfunction{About}{GDCM}
              \item About Sokar --- otwiera okno z informacjami o aplikacji, implementowane przez funkcje \sokarfunction{About}{Sokar}
          \end{itemize}
\end{itemize}