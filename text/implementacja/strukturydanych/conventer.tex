
\par
Każdy plik \DICOM posiada zbiór elementów danych.
Zapisane elementy danych należy przekonwertować na obiekty danych odpowiadające potrzebom programu.
Dlatego został zaimplementowany obiekt klasy \sokarclass{DataConverter} zajmujący się konwersją danych z pliku \DICOM na dane w formacie odpowiadającym programowi.

\par
Obiekt konwertera jest tworzony na podstawie pliku \DICOM i przy wywoływaniu konwersji należy podać tylko znacznik, który nas interesuje.
Takie rozwiązanie pozwala na przesłanie do wszystkich obiektów jednego względnie małego obiektu konwertera, co ułatwia zarządzanie dostępem do pliku \DICOM.

\par
Klasa \sokarclass{DataConverter} posiada następujące funkcje, pozwalające na konwertowanie danych:
\begin{itemize}
    \item \sokarfunction{DataConverter}{toString}

          Funkcja konwertuje element na obiekt tekstu \qtclass{QString}.

    \item \sokarfunction{DataConverter}{toAttributeTag}

          Funkcja konwertuje element o znaczniku typu \dicomvr{AT} na obiekt znacznika \gdcmclass{Tag}.

    \item \sokarfunction{DataConverter}{toAgeString}

          Funkcja konwertuje element o znaczniku typu \dicomvr{AS} na tekst w postaci czytelnej, np: „18 weeks” lub „3 years”.

    \item \sokarfunction{DataConverter}{toDate}

          Funkcja konwertuje element o znacznik typu \dicomvr{DA} na obiekt klasy \qtclass{QDate}, który ma w sobie wbudowaną konwersję na tekst zależny od ustawień językowych aplikacji.

    \item \sokarfunction{DataConverter}{toDecimalString}

          Funkcja konwertuje element o znacznik typu \dicomvr{DS} na obiekt wektora posiadającego liczby rzeczywiste.
          \cppcode{qreal} jest aliasem do typu zmiennoprzecinkowego, na systemach 64-bitowy jest to \cppcode{double}.

    \item \sokarfunction{DataConverter}{toIntegerString}

          Funkcja konwertuje element o znacznik typu \dicomvr{IS} na 32-bitową liczbę całkowitą (\cppcode{qint32}).

    \item \sokarfunction{DataConverter}{toPersonName}

          Funkcja konwertuje element o znacznik typu \dicomvr{PN} na obiekt tekst zawierający imię w formie pisanej.

    \item \sokarfunction{DataConverter}{toShort}

          Funkcja konwertuje element o znacznik typu \dicomvr{SS} na 16-bitowa liczbę całkowitą ze znakiem (\cppcode{qint16}).

    \item \sokarfunction{DataConverter}{toUShort}

          Funkcja konwertuje element o znacznik typu \dicomvr{US} na 16-bitowa liczbę całkowitą bez znaku (\cppcode{quint16}).

\end{itemize}
Oprócz powyższych funkcji jest jeszcze kilka innych funkcji pobocznych oraz kilka aliasów.

\par
Kilka rzeczy które się tyczą wszystkich danych i konwersji:
\begin{itemize}
    \item Większość VR jest to zapisane jako tekst, kodowanie i dekodowanie tekstu jest zapewniane przez bibliotekę.
    \item Większość danych może mieć kilka wartości oddzielonych backslashem \quotett{\textbackslash}, dlatego konwerter dla VR w, których standard przewiduje wiele wartości, zawsze zwraca wektor z tymi wartościami.
    \item Wszystkie dane są zapisane parzystą ilością bajtów, w przypadku tekstu dodaje się znak spacji na końcu danych, taka spacja jest pomijana w analizie danych.
\end{itemize}

