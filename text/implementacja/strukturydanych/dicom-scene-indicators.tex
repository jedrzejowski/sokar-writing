
Informacje na scenie są wyświetlane za pomocą obiektów, które dziedziczą po klasie \sokarclass{SceneIndicator}.
Obiekty te mają dostęp do obiektu konwertera.
Obiekty dziedziczące po \sokarclass{SceneIndicator} implementują róznież swoją pozycje na scenie i są wstanie ją zmieniać w raz ze zmianą wielkości sceny.

Domyślnie obiekty wyświetlające informacje (tytuły punktów to nazwy klas):
\paragraph{\sokarclass{PatientDataIndicator}}

Obiekt wyświetlający dane pacjenta, pojawia się zawsze na scenie w lewym górnym rogu i zawiera następujące linie:
\begin{itemize}
    \item Nazwa pacjenta oraz płeć

          Nazwa pacjenta znajduje się w \dicomtag{PatientName}{0010}{0010} o \dicomvr{PN}.

          Płeć, zapisana jest w \dicomtag{PatientSex}{0010}{0040} i może mieć następujące wartości:
          \begin{itemize}
              \item \dataword{M } - oznacza mężczyznę, wyświetlana jako O
              \item \dataword{F } - oznacza kobietę, wyświetlana jako O
              \item \dataword{O } - oznacza inną płeć i nie jest wyświetlana
          \end{itemize}

          W przypadku określenia inne płci niż jest w standardzie bądź braku tagu płeć nie będzie widoczna.

          Przykład: \dataword{Adam Jędrzejowski O}.

    \item Identyfikator pacjenta

          Unikalny identyfikator pacjenta z tagu \dicomtag{PatientID}{0010}{0020} wyświetlane w takiej formie jakiej jest zapisane, bez żadnej obróbki.
          W praktyce najczęściej jest to numer z systemu używanego w danym szpitalu, rzadziej numer PESEL.

          Przykład: \dataword{HIS/000000}.

    \item Data urodzenia oraz wiek pacjenta w trakcie badania

          Data urodzenia znajdująca się w \dicomtag{PatientBirthDate}{0010}{0030} i jest zamieniana na format \dataword{YYYY-MM-DD}.
          Dodatkowo, jeżeli tag \dicomtag{PatientAge}{0010}{1010} jest obecny, wyświetlany jest także wiek pacjenta w czasie badania.

          Przykład: \dataword{born 1982-08-09, 28 years}.

    \item Opis wykonany przez instytucję opis lub klasyfikację badania (komponentu)

          Tekst brany z \dicomtag{StudyDescription}{0008}{1030} i wyświetlany bez żadnej obróbki.

          UWAGA: Ta wartość jest wpisywana przez technika, operatora lub lekarza wykonującego badanie, więc wartość ta może być nie przewidywalna.

    \item Opis serii

          Tekst brany z \dicomtag{SeriesDescription}{0008}{103E} i wyświetlany bez żadnej obróbki.

          UWAGA: Ta wartość jest wpisywana przez technika, operatora lub lekarza wykonującego badanie, więc wartość ta może być nie przewidywalna.
\end{itemize}

Przykład pełnego teksu:

\texttt{\\
    \textbf{Adam Jędrzejowski} O\\
    HIS/123456\\
    born 1996-07-16, 19 years\\
    Kregoslup ledzwiowy a-p + boczne\\
    AP
}

\paragraph{\sokarclass{HospitalDataIndicator}}

Obiekt wyświetlający dane szpitala/instytucji, pojawia się zawsze na scenie w prawym górnym rogu i zawiera następujące linie:
\begin{itemize}
    \item Nazwa instytucji

          Tekst brany z \dicomtag{InstitutionalDepartmentName}{0008}{1040} i wyświetlany bez żadnej obróbki.

\end{itemize}

\paragraph{\sokarclass{ImageOrientationIndicator}}

\par
Obiekt wyświetlający cztery litery oznaczające orientacje obrazu w stosunku do pacjenta.
Obiekt posiada cztery pola: lewe, górne, prawe i dolne.

\par
Każda z sześciu możliwych liter oznacza kierunek oraz zwrot w jakim jest ułożony pacjent:
\begin{itemize}
    \item \dataword{R} - right - część prawa pacjenta
    \item \dataword{L} - left - część
    \item \dataword{A} - anterior - przód pacjenta
    \item \dataword{P} - posterior - tył pacjenta
    \item \dataword{F} - feet - część dolna
    \item \dataword{H} - head - część górna
\end{itemize}

\par
Pełeny opis implementacji algorytmu wyznaczania stron znajduje się w sekcji \label{sec:algorithm-imageorientationindicator}.

\paragraph{\sokarclass{PixelSpacingIndicator}}

Obiekt wyświetlający miarkę z podziałką informującą jakich rozmiarów jest obiekt na obrazie w rzeczywistości, pojawia się na dole i po prawie stronie sceny, gdy tag \dicomtag{PixelSpacing}{0028}{0030} jest obecny.
Wygląd podziałki można zaobserwować na rysunku \ref{fig:imageorientationindicator1}.

Podziałka dostosowuje swoją wielkość do obecnej sceny, jak i do innych elementów na scenie.
Wartości wyświetlane biorą pod uwagę transformatę skali i rotacji obrazu.

\paragraph{\sokarclass{ModalityIndicator}}

Obiekt wyświetla informacje o akwizycji obrazu.
Dane różnią się w zależności od modalności obrazu.
Domyślnie zawierają następujące linie:
\begin{itemize}
    \item bla bla bla
    \item bla bla bla
    \item bla bla bla
    \item bla bla bla
\end{itemize}

W przypadku następujących modalności zawierają również następujące informacje:
\begin{itemize}
    \item bla bla bla
    \item bla bla bla
    \item bla bla bla
    \item bla bla bla
\end{itemize}