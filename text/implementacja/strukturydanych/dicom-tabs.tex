\label{sec:sokar-dicomtabs}

\par
Obiekt zakładek, implementowany za pomocą klasy \sokarclass{DicomTabs}, odpowiada za wyświetlanie wielu obiektów zakładek w jednym obiekcie interfejsu.
Obsługuje również wczytanie nowych plików.

\subsubsection{Sposoby uzyskania nowych plików}

\par
Otworzenie nowego pliku może odbyć się z następujących źródeł: obiektu drzewa ze strukturą plików w systemie (opisanego w \ref{sec:sokar-framechooser}), menu programu (opisanego w \ref{sec:sokar-window-menu}), lub poprzez przeciągnięcie i upuszczenie pliku.
Z dwóch pierwszych można wczytać tylko po jednym pliku, natomiast trzecim sposobem można wczytać zarówno jeden jak i wiele plików.
Wysyłanie prośby odbywa się za pomocą dwóch funkcji: \sokarfunction{DicomTabs}{addDicomFile} i \sokarfunction{DicomTabs}{addDicomFiles}.
Każda z tych funkcji ma dwa przeciążenia, jedno z parametrem ścieżki a drugie z wczytanym plikiem.

\subsubsection{Wczytywanie plików}

\par
Po dostarczeniu ścieżek do obiektu, pliki zostają wczytane za pomocą funkcji \gdcmclass{ImageReader}.
W przypadku błędu proces wczytywania się kończy.
Po wczytaniu wszystkich plików zostaje utworzony obiekt kolekcji ramek obrazu lub kolekcji plików \DICOM za pomocą funkcji \sokarfunction{DicomFileSet}{create}, opisanej w sekcji \ref{sec:sokar-dicomfileset-create}.
