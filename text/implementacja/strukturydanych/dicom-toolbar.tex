\label{sec:sokar-dicomtoolbar}
\par
Pasek narzędzi znajdujący się na górze, implementowany jest przez klasę \sokarclass{DicomToolBar}, dziedziczącą po klasy \qtclass{QToolBar}.
Posiada on zespół ikonek z rozwijalnymi menu kontekstowymi.

\par
Kliknięcie odpowiedniej ikony spowoduje wysłanie sygnału do obecnie wyświetlanej sceny.
Są dwa sygnały możliwe do wysłania \sokarfunction{DicomToolBar}{stateToggleSignal} lub \sokarfunction{DicomToolBar}{actionTriggerSignal}.
Pierwszy sygnał oznacza zmianę stanu paska, czyli sposób obsługi myszki i zawiera jeden argument: stan (typu \cppcode{enum}).
Sygnał ten okazał się bezużyteczny i nie jest obecnie wykorzystywany przez scenę.
Drugi oznacza akcję, która powinna być wykonana przez scenę.
Zawiera dwa argumenty: typ akcji (typu \cppcode{enum}) i stan akcji (typu \cppcode{bool} z domyślną wartością \cppcode{false}).

Ikony na pasku:
\begin{itemize}
    \item Okienkowanie (1)

          Stan \cppcode{Windowing} oznacza, że horyzontalny ruch myszki powinien zmieniać szerokość okna, a wertykalny środek okna.
          Przycisk jest aktywny tylko wtedy, gdy obecna scena posiada obraz monochromatyczny.

    \item Przesuwanie (2)

          Stan \cppcode{Pan} oznacza, że ruch myszki powinien przesuwać obraz na scenie w prawo, lewo, górę lub dół, kiedy jest wciśnięty lewy klawisz myszy.

          Rozwijalne menu zawiera tylko jedne element \enquote{Move To Center} wysyłający sygnał akcji z argumentem \cppcode{ClearPan}.

    \item Skalowanie (3)

          Stan \cppcode{Zoom} oznacza, że ruch myszki powinien skalować obraz kiedy jest wciśnięty lewy klawisz myszy.

          Menu rozwijalne:
          \begin{itemize}
              \item Fit To Screen --- Dopasuj do ekranu

                    Akcja: \cppcode{Fit2Screen}.

                    Po otrzymaniu sygnału obraz na scenie powinien dopasować swoją wielkość do wielkości sceny

              \item Original Resolution --- Skala jeden do jednego

                    Akcja: \cppcode{OriginalResolution}.

                    Po otrzymaniu sygnału obraz na scenie powinien dopasować swoją wielkość jeden do jednego w stosunku do piksela na ekranie.

          \end{itemize}

    \item Rotacja (4)

          Stan \cppcode{Rotate} oznacza, że ruch myszki powinien obracać obrazem znajdującym się na scenie.

          Menu rozwijalne:
          \begin{itemize}
              \item Rotate Right --- Obróć w prawo

                    Akcja: \cppcode{RotateRight90}.

                    Po otrzymaniu sygnału obraz na scenie powinien obróć się o 90 stopni w prawo.

              \item Rotate Left --- Obróć w lewo

                    Akcja: \cppcode{RotateLeft90}.

                    Po otrzymaniu sygnału obraz na scenie powinien obróć się o 90 stopni w lewo.

              \item Flip Horizontal --- Odbij lustrzanie poziomo %https://sjp.pl/lustrzanie

                    Akcja: \cppcode{FlipHorizontal}.

                    Po otrzymaniu sygnału obraz na scenie powinien odbić się lustrzanie poziomo.

              \item Flip Vertical --- Odbij lustrzanie pionowo

                    Akcja: \cppcode{FlipVertical}.

                    Po otrzymaniu sygnału obraz na scenie powinien odbić się lustrzanie pionowo.

              \item Clear Transformation --- Wyczyść przekształcenia obrotu

                    Akcja: \cppcode{ClearRotate}.

                    Po otrzymaniu sygnału obraz na scenie powinien wyczyścić transformatę obrotu.

          \end{itemize}
          
    \item Informacje na obrazie (5)

          Ten element potrafi wyłączyć wyświetlanie niektórych elementów na scenie.
          Kliknięcie go odznacza lub zaznacza wszystkie pozycje w menu kontekstowym.
          Wszystkie pozycje są pozycjami odznaczanymi.

          Menu rozwijalne:
          \begin{itemize}
              \item Patient Data --- Dane pacjenta

                    Akcja: \cppcode{PatientData}.

                    Po otrzymaniu sygnału obiekt klasy \sokarclass{PatientDataIndicator} znajdujący się na scenie powinien pokazać lub ukryć się w zależności od stanu pozycji.

              \item Hospital Data --- Dane szpitala

                    Akcja: \cppcode{HospitalData}.

                    Po otrzymaniu sygnału obiekt klasy \sokarclass{HospitalDataIndicator} znajdujący się na scenie powinien pokazać lub ukryć się w zależności od stanu pozycji.
              \item Image Acquisition --- Dane akwizycji

                    Akcja: \cppcode{ModalityData}.

                    Po otrzymaniu sygnału obiekt klasy \sokarclass{ModalityIndicator} znajdujący się na scenie powinien pokazać lub ukryć się w zależności od stanu pozycji.

          \end{itemize}

    \item Tagi (5)

          Akcja: \cppcode{OpenDataSet}.

          Kliknięcie tego przycisku wyśle prośbę o otworzenie okna ze zbiorem elementów danych pliku obrazu, który jest obecnie wyświetlany na scenie.

\end{itemize}