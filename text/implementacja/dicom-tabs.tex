\label{sec:sokar-dicomtabs}

\par
Jest to obiekt odpowiadający za wyświetlanie wielu obiektów \sokarclass{DicomView} w jednym okienku w formie zakładek.
Obsługuje również prośby o wczytanie nowych plików.

\subsection{Sposoby uzyskania nowych plików}

\par
Prośba o otworzenie nowego pliku może przyjść z następujących źródeł: obiektem drzewa ze strukturą plików w systemie (opisanego w \ref{sec:sokar-framechooser}), menu programu (opisanego w \ref{sec:sokar-window-menu}), lub z możliwości przeciągnięcie i upuszczenia.
Z dwóch pierwszych można wczytać tylko po jednym pliku, natomiast z drugiego sposobu można wczytać zarówno jedne jak i wiele plików.
Wysyłanie prośby odbywa się za pomocą czterech funkcji a dokładniej dwóch, przeciążonych dwa razy: \sokarfunction{DicomTabs}{addDicomFile} i \sokarfunction{DicomTabs}{addDicomFiles}.
Każda z tych funkcji ma dwa przeciązęnia, jedno z parametrem ścieżki a drugie z wczytanym plikiem, dodatkowo funkcje te są slotami.

\subsection{Wczytywanie plików}

\par
Po dostarczeniu ścieżek do obiektu, pliki zostają wczytane za pomocą \gdcmclass{ImageReader}.
W przypadku błędu proces wczytywania się kończy.
Po wczytaniu wszystkich plików, zostaje utworzony obiekt zbiory ramek obrazu lub zbiór plików DICOM za pomocą funkcji \sokarfunction{DicomFileSet}{create}, opisanej w sekcji \ref{sec:sokar-dicomfileset-create}.
