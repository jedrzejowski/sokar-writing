
\par
Po uruchomieniu programu użytkownikowi ukazuje się głowne okno, pokazane na rysuneku \ref{sec:gui-window}, implemntowane przez klasę \sokarclass{MainWindow}.
Okno zawiera 3 elementy: menu (obiekt klasy \qtclass{QMenuBar}), drzewa plików (obiekt klasy \sokarclass{FileTree}), obiekt zakładek z obrazami (obiekt klasy \sokarclass{DicomTabs}).
\par
Użytkownik może otworzyć plik \DICOM na trzy sposoby: z menu na górze, z drzewa ze strukturą pików i poprzez przeciągnięcie.
\par
Obiekt wewnątrz zakładek odpowiada za wyświetlanie wszystkich elementów umożliwiających interakcje użytkownika z obrazem, jest implementowany przez klasę \sokarclass{DicomView}.
Jeden taki obiekt może posiadać wiele obrazów wyświetlanych w formie animacji.
Obrazy są wyświetlane na scenie implementowanej przez \sokarclass{DicomScene}.
Pod sceną znajduje się pasek filmu, za pomocą, którego użytkownik może zatrzymać lub wznowić animację.
Na prawo od sceny znajdują sie ikony i wszystkimi ramkami filmu.
Pasek filmu i ikony obrazów ukrywają się gdy jest wczytany tylko jeden obraz.

\subsection*{Informacje na scenie z obrazem}

Scena jest odpowiedzialna za wygenerowanie i wyświetlenie obrazu, ale również za 

\subsection*{Pasek filmu}

\subsection*{Struktura paska menu programu}
\par
\begin{itemize}
    \item File
          \begin{itemize}
              \item Open --- otwiera okienko wyboru plików
              \item Open Recent --- lista z ostatnio otworzonymi plikami
              \item Export as --- zapisanie obrazu w formacie JPEG, BMP, GIF lub PNG.
              \item Exit --- wyjście z aplikacji
          \end{itemize}
    \item Help
          \begin{itemize}
              \item About Qt --- otwiera okno informacji o bibliotece Qt.
              \item About GDCM --- otwiera okno z informacjami o bibliotece GDCM
              \item About Sokar --- otwiera okno z informacjami o aplikacji
          \end{itemize}
\end{itemize}