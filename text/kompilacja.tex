
% W przypadku pytań dotyczących kompilacji, jest możliwość skontaktowania się z autorem pod adresem \mailToMe..

\section{Narzędzia potrzebne do kompilacji}

\par
Do kompilacji wystarczą podstawowe narzędzia budowania dostosowane do systemu operacyjnego:

\begin{itemize}
    \item Windows --- Visual Studio w wersji 2017 lub nowszej,
    \item Linux --- pakiety zawierające następujące komendy: make; cmake (w wersji 3.10 lub nowszej), g++ (w wersji 8 lub nowszej),
          % sudo apt-get install build-essential libgl1-mesa-dev
          % sudo pacman -Syu qt5-base
    \item MacOS --- Xcode w wersji 10 lub nowszej.
\end{itemize}

Kod źródłowy został skompilowany za pomocą powyższych narzędzi.
W kodzie programu nie występują żadne elementy odbiegające od standardu C++17, więc nie powinno być problemów z użyciem niższych wersji wspierających standard C++17

\section{Biblioteki potrzebne do kompilacji}

Do kompilacji są potrzebne biblioteki Qt i GDCM.

\subsection{Instalacja Qt}

Program był testowany na wersji $5.12$, z tego powodu zalecanym jest używanie tej wersji.

\subsubsection*{Linux}

Biblioteka Qt jest dostępna w każdej standardowej dystrybucji Linuxa.
Dlatego instalacja Qt sprowadza się do pobrania z repozytoriów za pomocą menadżera pakietów.
% \par
% Komendy pozwalające zainstalować bibliotekę Qt na wybranych dystrybucjach:\\
% Ubuntu: \texttt{sudo apt-get install qt5-default}\\
% ArchLinux: \texttt{sudo pacman -Syu qt5-base}

\subsubsection*{Windows i MacOS}

\par
W celu instalacji Qt należy udać się na oficjalną stronę biblioteki Qt.
W prawym górnym rogu powinno się kliknąć zielony przycisk \enquote{Download. Try. Buy.}.
Na dole kolumny zatytułowanej \enquote{Open Source} należy kliknąć zielony przycisk \enquote{Go open source}.
Zostanie automatycznie pobrany plik instalacyjny.
Po pobraniu należy go otworzyć i postępować zgodnie z instalacją.
\par
W pewnym momencie użytkownik może zostać poproszony o dane kontaktowe.
Nie jest to wymagane i można kliknąć przycisk \enquote{Skip}.
\par
Następnie należy wybrać komponenty do zainstalowania.
W przypadku Windowsa należy zainstalować wersje \enquote{Qt 5.12.3 MSVC 2017 64-bit}.
Z kolei na MacOS należy zainstalować \enquote{Qt 5.12.3 clang\_x64}.
Można wyłączyć wszystkie inne opcje, nie są one wymagane do kompilacji programu.
Dalej należy postępować zgodnie z instrukcjami pojawiającymi się na ekranie.

\subsection{Pobranie kodu źródłowego GDCM}

Program był testowany na wersji $2.8.9$, z tego powodu zalecanym jest używanie tej wersji.
\par
Należy udać się na stronę \url{https://github.com/malaterre/GDCM/releases/tag/v2.8.9} i pobrać plik \enquote{Source code (zip)}, a następnie go rozpakować.

\subsection{Pobranie kodu źródłowego Sokar}

Kod źródłowy aplikacji można pobrać repozytorium git znajdującego się pod adresem \url{https://gl.ire.pw.edu.pl/ajedrzejowski/sokar-app}.

\section{Przygotowanie katalogów}

Należy utworzyć folder, w którym będą znajdowały się wszystkie foldery z plikami, dalej ten folder będzie nazywany \quotett{/path/}.
Kod źródłowy GDCM należy umieścić w katalogu \quotett{/path/gdcm/}.
Kod źródłowy Sokar należy umieścić w katalogu \quotett{/path/sokar-app/}.
Powinno się również utworzyć foldery \quotett{/path/gdcm-bin/} i \quotett{/path/sokar-app-bin/}.

\section{Kompilacja GDCM}

Kompilacja zawiera polecenia, które należy wykonać w następującej kolejności:
\begin{itemize}
    \item uruchom CMake z menu programów lub za pomocą polecenia \quotett{cmake-gui},
    \item w polu \enquote{Where is the source code:} wpisz \quotett{/path/gdcm/},
    \item w polu \enquote{Where to build the binnaries:} wpisz \quotett{/path/gdcm-bin/},
    \item kliknij przycisk \enquote{Configure} znajdujący się w dolnej lewej części okna,
    \item w oknie zatytułowanym \enquote{CMakeSetup} wybierz opcje: \enquote{Unix Makefiles} i \enquote{Use default native compilers} dla Linuxa i MacOS; \enquote{Visual Studio 15 2017}, \enquote{x64} i \enquote{Use default native compilers} dla Windowsa,
    \item zaznacz checkbox przy wartości \quotett{GDCM\_BUILD\_SHARED\_LIBS},
    \item kliknij przycisk \enquote{Finish},
    \item następnie kliknij przycisk \enquote{Generate},
    \item w przypadku Linuxa i MacOS, uruchom \quotett{make} w folderze \quotett{/path/gdcm-bin/}, zaś w przypadku Windowsa otwórz plik \quotett{/path/gdcm-bin/GDCM.sln} i kliknij przycisk \enquote{Lokalny debuger Windows}.
\end{itemize}

\section{Kompilacja Sokar}

Kompilacja zawiera polecenia, które należy wykonać w następującej kolejności:
\begin{itemize}
    \item uruchom CMake z menu programów lub za pomocą polecenia \quotett{cmake-gui},
    \item w polu \enquote{Where is the source code:} wpisz \quotett{/path/sokar-app/},
    \item w polu \enquote{Where to build the binnaries:} wpisz \quotett{/path/sokar-app-bin/},
    \item kliknij \enquote{Configure},
    \item w oknie zatytułowanym \enquote{CMakeSetup} wybierz takie same opcje na w GDCM,
    \item kliknij \enquote{Finish},
    \item ustaw parametr wartość \enquote{QtDir} na ścieżkę do skompilowanej biblioteki Qt,
    \item następnie kliknij przycisk \enquote{Generate},
    \item w przypadku Linuxa i MacOS, uruchom \quotett{make} w folderze \quotett{/path/sokar-app-bin/}, zaś w przypadku Windowsa otwórz plik \quotett{/path/gdcm-bin/Sokar.sln} i kliknij przycisk \enquote{Lokalny debuger Windows}.
\end{itemize}

\section{Uruchomienie}

W przypadku Linuxa i MacOS, plik binary znajduje się w katalogu \quotett{/path/sokar-app-bin/}.
Można go uruchomić go klikając lub z terminala za pomocą komendy \quotett{./Sokar}.
W przypadku Windows plik binarny znajduje się w folderze \quotett{/path/sokar-app-bin/Debug}.