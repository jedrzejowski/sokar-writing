\begin{center}
\large \bf
\thetitle
\end{center}

Praca składa się z sześciu rozdziałów: wstęp, obrazowanie diagnostyczne, biblioteki i narzędzia, implementacja, kompilacja oraz podsumowanie.
Wstęp jest wprowadzeniem do tematu i celu pracy.
\par
W drugim rozdziale jest opisane zagadnienie problemowe związane z obrazami w medycynie.
Wymienione są techniki diagnostyczne oraz ich podstawowe różnice.
Przedstawione są parametry cyfrowych obrazów w medycynie.
Ponadto opisano prezentacje obrazów medycznych oraz wyjaśniono czym są przeglądarki obrazów.
Omówione są posiadane przez nie funkcje.
Opisano format zapisu cyfrowych obrazów medycznych, standard \DICOM.
\par
Trzeci rozdział opisuje biblioteki i narzędzia użyte w czasie pisania pracy inżynierskiej.
Wyjaśnione są cele użycia narzędzia CMake i jego zalety.
Opisano bibliotekę Qt, jej możliwości, drzewa obiektów implementowane przez nią i sposób konstrukcji programowania zdarzeniowego w niej zawartego.
Przedstawiono i uzasadniono wybór biblioteki GDCM jako biblioteki do obsługi i wczytywania plików \DICOM.
\par
W czwartym rozdziale przedstawiono sposób implementacji pracy.
Określono przewidywany zakres implementowanych funkcji oprogramowania.
Opisano graficzny interfejs użytkownika i jego funkcje programu.
Wyjaśniono projekt struktury obiektowej programu.
Następnie szczegółowo opisano strukturę danych wraz z klasami C++.
Tam gdzie była możliwość załączony jest diagram UML.
Opisano wszystkie algorytmy przetwarzania danych w celu lepszej wizualizacji obrazu.
\par
W piątym rozdziale opisano przebieg kompilacji kodu źródłowego.

\bigskip
{\noindent\bf Słowa kluczowe:} \keywords

\vfill