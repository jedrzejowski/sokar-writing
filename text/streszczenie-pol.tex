\begin{center}
\large \bf
\thetitle
\end{center}

Praca składa się z sześciu rozdziałów: wstęp; obrazowanie diagnostyczne w medycynie; biblioteki i narzędzia; implementacja; kompilacja; podsumowanie.
Wstęp jest powierzchownym wprowadzeniem do tematu i celu pracy.
\par
W drugim rozdziale jest opisane zagadnienie problemowe związane z obrazami w medycznie.
Wymieniono techniki diagnostyczne oraz ich podstawowe różnice między sobą.
Przedstawiono parametry jakie cyfrowy obraz medycyny posiada.
Opisano prezentacje obrazów medycznych.
Wyjaśniono czym są przeglądarki obrazów, jakie funkcje mogą posiadać i jakie kryteria wyróżniono do ich porównywania.
Opisano format zapisu cyfrowych obrazów medycznych, standard \DICOM.
\par
Trzeci rozdział opisuje biblioteki i narzędzie użyte w czasie pisania pracy inżynierskiej.
Wyjaśniono cele użycia narzędzia CMake i jego zalety.
Opisano bibliotekę Qt, jej możliwości, drzewa obiektów implementowane przez nią i sposób konstrukcji programowania zdarzeniowego w niej zawartego.
Przedstawiono i uzasadniono wybór biblioteki GDCM jako biblioteki do obsługi i wczytywania plików \DICOM.
\par
W czwartym rozdziale przedstawiono sposób implementacji pracy.
Określono przewidywany zakres implementowanych funkcji oprogramowania.
Opisano graficzny interfejs użytkownika i jego funkcje oraz ogólną funkcjonalność programu.
Wyjaśniono powierzchownie projekt struktury obiektowej programu.
Następnie szczegółowo opisano strukturę danych wraz z klasami C++.
Tam gdzie była możliwość załączono diagramu UML.
Opisano wszystkie algorytmy przetwarzania danych w celu lepszej wizualizacji obrazu.
\par
W piątym rozdziale opisano przebieg kompilacji kodu źródłowego.
\par
Prace zakończono podsumowaniem i wnioskami.

\bigskip
{\noindent\bf Słowa kluczowe:} \keywords

\vfill