\par
W różnych systemach operacyjnych są różne kompilatory i wśród tej różnorodności pojawia się problem dotyczący zmiennych fundamentalnych.
Przykład jest zagadnienie: ile bitów ma zmienna \cppcode{int}?
Udając się do dokumentacji C++, dostępnej pod adresem \url{https://pl.cppreference.com/w/cpp/language/types}, możemy dowiedzieć się, że \cppcode{int} ma minimum 16 bitów.
Natomiast w dokumentacji MSVC, kompilatora firmy Microsoft, znajdującej się pod adresem \url{https://docs.microsoft.com/pl-pl/cpp/cpp/int8-int16-int32-int64?view=vs-2019}, widnieje informacja z której wynika, że aby mieć pewność o długości liczby całkowitej należy użyć takich typów: \cppcode{\_\_int8}, \cppcode{\_\_int16}, \cppcode{\_\_int32}, \cppcode{\_\_int64}.
\par
Jest to problem, który biblioteka Qt rozwiązała wprowadzając dodatkowe typy literałów, które dostosowują się do systemu i kompilatora oraz zapewniają pewność podczas deklaracji, że dana zmienna będzie zakładanej długości.
Dodatkowe typy literałów są dostępne w nagłówku \cppcode{<QtGlobal>}, dokumentacja dostępna pod adresem \url{https://doc.qt.io/qt-5/qtglobal.html}.

\par
Dlatego w pracy zostały użyte typy fundamentalne dostarczane przez bibliotekę Qt.
Kilka przykładów:
\begin{itemize}
    \item \cppcode{qint8} --- liczba całkowita, 8 bitowa, ze znakiem
    \item \cppcode{qint16} --- liczba całkowita, 16 bitowa, ze znakiem
    \item \cppcode{qint32} --- liczba całkowita, 32 bitowa, ze znakiem
    \item \cppcode{qint64} --- liczba całkowita, 64 bitowa, ze znakiem
    \item \cppcode{quint8} --- liczba całkowita, 8 bitowa, bez znaku
    \item \cppcode{quint16} --- liczba całkowita, 16 bitowa, bez znaku
    \item \cppcode{quint32} --- liczba całkowita, 32 bitowa, bez znaku
    \item \cppcode{quint64} --- liczba całkowita, 64 bitowa, bez znaku
    \item \cppcode{qreal} --- największa dostępna liczba zmiennoprzecinkowa
\end{itemize}