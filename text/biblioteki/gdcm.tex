

\subsection{Uzasadnienie wyboru}

\par
Znalezienie dobrej biblioteki do obsługi jest niebywale trudne, ponieważ jest ich bardzo dużo, a ich liczba wciąż rośnie.
Powstał nawet portal internetowy do ich indeksowania o nazwie \enquote{I DO IMAGING}, dostępny pod adresem \url{https://idoimaging.com/programs}.
Biblioteka, której szukałem powinna:
\begin{itemize}
    \item współpracować z językiem C++
    \item mieć licencje pozwalającą jej używać w potrzebnym mi zakresie
    \item darmowa, najlepiej otwarto źródłowa
    \item aktywnie rozwijana --- znaczna większość bibliotek charakteryzowała się tym, że była porzucona i ostatnia zmiana była wprowadzona x lat temu, a proces jej rozwoju trwał od 2 do 5 miesięcy
    \item dostępna na Linux'a, MacOS i Microsoft Windows
\end{itemize}
Ostateczna decyzja padła na bibliotekę o nazwie Grassroots DICOM (GDCM), dostępną pod adresem \url{http://gdcm.sourceforge.net/}.

\subsection{Opis}

\par
Przetłumaczony opis biblioteki z oficjalnej strony prezentuje się następująco:
Grassroots DICOM (GDCM) to implementacja standardu DICOM zaprojektowanego jako open source, dzięki czemu naukowcy mogą uzyskać bezpośredni dostęp do danych klinicznych.
GDCM zawiera definicję formatu pliku i protokół komunikacji sieciowej, z których oba powinny zostać rozszerzone w celu zapewnienia pełnego zestawu narzędzi dla badacza lub małego dostawcy obrazowania medycznego w celu połączenia z istniejącą bazą danych medycznych.

\par


\subsection{Licencja}
\subsection{Zalety}
\subsection{Interfejsy w innych językach}