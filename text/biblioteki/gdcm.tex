

\subsection{Uzasadnienie wyboru}

\par
Znalezienie dobrej biblioteki do obsługi jest trudne, ponieważ jest ich bardzo dużo, a ich liczba wciąż rośnie.
Powstał portal internetowy do ich indeksowania o nazwie \enquote{I DO IMAGING}, dostępny pod adresem \url{https://idoimaging.com/programs}.
\par
Biblioteka, której poszukiwano w tej pracy powinna:
\begin{itemize}
    \item współpracować z językiem C++
    \item mieć licencję pozwalającą jej używać w potrzebnym zakresie
    \item darmowa, najlepiej otwarto źródłowa
    \item aktywnie rozwijana --- znaczna większość bibliotek charakteryzowała się tym, że była porzucona i ostatnia zmiana była wprowadzona x lat temu, a proces jej rozwoju trwał od 2 do 5 miesięcy
    \item dostępna na Linux'a, MacOS i Microsoft Windows
\end{itemize}
Ostatecznie podjęto decyzję o wyborze biblioteki o nazwie Grassroots DICOM (GDCM), dostępną pod adresem \url{http://gdcm.sourceforge.net/}.

\subsection{Opis}

\par
Przetłumaczony opis biblioteki z oficjalnej strony prezentuje się następująco:
Grassroots DICOM (GDCM) to implementacja standardu \DICOM zaprojektowanego jako open source, dzięki czemu naukowcy mogą uzyskać bezpośredni dostęp do danych klinicznych.
GDCM zawiera definicję formatu pliku i protokół komunikacji sieciowej, z których oba powinny zostać rozszerzone dla zapewnienia pełnego zestawu narzędzi badaczowi lub małemu dostawcy obrazowania medycznego w celu połączenia z istniejącą bazą danych medycznych.

\par
GDCM jest biblioteką posiadającą możliwość wczytywania, edycji i zapisu plików w formacie \DICOM.
Obsługuje ona wiele kodowań obrazów jak i protokoły sieciowe.
Jest w całości napisana w C++, a do kompilacji używa CMake.
Dzięki temu w całym programie jest używany język C++ wraz z CMake, co ułatwia zarządzanie procesem kompilacji do jednego pliku.

\par
Główną zaletą biblioteki jest dobra dokumentacja wraz z przykładami jej użycia, które okazały się kluczowe przy wyborze.
Biblioteka została napisana w sposób obiektowy z usprawnieniami zawartymi w C++, takimi jak referencje i obiekty stałe, co ułatwia jej używanie.

\subsection{Licencja}

\par 
GDCM jest wydana na licencji BSD License, Apache License V2.0, która jest kompatybilna z GPLv3
Licencja ta dopuszcza użycie kodu źródłowego zarówno na potrzeby wolnego oprogramowania, jak i własnościowego oprogramowania.


\subsection{Podstawowe klasy}
\label{sec:gdcm-classes}
\gdcmclassExplanations

\begin{itemize}
    \item \gdcmclass{Reader} --- klasa służąca do wczytywania pliku DICOM
    \item \gdcmclass{ImageReader} --- klasa służąca do wczytywania obrazu DICOM, dziedziczy po \gdcmclass{Reader}, jest wstanie wygenerować obiekt obrazu
    \item \gdcmclass{Image} --- obiekt obrazu ułatwiający pobieranie informacji
    \item \gdcmclass{File} --- obiekt pliku DICOM
    \item \gdcmclass{DataSet} --- obiekt zbioru elementów
    \item \gdcmclass{DataElemet} --- obiekt elemntu
    \item \gdcmclass{Tag} --- obiekt znacznika
    \item \gdcmclass{StringFilter} --- pomocnicza klasa służąca do konwersji na obiekt tekstu
\end{itemize}

\subsection{Przykład użycia}
\label{sec:gdcm-use}
Poniżej zaprezentowano kilka przykłądów użycia biblioteki GDCM.

\newpage
\subsubsection{Przykład wczytania pliku}

W poniższym przykładzie mamy do czynienia z wczytaniem pliku oraz pobraniem kilku wartości z elementów o danych znacznikach.

\lstinputlisting{text/biblioteki/gdcm-reader.cpp}


\newpage
\subsubsection{Przykład wczytania obrazu}

W tym przykładzie widzimy usprawnione wczytywanie obrazu za pomocą klasy przystosowanej do tego.

\lstinputlisting{text/biblioteki/gdcm-image.cpp}
