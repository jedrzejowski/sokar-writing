\par
Biblioteka Qt, rozwijana przez organizacje Qt Project, jest zbiorem bibliotek i narzędzi pr programistycznych dedykowanych dla języków C++, QML i Java.
\par
Qt jest głownie znane jako biblioteka do tworzenia interfejsu graficznego, jednakże posiada ona wiele innych rozwiązań ułatwiających programowanie obiektowe i zdarzeniowe.
\par
Wybrałem Qt z uwagi na to, że posiada interfejs w C++.
Oraz kompilacja oprogramowania używającego Qt może odbywać się za pomocą dwóch narzędzi: CMake oraz dedykowanego narzędzia qmake, zrobionego specjalnie na potrzeby biblioteki Qt.
Dzięki czemu cały projekt przeglądarki używa tego samego języka oraz tego samego narzędzia zarządzania kompilacją.

\subsection{Wymowa}

\par
Według autorów, Qt powinno się czytać jak angielskie słowo \enquote{cute}, po polsku \enquote{kiut}.
Jednakże społeczność programistów nie jest co fo tego zgodna.
Ankiety zrobione na dwóch popularnych serwisach internetowych o tematyce programistycznej, pokazują, że najbardziej popularną wymową jest \enquote{Q.T.}, po polsku \enquote{ku te}.
\par
Odnośniki do przytoczonych ankiet:
\begin{itemize}
    \item \url{https://ubuntuforums.org/showthread.php?t=1605716}
    \item \url{https://www.qtcentre.org/threads/11347-How-do-you-pronounce-Qt}
\end{itemize}

\subsection{Licencja}

\par
Biblioteka Qt jest dystrybuowana w dwóch wersjach: komercyjnej i otwarto źródłowej.
Wersja otwarto źródłowa nie posiada wielu modułów, ale jest dystrybuowana na licencji \href{https://www.gnu.org/licenses/gpl.html}{GPL w wersji 3}.
Co sprawia, że bibliotekę można użyć w mojej pracy.

\subsection{Norma IEC 62304:2015}

\par
The Qt Company posiada szereg certyfikatów od FDA i UE, ułatwiające wprowadzenie produktów używających bibliotek Qt na rynek Europejski jak i Amerykański.
\par
Lista posiadanych norm:
\begin{itemize}
    \item IEC 62304:2015 (2006 + A1)
    \item IEC 61508:2010-3 7.4.4 (SIL 3)
    \item ISO 9001:2015 
\end{itemize}
Więcej informacji na temat certyfikatów można przeczytać na oficjalnej stronie Qt pod adresem \url{https://www.qt.io/qt-in-medical/}.

\subsection{Klasa QObject}

\par
Biblioteka Qt implementuje klasę \qtclass{QObect}, która jest bazą dla wszystkich obiektów Qt i wszystkie klasy współpracujące z biblioteką Qt powinny po niej dziedziczyć.
\qtclass{QObject} implementuje 2 podstawowe rzeczy: system drzewa obiektów (opisany w sekcji \ref{sec:qt-pareting}), system sygnałów (opisany w sekcji \ref{sec:qt-signals}).

\subsubsection{Drzewa obiektów}
\label{sec:qt-pareting}

\par
W C++ jednym z największych problemów jest wyciek pamięci, pojawia się wtedy gdy zaalokujemy na stercie ony obiekt za pomocą operatora \cppcode{new} i nie go usunąć gdy ten będzie potrzebny.
\par
\qtclass{QObject} zakłada, że obiekty mogą mięć jednego rodzica, a rodzic może mieć wiele dzieci.
Rodzica można przypisać podczas tworzenia obiektu oraz zmieniać go dowolnie w trackie działania programu.
Przypisanie rodzica dziecku oznacza to, że gdy wywołamy destruktor rodzica, ten wywoła destruktory dzieci i w ten sposób całe drzewo obiektów zostanie zniszczone.
\par
Mechanizm ten pozwala nam tworzyć nowe obiekty na stercie i nie martwić się o ich poźniejsze sprzątnięcie.
Jest to o tyle efektywne, że nie trzeba dla każdego obiektu tworzyć odrębnego wskaźnika lub wektora wskaźników w deklaracji klasy, a dzięki temu można mieć czystszy i czytelniejszy kod źródłowy.
Przykładowe użycie:
\par
\begin{lstlisting}[language=C++]
int main() {

    // Tworzymy obiekt przycisku
    auto *quit = new QPushButton("Quit");
    // Tworzymy obiekt okna
    auto *window = new QWidget();

    // Przypisujemy rodzica przyciskowi
    quit->setParent(window);
    
    ...

    // W tym momencie przycisk wraz z oknem zostaja usuniete
    delete window;
}
\end{lstlisting}

\subsubsection{Sygnały i sloty}
\label{sec:qt-signals}

\par
System sygnałów i slotów jest implementacją programowania zdarzeniowego.
Sygnał jest źródłem zdarzenia, a slot jest odbiornikiem zdarzenia.
Sygnał obiektu jest łączony do slotu obiektu dynamicznie w czasie działania programu.
Do jednego sygnału można podłączyć wiele slotów, jak i do jednego slotu można wprowadzić wiele sygnałów.
Gdy zdarzenie zostanie wyemitowane, to wszystkie sloty podłączone do sygnału zostaną powiadomione.
Sygnały i sloty są implementowane przez funkcje definiowane w deklaracji klasy.
System sygnałów Qt nie ma nic wspólnego w sygnałach pojawiających się w C, takich jak \enquote{SIGTERM}.
Dodatkowo sygnały w Qt są wstanie przenosić argumenty definiowane przez programistę.
Taka implementacja umożliwia programowanie zdarzeniowe.

\par
Przykład użycia sygnałów do propagacji zdarzenia.

\begin{lstlisting}
/* Tworzymy dwa obiekty klasy Counter (definicja w następnej sekcji) */
Counter a, b;

/* Łączymy sygnał Counter::valueChanged obiektu "a",
   do slotu Counter::setValue obiektu "b" */
QObject::connect(&a, &Counter::valueChanged,
                 &b, &Counter::setValue);

/* Ustawiamy wartość licznika obiektu "a" na 12 */
a.setValue(12);

/* W czasie ustawiania został wysłany sygnał z "a" do "b", więc:
   a.value() == 12    b.value() == 12 */

/* Ustawiamy wartość licznika obiektu "b" na 48 */
b.setValue(48);

/* Sygnał Counter::valueChanged obiektu "b" nie jest podłączony do
   żadnego slotu, więc:
   a.value() == 12     b.value() == 48 */

\end{lstlisting}

\par
Pełna dokumentacja na temat sygnałów i slotów znajduje się na oficjalne stronie Qt pod adresem \url{https://doc.qt.io/qt-5/signalsandslots.html}

\subsection{Graficzny interfejs użytkownika}

\subsection{Oddzielenie od platformy}

\par
