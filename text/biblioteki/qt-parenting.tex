
\par
W C++ jednym z największych problemów jest wyciek pamięci, pojawia się wtedy gdy zaalokujemy na stercie ony obiekt za pomocą operatora \cppcode{new} i nie go usunąć gdy ten będzie potrzebny.
\par
\qtclass{QObject} zakłada, że obiekty mogą mięć jednego rodzica, a rodzic może mieć wiele dzieci.
Rodzica można przypisać podczas tworzenia obiektu oraz zmieniać go dowolnie w trackie działania programu.
Przypisanie rodzica dziecku oznacza to, że gdy wywołamy destruktor rodzica, ten wywoła destruktory dzieci i w ten sposób całe drzewo obiektów zostanie zniszczone.
\par
Mechanizm ten pozwala nam tworzyć nowe obiekty na stercie i nie martwić się o ich poźniejsze sprzątnięcie.
Jest to o tyle efektywne, że nie trzeba dla każdego obiektu tworzyć odrębnego wskaźnika lub wektora wskaźników w deklaracji klasy, a dzięki temu można mieć czystszy i czytelniejszy kod źródłowy.
Przykładowe użycie:
\par
\begin{lstlisting}[language=C++]
int main() {

    // Tworzymy obiekt przycisku
    auto *quit = new QPushButton("Quit");
    // Tworzymy obiekt okna
    auto *window = new QWidget();

    // Przypisujemy rodzica przyciskowi
    quit->setParent(window);
    
    ...

    // W tym momencie przycisk wraz z oknem zostaja usuniete
    delete window;
}
\end{lstlisting}