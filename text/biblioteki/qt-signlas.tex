
\par
System sygnałów i slotów jest implementacją programowania zdarzeniowego.
W odróżnieniu od sygnałów pojawiających się w C, sygnały w Qt są wstanie przenosić argumenty definiowane przez programistę.
Sygnały i sloty są implementowane przez funkcje definiowane w deklaracji klasy.
Sygnał obiektu jest łączony do slotu obiektu dynamicznie w czasie działania programu.
Do jednego sygnału można podłączyć wiele slotów, jak i do jednego slotu można wprowadzić wiele sygnałów.
Taka implementacja umożliwia to tworzenie programowania zdarzeniowego.

\par
Przykład użycia sygnałów do propagacji zdarzenia.

\begin{lstlisting}
/* Tworzymy dwa obiekty klasy Counter (definicja w następnej sekcji) */
Counter a, b;

/* Łączymy sygnał Counter::valueChanged obiektu "a",
   do slotu Counter::setValue obiektu "b" */
QObject::connect(&a, &Counter::valueChanged,
                 &b, &Counter::setValue);

/* Ustawiamy wartość licznika obiektu "a" na 12 */
a.setValue(12);

/* W czasie ustawiania został wysłany sygnał z "a" do "b", więc:
   a.value() == 12    b.value() == 12 */

/* Ustawiamy wartość licznika obiektu "b" na 48 */
b.setValue(48);

/* Sygnał Counter::valueChanged obiektu "b" nie jest podłączony do
   żadnego slotu, więc:
   a.value() == 12     b.value() == 48 */

\end{lstlisting}

\par
Pełna dokumentacja na temat sygnałów i slotów znajduje się na oficjalne stronie Qt pod adresem \url{https://doc.qt.io/qt-5/signalsandslots.html}