
\begin{lstlisting}[language=C++]
#include <QObject>

class Counter : public QObject {
    // Każdy klasa dziedzicząca po QObject musi na samym początku swojej definicji mieć makro "Q_OBJECT"
    Q_OBJECT

public:
    Counter() { m_value = 0; }

    int value() const { return m_value; }

    // Sloty powinny być poprzedzone makrem "slots"
    // Widoczność slotów można zmieniać 
public slots:
    void setValue(int value){
        if (value != m_value) {
            m_value = value;

            // Podczas wywoływania sygnału należy poprzedzić to makrem "emit"
            emit valueChanged(value);
        }
    }

    // Sygnały powinny być poprzedzone makrem "signals"
    // Wszystkie sygnały są publiczne
signals:
    void valueChanged(int newValue);

private:
    int m_value;
};

\end{lstlisting}

\begin{lstlisting}[language=C++]
Counter a, b;
QObject::connect(&a, &Counter::valueChanged,
                    &b, &Counter::setValue);

a.setValue(12);     // a.value() == 12, b.value() == 12
b.setValue(48);     // a.value() == 12, b.value() == 48

\end{lstlisting}

\par
Pełna dokumentacja na temat sygnałów i slotów znajduje się na oficjalne stronie Qt pod adresem \url{https://doc.qt.io/qt-5/signalsandslots.html}