\par
CMake to wieloplatformowe narzędzie do automatycznego zarządzania procesem kompilacji programu.
Jest to niezależne od kompilatora narzędzie pozwalające napisać jeden plik, z którego można wygenerować odpowiednie pliki budowania dla dowolnej platformy.
\par
Z uwagi na to, że projekt musi mieć możliwość kompilacji na 3 platformy CMake jest idealnym rozwiązaniem.
Dodatkowo w pracy tej starano się wybrać biblioteki, które kompilują się za pomocą CMake.

\subsection*{Licencja}

CMake został wypuszczony na licencji BSD.
Licencji zgodnych z zasadami wolnego oprogramowania.
Powstałej początkowo na Uniwersytecie Kalifornijskim w Berkeley.
Licencje BSD skupiają się na prawach użytkownika.
Są bardzo liberalne, zezwalają nie tylko na modyfikacje kodu źródłowego i jego rozprowadzanie w takiej postaci, ale także na rozprowadzanie produktu bez postaci źródłowej czy włączenia do zamkniętego oprogramowania, pod warunkiem załączenia do produktu informacji o autorach oryginalnego kodu i treści licencji.
W programie została załączona informacja o użyciu CMake, więc jest możliwość użycia jej w pracy.