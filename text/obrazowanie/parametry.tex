\subsection{Podstawowe parametry obrazu cyfrowego}

\dicomtagExplanations

\par
Każdy obraz cyfrowy jest matrycą pikseli o ustalonych rozmiarach.
W przypadku standardu \DICOM obrazy są matrycami wokseli, posiadającymi wysokość (zapisaną w \dicomtag{Rows}{0028}{0010}) oraz szerokość (zapisaną w \dicomtag{Columns}{0028}{0011}).
Do poprawnej interpretacji znaczenia macierzy służy znacznik \dicomtag{Photometric Interpretation}{0028}{0004}, informujący o fotometrycznym znaczeniu wokseli.
Standard \DICOM definiuje następujące wartości tego tagu (wraz z wyjaśnieniem):
\begin{itemize}
    \item \dataword{MONOCHROME1} i \dataword{MONOCHROME2} --- ta wartość woksela odwzorowuje skale monochromatyczną, odpowiednio od jasnego do ciemnego i od ciemnego do jasnego.

    \item \dataword{PALETTE COLOR} --- ta wartość woksela jest używana jako indeks w każdej z tabel wyszukiwania kolorów palety czerwonej, niebieskiej i zielonej.
          Palety mają swoje własne tagi.
          Wartość raczej rzadka i nie spotykana.

    \item \dataword{RGB} --- oznacza, że woksel jest trzy-kanałowym pikselem RGB (kanały: czerwony, zielony i niebieski).

    \item \dataword{HSV} \fromEng{Hue Saturation Value} --- woksel reprezentuje piksel w modelu przestrzeni barw zaproponowany w 1978 roku przez Alveya Raya Smitha.
          Model ten nawiązuje do sposobu w jakim widzi oko człowieka.
          Wartość wycofana.

    \item \dataword{ARGB} --- ta wartość woksela to piksel RGB z dodatkowym kanałem przezroczystości.
          Wartość wycofana.

    \item \dataword{CMYK} --- ten woksel to piksel w modelu czterech podstawowych kolorów farb drukarskich stosowanych powszechnie w druku wielobarwnym w poligrafii: cyjan, magenta, żółty, czarny.
          Wartość wycofana.

    \item \dataword{YBR\_FULL} --- ten woksel to piksel w modelu przestrzeni barw nazwanej YC\textsubscript{b}C\textsubscript{r}.

          Dodatkowo standard zdefiniował pochodne tej wartości: \dataword{YBR\_RCT}, \dataword{YBR\_FULL\_422}, \dataword{YBR\_PARTIAL\_422}, \dataword{YBR\_PARTIAL\_420}, \dataword{YBR\_ICT}, ale wszystkie są już wycofane.
\end{itemize}

\dicomRetired

\par
Kwantyzacja obrazu, czyli liczba poziomów obrazu, jest zapisana na czterech znacznikach:
\begin{itemize}
    \item \dicomtag{Bits Allocated}{0028}{0100} --- informuje na jak wiele bitów zostało zaalokowanych do zapisania jednego piksela
    \item \dicomtag{Bits Stored}{0028}{0101} --- informuje jak wiele bitów z zaalokowanych posiada wartość piksela
    \item \dicomtag{High Bit}{0028}{0102} --- informuje gdzie znajduje się najstarszy bit
    \item \dicomtag{Pixel Representation}{0028}{0103} --- informuje czy poziomy są ze znakiem czy bez
\end{itemize}

\par
Obraz \DICOM również zawiera w sobie informacje o próbkowaniu.
Z uwagi na to, że próbkowanie wygląda inaczej w każdej technice, standard posiada odpowiedni zestaw znaczników dla każdej techniki.
Próbkowanie poszczególnych technik opisałem w sekcji \ref{sec:basic-imaging-technics}.

% \subsection{Wartość diagnostyczna obrazu}

% W obrazowaniu medycznym chodzi o wyciągnięcie wniosków z obrazów i postawienie diagnozy.
% Jest to kluczowy element obrazowania.
% Ocena wartości diagnostycznej to złożone zagadnienie z teorii hipotez statystycznych.
% Brak możliwości stwierdzenia co na obrazie się znajduje, stawia sens takiego obrazowania pod znakiem zapytania.
% Poco nam obraz w 4K na, którym można zobaczyć wyraźne plamy niczego.

% Wartość diagnostyczną można określić na podstawie następujących parametrów
% \begin{itemize}
%     \item Jakości obrazu

%           Parametry jakościowe obrazów są szczegółowo opisane w sekcji \ref{sec:image-quality}

%     \item Warunków obserwacji obrazu

%           W brew pozorom warunki obserwacji mają kluczowe znaczenie dla wartości diagnostycznej.
%           Jeżeli będziemy mieli dobry obraz, który wyświetlimy na budżetowym monitorze RGB, który w rzeczywistości posiada 6-bitowe kanały RGB i tworzy odcienie za pomocą techniki dithering'u, to niewiele zobaczymy.

%     \item wiarygodności diagnostycznej obrazów

%     \item charakterystyki pracy lekarza-specjalisty

% \end{itemize}

\subsection{Kontrast}

Jedną z wielu definicji kontrastu jest kontrast Michelsona wyrażony wzorem:
\[\frac{I_{max}-I_{min}}{I_{max}+I_{min}}\]
gdzie $I_{max}$ i $I_{min}$ to najwyższa i najniższa wartość luminancji.

\subsection{Rozdzielczość}

\subsubsection*{Przestrzenna}

\par
Rozdzielczość przestrzenna obrazu to najmniejsza odległość między dwoma punktami obrazu, które można rozróżnić.
Jest ona silnie związana z kontrastem obrazu za pomocą funkcji przenoszenia modulacji (MTF –-- Modulation Transfer Function).
Jest to krzywa ukazująca degradację kontrastowości w miarę zwiększania częstotliwości przestrzennej okresowego wzorca.
Funkcję MTF można wyznaczyć używając rozbieżnych tarcz rozdzielczości przestrzennej lub, w pewnych warunkach, przy pomocy norm wielopręcikowych.
W radiografii rozdzielczość określa się zazwyczaj jako liczbę równoległych linii, czarnych i białych, które można rozróżnić ma 1 milimetrze obrazu(paralinie na milimetr).

\par
Rozdzielczość przestrzenna jest zależna od kontrastu obrazu.
Zależność ta jest inna dla każdej techniki.

\subsubsection*{Czasowa}

Każdy pomiar wymaga pewnego czasu pobierania danych.
W nie których przypadkach ważna jest również zmiany zachodzące w organizmie w czasie wykonywania badania.
Rozdzielczość czasowa, jest istotna w obrazach dynamicznych, np. angioMR.
Kiedy mamy pomiar dokonywany w określonym czasie i ustalone są markery czasowe.
Rozdzielczość czasowa jest definiowana jako odległość w czasie od dwóch klatek obrazowania.

\subsection{Stosunek sygnału do szumu (SNR)}

Rodzaj i poziom szumu zależy od techniki obrazowania.
Stosunek sygnału do szumu ma decydujący wpływ na widoczności obiektów, kontrast oraz percepcję szczegółów w obrazie.

\subsection{Poziom artefaktów}

Artefakty to zjawiska fałszujące obraz poprzez tworzenie struktur w obrazie, nie istniejących w rzeczywistości.
Jest to problem występujący w różnych technikach obrazowania.
Najbardziej znanymi artefaktami są np. w badaniu USG tak zwany warkocz komety w przypadku obiektów o wysokiej różnicy impedancji w stosunku do otoczenia.

\subsection{Poziom zniekształceń przestrzennych}

Zniekształcenia przestrzenne powstają w wyniku geometrycznego ułożenia i kształtu obiektu badanego oraz aparatu pomiarowego.
Przykładem takiego zniekształcenia mogą być różne powiększenia obiektów zależne od głębokości ich ułożenia w USG, zmiana pozycji pacjenta(przez ruchy klatki piersiowej w czasie badania), czy deformacja obrazu spowodowana zmianami rozkładu pola magnetycznego przez metalowe obiekty w znajdujące się w tym samym pomieszczeniu w przypadku badań MRI.

