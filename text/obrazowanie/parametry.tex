\subsection{Podstawowe parametry obrazu cyfrowego}

\par
Każdy obraz cyfrowy jest matrycą pikseli o ustalonych rozmiarach.
Tak samo obrazy DICOM są matrycami pikseli, posiadającymi wysokość(zapisaną w \dicomtag{Rows}{0028}{0010}) oraz szerokość(zapisaną w \dicomtag{Columns}{0028}{0011}).

\par
Istnieje jeszcze tag \dicomtag{PhotometricInterpretation}{0028}{0004} informujące jak mamy interpretować piksel pod względem fotometrycznym.
Standard DICOM definiuje następujące wartości tego tagu:
\begin{itemize}
    \item \dataword{MONOCHROME1} i \dataword{MONOCHROME2} ---

    \item \dataword{PALETTE COLOR} --- wartość piksela jest używana jako indeks w każdej z tabel wyszukiwania kolorów palety czerwonej, niebieskiej i zielonej.
          Palety mają swoje własne tagi.
          Wartość raczej nie spotykana i rzadka.

    \item \dataword{RGB} --- piksel trzy-kanałowy, czerwony, zielony i niebieski

    \item \dataword{HSV} \fromEng{Hue Saturation Value} --- model opisu przestrzeni barw zaproponowany w 1978 roku przez Alveya Raya Smitha.
          Wartość wycofana.

    \item \dataword{ARGB} --- kolor RGB z dodatkowym kanałem przezroczystości.
          Wartość wycofana.

    \item \dataword{CMYK} --- zestaw czterech podstawowych kolorów farb drukarskich stosowanych powszechnie w druku wielobarwnym w poligrafii: cyjan, magenta, żółty, czarny.
          Wartość wycofana.

    \item \dataword{YBR\_FULL} --- Kolor

          Dodatkowo standard zdefiniował pochodne YBR: \dataword{YBR\_FULL\_422}, \dataword{YBR\_PARTIAL\_422}, \dataword{YBR\_PARTIAL\_420}, \dataword{YBR\_ICT}, \dataword{YBR\_RCT}, ale wszystkie są już wycofane.
\end{itemize}

\par
Kwantyzacja obrazu, czyli informacja mówiąca o tym ile poziomów jest na obrazie jest zapisana w czterech tagach:
\begin{itemize}
    \item \dicomtag{BitsAllocated}{0028}{0100} --- informuje na jak wiele bitów zostało zaalokowancyh do zapisany jedego piksela
    \item \dicomtag{BitsStored}{0028}{0101} --- informuje jak wiele bitów z zaalokowanych posiada wartość piksela
    \item \dicomtag{HighBit}{0028}{0102} --- informuje gdzie znajduje się najstarszy bit
    \item \dicomtag{PixelRepresentation}{0028}{0103} --- informuje czy poziomu są ze znakiem czy bez
\end{itemize}

\par
Obraz DICOM również zawiera w sobie informacje o próbkowaniu, ale z uwagi, że próbkowanie wygląda inaczej w każdej technice, dlatego standard posada oddzielne tagi informujące o próbkowania dla każdej techniki.
Próbkowanie poszczególnych technik opisałem w sekcji \ref{sec:basic-imaging-technics}, przytaczanie tagów dla każdej techniki jest bezcelowe.

\subsection{Wartość diagnostyczna obrazu}

W obrazowaniu medycznym chodzi o wyciągnięcie wniosków z obrazów i postawienie diagnozy.
Jest to kluczowy element obrazowania.
Ocena wartości diagnostycznej to złożone zagadnienie z teorii hipotez statystycznych.
Brak możliwości stwierdzenia co na obrazie się znajduje, stawia sens takiego obrazowania pod znakiem zapytania.
Poco nam obraz w 4K na, którym można zobaczyć wyraźne plamy niczego.

Wartość diagnostyczną można określić na podstawie następujących parametrów
\begin{itemize}
    \item Jakości obrazu

          Parametry jakościowe obrazów są szczegółowo opisane w sekcji \ref{sec:image-quality}

    \item Warunków obserwacji obrazu

          W brew pozorom warunki obserwacji mają kluczowe znaczenie dla wartości diagnostycznej.
          Jeżeli będziemy mieli dobry obraz, który wyświetlimy na budżetowym monitorze RGB, który w rzeczywistości posiada 6-bitowe kanały RGB i tworzy odcienie za pomocą techniki dithering'u, to niewiele zobaczymy.

    \item wiarygodności diagnostycznej obrazów

    \item charakterystyki pracy lekarza-specjalisty

\end{itemize}

\subsection{Rozdzielczość}

\subsubsection{Przestrzenna}

\par
Rozdzielczość przestrzenna obrazu to najmniejsza odległość między dwoma punktami obrazu, które można rozróżnić.
W radiografii rozdzielczość określa się zazwyczaj jako liczbę równoległych linii, czarnych i białych, które można rozróżnić ma 1 milimetrze obrazu(paralinie na milimetr).

\par
Rozdzielczość przestrzenna jest zależna od kontrastu obrazu.
Jednakże ta zależność jest troche inna dla każdej techniki.

\subsubsection{Czasowa}

Każdy pomiar wymaga pewnego czasu pobierania danych, ale w nie których przypadkach interesuje nas zmiana w czasie.
Rozdzielczość czasowa pojawia się w obrazach dynamicznych kiedy mamy pomiar w czasie i ustalone markery czasowe.
Jest definiowana jako odległość w czasie od dwóch klatek obrazowania.

\subsection{Jakość obrazów}
\label{sec:image-quality}

\begin{itemize}
    \item Kontrast

          ???

          kontrast mikelsona

          max-min / max+min

          amplidua sinusoidy do wartości średniej


    \item Stosunek sygnału do szumu (SNR)

          Rodzaj i poziom szumu zależy od techniki obrazowania.
          Stosunek sygnału ma decydujący wpływ na widoczności obiektów, kontrast oraz percepcję szczegółów w obrazie.

    \item Poziom artefaktów

          Artefakty to zjawiska fałszujące obraz poprzez tworzenie nie istniejących struktur w obrazie.
          Jest to problemem występujący w różnych technikach obrazowania.
          Najbardziej widocznymi artefaktami są warkocz komety i odbicie zwierciadlane w obrazach USG.

    \item Poziom zniekształceń przestrzennych

          Zniekształcenia przestrzenne powstają w wyniku geometrycznego ułożenia i kształtu obiektu badanego i aparat pomiarowego.
          Przykładem takiego zniekształcenia mogą być różne powiększenia obiektów zależne od głębokości ich ułożenia w USG, zmiana pozycji pacjenta(przez ruchy klatki piersiowej w czasie badania), czy deformacja obrazu spowodowana zmianami rozkładu pola magnetycznego przez metalowe obiekty w znajdujące się w tym samym pomieszczeniu, co MRI.

\end{itemize}