
Posiadanie wielu obrazów wiąże się z potrzebą ich przeglądania i porównywania.
Należy, więc posiadać jakieś narzędzie do wyświetlenia w sposób poprawny, najlepiej jednym i tym samym programem.

\subsection{Przeglądarki obrazów}

Przeglądarki obrazów to programy należące do kategorii przeglądarki plików.
Zwykłe przeglądarki obrazów takich jak jpg, png lub gif wyświetlają obraz w takiej postaci jakiej jest zapisany, oczywiście najpierw przeprowadzają dekompresje obrazu.
W przypadku obrazów medycznych najczęściej nie mammy do czynienia z danymi reprezentującymi kolory w spektrum światła widzialnego.
Przeglądarka obrazów DICOM musi wygenerować kolorowy obraz z danych na podstawie parametrów obrazu.


\subsection{Funkcje przeglądarki obrazów}

\subsubsection{Podstawowe operacje na obrazie}

\subsubsection{Analiza parametrów w celu lepszej informacji}

\subsubsection{Generowanie obrazów woliumetrcznych}

\subsubsection{Analiza i przetwarznie danych}

- mierzenie

- 

\subsubsection{Edycja danych}

- dodawanie obiektów

- rysowanie

- edycja parametrów


\subsection{Kryteria porównywania przeglądarek}

Trudno jest porównywać coś tak złożonego jak przeglądarka obrazów medycznych, nie można jednoznacznie powiedzieć, że jedna jest lepsza od drugiej.
W celu porównań wyróżniono 26 kryteriów do porównywania przeglądarek w postaci „tak” lub „nie”, podzielonych na 5 grup, platformy, interfejsu, wsparcie, obrazowanie dwu i trój wymiarowego.
Kryteria te w jasny sposób pozwalają na ocenę praktycznych aspektów użytkowania przeglądarki.

\subsection{Możliwości mojej przeglądarki}

Moją przeglądarkę nazwałem SokarXD