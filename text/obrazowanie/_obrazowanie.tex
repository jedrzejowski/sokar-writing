
\section{Obrazowe techniki diagnostyczne}
Istnieje wiele technik obrazowania wykorzystujące różne zjawiska fizyczne zachodzące w materii.
Podstawowe techniki obrazowania medycznego to:
\label{sec:basic-imaging-technics}
\begin{itemize}
    \item Radiografia - RTG

    Najstarsza i najbardziej rozpoznawalna technika obrazowania.
    Pierwsze zdjęcie analogowe zostało wykonane przez Röntgena w 1896 roku.
    Polega na przepuszczeniu przez obiekt badany promieniowania, a następnie detekcji tego promieniowania za obiektem badanym.
    W praktyce rejestrujemy współczynnik osłabienia promieniowania rentgenowskiego przez badany obiekt.
    Wyróżniamy dwa typu radiografii: analogowy i cyfrowy.
    Radiografia analogowa odchodzi powoli w zapomnienie.
    W radiografii cyfrowej obrazowana jest ilość promieniowania X przenikające przez badany obiekt.
    Kontrast zależy od położenia obiektu między źródłem a detektorem (położenie optymalne), napięcie anodowe, filtracja, grubość okładek wzmacniających.
    Rozdzielczość zależy od rozdzielczości detektora i rozmiaru ogniska lampy.

    W standardzie DICOM radiografia cyfrowa jest oznaczana jako \quotett{RT}.

    \item Tomografia rentgenowska - CT - Computer Tomography
    
    Agregacja w tomografii komputerowej jest podobna do badania RTG, ale w CT wykonujemy wiele pomiarów w różnych pozycjach względem obiektu badanego i pod różnym kontem.
    Następnie z tych pomiarów tworzymy obraz przez zastosowanie odpowiednich algorytmów tworzących obraz.
    Rejestrujemy współczynnik osłabienia promieniowania rentgenowskiego przez badany obiekt.
    Kontrast zależy od rozmiarów szczegółów badanego obiektu, napięcie anodowe, przyłożone masy (prąd katodowy i czas akwizycji).
    Rozdzielczość zależy od geometrii pomiaru, rozmiaru ogniska lampy rentgenowskiej, przestrzenna rozdzielczość matrycy detektora, liczby detektorów, dyskretyzację i filtru rekonstrukcyjnego.

    W standardzie DICOM obraz ultrasonograficzny jest oznaczana jako \quotett{CT}.

    \item Obrazowanie metodą rezonansu magnetycznego - MRI

    Sposób tworzenie obrazu MRI jest wysoce skomplikowanym procesem i ciężko opisać go w kilku zdaniach.
    Obrazowana jest sumaryczna gęstość atomów wodoru (protonów) w badanym obiekcie.
    Kontrast zależy od gęstości protonów, czasu relaksacji podłużnej i poprzecznej, prędkości przepływu płynu.
    Rozdzielczość zależy od parametrów skanera (rozmiar woksela).
    
    W standardzie DICOM obraz rezonansu magnetycznego jest oznaczana jako \quotett{MR}.
    
    \item Ultrasonografia
    
    Jest to badanie, które wszyscy kojarzą z badaniem płodu podczas ciąży z obrazem w kształcie łuku na, którym nic nie widać.
    Badanie ultrasonograficzne polega na wygenerowaniu fali akustycznej o wysokich częstotliwości, a następnie wprowadzeniu jej do ciała pacjenta.
    Następnie nasłuchuje się echa po tej fali.
    Obrazowana jest odbita fala ultradźwiękowa, osłabienia po odbiciach, zmienna częstotliwość i opóźnienie w czasie.
    Kontrast zależy od częstotliwości fali, głębokości badanego obiektu, ilości piezoelektryków w głowicy, obrazowanej struktury.
    Rozdzielczość zależy od czasu trwania impulsu zaburzenia oraz od szerokości wiązki ultradźwiękowej (powierzchnia czynna przetworników).

    W standardzie DICOM obraz ultrasonograficzny jest oznaczana jako \quotett{US}.

    \item Scyntygrafia
    
    Obrazowa technika diagnostyczna z gałęzi medycyny nuklearnej.
    Polega na wprowadzenia do organizmu ciał obcych, środków chemicznych zwanymi również radiofarmaceutykami, charakteryzującymi się krótkim czasie rozpadu i powinowactwem chemicznym z badanymi organami.
    Następnie wykrywanie rozpadów zachodzących w ciele poprzez rejestracje promieniowania wytwarzanego podczas rozpadu, a następnie przedstawienie to w formie graficznej.
    Kontrast zależy od długości trwania pomiaru, oraz od ilości wstrzykniętego radiofarmaceutyka.
    Rozdzielczość zależy od ułożenia i możliwości rozdzielczej kamer scyntylacyjnych, zwanymi także scyntykamerami, gammakamerami lub kamerami Angera.

    W standardzie DICOM obraz scyntygraficzny jest oznaczana jako \quotett{NM}.

    Radiofarmaceutyki to związki chemiczne zawierające radioizotop.

    \item Tomografia SPECT
    
    Technika obrazowania  z gałęzi medycyny nuklearnej. w której rejestruje się promieniowanie powstające rozpadu gamma.
    Źródłem promieniowania(fotonów) jest podana pacjentowi radiofarmaceutyk, ulegająca rozpadowi gamma.
    Rejestrujemy fotony powstające podczas anihilacji pozytonów.
    Kontrast zależy od wydajności detektorów, odległość detektora od obiektu oraz położenie obiektu.
    Na rozdzielczość ma wpływ przestrzenna rozdzielczość matrycy detektora, liczby detektorów.

    W standardzie DICOM obraz ultrasonograficzny jest oznaczana jako \quotett{PT}.

    \item Tomografii PET
    
    Technika obrazowania  z gałęzi medycyny nuklearnej. w której rejestruje się promieniowanie powstające podczas anihilacji pozytonów (antyelektronów).
    Źródłem promieniowania(pozytonów) jest podana pacjentowi substancja promieniotwórcza, ulegająca rozpadowi beta plus
    Rejestrujemy fotony powstające podczas anihilacji pozytonów.
    Kontrast zależy od wydajności detektorów, odległość detektora od obiektu oraz położenie obiektu.
    Na rozdzielczość ma wpływ przestrzenna rozdzielczość matrycy detektora, liczby detektorów.

    W standardzie DICOM obraz ultrasonograficzny jest oznaczana jako \quotett{PT}.
    
\end{itemize}

Istnieją też techniki, które są połączeniem kilku innych technik.
Takie jak:
\begin{itemize}
    \item PET-CT, PET/CT - połączenie PET z wielorzędowym tomografem komputerowym
    \item PET-MRI, PET/MRI - połączenie PET z rezonansem magnetycznym
\end{itemize}

Standard DICOM nazywa techniki obrazowania modalnościami(z ang. modality).

\section{Obrazy diagnostyczne}

\subsection{Parametry obrazów}

\subsubsection{Wartość diagnostyczna obrazu}

W obrazowaniu medycznym chodzi o wyciągnięcie wniosków z obrazów i postawienie diagnozy.
Jest to kluczowy element obrazowania.
Brak możliwości stwierdzenia co na obrazie się znajduje, stawia sens takiego obrazowania pod znakiem zapytania.
Poco nam obraz w 4K na, którym można zobaczyć wyraźne plamy niczego.

Wartość diagnostyczną można określić na podstawie następujących parametrów
\begin{itemize}
    \item Jakości obrazu
    
    Parametry jakościowe obrazów są szczegółowo opisane w sekcji \ref{sec:image-quality}

    \item Warunków obserwacji obrazu

    W brew pozorom warunki obserwacji mają kluczowe znaczenie dla wartości diagnostycznej.
    Jeżeli będziemy mieli dobry obraz, który wyświetlimy na budżetowym monitorze RGB, który w rzeczywistości posiada 6-bite kanały RGB i tworzy odcienie za pomocą techniki dithering'u, to niewiele zobaczymy.

    \item wiarygodności diagnostycznej obrazów

    \item charakterystyki pracy lekarza-specjalisty

\end{itemize}

\subsubsection{Jakość obrazów}
\label{sec:image-quality}

\begin{itemize}
    \item Kontrast
    
    ???

    kontrast mikelsona

    max-min / max+min
    
    amplidua sinusoidy do wartości średniej

    \item Rozdzielczość przestrzenna

    Rozdzielczość przestrzenna obrazu to najmniejsza odległość między dwoma punktami obrazu, które można rozróżnić.
    W radiografii rozdzielczość określa się zazwyczaj jako liczbę równoległych linii, czarnych i białych, które można rozróżnić ma 1 milimetrze obrazu(paralinie na milimetr).

    Porównanie zdolności rozdzielczych różnych technik obrazowania:
    \begin{itemize}
        \item scyntygrafia - 
        \item USG - 
        \item MRI -
        \item CT -
        \item radiografia -
    \end{itemize}
    TUTAJ COŚ WPISAĆ

    rozdzielczośc przy przy kontraście

    \item Stosunek sygnału użytecznego do szumu (SNR)

    W obrazach zawsze występuje szum, widoczny w różnych postaciach, na przykład w postaci cyfrowego ziarna.
    Rodzaj i poziom szumu zależy od techniki obrazowania.
    Stosunek sygnału użytecznego ma decydujący wpływ na widoczności obiektów, kontrast oraz percepcję szczegółów w obrazie.

    \item Poziom artefaktów
    
    Artefakty to zjawiska fałszujące obraz poprzez tworzeni nie istniejących struktur w obrazie.
    Problemem występującym w różnych technikach obrazowania.
    Najbardziej widocznymi artefaktami są warkocz komety i odbicie zwierciadlane w obrazach USG.

    \item Poziom zniekształceń przestrzennych
    
    Zniekształcenia przestrzenne powstają w wyniku geometrycznego ułożenia i kształtu obiektu badanego i aparat pomiarowego.
    Przykładem takiego zniekształcenia mogą być różne powiększenia obiektów zależne od głębokości ich ułożenia w USG, zmiana pozycji pacjenta(przez ruchy klatki piersiowej w czasie badani), czy deformacja obrazu spowodowana zmianami rozkładu pola magnetycznego przez metalowe obiekty w znaldujące się w tym samym pomieszczeniu, co MRI.

\end{itemize}

\section{Zapisywanie obrazów i standard DICOM}


Pierwsze tomografy komputerowe przeżyły swój rozkwit w latach siedemdziesiątych ubiegłego wieku.
Spowodowało to, że obrazu medyczne nie były bezpośrednim wynikiem badania, a jedynie wynikiem obróbki danych pomiarowych przez komputer.
Dodatkowo obrazy przedstawiały przekroje, co sprawiły wiele trudności w ich interpretacji personelowi medycznemu.
Zwyczajne pliki graficzne (jak np. jpg, png, gif), nie nadawały się do zapisu takich obrazów, ponieważ zapisywały obraz w spektrum światła widzialnego, a konkretniej w postaci pozwalającej na odtworzenie światła widzialnego.
Natomiast obrazy medyczne sa zapisywanie w spektrum rentgenowskim.
Nie ułatwiał fakt, że każdy producent stosował inne metryki oraz inne oznaczenia swojego sprzętu.

\subsection{Standard DICOM v3.0}

Standard DICOM wersji trzeciej to standard definiujący ujednolicony sposób zapisu i przekazywania danych medycznych reprezentujących lub związanych z obrazami diagnostycznymi w medycynie.
Standard został wydany w 1993 przez dwie agencje ACR (American College of Radiology) i NEMA (National Electrical Manufactures Association).
Wcześniejsze wersje nazywały się ACR/NEMA v1.0, wydana w 1983 roku i ACR/NEMA v2.0, wydana w 1990 roku, stąd wersja trzecia.
Od wydania wersji trzeciej w 1993, standard jest wciąż rozwijany i uzupełniany o nowe elementy.
W obecnej chwili standard DICOM definiuje 81 różnych typów badań.

UWAGA: Za każdym razem kiedy jest odniesienie do obecnego standardu DICOM, w domyśle jest to odsłona 2019a.

\subsection{Sposób zapisu danych w pliku DICOM}

Plik w formacie DICOM przypomina bazę danych z rekordami.
Baza danych nazywa się \keyword{Data Set} i składa się z rekordów, które nazywają się \keyword{Data Element}.

\begin{figure}[!htbp]
    \caption{Wizualizacja ułożenia \keyword{Data Element}(wraz z budową) w \keyword{Data Set}}
    \includegraphics[]{img/dicom-dataelement001.pdf}
    \centering
    \label{fig:dicom-dataelement}
\end{figure}

\subsubsection{Data Element}

\keyword{Data Element} jest rekordem, który przechowuje jakaś jedną informacje o czymś.
Składa się z czterem elementów:

\begin{itemize}

    \item \keyword{Tag} - to unikalny identyfikator, złożony z dwóch liczb: grupy(uint16) i elementu(uint16) grupy.
    Informuje o tym co dany rekord w sobie zawiera.
    W jednym \keyword{Data Set} nie mogą się pojawić dwa \keyword{Data Element} posiadających ten sam \keyword{Tag}
    
    Obiekt reprezentujący \gdcmclass{Tag}.

    Na przykład: jeżeli liczby \keyword{Tag} przyjmą wartości odpowiednio wartość $0010_{16}$ i $0010_{16}$ to oznacza, że jest to tag \dicomtag{PatientName}{0010}{0010}, czyli zwiera w sobie parametr zawierają nazwę pacjenta.

    Dokładne omówienie \keyword{Tag}-ów znajduje się w sekcji \ref{sec:dicom-tag}.

    \item \keyword{Value Representation}, w skrócie \keyword{VR} – to dwa bajty w postaci tekstu, informujący o formacie w jaki parametr został zapisany.
    
    Dokładne omówienie \keyword{VR}-ów znajduje się w sekcji \ref{sec:dicom-vr}.

    \item \keyword{Value Length}, w skrócie \keyword{VL} - 32-bitowa lub 16-bitowa liczba nieoznaczona, która informuj o długości pola danych(\keyword{Value Field}).
    
    Wartość \keyword{VL} zwykle jest liczbą parzystą.
    Standard DICOM zakłada, że wszystkie dane powinny być dopełniane do parzystej ilości bajtów.
    
    \item \keyword{Value Field} (opcjonalne) - pole z parametrem o długości VL.
    
\end{itemize}

Wizualizacja budowy \keyword{Data Element} jest na rysunku \ref{fig:dicom-dataelement}.

\subsubsection{Tag}
\label{sec:dicom-tag}

Znacznik


\subsubsection{VR - Value Representation}
\label{sec:dicom-vr}

Reprezencaja wartości danej

Obiekt używany do przechowywania taga to \gdcmclass{VR}.
Na przykład: Decimal String, w skrócie DS, oznacza liczbę zapisaną za pomocą teksu.
Czasami to pole może być puste, wtedy należy się odnieść do VR przypisanego do taga, który określa standard.

\subsection{DICOMDIR}

coś o dicomdir

\section{Wyświetlanie obrazów}

Posiadanie wielu obrazów wiąże się z potrzebą ich przeglądania i porównywania.
Należy, więc posiadać jakieś narzędzie do wyświetlenia w sposób poprawny, najlepiej jednym i tym samym programem.

\subsection{Przeglądarki obrazów}

Przeglądarki obrazów to programy należące do kategorii przeglądarki plików.
Zwykłe przeglądarki obrazów takich jak jpg, png lub gif wyświetlają obraz w takiej postaci jakiej jest zapisany, oczywiście najpierw przeprowadzają dekompresje obrazu.
W przypadku obrazów medycznych najczęściej nie mamy do czynienia z danymi reprezentującymi kolory w spektrum światła widzialnego.
Przeglądarka obrazów DICOM musi wygenerować kolorowy obraz z danych na podstawie parametrów obrazu.

\subsection{Porównanie przeglądarek obrazów}

Trudno jest porównywać coś tak złożonego jak przeglądarka obrazów medycznych, nie można jednoznacznie powiedzieć, że jedna jest lepsza od drugiej. W celu porównań wyróżniono 26 kryteriów do porównywania przeglądarek w postaci „tak” lub „nie”, podzielonych na 5 grup, platformy, interfejsu, wsparcie, obrazowanie dwu i trój wymiarowego.
Kryteria te w jasny sposób pozwalają na ocenę praktycznych aspektów użytkowania przeglądarki.

\paragraph{Platforma}

Samodzielność, aplikacje samodzielne są zaprojektowane tak, aby nie wymagały żadnego dodatkowego sprzętu fizycznego bądź infrastruktury do poprawnego działania(np. systemu Windows oraz serwisów przez niego dostarczanych).
Rozwiązania sieciowe, określają czy aplikacja jest usługą sieciową i można z przeglądarki korzystać jak ze strony WWW.
Wieloplatformowość, możliwość uruchomienia ich na różnych systemach operacyjnych Linux/MacOS/Windows
Rozwiązania mobilne, możliwość używania na urządzeniach mobilnych takich jak telefon.

\paragraph{Interfejs}

Przeglądarka powinna mieć możliwość komunikacji z interfejsami innych systemów.
Podstawowe interfejsy sieciowe to: C-STORE SCP DICOM C-STORE, C-STORE SCU, Query-Retrieve, WADO, Parameter Transfer.

\paragraph{Wsparcie techniczne}

Dokumentacja, dostępność pisemnej dokumentacji oprogramowania (np. podręczniki lub strony internetowej).
Wsparcie przez pocztę internetową, możliwość porozumienia się z twórcą lub opiekunem oprogramowania.
Forum, możliwość pytania się społeczności o opinie i ich wymiana.
Wiki, strona internetowa w formacie Wikipedii dostępna dla użytkownika.

\paragraph{Obrazowanie dwu-wymiarowe}

Przewijanie(\fromEng{scroll}), proces wyświetlania obrazów, można poprawić dzięki zmniejszeniu interakcji z klawiaturą oraz myszką. Można to osiągnąć na przykład, oferując możliwość przejścia do następnego lub poprzedniego obrazu przez przesunięcie kółkiem myszy lub używając przycisków góra/dół na klawiaturze.
Metadane, przeglądania powinna obejmować analizowanie i wyświetlanie metadanych obiektów DICOM, powinna obejmować wyświetlanie rozdzielczości obrazu, badanie (np. identyfikator podmiotu) oraz znaczniki DICOM specyficzne dla dostawcy (np. specjalne ustawienie urządzenia rejestrującego).
Warstwa informacyjna, najważniejsze informacje powinny powinny być wizualizowane w oknie wyświetlacza jako nakładka na obraz.
Na przykład aktualna pozycja lub nazwa podmiotu wykonującego badanie.
Okienkowanie (okna cyfrowe), sposób zamiany danych na skale szarości, okienkowanie jest opisane w sekcji \ref{sec:windowing}.
Pseudo-kolorowanie obrazu, tabele (LUT, \fromEng{LookUpTable}) odwzorowujące szare wartości obrazu na pseudo-kolory, poprawiaja one czytelność obrazu.
Histogram, histogramy wizualizują wystąpienia i rozkład wartości kolorów na obrazach, pozwalają opisywać istotne cechy obrazu
Wymiarowanie, możliwości rysowania bądź zaznaczania linii lub innych kształtów do analizy i wyznaczania odległości w jednostkach długości na obrazie.
Jest to możliwe gdyż nagłówki pliku DICOM zawierają parametry sprzętowe urządzenia (np. ilość pikseli na centymetr).
Adnotacje(opisy), które były wytworzone przez personel medyczny powinny być zapisywane w odpowiedni sposób w pliku.

\paragraph{Obrazowanie trój-wymiarowe}

Rekonstrukcja wtórna, zwykle dane dotyczące objętości medycznej są gromadzone wzdłuż jednej osi ciała (np. poprzecznej).
W wielu przypadkach ważne jest przeglądanie danych w innych kierunkach (np. strzałkowych lub czołowych), aby poprawić wizualizację niektórych struktur.
W tym celu należy zapewnić funkcjonalność rekonstrukcji osi pomocniczej na podstawie kierunku pierwotnego.
Plastry objętości kostki(\fromEng{Slice Cube Volume}), przekroje mogą być lepiej wyświetlane w określonej pozycji.
Funkcjonalność kostki plasterka umożliwia niezależną regulację położenia różnych osi wycinków (np. poprzecznych, strzałkowych lub czołowych) w modelu objętościowym.
Podczas tego przekroje są pokazane w osobnym oknie.
Renderowanie objętościowe – dane obrazu 3D są bezpośrednio wizualizowane jako objętość.
Użytkownik może wchodzić w interakcje z woluminem poprzez obracanie lub skalowanie.
Transfer Function(nie znam polskiej nazwy), służy do odwzorowania wartości szarości obrazów wokseli na wartości krycia typów tkanek (np. kości). Struktury obrazu pasujące do wzorców szarych wartości są podświetlone. Niewykorzystane szare wartości są wyświetlane jako
przezroczyste. Specyficzne struktury stają się lepiej widoczne.
Generowanie powierzchni, dzięki różnym algorytmom można generować powierzchnie w postaci wokselów. Reprezentacje powierzchni można również zastosować do poprawy wizualizacji niektórych struktur obrazu.

\subsection{Funkcje przeglądarki obrazów}

\subsubsection{Podstawowe operacje na obrazie}

\begin{itemize}
    \item Skalowaniu lub powiększenie.
          Możliwość powiększenia lub zmniejszenia wyświetlanego obrazu o pewny współczynnik skalujący.

    \item Przesuwanie(\fromEng{pan})
          Możliwość przesuwania obrazu o dowolny wektor.
          Przydatne gdy powiększymy obraz do takiego stopnia, że nie będzie mieścił się na ekranie lub w okienku programu.

    \item Lupa, skalowanie miejscowe
          Możliwość miejscowego powiększenia obrazu.
          Przykład użycia takiego narzędzia znajduje się na rysunku \ref{fig:wyswietlanie001}.

          \begin{figure}[!htbp]
              \caption{Przykład narzędzia Lupa w przeglądarce \href{https://www.softneta.com/products/meddream-dicom-viewer/}{MedDream DICOM Viewer}.}
              \includegraphics[width=\textwidth]{img/wyswietlanie001.png}
              \centering
              \label{fig:wyswietlanie001}
          \end{figure}

    \item Rotacja i odbicia lustrzane
          Możliwość obrócenia obrazu o zadany kąt.
          Oraz możliwość odbicia lustrzanego obrazu w dwóch osiach X i Y.

\end{itemize}

\subsubsection{Analiza parametrów w celu lepszej informacji}

\begin{itemize}
    \item Okienkowanie
    \item Maski lub nakładki(\fromEng{overlay})
\end{itemize}

\subsubsection{Generowanie obrazów woliumetrcznych}

Jeżeli mamy do dyspozycji wiele obrazów tomograficznych o znanych parametrach

\subsubsection{Analiza i przetwarznie danych}

\begin{itemize}
    \item Histogram
          Możliwość wygenerowania histogramu obrazu.

          Histogram to wykres przedstawiający dystrybucje wartości numerycznych obrazu.


          \begin{figure}[!htbp]
            \caption{Przykładowy histogram.}
            \includegraphics[width=\textwidth]{img/wyswietlanie002.svg}
            \centering
            \label{fig:wyswietlanie002}
        \end{figure}

    \item Mierzenie obrazu, wykonywanie pomiarów
          Możliwość zmierzenia odległości pomiędzy dwoma punktami przez lekarza lub zmierzenia wielkości/pola zadanego kształtu.11
\end{itemize}

\subsubsection{Edycja danych}

\begin{itemize}
    \item Dodawanie nowych obiektów.
          Możliwość rysowania, dodawania figur geometrycnych lub tekstu przez lekarza i możliwość zapisu tych informacji w pliku DICOM.

    \item Mierzenie obrazu, wykonywanie pomiarów.
\end{itemize}


- dodawanie obiektów

- rysowanie

- edycja parametrów


\subsection{Możliwości mojej przeglądarki}

Po analizie możliwości przeglądarek plików DICOM dostępnych na rynku postanowiłem zaimplementować następujące komponenty w mojej przeglądarce:

\begin{itemize}
    \item Przesuwanie(\fromEng{pan})

    \item Skalowaniu lub powiększenie

    \item Rotacja i odbicia lustrzane

    \item Okienkowanie

    \item Wczytywanie wielu plików na raz
\end{itemize}

Moją przeglądarkę nazwałem Sokar.

