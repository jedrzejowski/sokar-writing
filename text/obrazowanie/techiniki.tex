Istnieje wiele technik obrazowania wykorzystujące różne zjawiska fizyczne zachodzące w materii.
Podstawowe techniki obrazowania medycznego to:
\label{sec:basic-imaging-technics}
\begin{itemize}
    \item Radiografia - RTG

    Najstarsza i najbardziej rozpoznawalna technika obrazowania.
    Pierwsze zdjęcie analogowe zostało wykonane przez Röntgena w 1896 roku.
    Polega na przepuszczeniu przez obiekt badany promieniowania, a następnie detekcji tego promieniowania za obiektem badanym.
    W praktyce rejestrujemy współczynnik osłabienia promieniowania rentgenowskiego przez badany obiekt.
    Wyróżniamy dwa typu radiografii: analogowy i cyfrowy.
    Radiografia analogowa odchodzi powoli w zapomnienie.
    W radiografii cyfrowej obrazowana jest ilość promieniowania X przenikające przez badany obiekt.
    Kontrast zależy od położenia obiektu między źródłem a detektorem (położenie optymalne), napięcie anodowe, filtracja, grubość okładek wzmacniających.
    Rozdzielczość zależy od rozdzielczości detektora i rozmiaru ogniska lampy.

    W standardzie DICOM radiografia cyfrowa jest oznaczana jako \quotett{RT}.

    \item Tomografia rentgenowska - CT - Computer Tomography
    
    Agregacja w tomografii komputerowej jest podobna do badania RTG, ale w CT wykonujemy wiele pomiarów w różnych pozycjach względem obiektu badanego i pod różnym kontem.
    Następnie z tych pomiarów tworzymy obraz przez zastosowanie odpowiednich algorytmów tworzących obraz.
    Rejestrujemy współczynnik osłabienia promieniowania rentgenowskiego przez badany obiekt.
    Kontrast zależy od rozmiarów szczegółów badanego obiektu, napięcie anodowe, przyłożone masy (prąd katodowy i czas akwizycji).
    Rozdzielczość zależy od geometrii pomiaru, rozmiaru ogniska lampy rentgenowskiej, przestrzenna rozdzielczość matrycy detektora, liczby detektorów, dyskretyzację i filtru rekonstrukcyjnego.

    W standardzie DICOM obraz ultrasonograficzny jest oznaczana jako \quotett{CT}.

    \item Obrazowanie metodą rezonansu magnetycznego - MRI

    Sposób tworzenie obrazu MRI jest wysoce skomplikowanym procesem i ciężko opisać go w kilku zdaniach.
    Obrazowana jest sumaryczna gęstość atomów wodoru (protonów) w badanym obiekcie.
    Kontrast zależy od gęstości protonów, czasu relaksacji podłużnej i poprzecznej, prędkości przepływu płynu.
    Rozdzielczość zależy od parametrów skanera (rozmiar woksela).
    
    W standardzie DICOM obraz rezonansu magnetycznego jest oznaczana jako \quotett{MR}.
    
    \item Ultrasonografia
    
    Jest to badanie, które wszyscy kojarzą z badaniem płodu podczas ciąży z obrazem w kształcie łuku na, którym nic nie widać.
    Badanie ultrasonograficzne polega na wygenerowaniu fali akustycznej o wysokich częstotliwości, a następnie wprowadzeniu jej do ciała pacjenta.
    Następnie nasłuchuje się echa po tej fali.
    Obrazowana jest odbita fala ultradźwiękowa, osłabienia po odbiciach, zmienna częstotliwość i opóźnienie w czasie.
    Kontrast zależy od częstotliwości fali, głębokości badanego obiektu, ilości piezoelektryków w głowicy, obrazowanej struktury.
    Rozdzielczość zależy od czasu trwania impulsu zaburzenia oraz od szerokości wiązki ultradźwiękowej (powierzchnia czynna przetworników).

    W standardzie DICOM obraz ultrasonograficzny jest oznaczana jako \quotett{US}.

    \item Scyntygrafia
    
    Obrazowa technika diagnostyczna z gałęzi medycyny nuklearnej.
    Polega na wprowadzenia do organizmu ciał obcych, środków chemicznych zwanymi również radiofarmaceutykami, charakteryzującymi się krótkim czasie rozpadu i powinowactwem chemicznym z badanymi organami.
    Następnie wykrywanie rozpadów zachodzących w ciele poprzez rejestracje promieniowania wytwarzanego podczas rozpadu, a następnie przedstawienie to w formie graficznej.
    Kontrast zależy od długości trwania pomiaru, oraz od ilości wstrzykniętego radiofarmaceutyka.
    Rozdzielczość zależy od ułożenia i możliwości rozdzielczej kamer scyntylacyjnych, zwanymi także scyntykamerami, gammakamerami lub kamerami Angera.

    W standardzie DICOM obraz scyntygraficzny jest oznaczana jako \quotett{NM}.

    Radiofarmaceutyki to związki chemiczne zawierające radioizotop.

    \item Tomografia SPECT
    
    Technika obrazowania  z gałęzi medycyny nuklearnej. w której rejestruje się promieniowanie powstające rozpadu gamma.
    Źródłem promieniowania(fotonów) jest podana pacjentowi radiofarmaceutyk, ulegająca rozpadowi gamma.
    Rejestrujemy fotony powstające podczas anihilacji pozytonów.
    Kontrast zależy od wydajności detektorów, odległość detektora od obiektu oraz położenie obiektu.
    Na rozdzielczość ma wpływ przestrzenna rozdzielczość matrycy detektora, liczby detektorów.

    W standardzie DICOM obraz ultrasonograficzny jest oznaczana jako \quotett{PT}.

    \item Tomografii PET
    
    Technika obrazowania  z gałęzi medycyny nuklearnej. w której rejestruje się promieniowanie powstające podczas anihilacji pozytonów (antyelektronów).
    Źródłem promieniowania(pozytonów) jest podana pacjentowi substancja promieniotwórcza, ulegająca rozpadowi beta plus
    Rejestrujemy fotony powstające podczas anihilacji pozytonów.
    Kontrast zależy od wydajności detektorów, odległość detektora od obiektu oraz położenie obiektu.
    Na rozdzielczość ma wpływ przestrzenna rozdzielczość matrycy detektora, liczby detektorów.

    W standardzie DICOM obraz ultrasonograficzny jest oznaczana jako \quotett{PT}.
    
\end{itemize}

Istnieją też techniki, które są połączeniem kilku innych technik.
Takie jak:
\begin{itemize}
    \item PET-CT, PET/CT - połączenie PET z wielorzędowym tomografem komputerowym
    \item PET-MRI, PET/MRI - połączenie PET z rezonansem magnetycznym
\end{itemize}

Standard DICOM nazywa techniki obrazowania modalnościami(z ang. modality).