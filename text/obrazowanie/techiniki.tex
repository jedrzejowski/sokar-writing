Istnieje wiele technik obrazowania wykorzystujące różne zjawiska fizyczne zachodzące w materii.
Podstawowe techniki obrazowania medycznego to:
\label{sec:basic-imaging-technics}
\begin{itemize}
    \item Radiografia --- RTG

          Radiografia to najstarsza i najbardziej rozpoznawalna technika obrazowania.
          Pierwsze zdjęcie analogowe zostało wykonane przez Röntgena w 1896 roku.
          Polega na transmisji promieniowania X przez badany obiekt, a następnie detekcji tego promieniowania za obiektem badanym.
          Promieniowanie za obiektem jest funkcją współczynnika osłabiania promieniowania rentgenowskiego dla materii znajdującej się na drodze tego promieniowania.
          Wyróżniamy dwa typy radiografii: analogową i cyfrową.
          Radiografia analogowa wykorzystująca naświetlanie filmów światłoczułych odchodzi powoli w zapomnienie ze względu na koszt i uciążliwość wywoływania filmów.
          W radiografii cyfrowej do detekcji są wykorzystywane różne typy detektorów.
          Detektory z konwersją bezpośrednią, w których kwanty X konwertowane są na elektrony w grubej warstwie odpowiednio dobranego półprzewodnika (np. selenu).
          Detektory z konwersją pośrednią, w których kwanty X konwertowane są w scyntylatorze na fotony światła widzialnego, które z kolei rejestrowane są przez fotodiody krzemowe.

          W radiografii obrazowana jest ilość promieniowania X przenikającego przez badany obiekt.
          Piksel w obrazie jest uzyskiwany przez zliczanie ilości rozbłysków i reprezentuje współczynnik przenikania promieniowania X, dlatego zdjęcie jest negatywem i w takiej formie zdjęcie jest analizowane przez lekarza.
          Wielkość obrazu zależy od matrycy wliczającej rozbłyski.
          Kontrast zależy od położenia obiektu między źródłem a detektorem (położenie optymalne), od napięcia anodowego, filtracji, grubości okładek wzmacniających.
          Rozdzielczość zależy od rozdzielczości detektora, rozmiaru ogniska lampy, położenia obiektu względem detektora a lampą i wielkości obiektu.
          Miarą rozdzielczości jest liczba rozróżnialnych linii na jednostkę długości.

          W standardzie \DICOM radiografia cyfrowa jest oznaczana jako \enquote{RT}.

    \item Tomografia komputerowa (Computer Tomography --- CT)

          Akwizycja w tomografii komputerowej jest podobna do badania RTG, ale w CT wykonujemy wiele pomiarów w różnych pozycjach względem obiektu badanego i pod różnym kątem.
          W tomografii komputerowej podobnie jak w radiografii wykorzystuje się promieniowanie X do pomiaru projekcji (stąd inna nazwa tomografia rentgenowska).
          W wybranej płaszczyźnie dokonuje się pomiarów projekcji po liniach biegnących pod różnym kątem i w różnych odległościach od badanego obiektu.
          Przekrój obiektu jest rekonstruowany numerycznie na podstawie zmierzonych projekcji wstecznej.

          Obrazowany jest współczynnik przenikalności promieniowania X przez obiekt.
          Wielkość obrazu może być różna i jest zależna od ustawień tomografu, najczęściej jest to 512 na 512 wokseli.
          Piksel obrazu jest uzyskiwany podczas rekonstrukcji obrazu i reprezentuje przenikalność promieniowania X.
          Kontrast i rozdzielczość zależy od tych samych parametrów co w klasycznej radiografii.

          W standardzie \DICOM technika jest oznaczana skrótowcem \enquote{CT}.

    \item Obrazowanie metodą rezonansu magnetycznego --- MRI

          Sposób tworzenie obrazu MRI jest wysoce skomplikowanym procesem, którego szczegółowy opis przekracza zakres niniejszego opracowania.
          Obrazowana jest sumaryczna gęstość atomów wodoru (protonów) w badanym obiekcie.
          W zależności od sekwencji pobudzeń polem elektromagnetycznym, wyróżniamy trzy typy obrazów: PD, T1 i T2.
          Kontrast zależy od gęstości protonów, czasu relaksacji podłużnej i poprzecznej, prędkości przepływu płynu.
          Rozdzielczość zależy od parametrów skanera (rozmiar woksela).

          W standardzie \DICOM modalność rezonansu magnetycznego jest oznaczana jako \enquote{MR}.

    \item Ultrasonografia

          Podczas badania ultrasonograficznego generujemy fale akustyczne o wysokich częstotliwości skierowane w stronę obiektu, następnie rejestrujemy fale odbite.
          Obrazowana jest różnica gęstości poszczególnych warstw znajdujących się w obiekcie.

          Zbieranie danych odbywa się przez cyklicznie wysyłanie i odbieranie fali ultradźwiękowej pod różnymi kątami.
          Z każdego cyklu jest tworzona jedna linia, obraz jest tworzony z wielu lini, które następnie są układane pod różnymi kątami, odpowiadającym ich rzeczywistemu ułożeniu na głowicy.
          Wielkość obrazu jest zależna od algorytmu rekonstrukcji i jest z góry ustawiona przez producenta aparatu.
          Piksel w obrazie nie przedstawia żadnej wartości fizycznej, różnice pomiędzy pikselami definiują umowną różnicę gęstości zależną od aparatu.
          Kontrast zależy od częstotliwości fali, głębokości badanego obiektu, ilości piezoelektryków w głowicy, obrazowanej struktury.
          Rozdzielczość zależy od czasu trwania impulsu zaburzenia oraz od szerokości wiązki ultradźwiękowej (powierzchnia czynna przetworników).

          W standardzie \DICOM obraz ultrasonograficzny jest oznaczana jako \enquote{US}.
          Obrazy dopplerowskie \enquote{Color flow Doppler(CD)} i \enquote{Duplex Doppler(DD)} były kiedyś w standardzie, ale zdecydowano się je wycofać.

    \item Scyntygrafia

          Obrazowa technika diagnostyczna z gałęzi medycyny nuklearnej.
          Polega na wprowadzeniu do organizmu radiofarmaceutyku, czyli związku chemicznego zawierającego izotop.
          Charakteryzuje się on krótkim czasem rozpadu i powinowactwem chemicznym z badanymi organami.
          Wykrywa się rozpad zachodzący w ciele poprzez rejestrację promieniowania wytwarzanego podczas tego rozpadu, a następnie przedstawia się go w formie graficznej.

          Detekcja odbywa się za pomocą scyntylatora, fotopowielacza i układu liniowego sumowania.
          Wielkość obrazu zależy od rozróżnialnych współrzędnych przez detektor.
          Piksel reprezentuje ilość zliczeń na jednej współrzędniej.
          Kontrast zależy od czasu trwania pomiaru, oraz od aktywności wstrzykniętego radiofarmaceutyka.
          Rozdzielczość zależy od możliwości kamer scyntylacyjnych, zwanymi także scyntykamerami, gammakamerami lub kamerami Angera.

          W standardzie \DICOM obraz scyntygraficzny jest oznaczana jako \enquote{NM}.

    \item Tomografia SPECT

          Jest to technika obrazowania  z gałęzi medycyny nuklearnej, w której rejestruje się promieniowanie powstające rozpadu gamma.
          Źródłem promieniowania(fotonów) jest radiofarmaceutyk, którego izotop ulega rozpadowi z emisją promieniowania gamma.
          Kontrast zależy od wydajności detektorów, odległość detektora od obiektu oraz położenie obiektu.
          Na rozdzielczość ma wpływ przestrzenna rozdzielczość matrycy detektora oraz liczba detektorów.

          W standardzie \DICOM obraz jest oznaczany jako \enquote{PT}.

    \item Tomografii PET

          Technika obrazowania  z gałęzi medycyny nuklearnej, w której rejestruje się promieniowanie powstające podczas anihilacji pozytonów (antyelektronów).
          Źródłem promieniowania(pozytonów) jest podana pacjentowi substancja promieniotwórcza, ulegająca rozpadowi beta plus.
          Rejestrujemy fotony powstające podczas anihilacji pozytonów.
          Kontrast zależy od wydajności detektorów, odległości detektora od obiektu oraz położenia obiektu.
          Na rozdzielczość ma wpływ przestrzenna rozdzielczość matrycy detektora oraz liczba detektorów.

          W standardzie \DICOM obraz jest oznaczana jako \enquote{PT}.

\end{itemize}

Istnieją badania łączące w sobie różne techniki, takie jak:
\begin{itemize}
    \item PET-CT, PET/CT --- połączenie PET z wielorzędowym tomografem komputerowym
    \item PET-MRI, PET/MRI --- połączenie PET z rezonansem magnetycznym
\end{itemize}

Standard \DICOM nazywa techniki obrazowania modalnościami \fromEng{modality}.