\par
Celem pracy inżynierskiej było napisanie aplikacji do obsługi obrazów \DICOM w C++ z możliwością kompilacji na wiele platform.
Cel udało się osiągnąć.
Użycie języka C++ umożliwiło wykorzystanie całego potencjału obliczeniowego maszyny.
Zastosowano biblioteki dostępne na różnych platformach: Qt i GDCM, które również zostały napisane w C++, dzięki czemu uzyskano jednolity program napisany w jednym języku.
Zapewniono jednolity sposób kompilacji na platformach przy użyciu narzędzia CMake.
Aplikacja działa w ten sam sposób na wszystkich testowanych platformach: Linux, MacOS i Windows.
Jednolity wygląd aplikacji zapewniła biblioteka Qt, co sprawia, że interfejs aplikacji jest prawie taki sam na każdym systemie.
\par
Zaplanowano i dodano obsługę podstawowych operacji na obrazie ułatwiających jego oglądanie i ocenienie, takich jak: przenoszenie; skalowanie; obrót.
Zaimplementowano pseudokolorowanie obrazów monochromatycznych z możliwością dodawania nowych palet.
Wprowadzono obsługę serii obrazów jako całości, włączając w to przegląd obrazów w serii, animacje, wspólne okna w skali barwnej oraz wspólne przekształcenia macierzowe.
\par
Napotkano problem z biblioteką GDCM w postaci braku możliwości używania plików binarnych dostarczonych przez twórców.
Te pliki binarne zostały skompilowane za pomocą innego kompilatora niż pliki binarne Qt.
Spowodowało to, że typ \stdclass{string}{string/basic_string} z jedenej biblioteki nie jest kompatybilny z \stdclass{string}{string/basic_string} z drugiej biblioteki.
Wynika to z użycia innych interfejsów binarnych aplikacji \fromEng{application binary interface} w różnych kompilatorach.
Problem można rozwiązać kompilując bibliotekę GDCM własnoręcznie.