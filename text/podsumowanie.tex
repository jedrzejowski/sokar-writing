\par
Celem pracy inżynierskiej było napisanie aplikacji do obsługi obrazów \DICOM w C++ z możliwością kompilacji na wiele platform.
Cel udało się osiągnąć.
Zniesienie ograniczeń wirtualizacji kodu rozwiązano użyciem C++ jako język programowania.
Zastosowano biblioteki dostępne na różnych platformach: Qt i GDCM, które również zostały napisane w C++, dzięki czemu uzyskano jednolity program napisany w jednym języku.
Zapewniono jednolity sposób kompilacji na platformach przy użyciu narzędzia CMake.
Dzięki czemu aplikacja działa w ten sam sposób na wszystkich testowanych platformach: Linux, MacOS i Windows.
Jednolity wygląda aplikacji zapewniła biblioteka Qt, dzięki czemu interfejs aplikacji jest prawie taki sam na każdym systemie.
\par
Zaplanowano i dodano obsługę podstawowych operacji na obrazie ułatwiających jego oglądanie i ocenienie, takich jak: przenoszenie; skalowanie; obrót.
Zaimplementowano kolorowe pseudokolorowanie obrazów monochromatycznych z możliwością dodawania nowych palet.
Wprowadzono obsługę serii obrazów jako całości, włączając w to przegląd obrazów w serii, animacje, wspólne okna w skali barwnej oraz wspólne przekształceniami macierzowymi
\par
Napotkano problem z biblioteką GDCM w postaci braku możliwości używania plików binarnych dostarczonych przez twórców.
Te pliki binarne zostały skompilowane za pomocą innego kompilatora niż pliki binarne Qt.
Co sprawia, że typ \stdclass{string}{string/basic_string} z jedenej biblioteki nie jest kompatybilny z \stdclass{string}{string/basic_string} z drugiej biblioteki.
Wynika to z użycia innych ABI w różnych kompilatorach.
Problem można rozwiązać kompilując bibliotekę GDCM własnoręcznie.