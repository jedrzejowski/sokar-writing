\par
Celem pracy było napisanie aplikacji do obsługi obrazów \DICOM w C++ z możliwością kompilacji na wiele platform.
Cel udało się osiągnąć.
Wieloplatformowość uzyskano poprzez użycie bibliotek dostępnych na różnych platformach: Qt i GDCM, które również zostały napisane w C++.
Aplikacja działa w ten sam sposób na wszystkich testowanych platformach: Linux, MacOS i Windows.
Zapewniono jednolity sposób kompilacji na platformach przy użyciu narzędzia CMake.
Zaplanowano i dodano obsługę podstawowych operacji na obrazie ułatwiających jego oglądanie i ocenienie.
\par
Napotkano problem z biblioteką GDCM w postaci braku możliwości używania plików binarnych dostarczonych przez twórców.
Te pliki binarne zostały skompilowane za pomocą innego kompilatora niż pliki binarne Qt.
Co sprawia, że typ \stdclass{string}{string/basic_string} z jedenej biblioteki nie jest kompatybilny z \stdclass{string}{string/basic_string} z drugiej biblioteki.
Wynika to z użycia innych ABI w różnych kompilatorach.
Problem można rozwiązać kompilując bibliotekę GDCM własnoręcznie.