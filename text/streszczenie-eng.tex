\begin{center}
\large \bf
Cross-platform DICOM image viewer in C++
\end{center}

The work consists of six chapters: introduction, diagnostic imaging, libraries and tools, implementation, compilation and summary.
Introduction introduces reader to the subject and purpose of the work.
\par
The second chapter describes the problem related to medical images.
The diagnostic techniques and their basic differences are mentioned.
The parameters of digital images in medicine are presented.
In addition, presentations of medical images are described, and image browsers are explained.
The functions of medical image browsers are discussed.
The storing format of digital medical images is described, standard \DICOM.
\par
The third chapter describes the libraries and tools used when writing engineering work.
The purpose of using the CMake tool and its advantages are explained.
The Qt library, its possibilities, object trees implemented by it and the way of constructing event programming implemented in it have been described.
The selection of the GDCM as a library for handling and loading of \DICOM files has been presented and justified.
\par
The fourth chapter presents the method of work implementation.
The expected range of implemented software functions has been specified.
The graphical user interface and its program functions are described.
The design of the object structure of the program has been explained.
Then, the data structure with the C ++ classes is described in detail.
Where there was a possibility, a UML diagram is attached.
All data processing algorithms are described for better visualization of the image.
\par
The fifth chapter describes the compilation process of the source code.

\bigskip
{\noindent\bf Keywords:} \keywordsEng

\vfill

