\begin{center}
\large \bf
Multi-platform DICOM image viewer in C++
\end{center}


The work consists of six chapters: introduction, diagnostic imaging, libraries and tools, implementation, compilation and summary.
The first chapter is an introduction to the subject and the purpose of the work.
\par
The second chapter describes problems that are related to images in medicine.
Diagnostic techniques and their basic differences are listed in this part.
There are presented the parameters of digital imagines in medicine.
In addition the presentations of the images are described and it is explained what image viewers are.
Their functions are depicted.
The format for recording digital medical images, DICOM standard, is described.
\par
The third chapter describes the libraries and tools that were used to write the engineering work.
The purpose of using the CMake tool and its advantages are explained.
The Qt library, its capabilities, object trees implemented by it and the way of programming construction have been described for the events contained in it.
The choice of the GDCM library was presented and justified as a library for handling  and loading DICOM files.
\par
The fourth chapter presents the way in which the work is implemented.
The expected range of the implemented functions of the software has been determined.
The graphical user interface and its program functions are described.
The design of the object structure of the program has been explained.
The structure of the data, together with the C++ classes is then described in details.
Where it was possibile a UML diagram was included.
All data processing algorithms are described for better  visualization of the images.
\par
The fifth chapter describes the process of compilation of the source code.

\bigskip
{\noindent\bf Keywords:} \keywordsEng

\vfill

