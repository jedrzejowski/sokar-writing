Skórt YBR odpowiada skrótowi YCbCr.
Wartości są ułożone w taki sposób.
\[Y1, B1, R1, Y2, B2, R2, Y3, B3, R3, Y4, B4, R4,  ...\]

Ponieważ wartości te reprezentują kolory, są już w pewnym sensie są obrazem, ale nie można go wyświetlić, ponieważ komputery bazują na kolorach RGB.
Dlatego odpowieni algorytm konwertuje kolor YBR na kolor RGB, iterując po wszystkich wartościach obrazu.

\paragraph{Konwersja koloru YBR na kolor RGB}

YBR albo YCbCr to model przestrzeni kolorów do przechowywania obrazów i wideo.
Wykorzystuje do tego trzy typy danych: Y – składową luminancji, B lub Cb – składową różnicową chrominancji Y-B, stanowiącą różnicę między luminancją a niebieskim, oraz R lub Cr – składową chrominancji Y-R, stanowiącą różnicę między luminancją a czerwonym.
Kolor zielony jest uzyskiwany na podstawie tych trzech wartości.
YBR nie pokrywa w całości RGB, tak jak RGB nie pokrywa YBR.
Posiadają one część wspólną, co uniemożliwia wyświetlenie obrazu w stu procentach bez zniekształceń.
