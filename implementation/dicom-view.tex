
Każda zakładka z obrazem lub obrazami jest implementowana przez klasę \sokarclass{DicomView}.

Interfejs graficzny \sokarclass{DicomView} posiada następujące elementy:
\begin{itemize}
    \item pasek narzędzi znajdujący się na górze - implementowany za pomocą klasy \sokarclass{DicomToolBar}
    \item miejsce na scene z obrazem DICOM na środku - implementowany za pomocą klasy \sokarclass{DicomGraphics}
    \item suwak filmu w dolnej części - implementowany za pomocą klasy \sokarclass{MovieBar}
    \item podgląd miniaturek obrazów w prawej części - implementowany za pomocą klasy \sokarclass{FrameChooser}
\end{itemize}

Dodatkowo posiada obiekt \sokarclass{DicomSceneSet}, który jest zbiorem obrazów opisany w sekcji \ref{sec:scene-sets}.
\sokarclass{DicomView} łączy zdarzenia wysyłane przez wszystkie obiekty.

Poniżej jest opisane zachowanie tych elementów:

\subsection{Elementy interfejsu graficznego}
\subsubsection{\sokarclass{DicomToolBar}}
\subsubsection{\sokarclass{DicomGraphics}}
\subsubsection{\sokarclass{MovieBar}}
\subsubsection{\sokarclass{FrameChooser}}

\subsection{Zależności pomiędzy elementami graficznymi}