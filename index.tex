\documentclass[a4paper,12pt,twoside,openany]{report}

\usepackage{polski}

\usepackage{titlesec}
\usepackage[T1]{fontenc}
\usepackage[utf8]{inputenc}
\usepackage{helvet}


\usepackage[autostyle]{csquotes}
\usepackage{xcolor}

\usepackage[pdftex]{hyperref}
\usepackage{listings}
\usepackage{array}

\usepackage[polish]{babel}
\usepackage{graphicx}
\usepackage{amsmath}

\usepackage{indentfirst}

\usepackage{prmag2017}


\newcommand{\dicomtag}[3] {#1(#2, #3)}

\newcommand{\wiki}[1]{https://pl.wikipedia.org/wiki/Teksel}

\newcommand{\gdcmclass}[1]{gdcm::#1}

\newcommand{\sokarclass}[1]{Sokar::#1}

\newcommand{\qtclass}[1]{Qt::#1}

\newcommand{\dicomvr}[1]{VR #1}

\newenvironment{conditions}
  {\par\vspace{\abovedisplayskip}\noindent\begin{tabular}{>{$}l<{$} @{${}={}$} l}}
  {\end{tabular}\par\vspace{\belowdisplayskip}}

\newcommand{\quotett}[1]{\enquote{\texttt{#1}}}

\newcommand{\cppcode}[1]{{\color{blue}\texttt{#1}}}
\inputencoding{utf8}

\def\utfMaleSign{\includegraphics[height=1em]{utf8char/malesign.pdf}}
\def\utfFemaleSign{\includegraphics[height=1em]{utf8char/femalesign.pdf}}

% Aby komendy nie wychodziły poza margines
% \sloppy 

\title{Wieloplatformowa przeglądarka obrazów DICOM w C++}

\author{Adam Jędrzejowski}
\nrindeksu{277417}

\opiekun{prof. nzw. dr hab. inż. Waldemar Smolik}
\terminwykonania{30 lutego 2019}
\rok{2019}

\miasto{Warszawa}
\uczelnia{POLITECHNIKA WARSZAWSKA}
\wydzial{Wydział Elektroniki i Technik Informacyjnych Politechniki Warszawskiej}
\instytut{Instytut Radioelektroniki i Technik Multimedialnych}
\zaklad{Zakład Elektroniki Jądrowej i Medycznej}
\kierunekstudiow{Elektronika}

\def\keywords{DICOM; przeglądarka DICOM; obrazy; obrazowanie; C++; Qt; GDCM; programowanie}
\def\keywordsEng{DICOM; DICOM viewer; images; imaging; C++; Qt; GDCM; programing}

% Ustawienie metadanych pdfu
\makeatletter
\hypersetup{
    unicode=true,
    pdfauthor={\@author},
    pdftitle={\@title},
    pdfsubject={Praca Inżynierska},
    pdfkeywords={\keywords},
    pdfproducer={\@author},
    pdfcreator={\@author}
}
\makeatother

\opinie{%
    \input{opiniaopiekuna.tex}
    \newpage
    \input{recenzja.tex}
}

\streszczenia{
    \newpage
    \begin{center}
\large \bf
\thetitle
\end{center}

Praca składa się z sześciu rozdziałów: wstęp; obrazowanie diagnostyczne w medycynie; biblioteki i narzędzia; implementacja; kompilacja; podsumowanie.
Wstęp jest powierzchownym wprowadzeniem do tematu i celu pracy.
\par
W drugim rozdziale jest opisane zagadnienie problemowe związane z obrazami w medycznie.
Wymieniono techniki diagnostyczne oraz ich podstawowe różnice między sobą.
Przedstawiono parametry jakie cyfrowy obraz medycyny posiada.
Opisano prezentacje obrazów medycznych.
Wyjaśniono czym są przeglądarki obrazów, jakie funkcje mogą posiadać i jakie kryteria wyróżniono do ich porównywania.
Opisano format zapisu cyfrowych obrazów medycznych, standard \DICOM.
\par
Trzeci rozdział opisuje biblioteki i narzędzie użyte w czasie pisania pracy inżynierskiej.
Wyjaśniono cele użycia narzędzia CMake i jego zalety.
Opisano bibliotekę Qt, jej możliwości, drzewa obiektów implementowane przez nią i sposób konstrukcji programowania zdarzeniowego w niej zawartego.
Przedstawiono i uzasadniono wybór biblioteki GDCM jako biblioteki do obsługi i wczytywania plików \DICOM.
\par
W czwartym rozdziale przedstawiono sposób implementacji pracy.
Określono przewidywany zakres implementowanych funkcji oprogramowania.
Opisano graficzny interfejs użytkownika i jego funkcje oraz ogólną funkcjonalność programu.
Wyjaśniono powierzchownie projekt struktury obiektowej programu.
Następnie szczegółowo opisano strukturę danych wraz z klasami C++.
Tam gdzie była możliwość załączono diagramu UML.
Opisano wszystkie algorytmy przetwarzania danych w celu lepszej wizualizacji obrazu.
\par
W piątym rozdziale opisano przebieg kompilacji kodu źródłowego.
\par
Prace zakończono podsumowaniem i wnioskami.

\bigskip
{\noindent\bf Słowa kluczowe:} \keywords

\vfill
    \newpage
    \begin{center}
\large \bf
\thetitle
\end{center}

Na angielski przetłumacze jak będzie po polsku gotowe.

\bigskip
{\noindent\bf Keywords:} \keywordsEng

\vfill


}

\begin{document}

\maketitle

\chapter{Wstęp}
\input{text/wstep.tex}

\chapter{Obrazowanie diagnostyczne w medycynie}

\section{Obrazowe techniki diagnostyczne}
Istnieje wiele technik obrazowania wykorzystujące różne zjawiska fizyczne zachodzące w materii.
Podstawowe techniki obrazowania medycznego to:
\label{sec:basic-imaging-technics}
\begin{itemize}
    \item Radiografia - RTG

    Najstarsza i najbardziej rozpoznawalna technika obrazowania.
    Pierwsze zdjęcie analogowe zostało wykonane przez Röntgena w 1896 roku.
    Polega na przepuszczeniu przez obiekt badany promieniowania, a następnie detekcji tego promieniowania za obiektem badanym.
    W praktyce rejestrujemy współczynnik osłabienia promieniowania rentgenowskiego przez badany obiekt.
    Wyróżniamy dwa typu radiografii: analogowy i cyfrowy.
    Radiografia analogowa odchodzi powoli w zapomnienie.
    W radiografii cyfrowej obrazowana jest ilość promieniowania X przenikające przez badany obiekt.
    Kontrast zależy od położenia obiektu między źródłem a detektorem (położenie optymalne), napięcie anodowe, filtracja, grubość okładek wzmacniających.
    Rozdzielczość zależy od rozdzielczości detektora i rozmiaru ogniska lampy.

    W standardzie DICOM radiografia cyfrowa jest oznaczana jako \quotett{RT}.

    \item Tomografia rentgenowska - CT - Computer Tomography
    
    Agregacja w tomografii komputerowej jest podobna do badania RTG, ale w CT wykonujemy wiele pomiarów w różnych pozycjach względem obiektu badanego i pod różnym kontem.
    Następnie z tych pomiarów tworzymy obraz przez zastosowanie odpowiednich algorytmów tworzących obraz.
    Rejestrujemy współczynnik osłabienia promieniowania rentgenowskiego przez badany obiekt.
    Kontrast zależy od rozmiarów szczegółów badanego obiektu, napięcie anodowe, przyłożone masy (prąd katodowy i czas akwizycji).
    Rozdzielczość zależy od geometrii pomiaru, rozmiaru ogniska lampy rentgenowskiej, przestrzenna rozdzielczość matrycy detektora, liczby detektorów, dyskretyzację i filtru rekonstrukcyjnego.

    W standardzie DICOM obraz ultrasonograficzny jest oznaczana jako \quotett{CT}.

    \item Obrazowanie metodą rezonansu magnetycznego - MRI

    Sposób tworzenie obrazu MRI jest wysoce skomplikowanym procesem i ciężko opisać go w kilku zdaniach.
    Obrazowana jest sumaryczna gęstość atomów wodoru (protonów) w badanym obiekcie.
    Kontrast zależy od gęstości protonów, czasu relaksacji podłużnej i poprzecznej, prędkości przepływu płynu.
    Rozdzielczość zależy od parametrów skanera (rozmiar woksela).
    
    W standardzie DICOM obraz rezonansu magnetycznego jest oznaczana jako \quotett{MR}.
    
    \item Ultrasonografia
    
    Jest to badanie, które wszyscy kojarzą z badaniem płodu podczas ciąży z obrazem w kształcie łuku na, którym nic nie widać.
    Badanie ultrasonograficzne polega na wygenerowaniu fali akustycznej o wysokich częstotliwości, a następnie wprowadzeniu jej do ciała pacjenta.
    Następnie nasłuchuje się echa po tej fali.
    Obrazowana jest odbita fala ultradźwiękowa, osłabienia po odbiciach, zmienna częstotliwość i opóźnienie w czasie.
    Kontrast zależy od częstotliwości fali, głębokości badanego obiektu, ilości piezoelektryków w głowicy, obrazowanej struktury.
    Rozdzielczość zależy od czasu trwania impulsu zaburzenia oraz od szerokości wiązki ultradźwiękowej (powierzchnia czynna przetworników).

    W standardzie DICOM obraz ultrasonograficzny jest oznaczana jako \quotett{US}.

    \item Scyntygrafia
    
    Obrazowa technika diagnostyczna z gałęzi medycyny nuklearnej.
    Polega na wprowadzenia do organizmu ciał obcych, środków chemicznych zwanymi również radiofarmaceutykami, charakteryzującymi się krótkim czasie rozpadu i powinowactwem chemicznym z badanymi organami.
    Następnie wykrywanie rozpadów zachodzących w ciele poprzez rejestracje promieniowania wytwarzanego podczas rozpadu, a następnie przedstawienie to w formie graficznej.
    Kontrast zależy od długości trwania pomiaru, oraz od ilości wstrzykniętego radiofarmaceutyka.
    Rozdzielczość zależy od ułożenia i możliwości rozdzielczej kamer scyntylacyjnych, zwanymi także scyntykamerami, gammakamerami lub kamerami Angera.

    W standardzie DICOM obraz scyntygraficzny jest oznaczana jako \quotett{NM}.

    Radiofarmaceutyki to związki chemiczne zawierające radioizotop.

    \item Tomografia SPECT
    
    Technika obrazowania  z gałęzi medycyny nuklearnej. w której rejestruje się promieniowanie powstające rozpadu gamma.
    Źródłem promieniowania(fotonów) jest podana pacjentowi radiofarmaceutyk, ulegająca rozpadowi gamma.
    Rejestrujemy fotony powstające podczas anihilacji pozytonów.
    Kontrast zależy od wydajności detektorów, odległość detektora od obiektu oraz położenie obiektu.
    Na rozdzielczość ma wpływ przestrzenna rozdzielczość matrycy detektora, liczby detektorów.

    W standardzie DICOM obraz ultrasonograficzny jest oznaczana jako \quotett{PT}.

    \item Tomografii PET
    
    Technika obrazowania  z gałęzi medycyny nuklearnej. w której rejestruje się promieniowanie powstające podczas anihilacji pozytonów (antyelektronów).
    Źródłem promieniowania(pozytonów) jest podana pacjentowi substancja promieniotwórcza, ulegająca rozpadowi beta plus
    Rejestrujemy fotony powstające podczas anihilacji pozytonów.
    Kontrast zależy od wydajności detektorów, odległość detektora od obiektu oraz położenie obiektu.
    Na rozdzielczość ma wpływ przestrzenna rozdzielczość matrycy detektora, liczby detektorów.

    W standardzie DICOM obraz ultrasonograficzny jest oznaczana jako \quotett{PT}.
    
\end{itemize}

Istnieją też techniki, które są połączeniem kilku innych technik.
Takie jak:
\begin{itemize}
    \item PET-CT, PET/CT - połączenie PET z wielorzędowym tomografem komputerowym
    \item PET-MRI, PET/MRI - połączenie PET z rezonansem magnetycznym
\end{itemize}

Standard DICOM nazywa techniki obrazowania modalnościami(z ang. modality).

\section{Parametry obrazów}
\subsection{Podstawowe parametry obrazu cyfrowego}

\dicomtagExplanations

\par
Każdy obraz cyfrowy jest matrycą pikseli o ustalonych rozmiarach.
W przypadku standardu \DICOM obrazy są matrycami wokseli, posiadającymi wysokość (zapisaną w \dicomtag{Rows}{0028}{0010}) oraz szerokość (zapisaną w \dicomtag{Columns}{0028}{0011}).
Do poprawnej interpretacji znaczenia macierzy służy znacznik \dicomtag{Photometric Interpretation}{0028}{0004}, informujący o fotometrycznym znaczeniu wokseli.
Standard \DICOM definiuje następujące wartości tego tagu (wraz z wyjaśnieniem):
\begin{itemize}
    \item \dataword{MONOCHROME1} i \dataword{MONOCHROME2} --- ta wartość woksela odwzorowuje skale monochromatyczną, odpowiednio od jasnego do ciemnego i od ciemnego do jasnego.

    \item \dataword{PALETTE COLOR} --- ta wartość woksela jest używana jako indeks w każdej z tabel wyszukiwania kolorów palety czerwonej, niebieskiej i zielonej.
          Palety mają swoje własne tagi.
          Wartość raczej rzadka i nie spotykana.

    \item \dataword{RGB} --- oznacza, że woksel jest trzy-kanałowym pikselem RGB (kanały: czerwony, zielony i niebieski).

    \item \dataword{HSV} \fromEng{Hue Saturation Value} --- woksel reprezentuje piksel w modelu przestrzeni barw zaproponowany w 1978 roku przez Alveya Raya Smitha.
          Model ten nawiązuje do sposobu w jakim widzi oko człowieka.
          Wartość wycofana.

    \item \dataword{ARGB} --- ta wartość woksela to piksel RGB z dodatkowym kanałem przezroczystości.
          Wartość wycofana.

    \item \dataword{CMYK} --- ten woksel to piksel w modelu czterech podstawowych kolorów farb drukarskich stosowanych powszechnie w druku wielobarwnym w poligrafii: cyjan, magenta, żółty, czarny.
          Wartość wycofana.

    \item \dataword{YBR\_FULL} --- ten woksel to piksel w modelu przestrzeni barw nazwanej YC\textsubscript{b}C\textsubscript{r}.

          Dodatkowo standard zdefiniował pochodne tej wartości: \dataword{YBR\_RCT}, \dataword{YBR\_FULL\_422}, \dataword{YBR\_PARTIAL\_422}, \dataword{YBR\_PARTIAL\_420}, \dataword{YBR\_ICT}, ale wszystkie są już wycofane.
\end{itemize}

\dicomRetired

\par
Kwantyzacja obrazu, czyli liczba poziomów obrazu, jest zapisana na czterech znacznikach:
\begin{itemize}
    \item \dicomtag{Bits Allocated}{0028}{0100} --- informuje na jak wiele bitów zostało zaalokowanych do zapisania jednego piksela
    \item \dicomtag{Bits Stored}{0028}{0101} --- informuje jak wiele bitów z zaalokowanych posiada wartość piksela
    \item \dicomtag{High Bit}{0028}{0102} --- informuje gdzie znajduje się najstarszy bit
    \item \dicomtag{Pixel Representation}{0028}{0103} --- informuje czy poziomy są ze znakiem czy bez
\end{itemize}

\par
Obraz \DICOM również zawiera w sobie informacje o próbkowaniu.
Z uwagi na to, że próbkowanie wygląda inaczej w każdej technice, standard posiada odpowiedni zestaw znaczników dla każdej techniki.
Próbkowanie poszczególnych technik opisałem w sekcji \ref{sec:basic-imaging-technics}.

% \subsection{Wartość diagnostyczna obrazu}

% W obrazowaniu medycznym chodzi o wyciągnięcie wniosków z obrazów i postawienie diagnozy.
% Jest to kluczowy element obrazowania.
% Ocena wartości diagnostycznej to złożone zagadnienie z teorii hipotez statystycznych.
% Brak możliwości stwierdzenia co na obrazie się znajduje, stawia sens takiego obrazowania pod znakiem zapytania.
% Poco nam obraz w 4K na, którym można zobaczyć wyraźne plamy niczego.

% Wartość diagnostyczną można określić na podstawie następujących parametrów
% \begin{itemize}
%     \item Jakości obrazu

%           Parametry jakościowe obrazów są szczegółowo opisane w sekcji \ref{sec:image-quality}

%     \item Warunków obserwacji obrazu

%           W brew pozorom warunki obserwacji mają kluczowe znaczenie dla wartości diagnostycznej.
%           Jeżeli będziemy mieli dobry obraz, który wyświetlimy na budżetowym monitorze RGB, który w rzeczywistości posiada 6-bitowe kanały RGB i tworzy odcienie za pomocą techniki dithering'u, to niewiele zobaczymy.

%     \item wiarygodności diagnostycznej obrazów

%     \item charakterystyki pracy lekarza-specjalisty

% \end{itemize}

\subsection{Kontrast}

Jedną z wielu definicji kontrastu jest kontrast Michelsona wyrażony wzorem:
\[\frac{I_{max}-I_{min}}{I_{max}+I_{min}}\]
gdzie $I_{max}$ i $I_{min}$ to najwyższa i najniższa wartość luminancji.

\subsection{Rozdzielczość}

\subsubsection*{Przestrzenna}

\par
Rozdzielczość przestrzenna obrazu to najmniejsza odległość między dwoma punktami obrazu, które można rozróżnić.
Jest ona silnie związana z kontrastem obrazu za pomocą funkcji przenoszenia modulacji (MTF –-- Modulation Transfer Function).
Jest to krzywa ukazująca degradację kontrastowości w miarę zwiększania częstotliwości przestrzennej okresowego wzorca.
Funkcję MTF można wyznaczyć używając rozbieżnych tarcz rozdzielczości przestrzennej lub, w pewnych warunkach, przy pomocy norm wielopręcikowych.
W radiografii rozdzielczość określa się zazwyczaj jako liczbę równoległych linii, czarnych i białych, które można rozróżnić ma 1 milimetrze obrazu (paralinie na milimetr).

\par
Rozdzielczość przestrzenna jest zależna od kontrastu obrazu.
Zależność ta jest inna dla każdej techniki.

\subsubsection*{Czasowa}

Każdy pomiar wymaga pewnego czasu pobierania danych.
W nie których przypadkach ważna jest również zmiany zachodzące w organizmie w czasie wykonywania badania.
Rozdzielczość czasowa, jest istotna w obrazach dynamicznych, np. angioMR.
Kiedy mamy pomiar dokonywany w określonym czasie i ustalone są markery czasowe.
Rozdzielczość czasowa jest definiowana jako odległość w czasie od dwóch klatek obrazowania.

\subsection{Stosunek sygnału do szumu (SNR)}

Rodzaj i poziom szumu zależy od techniki obrazowania.
Stosunek sygnału do szumu ma decydujący wpływ na widoczności obiektów, kontrast oraz percepcję szczegółów w obrazie.

\subsection{Poziom artefaktów}

Artefakty to zjawiska fałszujące obraz poprzez tworzenie struktur w obrazie, nie istniejących w rzeczywistości.
Jest to problem występujący w różnych technikach obrazowania.
Najbardziej znanymi artefaktami są np. w badaniu USG tak zwany warkocz komety w przypadku obiektów o wysokiej różnicy impedancji w stosunku do otoczenia.

\subsection{Poziom zniekształceń przestrzennych}

Zniekształcenia przestrzenne powstają w wyniku geometrycznego ułożenia i kształtu obiektu badanego oraz aparatu pomiarowego.
Przykładem takiego zniekształcenia mogą być różne powiększenia obiektów zależne od głębokości ich ułożenia w USG, zmiana pozycji pacjenta(przez ruchy klatki piersiowej w czasie badania), czy deformacja obrazu spowodowana zmianami rozkładu pola magnetycznego przez metalowe obiekty w znajdujące się w tym samym pomieszczeniu w przypadku badań MRI.



\section{Zapisywanie obrazów i standard DICOM}


Pierwsze tomografy komputerowe przeżyły swój rozkwit w latach siedemdziesiątych ubiegłego wieku.
Spowodowało to, że obrazu medyczne nie były bezpośrednim wynikiem badania, a jedynie wynikiem obróbki danych pomiarowych przez komputer.
Dodatkowo obrazy przedstawiały przekroje, co sprawiły wiele trudności w ich interpretacji personelowi medycznemu.
Zwyczajne pliki graficzne (jak np. jpg, png, gif), nie nadawały się do zapisu takich obrazów, ponieważ zapisywały obraz w spektrum światła widzialnego, a konkretniej w postaci pozwalającej na odtworzenie światła widzialnego.
Natomiast obrazy medyczne sa zapisywanie w spektrum rentgenowskim.
Nie ułatwiał fakt, że każdy producent stosował inne metryki oraz inne oznaczenia swojego sprzętu.

\subsection{Standard DICOM v3.0}

Standard DICOM wersji trzeciej to standard definiujący ujednolicony sposób zapisu i przekazywania danych medycznych reprezentujących lub związanych z obrazami diagnostycznymi w medycynie.
Standard został wydany w 1993 przez dwie agencje ACR (American College of Radiology) i NEMA (National Electrical Manufactures Association).
Wcześniejsze wersje nazywały się ACR/NEMA v1.0, wydana w 1983 roku i ACR/NEMA v2.0, wydana w 1990 roku, stąd wersja trzecia.
Od wydania wersji trzeciej w 1993, standard jest wciąż rozwijany i uzupełniany o nowe elementy.
W obecnej chwili standard DICOM definiuje 81 różnych typów badań.

UWAGA: Za każdym razem kiedy jest odniesienie do obecnego standardu DICOM, w domyśle jest to odsłona 2019a.

\subsection{Sposób zapisu danych w pliku DICOM}

Plik w formacie DICOM przypomina bazę danych z rekordami.
Baza danych nazywa się \keyword{Data Set} i składa się z rekordów, które nazywają się \keyword{Data Element}.

\begin{figure}[!htbp]
    \caption{Wizualizacja ułożenia \keyword{Data Element}(wraz z budową) w \keyword{Data Set}}
    \includegraphics[]{img/dicom-dataelement001.pdf}
    \centering
    \label{fig:dicom-dataelement}
\end{figure}

\subsubsection{Data Element}

\keyword{Data Element} jest rekordem, który przechowuje jakaś jedną informacje o czymś.
Składa się z czterem elementów:

\begin{itemize}

    \item \keyword{Tag} - to unikalny identyfikator, złożony z dwóch liczb: grupy(uint16) i elementu(uint16) grupy.
    Informuje o tym co dany rekord w sobie zawiera.
    W jednym \keyword{Data Set} nie mogą się pojawić dwa \keyword{Data Element} posiadających ten sam \keyword{Tag}
    
    Obiekt reprezentujący \gdcmclass{Tag}.

    Na przykład: jeżeli liczby \keyword{Tag} przyjmą wartości odpowiednio wartość $0010_{16}$ i $0010_{16}$ to oznacza, że jest to tag \dicomtag{PatientName}{0010}{0010}, czyli zwiera w sobie parametr zawierają nazwę pacjenta.

    Dokładne omówienie \keyword{Tag}-ów znajduje się w sekcji \ref{sec:dicom-tag}.

    \item \keyword{Value Representation}, w skrócie \keyword{VR} – to dwa bajty w postaci tekstu, informujący o formacie w jaki parametr został zapisany.
    
    Dokładne omówienie \keyword{VR}-ów znajduje się w sekcji \ref{sec:dicom-vr}.

    \item \keyword{Value Length}, w skrócie \keyword{VL} - 32-bitowa lub 16-bitowa liczba nieoznaczona, która informuj o długości pola danych(\keyword{Value Field}).
    
    Wartość \keyword{VL} zwykle jest liczbą parzystą.
    Standard DICOM zakłada, że wszystkie dane powinny być dopełniane do parzystej ilości bajtów.
    
    \item \keyword{Value Field} (opcjonalne) - pole z parametrem o długości VL.
    
\end{itemize}

Wizualizacja budowy \keyword{Data Element} jest na rysunku \ref{fig:dicom-dataelement}.

\subsubsection{Tag}
\label{sec:dicom-tag}

Znacznik


\subsubsection{VR - Value Representation}
\label{sec:dicom-vr}

Reprezencaja wartości danej

Obiekt używany do przechowywania taga to \gdcmclass{VR}.
Na przykład: Decimal String, w skrócie DS, oznacza liczbę zapisaną za pomocą teksu.
Czasami to pole może być puste, wtedy należy się odnieść do VR przypisanego do taga, który określa standard.

\subsection{DICOMDIR}

coś o dicomdir

\section{Wyświetlanie obrazów}

Posiadanie wielu obrazów wiąże się z potrzebą ich przeglądania i porównywania.
Należy, więc posiadać jakieś narzędzie do wyświetlenia w sposób poprawny, najlepiej jednym i tym samym programem.

\subsection{Przeglądarki obrazów}

Przeglądarki obrazów to programy należące do kategorii przeglądarki plików.
Zwykłe przeglądarki obrazów takich jak jpg, png lub gif wyświetlają obraz w takiej postaci jakiej jest zapisany, oczywiście najpierw przeprowadzają dekompresje obrazu.
W przypadku obrazów medycznych najczęściej nie mamy do czynienia z danymi reprezentującymi kolory w spektrum światła widzialnego.
Przeglądarka obrazów DICOM musi wygenerować kolorowy obraz z danych na podstawie parametrów obrazu.

\subsection{Porównanie przeglądarek obrazów}

Trudno jest porównywać coś tak złożonego jak przeglądarka obrazów medycznych, nie można jednoznacznie powiedzieć, że jedna jest lepsza od drugiej. W celu porównań wyróżniono 26 kryteriów do porównywania przeglądarek w postaci „tak” lub „nie”, podzielonych na 5 grup, platformy, interfejsu, wsparcie, obrazowanie dwu i trój wymiarowego.
Kryteria te w jasny sposób pozwalają na ocenę praktycznych aspektów użytkowania przeglądarki.

\paragraph{Platforma}

Samodzielność, aplikacje samodzielne są zaprojektowane tak, aby nie wymagały żadnego dodatkowego sprzętu fizycznego bądź infrastruktury do poprawnego działania(np. systemu Windows oraz serwisów przez niego dostarczanych).
Rozwiązania sieciowe, określają czy aplikacja jest usługą sieciową i można z przeglądarki korzystać jak ze strony WWW.
Wieloplatformowość, możliwość uruchomienia ich na różnych systemach operacyjnych Linux/MacOS/Windows
Rozwiązania mobilne, możliwość używania na urządzeniach mobilnych takich jak telefon.

\paragraph{Interfejs}

Przeglądarka powinna mieć możliwość komunikacji z interfejsami innych systemów.
Podstawowe interfejsy sieciowe to: C-STORE SCP DICOM C-STORE, C-STORE SCU, Query-Retrieve, WADO, Parameter Transfer.

\paragraph{Wsparcie techniczne}

Dokumentacja, dostępność pisemnej dokumentacji oprogramowania (np. podręczniki lub strony internetowej).
Wsparcie przez pocztę internetową, możliwość porozumienia się z twórcą lub opiekunem oprogramowania.
Forum, możliwość pytania się społeczności o opinie i ich wymiana.
Wiki, strona internetowa w formacie Wikipedii dostępna dla użytkownika.

\paragraph{Obrazowanie dwu-wymiarowe}

Przewijanie(\fromEng{scroll}), proces wyświetlania obrazów, można poprawić dzięki zmniejszeniu interakcji z klawiaturą oraz myszką. Można to osiągnąć na przykład, oferując możliwość przejścia do następnego lub poprzedniego obrazu przez przesunięcie kółkiem myszy lub używając przycisków góra/dół na klawiaturze.
Metadane, przeglądania powinna obejmować analizowanie i wyświetlanie metadanych obiektów DICOM, powinna obejmować wyświetlanie rozdzielczości obrazu, badanie (np. identyfikator podmiotu) oraz znaczniki DICOM specyficzne dla dostawcy (np. specjalne ustawienie urządzenia rejestrującego).
Warstwa informacyjna, najważniejsze informacje powinny powinny być wizualizowane w oknie wyświetlacza jako nakładka na obraz.
Na przykład aktualna pozycja lub nazwa podmiotu wykonującego badanie.
Okienkowanie (okna cyfrowe), sposób zamiany danych na skale szarości, okienkowanie jest opisane w sekcji \ref{sec:windowing}.
Pseudo-kolorowanie obrazu, tabele (LUT, \fromEng{LookUpTable}) odwzorowujące szare wartości obrazu na pseudo-kolory, poprawiaja one czytelność obrazu.
Histogram, histogramy wizualizują wystąpienia i rozkład wartości kolorów na obrazach, pozwalają opisywać istotne cechy obrazu
Wymiarowanie, możliwości rysowania bądź zaznaczania linii lub innych kształtów do analizy i wyznaczania odległości w jednostkach długości na obrazie.
Jest to możliwe gdyż nagłówki pliku DICOM zawierają parametry sprzętowe urządzenia (np. ilość pikseli na centymetr).
Adnotacje(opisy), które były wytworzone przez personel medyczny powinny być zapisywane w odpowiedni sposób w pliku.

\paragraph{Obrazowanie trój-wymiarowe}

Rekonstrukcja wtórna, zwykle dane dotyczące objętości medycznej są gromadzone wzdłuż jednej osi ciała (np. poprzecznej).
W wielu przypadkach ważne jest przeglądanie danych w innych kierunkach (np. strzałkowych lub czołowych), aby poprawić wizualizację niektórych struktur.
W tym celu należy zapewnić funkcjonalność rekonstrukcji osi pomocniczej na podstawie kierunku pierwotnego.
Plastry objętości kostki(\fromEng{Slice Cube Volume}), przekroje mogą być lepiej wyświetlane w określonej pozycji.
Funkcjonalność kostki plasterka umożliwia niezależną regulację położenia różnych osi wycinków (np. poprzecznych, strzałkowych lub czołowych) w modelu objętościowym.
Podczas tego przekroje są pokazane w osobnym oknie.
Renderowanie objętościowe – dane obrazu 3D są bezpośrednio wizualizowane jako objętość.
Użytkownik może wchodzić w interakcje z woluminem poprzez obracanie lub skalowanie.
Transfer Function(nie znam polskiej nazwy), służy do odwzorowania wartości szarości obrazów wokseli na wartości krycia typów tkanek (np. kości). Struktury obrazu pasujące do wzorców szarych wartości są podświetlone. Niewykorzystane szare wartości są wyświetlane jako
przezroczyste. Specyficzne struktury stają się lepiej widoczne.
Generowanie powierzchni, dzięki różnym algorytmom można generować powierzchnie w postaci wokselów. Reprezentacje powierzchni można również zastosować do poprawy wizualizacji niektórych struktur obrazu.

\subsection{Funkcje przeglądarki obrazów}

\subsubsection{Podstawowe operacje na obrazie}

\begin{itemize}
    \item Skalowaniu lub powiększenie.
          Możliwość powiększenia lub zmniejszenia wyświetlanego obrazu o pewny współczynnik skalujący.

    \item Przesuwanie(\fromEng{pan})
          Możliwość przesuwania obrazu o dowolny wektor.
          Przydatne gdy powiększymy obraz do takiego stopnia, że nie będzie mieścił się na ekranie lub w okienku programu.

    \item Lupa, skalowanie miejscowe
          Możliwość miejscowego powiększenia obrazu.
          Przykład użycia takiego narzędzia znajduje się na rysunku \ref{fig:wyswietlanie001}.

          \begin{figure}[!htbp]
              \caption{Przykład narzędzia Lupa w przeglądarce \href{https://www.softneta.com/products/meddream-dicom-viewer/}{MedDream DICOM Viewer}.}
              \includegraphics[width=\textwidth]{img/wyswietlanie001.png}
              \centering
              \label{fig:wyswietlanie001}
          \end{figure}

    \item Rotacja i odbicia lustrzane
          Możliwość obrócenia obrazu o zadany kąt.
          Oraz możliwość odbicia lustrzanego obrazu w dwóch osiach X i Y.

\end{itemize}

\subsubsection{Analiza parametrów w celu lepszej informacji}

\begin{itemize}
    \item Okienkowanie
    \item Maski lub nakładki(\fromEng{overlay})
\end{itemize}

\subsubsection{Generowanie obrazów woliumetrcznych}

Jeżeli mamy do dyspozycji wiele obrazów tomograficznych o znanych parametrach

\subsubsection{Analiza i przetwarznie danych}

\begin{itemize}
    \item Histogram
          Możliwość wygenerowania histogramu obrazu.

          Histogram to wykres przedstawiający dystrybucje wartości numerycznych obrazu.


          \begin{figure}[!htbp]
            \caption{Przykładowy histogram.}
            \includegraphics[width=\textwidth]{img/wyswietlanie002.svg}
            \centering
            \label{fig:wyswietlanie002}
        \end{figure}

    \item Mierzenie obrazu, wykonywanie pomiarów
          Możliwość zmierzenia odległości pomiędzy dwoma punktami przez lekarza lub zmierzenia wielkości/pola zadanego kształtu.11
\end{itemize}

\subsubsection{Edycja danych}

\begin{itemize}
    \item Dodawanie nowych obiektów.
          Możliwość rysowania, dodawania figur geometrycnych lub tekstu przez lekarza i możliwość zapisu tych informacji w pliku DICOM.

    \item Mierzenie obrazu, wykonywanie pomiarów.
\end{itemize}


- dodawanie obiektów

- rysowanie

- edycja parametrów


\subsection{Możliwości mojej przeglądarki}

Po analizie możliwości przeglądarek plików DICOM dostępnych na rynku postanowiłem zaimplementować następujące komponenty w mojej przeglądarce:

\begin{itemize}
    \item Przesuwanie(\fromEng{pan})

    \item Skalowaniu lub powiększenie

    \item Rotacja i odbicia lustrzane

    \item Okienkowanie

    \item Wczytywanie wielu plików na raz
\end{itemize}

Moją przeglądarkę nazwałem Sokar.


\chapter{Biblioteki i narzędzia}



\section{CMake}
\par
CMake to wieloplatformowe narzędzie do automatycznego zarządzania procesem kompilacji programu.
Jest to niezależne od kompilatora narzędzie pozwalające napisać jeden plik, z którego można wygenerować odpowiednie pliki budowania dla dowolnej platformy.
\par
Z uwagi na to, że projekt musi mieć możliwość kompilacji na 3 platformy CMake jest idealnym rozwiązaniem.
Dodatkowo starałem się wybrać biblioteki, które kompilują się za pomocą CMake.

\subsection{Przebieg kompilacji za pomocą narzędzia CMake}

\subsubsection{Linux}

\subsubsection{MacOS}

\subsubsection{Microsoft Windows}

\section{QT}
\par
Biblioteka Qt, rozwijana przez organizacje Qt Project, jest zbiorem bibliotek i narzędzi pr programistycznych dedykowanych dla języków C++, QML i Java.
\par
Qt jest głownie znane jako biblioteka do tworzenia interfejsu graficznego, jednakże posiada ona wiele innych rozwiązań ułatwiających programowanie obiektowe i zdarzeniowe.
\par
Wybrałem Qt z uwagi na to, że posiada interfejs w C++.
Oraz kompilacja oprogramowania używającego Qt może odbywać się za pomocą dwóch narzędzi: CMake oraz dedykowanego narzędzia qmake, zrobionego specjalnie na potrzeby biblioteki Qt.
Dzięki czemu cały projekt przeglądarki używa tego samego języka oraz tego samego narzędzia zarządzania kompilacją.

\subsection{Wymowa}

\par
Według autorów, Qt powinno się czytać jak angielskie słowo \enquote{cute}, po polsku \enquote{kiut}.
Jednakże społeczność programistów nie jest co fo tego zgodna.
Ankiety zrobione na dwóch popularnych serwisach internetowych o tematyce programistycznej, pokazują, że najbardziej popularną wymową jest \enquote{Q.T.}, po polsku \enquote{ku te}.
\par
Odnośniki do przytoczonych ankiet:
\begin{itemize}
    \item \url{https://ubuntuforums.org/showthread.php?t=1605716}
    \item \url{https://www.qtcentre.org/threads/11347-How-do-you-pronounce-Qt}
\end{itemize}

\subsection{Licencja}

\par
Biblioteka Qt jest dystrybuowana w dwóch wersjach: komercyjnej i otwarto źródłowej.
Wersja otwarto źródłowa nie posiada wielu modułów, ale jest dystrybuowana na licencji \href{https://www.gnu.org/licenses/gpl.html}{GPL w wersji 3}.
Co sprawia, że bibliotekę można użyć w mojej pracy.

\subsection{Norma IEC 62304:2015}

\par
The Qt Company posiada szereg certyfikatów od FDA i UE, ułatwiające wprowadzenie produktów używających bibliotek Qt na rynek Europejski jak i Amerykański.
\par
Lista posiadanych norm:
\begin{itemize}
    \item IEC 62304:2015 (2006 + A1)
    \item IEC 61508:2010-3 7.4.4 (SIL 3)
    \item ISO 9001:2015 
\end{itemize}
Więcej informacji na temat certyfikatów można przeczytać na oficjalnej stronie Qt pod adresem \url{https://www.qt.io/qt-in-medical/}.

\subsection{Klasa QObject}

\par
Biblioteka Qt implementuje klasę \qtclass{QObect}, która jest bazą dla wszystkich obiektów Qt i wszystkie klasy współpracujące z biblioteką Qt powinny po niej dziedziczyć.
\qtclass{QObject} implementuje 2 podstawowe rzeczy: system drzewa obiektów (opisany w sekcji \ref{sec:qt-pareting}), system sygnałów (opisany w sekcji \ref{sec:qt-signals}).

\subsubsection{Drzewa obiektów}
\label{sec:qt-pareting}

\par
W C++ jednym z największych problemów jest wyciek pamięci, pojawia się wtedy gdy zaalokujemy na stercie ony obiekt za pomocą operatora \cppcode{new} i nie go usunąć gdy ten będzie potrzebny.
\par
\qtclass{QObject} zakłada, że obiekty mogą mięć jednego rodzica, a rodzic może mieć wiele dzieci.
Rodzica można przypisać podczas tworzenia obiektu oraz zmieniać go dowolnie w trackie działania programu.
Przypisanie rodzica dziecku oznacza to, że gdy wywołamy destruktor rodzica, ten wywoła destruktory dzieci i w ten sposób całe drzewo obiektów zostanie zniszczone.
\par
Mechanizm ten pozwala nam tworzyć nowe obiekty na stercie i nie martwić się o ich poźniejsze sprzątnięcie.
Jest to o tyle efektywne, że nie trzeba dla każdego obiektu tworzyć odrębnego wskaźnika lub wektora wskaźników w deklaracji klasy, a dzięki temu można mieć czystszy i czytelniejszy kod źródłowy.
Przykładowe użycie:
\par
\begin{lstlisting}[language=C++]
int main() {

    // Tworzymy obiekt przycisku
    auto *quit = new QPushButton("Quit");
    // Tworzymy obiekt okna
    auto *window = new QWidget();

    // Przypisujemy rodzica przyciskowi
    quit->setParent(window);
    
    ...

    // W tym momencie przycisk wraz z oknem zostaja usuniete
    delete window;
}
\end{lstlisting}

\subsubsection{Sygnały i sloty}
\label{sec:qt-signals}
\input{text/biblioteki/qt-signlas.tex}

\subsection{Graficzny interfejs użytkownika}

\subsection{Oddzielenie od platformy}

\par


\section{GDCM}


\subsection{Uzasadnienie wyboru}

\par
Znalezienie dobrej biblioteki do obsługi jest niebywale trudne, ponieważ jest ich bardzo dużo, a ich liczba wciąż rośnie.
Powstał nawet portal internetowy do ich indeksowania o nazwie \enquote{I DO IMAGING}, dostępny pod adresem \url{https://idoimaging.com/programs}.
Biblioteka, której szukałem powinna:
\begin{itemize}
    \item współpracować z językiem C++
    \item mieć licencje pozwalającą jej używać w potrzebnym mi zakresie
    \item darmowa, najlepiej otwarto źródłowa
    \item aktywnie rozwijana --- znaczna większość bibliotek charakteryzowała się tym, że była porzucona i ostatnia zmiana była wprowadzona x lat temu, a proces jej rozwoju trwał od 2 do 5 miesięcy
    \item dostępna na Linux'a, MacOS i Microsoft Windows
\end{itemize}
Ostateczna decyzja padła na bibliotekę o nazwie Grassroots DICOM (GDCM), dostępną pod adresem \url{http://gdcm.sourceforge.net/}.

\subsection{Opis}

\par
Przetłumaczony opis biblioteki z oficjalnej strony prezentuje się następująco:
Grassroots DICOM (GDCM) to implementacja standardu DICOM zaprojektowanego jako open source, dzięki czemu naukowcy mogą uzyskać bezpośredni dostęp do danych klinicznych.
GDCM zawiera definicję formatu pliku i protokół komunikacji sieciowej, z których oba powinny zostać rozszerzone w celu zapewnienia pełnego zestawu narzędzi dla badacza lub małego dostawcy obrazowania medycznego w celu połączenia z istniejącą bazą danych medycznych.

\par
GDCM jest biblioteką posiadającą możliwość wczytywania, edycji i zpaisu plików w formacie DICOM.
Obsługuje wiele kodowań obrazów jak i protokoły sieciowe.
Jest w całości napisana w C++, a do kompilacji używa CMake.
Dzięki temu w całym programie jest używany język C++ wraz z CMake, co ułatwia zarządzaniem procesem kompilacji do jednego pliku.

\par
Główną zaletą biblioteki, jest dobra dokumentacja wraz z przykładami jej użycia, które okazały się kluczowe przy wyborze.
Biblioteka została napisana w sposób obiektowy z usprawnieniami zawartymi w C++, takimi jak referencje i obiekty stałe, co ułatwia jej używanie.

\subsection{Licencja}

\par 
GDCM jest wydana na licencji BSD License, Apache License V2.0, która jest kompatybilna z GPLv3
Licencja ta dopuszcza użycie kodu źródłowego zarówno na potrzeby wolnego oprogramowania, jak i własnościowego oprogramowania.


\subsection{Podstawowe klasy}
\label{sec:gdcm-classes}
\input{text/biblioteki/gdcm-classes.tex}

\subsection{Przykład użycia}
\label{sec:gdcm-use}
\input{text/biblioteki/gdcm-use.tex}

\section{Git}
\par
Git to system kontroli wersji.
Cała praca została wykonana przy asyście tego narzędzia, a repozytorium z programem znajduje się pod adresem \url{https://gl.ire.pw.edu.pl/ajedrzejowski/sokar-app}.
Źródło pracy pisemnej napisanej w \href{https://www.latex-project.org/}{LaTeX} można znaleźć pod adresem \url{https://gl.ire.pw.edu.pl/ajedrzejowski/sokar-writing}.



\chapter{Implementacja}

\par
Najbardziej rozpoznawalne dwie przeglądarki to Osirix i Horus.
Ich nazwy zaczerpnięto od nazw egipskich bogów: odpowiednio od Ozyrysa, boga śmierci i Horusa, boga nieba.
Nazwa przeglądarki omawianej w pracy będzie miała nazwę: Sokar.
\par
Sokar w mitologii egipskiej to bóstwo dokonujące przyjęcia i oczyszczenia zmarłego władcy oraz przenoszący go na swej barce do niebios, patron metalurgów, rzemieślników i tragarzy (nosicieli lektyk) oraz wszelkich przewoźników.

\section{Zakres implementacji}

\par
Po analizie możliwości przeglądarek plików \DICOM dostępnych na rynku postanowiono zaimplementować następujące komponenty w opracowywanej przeglądarce:
\begin{itemize}
    \item Obsługa obrazów bez względu na ich modalność, ale z ograniczeniem do następujących interpretacji fotometrycznej:

          \begin{itemize}
              \item \dataword{MONOCHROME1}
              \item \dataword{MONOCHROME2}
              \item \dataword{RGB}
              \item \dataword{YBR}
          \end{itemize}

    \item Przesuwanie \fromEng{pan}.

    \item Skalowanie lub powiększenie poprzez decymacje i interpolacje liniowe.

    \item Rotacja i odbicia lustrzane.

    \item Okienkowanie i pseudokolorowanie, zarówno w skali szarości jak i z użyciem wielokolorowych palet.

    \item Obsługa serii obrazów jako całości
          \begin{itemize}
              \item przegląd obrazów w serii
              \item animacje
              \item wspólne okna w skali barwnej
              \item wspólne przekształcenia macierzowe
          \end{itemize}
\end{itemize}


\section{Wieloplatformowość}
Przeglądarka jest napisana w taki sposób, że jej implementacja nie ogranicza możliwości kompilacji na konkretny systemy operacyjnego.


\subsection{Język programowania}

Przeglądarka została napisana w C++ w standardzie ISO/IEC 14882 z 2018, w skrócie C++17

\subsection{Środowisko programistyczne}

Do pisania kodu oraz debugowania używałem głównie CLion, IDE stworzonego przez firmę JetBrians.
Zdecydowaną większość czasu przeglądarka była testowana i debugowana na aktualizowanym systemie ArchLinux.

\subsection{Obiektowy model w oprogramowaniu}

Praca jest zaprojektowany w sposób obiektowy, w taki sposób aby była możliwość jej rozbudowy i dodawania nowych funkcji.

\section{Graficzny interfejs użytkownika}
\sokarclassExplanations

\par
Po uruchomieniu programu użytkownikowi ukazuje się jedno okno (rysunek \ref{sec:gui-window}), które zawiera 3 elementy: menu (obiekt klasy \qtclass{QMenuBar}), drzewa plików (obiekt klasy \sokarclass{FileTree}), obiekt zakładek z obrazami (obiekt klasy \sokarclass{DicomTabs}).
\par
Użytkownik może otworzyć plik DICOM na trzy sposoby: z menu na górze, z drzewem ze strukturą pików i poprzez przeciągnięcie.

\section{Projekt struktury obiektowej programu}

\par
W tej sekcji wyjaśniona jest ogólna struktura programu, z pominięciem dokładnych opisów poszczególnych elementów.
Ich szczegółowy opis znajduje się w następnych sekcjach.
\par
Obiekt okna, klasy \sokarclass{MainWindow} posiada 3 elementy: menu (klasy \qtclass{QMenuBar}), drzewa plików (klasy \sokarclass{FileTree}), obiekt zakładek (klasy \sokarclass{DicomTabs}).
Zakładki obiektu zakładek są implementowane prze klasę \sokarclass{DicomView}.
Obiekt zakładki posiada abstrakcyjną kolekcję scen, implementowaną przez \sokarclass{DicomSceneSet}.
Kolekcja scen odpowiada za przechowywanie obrazów i scen, obiektów klasy \sokarclass{DicomScene}.
Sceny nie posiadają bezpośredniego dostępu do pliku, a jednie wskaźniki do odpowiednich miejsc w pamięci, gdzie obrazy są przechowywane.
Ogólny diagram klas znajduje się na rysunku \ref{fig:uml-global-sturcture}.

\begin{figure}[!htbp]
    \centering
    \includegraphics[width=\textwidth]{img/uml/global-sturcture.png}
    \caption{Diagram klas UML globalnej struktury programu.}
    \label{fig:uml-global-sturcture}
\end{figure}


\section{Struktury danych}

\subsection{Konwertowanie danych ze znaczników}
\input{text/implementacja/strukturydanych/conventer.tex}

\subsection{Scena}
\input{text/implementacja/strukturydanych/dicom-scene.tex}

\subsection{Kolekcje scen}
\input{text/implementacja/strukturydanych/dicom-scene-sets.tex}

\subsection{Zakładka}
\input{text/implementacja/strukturydanych/dicom-view.tex}

\subsection{Obiekt zakładek}
\input{text/implementacja/strukturydanych/dicom-tabs.tex}

\subsection{Okno główne programu}
\input{text/implementacja/strukturydanych/sokar-window.tex}


\section{Algorytmy}


\subsection{Cykl generowania obrazów}
\input{text/implementacja/algorytmy/pixmap-generate.tex}

\subsection{Generowania obraz monochromatycznego}
\input{text/implementacja/algorytmy/pixmap-monochrome.tex}

\subsection{Generowania obraz YBR}
\input{text/implementacja/algorytmy/pixmap-ybr.tex}

\subsection{Tworzenie transformat i ich użycie na obrazie}
\input{text/implementacja/algorytmy/pixmap-transformat.tex}

\subsection{Ustalanie pozycji pacjenta względem sceny}
\input{text/implementacja/algorytmy/image-orientation-indicator.tex}

\chapter{Kompilacja}

\section{Narzędzia potrzebne do kompilacji}

\section{Biblioteki potrzebne do kompilacji}

\section{Przegotowanie katalogów}

\section{Utworzenie plików budowania}

\section{Uruchomienie kompilacji}

\section{Przeniesienie plików do jednego folderu}

\chapter{Podsumowanie}
\par
Celem pracy było napisanie aplikacji do obsługi obrazów \DICOM w C++ z możliwością kompilacji na wiele platform.


\listoffigures

\bibliographystyle{plainnat}
\bibliography{bibliography} 

\end{document}
