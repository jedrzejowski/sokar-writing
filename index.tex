\documentclass{article}

\usepackage{polski}

\usepackage{titlesec}
\usepackage[T1]{fontenc}
\usepackage[utf8]{inputenc}
\usepackage{helvet}


\usepackage[autostyle]{csquotes}
\usepackage{xcolor}

\usepackage[pdftex]{hyperref}
\usepackage{listings}
\usepackage{array}

\usepackage[polish]{babel}
\usepackage{graphicx}
\usepackage{amsmath}

\usepackage{indentfirst}

\usepackage{prmag2017}


\newcommand{\dicomtag}[3] {#1(#2, #3)}

\newcommand{\wiki}[1]{https://pl.wikipedia.org/wiki/Teksel}

\newcommand{\gdcmclass}[1]{gdcm::#1}

\newcommand{\sokarclass}[1]{Sokar::#1}

\newcommand{\qtclass}[1]{Qt::#1}

\newcommand{\dicomvr}[1]{VR #1}

\newenvironment{conditions}
  {\par\vspace{\abovedisplayskip}\noindent\begin{tabular}{>{$}l<{$} @{${}={}$} l}}
  {\end{tabular}\par\vspace{\belowdisplayskip}}

\newcommand{\quotett}[1]{\enquote{\texttt{#1}}}

\newcommand{\cppcode}[1]{{\color{blue}\texttt{#1}}}

\inputencoding{utf8}


\def\utfMaleSign{\includegraphics[height=1em]{utf8char/malesign.pdf}}
\def\utfFemaleSign{\includegraphics[height=1em]{utf8char/femalesign.pdf}}


\author{Adam Jędrzejowski}
\title{Wieloplatformowa przeglądarka obrazów DICOM w C++}
\date{\today}

\begin{document}

\maketitle

cel pracy

rozdził wprowadzenie

obrazowe techniki diagnostyczne

coto są obrazy diagnostyczne, jakie one mogą być

jakie cechy posinna spełniać przglądrka obrazów

dlaczego tworzymy wieloplatformową w c++

Układ pracy

\part{Wstęp}
\input{1.medical-diagnostics}
\input{2.browsers}

\part{Standard DICOM}

\part{Biblioteki}

\part{Interfejs graficzny}

\part{Implementacja}

\section{Wieloplatformowość}
Przeglądarka jest napisana w taki sposób, że jej implementacja nie uwzględnia systemu operacyjnego na którym pracuje

Zróżnych perspektyw

\subsection{Język programowania}

Przeglądarka została napisana w C++ w standardzie z 2017 roku w skrócie C++17

\subsection{Środowisko programistyczne}

Do programowania głównie używałem CLion, IDE stworzonego przez firmę JetBrians.
Zdecydowaną większość czasu przeglądarka była testowana i debugowana na aktualizowanym systemie ArchLinux.

\subsection{Obiektowy model w oprogramowaniu}

Praca jest zaprojektowany w sposób obiektowy, za wyjątkiem kilku pomniejszych elementów.

\section{Konwertowanie i analiza danych w tagach}

\par
Każdy plik \DICOM posiada zbiór elementów danych.
Zapisane elementy danych należy przekonwertować na obiekty danych odpowiadające potrzebom programu.
Dlatego został zaimplementowany obiekt klasy \sokarclass{DataConverter} zajmujący się konwersją danych z pliku \DICOM na dane w formacie odpowiadającym programowi.

\par
Obiekt konwertera jest tworzony na podstawie pliku \DICOM i przy wywoływaniu konwersji należy podać tylko znacznik, który nas interesuje.
Takie rozwiązanie pozwala na przesłanie do wszystkich obiektów jednego względnie małego obiektu konwertera, co ułatwia zarządzanie dostępem do pliku \DICOM.

\par
Klasa \sokarclass{DataConverter} posiada następujące funkcje, pozwalające na konwertowanie danych:
\begin{itemize}
    \item \sokarfunction{DataConverter}{toString}

          Funkcja konwertuje element na obiekt tekstu \qtclass{QString}.

    \item \sokarfunction{DataConverter}{toAttributeTag}

          Funkcja konwertuje element o znaczniku typu \dicomvr{AT} na obiekt znacznika \gdcmclass{Tag}.

    \item \sokarfunction{DataConverter}{toAgeString}

          Funkcja konwertuje element o znaczniku typu \dicomvr{AS} na tekst w postaci czytelnej, np: „18 weeks” lub „3 years”.

    \item \sokarfunction{DataConverter}{toDate}

          Funkcja konwertuje element o znacznik typu \dicomvr{DA} na obiekt klasy \qtclass{QDate}, który ma w sobie wbudowaną konwersję na tekst zależny od ustawień językowych aplikacji.

    \item \sokarfunction{DataConverter}{toDecimalString}

          Funkcja konwertuje element o znacznik typu \dicomvr{DS} na obiekt wektora posiadającego liczby rzeczywiste.
          \cppcode{qreal} jest aliasem do typu zmiennoprzecinkowego, na systemach 64-bitowy jest to \cppcode{double}.

    \item \sokarfunction{DataConverter}{toIntegerString}

          Funkcja konwertuje element o znacznik typu \dicomvr{IS} na 32-bitową liczbę całkowitą (\cppcode{qint32}).

    \item \sokarfunction{DataConverter}{toPersonName}

          Funkcja konwertuje element o znacznik typu \dicomvr{PN} na obiekt tekst zawierający imię w formie pisanej.

    \item \sokarfunction{DataConverter}{toShort}

          Funkcja konwertuje element o znacznik typu \dicomvr{SS} na 16-bitowa liczbę całkowitą ze znakiem (\cppcode{qint16}).

    \item \sokarfunction{DataConverter}{toUShort}

          Funkcja konwertuje element o znacznik typu \dicomvr{US} na 16-bitowa liczbę całkowitą bez znaku (\cppcode{quint16}).

\end{itemize}
Oprócz powyższych funkcji jest jeszcze kilka innych funkcji pobocznych oraz kilka aliasów.

\par
Ogólne zasady konwersji, które się tyczą wszystkich danych:
\begin{itemize}
    \item Większość VR jest to zapisanych jako tekst, kodowanie i dekodowanie tekstu jest zapewniane przez bibliotekę.
    \item Większość danych może mieć kilka wartości oddzielonych backslashem \quotett{\textbackslash}, dlatego konwerter dla VR, w których standard przewiduje wiele wartości, zawsze zwraca wektor z tymi wartościami.
    \item Wszystkie dane są zapisane parzystą ilością bajtów, w przypadku tekstu dodaje się znak spacji na końcu danych.
          Taka spacja jest pomijana w analizie danych.
\end{itemize}



\section{DicomScene}

Jest to obiektem jednej ramki obrazu i jest odpowiedzialna za pośrednie wygenerowanie obrazu oraz jego wyświetlenie na ekranie.
Klasa dziedzicząca pośrednio po \qtclass{QGraphicsScene} przez \sokarclass{Scene}.

\subsection{Informacje wyświetlane na scenie}

Informacje na scenie są wyświetlane za pomocą obiektów, które dziedziczą po klasie \sokarclass{SceneIndicator}.
Obiekty te mają dostęp do obiektu konwertera.
Obiekty dziedziczące po \sokarclass{SceneIndicator} implementują róznież swoją pozycje na scenie i są wstanie ją zmieniać w raz ze zmianą wielkości sceny.


Domyślnie obiekty wyświetlające informacje (tytuły punktów to nazwy klas):
\subsubsection{\sokarclass{PatientDataIndicator}}

Obiekt wyświetlający dane pacjenta, pojawia się zawsze na scenie w lewym górnym rogu i zawiera następujące linie:
\begin{itemize}
    \item Nazwa pacjenta oraz płeć

          Nazwa pacjenta znajduje się w \dicomtag{PatientName}{0010}{0010} o \dicomvr{PN}.

          Płeć, zapisana jest w \dicomtag{PatientSex}{0010}{0040} i może mieć następujące wartości:
          \begin{itemize}
              \item \dataword{M } - oznacza mężczyznę, wyświetlana jako O
              \item \dataword{F } - oznacza kobietę, wyświetlana jako O
              \item \dataword{O } - oznacza inną płeć i nie jest wyświetlana
          \end{itemize}

          W przypadku określenia inne płci niż jest w standardzie bądź braku tagu płeć nie będzie widoczna.

          Przykład: \dataword{Adam Jędrzejowski O}.

    \item Identyfikator pacjenta

          Unikalny identyfikator pacjenta z tagu \dicomtag{PatientID}{0010}{0020} wyświetlane w takiej formie jakiej jest zapisane, bez żadnej obróbki.
          W praktyce najczęściej jest to numer z systemu używanego w danym szpitalu, rzadziej numer PESEL.

          Przykład: \dataword{HIS/000000}.

    \item Data urodzenia oraz wiek pacjenta w trakcie badania

          Data urodzenia znajdująca się w \dicomtag{PatientBirthDate}{0010}{0030} i jest zamieniana na format \dataword{YYYY-MM-DD}.
          Dodatkowo, jeżeli tag \dicomtag{PatientAge}{0010}{1010} jest obecny, wyświetlany jest także wiek pacjenta w czasie badania.

          Przykład: \dataword{born 1982-08-09, 28 years}.

    \item Opis wykonany przez instytucję opis lub klasyfikację badania (komponentu)

          Tekst brany z \dicomtag{StudyDescription}{0008}{1030} i wyświetlany bez żadnej obróbki.

          UWAGA: Ta wartość jest wpisywana przez technika, operatora lub lekarza wykonującego badanie, więc wartość ta może być nie przewidywalna.

    \item Opis serii

          Tekst brany z \dicomtag{SeriesDescription}{0008}{103E} i wyświetlany bez żadnej obróbki.

          UWAGA: Ta wartość jest wpisywana przez technika, operatora lub lekarza wykonującego badanie, więc wartość ta może być nie przewidywalna.
\end{itemize}

Przykład pełnego teksu:

\texttt{\\
    \textbf{Adam Jędrzejowski} O\\
    HIS/123456\\
    born 1996-07-16, 19 years\\
    Kregoslup ledzwiowy a-p + boczne\\
    AP
}

\subsubsection{\sokarclass{HospitalDataIndicator}}

Obiekt wyświetlający dane szpitala/instytucji, pojawia się zawsze na scenie w prawym górnym rogu i zawiera następujące linie:
\begin{itemize}
    \item Nazwa instytucji

          Tekst brany z \dicomtag{InstitutionalDepartmentName}{0008}{1040} i wyświetlany bez żadnej obróbki.

\end{itemize}

\subsubsection{\sokarclass{ImageOrientationIndicator}}

Obiekt wyświetlający cztery litery oznaczające orientacje obrazu w stosunku do pacjenta.
Obiekt posiada cztery pola: lewe, górne, prawe i dolne.

Każda z sześciu możliwych liter oznacza kierunek oraz zwrot w jakim jest ułożony pacjent:
\begin{itemize}
    \item \dataword{R} --- right --- część prawa pacjenta
    \item \dataword{L} --- left --- część
    \item \dataword{A} --- anterior --- przód pacjenta
    \item \dataword{P} --- posterior --- tył pacjenta
    \item \dataword{F} --- feet --- część dolna
    \item \dataword{H} --- head --- część górna
\end{itemize}

Pary poszczególnych liter tworzą osie:
\begin{itemize}
    \item \quotett{x} --- oś przechodząca od prawej do lewej strony pacjenta, \dataword{L} oznacza zwrot zgodny z osią, a \dataword{R} oznacza zwrot przeciwny

    \item \quotett{y} --- oś przechodząca od przodu do tyłu pacjenta, \dataword{P} oznacza zwrot zgodny z osią, a \dataword{A} oznacza zwrot przeciwny

    \item \quotett{z} --- oś przechodząca od dołu do góry pacjenta, \dataword{H} oznacza zwrot zgodny z osią, a \dataword{F} oznacza zwrot przeciwny

\end{itemize}

\begin{figure}[!htbp]
    \centering
    \includegraphics[width=0.7\textwidth]{img/imageorientationindicator-003.pdf}
    \caption{Wizualizacja układu osi współrzędnych pacjenta. Zdjęcie własne.}
    \label{fig:imageorientationindicator2}
\end{figure}

Informacje o orientacji oraz pozycji względem pacjenta znajdują się w odpowiednio w tagach \dicomtag{ImageOrientation}{0020}{0037} i \dicomtag{ImagePosition}{0020}{0032}.
Wartość \dicomtag{ImageOrientation}{0020}{0037} składa się z sześciu liczb, opowiednio oznaczanych dalej X\textsubscript{x}, X\textsubscript{y}, X\textsubscript{z}, Y\textsubscript{x}, Y\textsubscript{y}, Y\textsubscript{z}.

Standard DICOM definiuje, że te dane mają być z interpretowane w następujący sposób:
\[
    \begin{bmatrix}
        P_x \\ P_y \\ P_z \\ 1
    \end{bmatrix}
    =
    \begin{bmatrix}
        X_x\Delta_i & Y_x\Delta_j & 0 & S_x \\
        X_y\Delta_i & Y_y\Delta_j & 0 & S_y \\
        X_z\Delta_i & Y_z\Delta_j & 0 & S_z \\
        0           & 0           & 0 & 1
    \end{bmatrix}
    \begin{bmatrix}
        i \\ j \\ 0 \\ 1
    \end{bmatrix}
    =
    M
    \begin{bmatrix}
        i \\ j \\ 0 \\ 1
    \end{bmatrix}
\]
gdzie:
\begin{itemize}
    \item $P_{xyz}$ --- koordynaty woksela (i,j) w macierzy obrazu wyrażone w milimetrach
    \item $S_{xyz}$ --- trzy wartości z elementu ze znacznikiem \dicomtag{ImagePosition}{0020}{0032}. Oznacza punkt pozycji pacjenta wyrażony w milimetrach w stosunku do urządzenia wykonującego pomiar.
    \item $X_{xyz}$ --- trzy pierwsze wartości z \dicomtag{ImageOrientation}{0020}{0037}
    \item $Y_{xyz}$ --- trzy ostatnie wartości z \dicomtag{ImageOrientation}{0020}{0037}
    \item $i$ i $j$ --- oznaczają współrzędne na macierzy obrazu, odpowiednio kolumnę i wiersz. Zero oznacza początek.
    \item $\Delta_i$ i $\Delta_j$ --- rzeczywista wielkość piksela obrazu wyrażoną w milimetrach, w algorytmie wyznaczania strony pacjenta ta wartość, może wynosić 1, ponieważ odpowiada za skale
\end{itemize}

Praktycznie rzecz biorąc, pierwsza macierz to wektor reprezentujący pozycję pacjenta.
Druga jest to transformata.
Trzecia to pozycja na obrazie.

Interesuje nas wyznaczenie pozycji sześciu (punktów) na płaszczyźnie obrazu, o następujących współrzędnych, dalej używanych pod nazwą $PatientPosition$:
\begin{itemize}
    \item \quotett{R} - $[-1, 0, 0, 1]$
    \item \quotett{L} - $[+1, 0, 0, 1]$
    \item \quotett{A} - $[0, -1, 0, 1]$
    \item \quotett{P} - $[0, +1, 0, 1]$
    \item \quotett{F} - $[0, 0, -1, 1]$
    \item \quotett{H} - $[0, 0, +1, 1]$
\end{itemize}

UWAGA: Wszystkie obliczenia odbywają się w współrzędnych jednorodnych.

Wykonuje takie przekształcenie:
\[PatientPosition = imgMatrix * ScenePosition\]
\[imgMatrix^{-1} * PatientPosition = imgMatrix^{-1} * imgMatrix * ScenePosition\]
\[imgMatrix^{-1} * PatientPosition = ScenePosition\]
\[ScenePosition = imgMatrix^{-1} * PatientPosition\]
gdzie:
\begin{itemize}
    \item $imgMatrix$ --- macierz przekształcenia obrazu, o której będzie dalej
    \item $ScenePosition$ --- pozycja na obrazie, która naz interesuje
    \item $PatientPosition$ --- któryś z punktów względem pacjenta.
\end{itemize}

Wygląd macierzy $imgMatrix$:
\[
    \begin{bmatrix}
        X_x & Y_x & 0 & 0 \\
        X_y & Y_y & 0 & 0 \\
        X_z & Y_z & 0 & 0 \\
        0   & 0   & 0 & 1
    \end{bmatrix}
\]
Powyższa macierz różni się od macierzy definiowanej w standardzie.
Po pierwsze PikselSpacing został pominięty, a konkretniej nadałem mu wartość 1.
Po drugie pozycja z \dicomtag{ImagePosition}{0020}{0032} została zrównana do punktu zerowego, dzięki temu, wynik też będzie względem punktu zero.
Wyznaczenie macierzy $imgMatrix$ jest jednorazowe.

Po wyznaczeniu sześciu punktów $ScenePosition$, po jednej dla każdego punktu względem pacjenta są zapisywane. $ScenePosition$ odpowiada pozycji punktów na obrazie w pozycji startowej.

Na scenie, której jest wyświetlany obraz, użytkownik, może obracać obraz o dowolny kąt, według własnego uznania.
Te przekształcenia, są realizowane za pomocą macierzy rotacji, dalej znana jako $rotateTransform$.
Macierz $rotateTransform$ jest przesyłana do naszego obiektu \sokarclass{ImageOrientationIndicator} za każdym razem kiedy zostanie zmieniona.

Ostateczne wyznaczenie pozycji punktów pacjent na obrazie odbywa sie przez przemnożenie lewostronne $rotateTransform$ i $ScenePosition$.
\[rotateTransform * ScenePosition\]
Wyznaczane jest w ten sposób pozycja sześciu punktów pacjenta na płaszczyźnie sceny wyświetlanej.
Następnie określane jest na, której z ośmiu części płaszczyzny jest umieszczony dany punkt, podział płaszczyzny jest widoczny na rysunku \ref{fig:imageorientationindicator4}.
Tej płaszczyźnie nadawany jest tytuł w postaci litery, która oznacza stronę pacjenta.
Jeżeli punkt znajduje się w centrum, na przecięciu osi, to oznacza, że punkt znajduje się za lub przed ekranem, więc jest pomijany.
Następnie do czterech pól wyświetlających zostają wstawione następujące teksty:
\begin{itemize}
    \item lewe pole: tytuł części 7, tytuł części 0 i tytuł części 1
    \item górne pole: tytuł części 1, tytuł części 2 i tytuł części 3
    \item prawe pole: tytuł części 3, tytuł części 4 i tytuł części 5
    \item dolne pole: tytuł części 7, tytuł części 6 i tytuł części 5
\end{itemize}

Przykład:\\
Punkt \quotett{H}, czyli punkt reprezentujący kierunek głowy, został przypisany do części 1 i odpowiednio \quotett{L} do części 7, \quotett{R} do części 3 i \quotett{F} do części 5.
Punkty \quotett{A} i \quotett{P} zostały pominięte ponieważ znalazły się na środku.
Do lewego pola wstawiany jest tekst \quotett{HL}, do górnego \quotett{HR}, do prawego \quotett{RF} i do dolnego \quotett{LF}.

\begin{figure}[!htbp]
    \centering
    \includegraphics[width=\textwidth]{img/imageorientationindicator-004.png}
    \caption{Podział płaszczyzny sceny. Wyróżniono osiem części. Zdjęcie własne.} 
    \label{fig:imageorientationindicator4}
\end{figure}

Przykład można zobaczyć na rysunku \ref{fig:imageorientationindicator1}.
Na obrazie widzimy, że lewa strona pacjenta znajduje się po prawej stronie obrazu, prawa strona pacjenta po lewej, góra pacjenta na górnej części obrazu.
Wynika z tego, że obraz przedstawia pacjenta skierowanego twarzą do nas.

\begin{figure}[!htbp]
    \centering
    \includegraphics[width=100mm]{img/imageorientationindicator-002.png}
    \caption{Przykładowy obraz medyczny (przekrój głowy MR) z oznaczeniem orientacji obrazu z apomocą liter A, P, R, L, F, H. Zdjęcie własne.}
    \label{fig:imageorientationindicator1}
\end{figure}

\subsubsection{\sokarclass{PixelSpacingIndicator}}

Obiekt wyświetlający podziałkę informującą jakich rozmiarów jest obiekt na obrazie w rzeczywistości, pojawia się na dole i po prawie stronie sceny, gdy tag \dicomtag{PixelSpacing}{0028}{0030} jest obecny.
Wygląd podziałki można zaobserwować na rysunku \ref{fig:imageorientationindicator1}.

Podziałka dostosowuje swoją wielkość do obecnej sceny, jak i do innych elementów na scenie.
Wartości wyświetlane biorą pod uwagę transformatę skali i rotacji obrazu.

\subsubsection{\sokarclass{ModalityIndicator}}

Obiekt wyświetla informacje o akwizycji obrazu.
Dane różnią się w zależności od modalności obrazu.
Domyślnie zawierają następujące linie:
\begin{itemize}
    \item bla bla bla
    \item bla bla bla
    \item bla bla bla
    \item bla bla bla
\end{itemize}

W przypadku następujących modalności zawierają również następujące informacje:
\begin{itemize}
    \item bla bla bla
    \item bla bla bla
    \item bla bla bla
    \item bla bla bla
\end{itemize}

\subsection{Generowanie obrazów z danych}

Klasa \sokarclass{DicomScene} jest klasą abstrakcyjną i nie generuje obrazu, pozostawia do klasą dziedziczących po niej.

\paragraph{Cykl generowania obrazu}

Klasa \sokarclass{DicomScene} dostarcza następujące obiekty do generowania obrazu:
\begin{itemize}
    \item \cppcode{processing}, obiekt klasy \qtclass{QMutex} muteks do zablokowania podczas generowania obrazu, aby parametry obrazu nie mogły być zmienianie podczas jego generowania.

    \item \cppcode{imgDimX} zmienna typu \cppcode{uint}, oznacza szerokość obrazu w pikselach.

    \item \cppcode{imgDimY} zmienna typu \cppcode{uint}, oznacza wysokość obrazu w pikselach.

    \item \cppcode{targetBuffer} wektor docelowego obrazu RGB o długości $imgDimX*imgDimY$, typu \cppcode{std::vector<Pixel>}.

          \sokarclass{Pixel} to struktura reprezentujące piksel.
          Nie jest to w żadnym wypadku obiekt, a jedynie twór ułatwiający zarządzanie kodem.

          \begin{lstlisting}
struct Pixel {
    quint8 red = 0;   
    quint8 green = 0;    
    quint8 blue = 0;   
}\end{lstlisting}

    \item \cppcode{originBuffer} wektor danych wypełniona danymi z jednej ramki o długośći iloczynu $imgDimX*imgDimY$ i ilości bajtów jednego piksela obrazu.

    \item \cppcode{qImage} obiekt obrazu klasy \qtclass{QImage}.

          \qtclass{QImage} można zrobić z istniejącego bufora, w tym przypadku jest to \cppcode{targetBuffer}.
          Format obrazu to \qtclass{QImage::Format\_RGB888}, czyli trzy bajty, każdy na jeden kanał.
          Proszę zwrócić uwagę, że struktura \sokarclass{Pixel} odpowiada temu formatowi.
          Według dokumentacji Qt obiekt ten po utworzeniu z istniejącego bufora powinien z niego dalej korzystać, dlatego zmiany \cppcode{targetBuffer} nie wymagają odświeżania \cppcode{qImage}.

    \item \cppcode{pixmap} obiekt obrazu do wyświetlania, klasy \qtclass{QPixmap}.

          Obiektów klasy \qtclass{QImage} nie da się wyświetlić, nie jest on przystosowany do wyświetlania.
          Natomiast klasa \qtclass{QPixmap} to reprezentacja obrazu dostosowana do wyświetlania ekranie, która może być używana jako urządzenie do malowania w bibliotece Qt.

    \item \cppcode{iconPixmap} obiekt obrazu ikonu, klasy \qtclass{QPixmap}, docelowo powinien mieć 128 pikseli na 128 pikseli.

\end{itemize}

Generowanie obrazu jest robione przez czysto wirtualną funkcje \sokarfunction{DicomScene}{generatePixmap}.
Po wywołaniu funkcji obiekt \cppcode{targetBuffer} powinien zawierać obraz wygenerowany z obecnymi parametrami.
Funkcja zwraca również wartość logiczną, który informuje nas czy \cppcode{targetBuffer} rzeczywiście został zmieniony.
Następnie z obiekt \cppcode{pixmap} jest odświeżany na bazie \cppcode{qImage}.

Całe odświeżanie obrazu jest implementowane w funkcji \sokarfunction{DicomScene}{reloadPixmap}.
Funkcja wywołuje \sokarfunction{DicomScene}{generatePixmap} i odświeża \cppcode{pixmapItem} kiedy zajdzie taka potrzeba

Generowanie poszczególnych typów obrazów jest wyjaśnione poniżej.

\subsubsection{Monochorme}
\input{text/implementacja/pixmap-monochrome.tex}

\subsubsection{RGB}
\input{text/implementacja/pixmap-rgb.tex}

\subsubsection{YBR}
\input{text/implementacja/pixmap-ybr.tex}


\section{DicomView}

Każda zakładka z obrazem lub obrazami jest implementowana przez klasę \sokarclass{DicomView}.

Interfejs graficzny \sokarclass{DicomView} posiada następujące elementy:
\begin{itemize}
    \item pasek narzędzi znajdujący się na górze - implementowany za pomocą klasy \sokarclass{DicomToolBar}
    \item miejsce na scene z obrazem DICOM na środku - implementowany za pomocą klasy \sokarclass{DicomGraphics}
    \item suwak filmu w dolnej części - implementowany za pomocą klasy \sokarclass{MovieBar}
    \item podgląd miniaturek obrazów w prawej części - implementowany za pomocą klasy \sokarclass{FrameChooser}
\end{itemize}

Dodatkowo posiada obiekt \sokarclass{DicomSceneSet}, który jest zbiorem obrazów opisany w sekcji \ref{sec:scene-sets}.
\sokarclass{DicomView} łączy zdarzenia wysyłane przez wszystkie obiekty.

Poniżej jest opisane zachowanie tych elementów:

\subsection{Elementy interfejsu graficznego}

\subsubsection{\sokarclass{DicomToolBar}}

Jest to pasek narzędzi znajdujący się na górze \sokarclass{DicomView}.
Posiada on zespół ikonek z rozwijalnymi menu kontekstowymi.
Kliknięcie odpowiedniej ikony spowoduje wysłanie sygnału do obecnie wyświetlanej sceny.

Są dwa sygnału Qt wysyłane przez klase:
\begin{itemize}
    \item \cppcode{void stateToggleSignal(State state);}

    Sygnał ten oznacza zmianę stanu interfejsu

    \item \cppcode{void actionTriggerSignal(Action action, bool state = false);}
\end{itemize}


Ikony na pasku:
\begin{itemize}
    \item Okienkowanie
    \item Przesuwanie
    \item Skalowanie
    \item Rotacja
    \item Indicators
    \item Tagi - czerwona ikonka dwóch metek

    Kliknięcie 
\end{itemize}

\subsubsection{\sokarclass{DicomGraphics}}
\subsubsection{\sokarclass{MovieBar}}
\subsubsection{\sokarclass{FrameChooser}}

\subsection{Zależności pomiędzy elementami graficznymi}

\section{Tryb filmu - MovieMode}
\label{sec:sokar-scenesets}

Abstrakcyjna klasa \sokarclass{DicomSceneSet} implementuje kolekcje scen za pomocą wektora \qtclass{QVector}.
Jest to obiekt, który grupuje w jakiś sposób sceny a następnie tworzy obiekt \sokarclass{SceneSequence}, który jest rzeczywistą sekwencją scen, ułożoną w taki sposób, jaki obrazy powinny być wyświetlane.
Są dwie implementacje zbioru scen: zbiór plików i zbiór ramek z jednego pliku

\subsection{Sekwencja scen}
\label{sec:sokar-scenesequence}

\par
Sekwencja scen implementuje strukturę danych informującą o przejściach pomiędzy scenami poprzez klasę \sokarclass{SceneSequence}.
Sekwencja to wektor zawierającą kroki z dodatkowymi informacjami o stanie sekwencji.
Indeksem w którym obecnie znajduje się sekwencja.
Kierunkiem sekwencji, sekwencja może iść w stronę początku lub kocań.
Rodzajem przemiatania, jest to wartość logiczna informująca w jaki sposób ma zachować się gdy sekwencja dojdzie do końca, lub początku.
Po dojściu do końca sekwencja skoczy do pierwszego elementu lub może zmienić kierunek i zacząć iść do tyłu.

\par
Kroki, implementowane, przez klasą \sokarclass{Step}, zawierają następujące informacje: wskaźnik do sceny oraz czas trwania sceny.

\par
Sekwencja ma wbudowane funkcje zapewniające przesuwanie się po indeksie na wektorze:
\begin{itemize}
    \item \sokarfunction{SceneSequence}{stepForward} --- krok do przodu, zwiększa indeks tym samym wykonuje krok w stronę końca sekwencji
    \item \sokarfunction{SceneSequence}{stepBackward} --- krok do tyłu, zmniejsza indeks tym samym wykonując krok w stronę początku sekwencji
    \item \sokarfunction{SceneSequence}{step} --- wykonuje krok w tył lub przód w zależności od kierunku sekwencji
\end{itemize}
Wszystkie powyższe funkcje są zarazem slotami dla sygnałów oraz emitują sygnał \sokarfunction{SceneSequence}{steped}.

\subsection{Zbiór ramek DICOM}
\label{sec:sokar-dicomframeset}

\par
Zbiory ramek są implementowane przez \sokarclass{DicomFrameSet} i są tworzone z jednego wczytanego pliku DICOM.
Klasa tworzy obiekt konwertera i pobiera liczbę ramek w obrazie.
Tworzy jeden buffor na wszystkie ramki obrazów, a następnie dzieli go na ilość ramek.
Biblioteka GDCM nie daje dostępu do oryginalnego bufora, dlatego wymagany jest bufor pośredni.
Następnie jest tworzonych tyle obiektów scen ile jest ramek.
\par
Kolejność sekwencji scen jest taka sama jak kolejność ramek.
Natomiast czas wyświetlania ramki może być zapisany w różnych znacznikach.
To w którym tagu został zapisany informuje element o znaczniku \dicomtag{FrameIncrementPointer}{0028}{0009}, zawiera on wskaźnik do elementu o zadanym znaczniku i w zależności od znacznika.
Została zaimplementowana obsługa poniższy znaczników:
\begin{itemize}
    \item \dicomtag{FrameTime}{0018}{1063} --- element z tym znacznikiem zawiera czas trwania jednej ramki w milisekundach, każdemu krokowi jest przypisywana ta wartość trwania

    \item \dicomtag{FrameTimeVector}{0018}{1065} --- zawiera tablice z przyrostami czasu w milisekundach między n-tą ramką a poprzednią klatką. Pierwsza ramka ma zawsze przyrost czasu równy 0.
    
    \item \dicomtag{CineRate}{0018}{0040} --- zawiera ilość klatek wyświetlanych na sekunda, każdemu krokowi jest przypisywana wartość odwrotna do tej
\end{itemize}
W przypadku braku znacznika lub gdy zostaje wskazany znacznik nieznany, czas trwania ramki wynosi $83.3$ milisekundy, co odpowiada 12 klatkom na sekundę.


\subsection{Zbiór plików DICOM}
\label{sec:sokar-dicomfileset}
\par
Zbiory plików są implementowane prze \sokarclass{DicomFileSet} i służą do przechowywania wielu wczytanych plików DICOM.
Na początku pliki są sortowane na podstawie liczby zawartej w elemencie o znaczniku \dicomtag{TagInstanceNumber}{0020}{0x0013}.
Dla każdego pliku jest tworzony obiekt \sokarclass{DicomFrameSet}.
\par
Sekwencja jest tworzona na połączenie sekwencji poszczególnych obrazów.

\subsubsection{Segregowanie obrazów}
\label{sec:sokar-dicomfileset-create}
\par
W przypadku kiedy mamy do czynienia z wieloma plikami, należy jest rozdzielić na serie i uporządkować w odpowiedniej kolejności.
Unikalny identyfikator serii jest zawarty w elemencie danych znaczniku \dicomtag{TagSeriesInstanceUID}{0020}{000E}.
Kolejności obrazów w serii to liczba zawarta w elemencie danych o znaczniku \dicomtag{TagInstanceNumber}{0020}{0x0013}.
\par
Segregacja odbywa się za pomocą funkcji \sokarfunction{DicomFileSet}{create}.
Do funkcji jest przesyłany wektor z wczytanymi plikami DICOM, następnie dzieli ona pliki na zbiory zawierające zdjęcia tej samej serii, tworzy obiekty zbiorów plików DICOM, ostatecznie zwraca ona wektor z gotowymi obiektami zbiorów plików DICOM.
Sortowanie plików DICOM według ich kolejności odbywa się za pomocą funkcji \stdclass{sort} wewnątrz konstruktora klasy \sokarclass{DicomFileSet}, który nie jest publiczny.


\end{document}
